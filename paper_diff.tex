%DIF 1c1
%DIF LATEXDIFF DIFFERENCE FILE
%DIF DEL /tmp/original_draft.tex   Sun Dec 14 23:18:32 2025
%DIF ADD paper.tex                 Sun Dec 14 23:09:46 2025
%DIF < \documentclass[12pt,a4paper]{article}
%DIF -------
\documentclass[11pt,a4paper]{article} %DIF > 
%DIF -------

%==============================================================================
% PACKAGES AND CONFIGURATION
%==============================================================================
\usepackage[utf8]{inputenc}
\usepackage[T1]{fontenc}
%DIF 8c8
%DIF < \usepackage{mathpazo}           % Palatino: A standard, serious font for economic theory
%DIF -------
\usepackage{mathpazo}           % Palatino: professional font for economic theory %DIF > 
%DIF -------
\usepackage{microtype}          % Micro-typography for professional appearance
\usepackage{amsmath,amssymb,amsthm,mathtools}
\usepackage{bbm}
\usepackage{enumitem}
%DIF 13c13
%DIF < \usepackage[margin=1.25in]{geometry} % Generous margins for readability
%DIF -------
\usepackage[margin=1in]{geometry} % Tight Econometrica margins %DIF > 
%DIF -------
\usepackage{setspace}
\usepackage{graphicx}
\usepackage{booktabs}
\usepackage[authoryear]{natbib}
\usepackage{titlesec}
\usepackage{xcolor}
\usepackage[colorlinks=true, linkcolor=DarkBlue, citecolor=DarkBlue, urlcolor=DarkBlue]{hyperref}
\usepackage{cleveref}

% Define Econometrica-style dark blue
\definecolor{DarkBlue}{rgb}{0.0, 0.0, 0.55}

% Section formatting: Sober and academic
\titleformat{\section}{\large\bfseries\scshape}{\thesection.}{1em}{}
\titleformat{\subsection}{\bfseries\itshape}{\thesubsection.}{1em}{}
%DIF 29c29
%DIF < \titlespacing*{\section}{0pt}{3.5ex plus 1ex minus .2ex}{2.3ex plus .2ex}
%DIF -------
\titlespacing*{\section}{0pt}{2.5ex plus 1ex minus .2ex}{1.5ex plus .2ex} %DIF > 
%DIF -------

%DIF 31-34c31-34
%DIF < % Spacing configuration
%DIF < \onehalfspacing
%DIF < \setlength{\parskip}{0.6em}
%DIF < \setlength{\parindent}{1.5em}
%DIF -------
% Tight spacing configuration %DIF > 
\setstretch{1.1} %DIF > 
\setlength{\parskip}{0.4em} %DIF > 
\setlength{\parindent}{1.2em} %DIF > 
%DIF -------

%==============================================================================
% THEOREM ENVIRONMENTS
%==============================================================================
\newtheorem{theorem}{Theorem}
%DIF 40-43c40-42
%DIF < \newtheorem{lemma}{Lemma}
%DIF < \newtheorem{proposition}{Proposition}
%DIF < \newtheorem{corollary}{Corollary}
%DIF < 
%DIF -------
\newtheorem{lemma}[theorem]{Lemma} %DIF > 
\newtheorem{proposition}[theorem]{Proposition} %DIF > 
\newtheorem{corollary}[theorem]{Corollary} %DIF > 
%DIF -------
\theoremstyle{definition}
%DIF 45c44
%DIF < \newtheorem{definition}{Definition}
%DIF -------
\newtheorem{definition}[theorem]{Definition} %DIF > 
%DIF -------
\newtheorem{assumption}{Assumption}
%DIF 47a46
\newtheorem{remark}{Remark} %DIF > 
%DIF -------
\newtheorem{example}{Example}

%DIF 49-51d49
%DIF < \theoremstyle{remark}
%DIF < \newtheorem{remark}{Remark}
%DIF < 
%DIF -------
%==============================================================================
% MACROS
%==============================================================================
\newcommand{\E}{\mathbb{E}}
\newcommand{\R}{\mathbb{R}}
\newcommand{\N}{\mathcal{N}}
%DIF 58-63c55-60
%DIF < \newcommand{\I}{\mathcal{I}}         % Set of industries
%DIF < \newcommand{\Sset}{\mathcal{S}}      % Signal space
%DIF < \newcommand{\Lcal}{\mathcal{L}}      % Lattice
%DIF < \newcommand{\Mcal}{\mathcal{M}}      % Multiplier
%DIF < \newcommand{\s}{\mathbf{s}}          % Signal vector
%DIF < \newcommand{\p}{\mathbf{p}}          % Price vector
%DIF -------
\newcommand{\1}{\mathbbm{1}} %DIF > 
\newcommand{\I}{\mathcal{I}} %DIF > 
\newcommand{\Sset}{\mathcal{S}} %DIF > 
\newcommand{\Pcal}{\mathcal{P}} %DIF > 
\newcommand{\BR}{\mathrm{BR}} %DIF > 
\newcommand{\s}{\mathbf{s}} %DIF > 
%DIF -------
\DeclareMathOperator*{\argmin}{arg\,min}
\DeclareMathOperator*{\argmax}{arg\,max}
\DeclareMathOperator{\Cov}{Cov}
\DeclareMathOperator{\Var}{Var}

%==============================================================================
% TITLE PAGE
%==============================================================================
\title{\DIFdelbegin \DIFdel{\textbf{Sentiment and Supply Chains:\\ Endogenous Production Networks under Uncertainty}}%DIFDELCMD < \thanks{We are grateful to Daron Acemoglu, Pablo Azar, and seminar participants for detailed comments. This paper generalizes the framework of \citet{acemoglu2020endogenous} to environments with dispersed information. All errors are our own.}%%%
\DIFdelend \DIFaddbegin \DIFadd{\textbf{Sentiment and Supply Chains:\\ How Beliefs Shape Production Networks}}\thanks{We are grateful to Daron Acemoglu, Pablo Azar, and seminar participants for detailed comments. This paper generalizes the framework of \citet{acemoglu2020endogenous} to environments with dispersed information.}\DIFaddend }
%DIF 74d71
%DIF < 
%DIF -------
\author{
    \textsc{Author Name}\thanks{Department of Economics, University Name. Email: author@university.edu}
}
%DIF 78d74
%DIF < 
%DIF -------
\date{\today}
%DIF PREAMBLE EXTENSION ADDED BY LATEXDIFF
%DIF UNDERLINE PREAMBLE %DIF PREAMBLE
\RequirePackage[normalem]{ulem} %DIF PREAMBLE
\RequirePackage{color}\definecolor{RED}{rgb}{1,0,0}\definecolor{BLUE}{rgb}{0,0,1} %DIF PREAMBLE
\providecommand{\DIFaddtex}[1]{{\protect\color{blue}\uwave{#1}}} %DIF PREAMBLE
\providecommand{\DIFdeltex}[1]{{\protect\color{red}\sout{#1}}}                      %DIF PREAMBLE
%DIF SAFE PREAMBLE %DIF PREAMBLE
\providecommand{\DIFaddbegin}{} %DIF PREAMBLE
\providecommand{\DIFaddend}{} %DIF PREAMBLE
\providecommand{\DIFdelbegin}{} %DIF PREAMBLE
\providecommand{\DIFdelend}{} %DIF PREAMBLE
\providecommand{\DIFmodbegin}{} %DIF PREAMBLE
\providecommand{\DIFmodend}{} %DIF PREAMBLE
%DIF FLOATSAFE PREAMBLE %DIF PREAMBLE
\providecommand{\DIFaddFL}[1]{\DIFadd{#1}} %DIF PREAMBLE
\providecommand{\DIFdelFL}[1]{\DIFdel{#1}} %DIF PREAMBLE
\providecommand{\DIFaddbeginFL}{} %DIF PREAMBLE
\providecommand{\DIFaddendFL}{} %DIF PREAMBLE
\providecommand{\DIFdelbeginFL}{} %DIF PREAMBLE
\providecommand{\DIFdelendFL}{} %DIF PREAMBLE
%DIF HYPERREF PREAMBLE %DIF PREAMBLE
\providecommand{\DIFadd}[1]{\texorpdfstring{\DIFaddtex{#1}}{#1}} %DIF PREAMBLE
\providecommand{\DIFdel}[1]{\texorpdfstring{\DIFdeltex{#1}}{}} %DIF PREAMBLE
\newcommand{\DIFscaledelfig}{0.5}
%DIF HIGHLIGHTGRAPHICS PREAMBLE %DIF PREAMBLE
\RequirePackage{settobox} %DIF PREAMBLE
\RequirePackage{letltxmacro} %DIF PREAMBLE
\newsavebox{\DIFdelgraphicsbox} %DIF PREAMBLE
\newlength{\DIFdelgraphicswidth} %DIF PREAMBLE
\newlength{\DIFdelgraphicsheight} %DIF PREAMBLE
% store original definition of \includegraphics %DIF PREAMBLE
\LetLtxMacro{\DIFOincludegraphics}{\includegraphics} %DIF PREAMBLE
\newcommand{\DIFaddincludegraphics}[2][]{{\color{blue}\fbox{\DIFOincludegraphics[#1]{#2}}}} %DIF PREAMBLE
\newcommand{\DIFdelincludegraphics}[2][]{% %DIF PREAMBLE
\sbox{\DIFdelgraphicsbox}{\DIFOincludegraphics[#1]{#2}}% %DIF PREAMBLE
\settoboxwidth{\DIFdelgraphicswidth}{\DIFdelgraphicsbox} %DIF PREAMBLE
\settoboxtotalheight{\DIFdelgraphicsheight}{\DIFdelgraphicsbox} %DIF PREAMBLE
\scalebox{\DIFscaledelfig}{% %DIF PREAMBLE
\parbox[b]{\DIFdelgraphicswidth}{\usebox{\DIFdelgraphicsbox}\\[-\baselineskip] \rule{\DIFdelgraphicswidth}{0em}}\llap{\resizebox{\DIFdelgraphicswidth}{\DIFdelgraphicsheight}{% %DIF PREAMBLE
\setlength{\unitlength}{\DIFdelgraphicswidth}% %DIF PREAMBLE
\begin{picture}(1,1)% %DIF PREAMBLE
\thicklines\linethickness{2pt} %DIF PREAMBLE
{\color[rgb]{1,0,0}\put(0,0){\framebox(1,1){}}}% %DIF PREAMBLE
{\color[rgb]{1,0,0}\put(0,0){\line( 1,1){1}}}% %DIF PREAMBLE
{\color[rgb]{1,0,0}\put(0,1){\line(1,-1){1}}}% %DIF PREAMBLE
\end{picture}% %DIF PREAMBLE
}\hspace*{3pt}}} %DIF PREAMBLE
} %DIF PREAMBLE
\LetLtxMacro{\DIFOaddbegin}{\DIFaddbegin} %DIF PREAMBLE
\LetLtxMacro{\DIFOaddend}{\DIFaddend} %DIF PREAMBLE
\LetLtxMacro{\DIFOdelbegin}{\DIFdelbegin} %DIF PREAMBLE
\LetLtxMacro{\DIFOdelend}{\DIFdelend} %DIF PREAMBLE
\DeclareRobustCommand{\DIFaddbegin}{\DIFOaddbegin \let\includegraphics\DIFaddincludegraphics} %DIF PREAMBLE
\DeclareRobustCommand{\DIFaddend}{\DIFOaddend \let\includegraphics\DIFOincludegraphics} %DIF PREAMBLE
\DeclareRobustCommand{\DIFdelbegin}{\DIFOdelbegin \let\includegraphics\DIFdelincludegraphics} %DIF PREAMBLE
\DeclareRobustCommand{\DIFdelend}{\DIFOaddend \let\includegraphics\DIFOincludegraphics} %DIF PREAMBLE
\LetLtxMacro{\DIFOaddbeginFL}{\DIFaddbeginFL} %DIF PREAMBLE
\LetLtxMacro{\DIFOaddendFL}{\DIFaddendFL} %DIF PREAMBLE
\LetLtxMacro{\DIFOdelbeginFL}{\DIFdelbeginFL} %DIF PREAMBLE
\LetLtxMacro{\DIFOdelendFL}{\DIFdelendFL} %DIF PREAMBLE
\DeclareRobustCommand{\DIFaddbeginFL}{\DIFOaddbeginFL \let\includegraphics\DIFaddincludegraphics} %DIF PREAMBLE
\DeclareRobustCommand{\DIFaddendFL}{\DIFOaddendFL \let\includegraphics\DIFOincludegraphics} %DIF PREAMBLE
\DeclareRobustCommand{\DIFdelbeginFL}{\DIFOdelbeginFL \let\includegraphics\DIFdelincludegraphics} %DIF PREAMBLE
\DeclareRobustCommand{\DIFdelendFL}{\DIFOaddendFL \let\includegraphics\DIFOincludegraphics} %DIF PREAMBLE
%DIF LISTINGS PREAMBLE %DIF PREAMBLE
\RequirePackage{listings} %DIF PREAMBLE
\RequirePackage{color} %DIF PREAMBLE
\lstdefinelanguage{DIFcode}{ %DIF PREAMBLE
%DIF DIFCODE_UNDERLINE %DIF PREAMBLE
  moredelim=[il][\color{red}\sout]{\%DIF\ <\ }, %DIF PREAMBLE
  moredelim=[il][\color{blue}\uwave]{\%DIF\ >\ } %DIF PREAMBLE
} %DIF PREAMBLE
\lstdefinestyle{DIFverbatimstyle}{ %DIF PREAMBLE
	language=DIFcode, %DIF PREAMBLE
	basicstyle=\ttfamily, %DIF PREAMBLE
	columns=fullflexible, %DIF PREAMBLE
	keepspaces=true %DIF PREAMBLE
} %DIF PREAMBLE
\lstnewenvironment{DIFverbatim}{\lstset{style=DIFverbatimstyle}}{} %DIF PREAMBLE
\lstnewenvironment{DIFverbatim*}{\lstset{style=DIFverbatimstyle,showspaces=true}}{} %DIF PREAMBLE
%DIF END PREAMBLE EXTENSION ADDED BY LATEXDIFF

\begin{document}


\maketitle

\begin{abstract}
\noindent We study \DIFdelbegin \DIFdel{the formation of production networks when firms possess dispersed, affiliated information }\DIFdelend \DIFaddbegin \DIFadd{production network formation when firms have private, correlated signals }\DIFaddend about aggregate productivity. \DIFdelbegin \DIFdel{Firms make extensive margindecisions—choosing supplier sets—under uncertainty, anticipating that input prices will reflect the aggregate network structure. We characterize this economy as a supermodular Bayesian game }\DIFdelend \DIFaddbegin \DIFadd{Each firm chooses supplier links (the extensive margin) and input quantities (the intensive margin) under general production technologies satisfying homogeneity and labor essentiality. When technologies exhibit input complementarities and signals are affiliated, the induced Bayesian game features strategic complementarities}\DIFaddend . Using lattice-theoretic methods, we prove \DIFdelbegin \DIFdel{the }\DIFdelend existence of extremal monotone Bayesian Nash equilibria \DIFdelbegin \DIFdel{where firms with optimistic private signals adopt denser setsof inputs. We identify a ``sentiment multiplier'' mechanism: because signals are affiliated, a firm's optimism rationally raises its expectation of others' optimism, leading to coordinated network expansions that lower equilibrium prices and validate the initial sentiment.
We decompose network volatility into fundamental and strategic components, showing that higher signal correlation amplifies volatility while higher signal precision dampens it.
}\DIFdelend \DIFaddbegin \DIFadd{in which firms with higher signals adopt weakly larger supplier sets. We then derive comparative statics that keep beliefs at the center of the analysis: adoption costs shift networks in the standard direction, while first-order shifts in interim beliefs generate an additional strategic channel operating through expectations about others' expansion and the resulting equilibrium price system.
}\end{abstract}
\DIFaddend 

\DIFdelbegin %DIFDELCMD < \bigskip
%DIFDELCMD < %%%
\DIFdelend \DIFaddbegin \medskip
\DIFaddend \noindent\textbf{Keywords:} Production networks, dispersed information, strategic complementarities, supermodular games, \DIFdelbegin \DIFdel{affiliation, sentiment multiplier.
}\DIFdelend \DIFaddbegin \DIFadd{P-matrices, affiliation
}\DIFaddend 

\DIFaddbegin \medskip
\DIFaddend \noindent\textbf{JEL Codes:} D21, D83, D85, L14, \DIFdelbegin \DIFdel{E32.
}%DIFDELCMD < \end{abstract}
%DIFDELCMD < %%%
\DIFdelend \DIFaddbegin \DIFadd{L23
}\DIFaddend 

\DIFdelbegin %DIFDELCMD < \thispagestyle{empty}
%DIFDELCMD < \newpage
%DIFDELCMD < \setcounter{page}{1}
%DIFDELCMD < %%%
\DIFdelend \DIFaddbegin \onehalfspacing
\DIFaddend 

%==============================================================================
\section{Introduction}\DIFaddbegin \label{sec:introduction}
\DIFaddend %==============================================================================

\DIFdelbegin \DIFdel{Supply chains are constructed, not endowed. When a firm decides to source specialized inputs or adopt a logistics technology, it does so based on imperfect information }\DIFdelend \DIFaddbegin \DIFadd{Supplier relationships are formed under uncertainty. Establishing a new input relationship typically requires search, contracting, and relationship-specific investments. These choices are rarely made with complete information about aggregate conditions or about upstream capacity. Instead, firms rely on partial and noisy indicators---delivery delays, procurement quotes, and local shortages}\DIFaddend . \DIFdelbegin \DIFdel{Procurement managers observe local signals—order book depth, supplier quotes, industry chatter—that reflect both idiosyncratic factors and aggregate conditions. }\DIFdelend Because these \DIFaddbegin \DIFadd{indicators reflect common macroeconomic and sectoral forces, different firms' }\DIFaddend signals are correlated\DIFdelbegin \DIFdel{across firms, they generate a complex inference problem: a firm must forecast not only the fundamental productivity of the economy but also the sourcing decisions of other firms, which determine the equilibrium price of inputs}\DIFdelend .

This paper \DIFdelbegin \DIFdel{investigates how dispersed , affiliated }\DIFdelend \DIFaddbegin \DIFadd{studies how dispersed and correlated }\DIFaddend information shapes the endogenous formation of production networks. \DIFdelbegin \DIFdel{We depart from the standard literature, which typically treats networks as passive transmitters of shocks, by modeling the network as an equilibrium outcome of belief-driven investment. We show that when firms cannot perfectly }\DIFdelend \DIFaddbegin \DIFadd{In our setting, networks are not passive objects that merely transmit shocks. They are equilibrium outcomes of firms' technology choices. When firms cannot }\DIFaddend disentangle fundamentals from correlated noise, \DIFdelbegin \DIFdel{``sentiment '' becomes a driver of real economic structure}\DIFdelend \DIFaddbegin \DIFadd{changes in sentiment can reorganize the network itself, and the resulting change in input prices feeds back into further technology adoption}\DIFaddend .

We \DIFdelbegin \DIFdel{develop a model of network formation under private information, building on the cost-minimization }\DIFdelend \DIFaddbegin \DIFadd{build on the endogenous network }\DIFaddend framework of \citet{acemoglu2020endogenous}. \DIFaddbegin \DIFadd{Firms operate constant-returns technologies with labor essentiality. Given a supplier set, a firm chooses labor and intermediate inputs to minimize unit costs; it then chooses the supplier set that minimizes unit costs at prevailing prices, taking into account per-link adoption costs. }\DIFaddend We depart from the \DIFdelbegin \DIFdel{complete information }\DIFdelend \DIFaddbegin \DIFadd{complete-information }\DIFaddend benchmark by assuming \DIFdelbegin \DIFdel{aggregate productivity }\DIFdelend \DIFaddbegin \DIFadd{that the aggregate productivity state }\DIFaddend is unobserved. \DIFdelbegin \DIFdel{Instead, firms receive private signals that }\DIFdelend \DIFaddbegin \DIFadd{Each firm instead observes a private signal about the state. Signals }\DIFaddend are \emph{affiliated} in the sense of \citet{milgrom1982theory}\DIFaddbegin \DIFadd{, so a higher signal makes a firm more optimistic about fundamentals and, crucially, more optimistic about other firms' signals.
}

\DIFadd{Affiliation introduces a strategic channel that is distinct from the technological channel emphasized in complete-information models. With input complementarities, a firm's gain from expanding its supplier set is higher when other firms expand because expansion lowers equilibrium input prices. Under affiliation, a firm's signal is also informative about others' expansion. The interaction of these forces generates strategic complementarities in the Bayesian game of network formation}\DIFaddend .
\DIFdelbegin \DIFdel{This information structure introduces a distinct strategic channel . In standard production network models, complementarities are technological : low upstream prices encourage downstream expansion. In our environment, complementaritiesare also inferential. An optimistic firmexpands not only because it believes fundamentals are strong, but because it expects othersto be optimistic—and thus to expand, lowering the general price level.
}\DIFdelend 

Our analysis \DIFdelbegin \DIFdel{proceeds in four steps. }%DIFDELCMD < 

%DIFDELCMD < %%%
\DIFdel{First, in \mbox{%DIFAUXCMD
\cref{sec:model}}\hspace{0pt}%DIFAUXCMD
, we integrate the physical production environment with the information structure. We define a Bayesian game where firms choose supplier sets to minimize expected unit costs, conditional on their private signals and the anticipated equilibrium price system. We impose }\DIFdelend \DIFaddbegin \DIFadd{delivers two sets of results. First, we show that the network formation problem is a supermodular Bayesian game under }\DIFaddend general conditions on production technology \DIFdelbegin \DIFdel{(homogeneity and labor essentiality) that ensure the price system is well-behaved. }%DIFDELCMD < 

%DIFDELCMD < %%%
\DIFdel{Second, in \mbox{%DIFAUXCMD
\cref{sec:equilibrium}}\hspace{0pt}%DIFAUXCMD
, we establish the existence of }\textbf{\DIFdel{monotone Bayesian Nash equilibria}}%DIFAUXCMD
\DIFdel{. Although the space of potential networks is high-dimensional and discrete, we show that the game is supermodular. Firms with more optimistic signals monotonically expand their supplier sets, generating distinct ``high density'' (optimistic) and ``low density'' (pessimistic)regimes for the same underlying fundamentals.
This existence proof relies on lattice-theoretic methods (Tarski's Fixed Point Theorem) applied to the interim strategy space. }\DIFdelend \DIFaddbegin \DIFadd{and on the information structure. We then apply lattice methods to prove existence of extremal monotone Bayesian Nash equilibria. In these equilibria, firms with higher signals adopt weakly larger supplier sets. Second, we use the same monotone structure to derive comparative statics. Lower adoption costs expand equilibrium networks in the standard sense. More importantly for our setting, a first-order shift toward more optimistic interim beliefs expands equilibrium networks through both a direct channel (higher expected fundamentals) and a strategic channel (higher expected expansion by others and therefore lower expected input prices).
}\DIFaddend 

\DIFdelbegin \DIFdel{Third, in \mbox{%DIFAUXCMD
\cref{sec:statics}}\hspace{0pt}%DIFAUXCMD
, we identify a }\textbf{\DIFdel{sentiment multiplier}}%DIFAUXCMD
\DIFdel{. We decompose the network response to a belief shock into a direct fundamental effect and a strategic amplification effect.
Because beliefs are affiliated, optimism begets optimism.
The strategic channel amplifies the fundamental response, generating excess volatility in networkdensity relative to TFP shocks.
We further provide comparative statics, showing that increasing the precision of private signals reduces volatility, while increasing the correlation of forecast errors amplifies it.
}\DIFdelend \DIFaddbegin \paragraph{\DIFadd{Related literature.}}
\DIFadd{This paper contributes to the production networks literature \mbox{%DIFAUXCMD
\citep{long1983real, horvath2000sectoral, gabaix2011granular, acemoglu2012network} }\hspace{0pt}%DIFAUXCMD
and to models of endogenous network formation \mbox{%DIFAUXCMD
\citep{oberfield2018theory, acemoglu2020endogenous}}\hspace{0pt}%DIFAUXCMD
. On supply-chain disruptions and propagation, see \mbox{%DIFAUXCMD
\citet{barrot2016input, carvalho2021supply}}\hspace{0pt}%DIFAUXCMD
. On uncertainty and dispersed information, see \mbox{%DIFAUXCMD
\citet{bloom2009impact, bloom2018really, jurado2015measuring}}\hspace{0pt}%DIFAUXCMD
. Methodologically, we rely on the theory of supermodular games \mbox{%DIFAUXCMD
\citep{topkis1998supermodularity, milgrom1994monotone} }\hspace{0pt}%DIFAUXCMD
and on existence results for monotone equilibria in Bayesian games \mbox{%DIFAUXCMD
\citep{van2007monotone}}\hspace{0pt}%DIFAUXCMD
.
}\DIFaddend 

\DIFdelbegin \DIFdel{Fourth, in \mbox{%DIFAUXCMD
\cref{sec:dynamic} }\hspace{0pt}%DIFAUXCMD
, we sketch }\DIFdelend \DIFaddbegin \paragraph{\DIFadd{Roadmap.}}
\DIFadd{\mbox{%DIFAUXCMD
\cref{sec:information} }\hspace{0pt}%DIFAUXCMD
introduces the information structure.
\mbox{%DIFAUXCMD
\cref{sec:environment} }\hspace{0pt}%DIFAUXCMD
defines the production environment.
\mbox{%DIFAUXCMD
\cref{sec:equilibrium_characterization} }\hspace{0pt}%DIFAUXCMD
collects the equilibrium definition and characterizes equilibrium prices conditional on a network.
\mbox{%DIFAUXCMD
\cref{sec:complementarities} }\hspace{0pt}%DIFAUXCMD
establishes strategic complementarities.
\mbox{%DIFAUXCMD
\cref{sec:monotone_equilibria} }\hspace{0pt}%DIFAUXCMD
proves existence of extremal monotone equilibria.
\mbox{%DIFAUXCMD
\cref{sec:comparative} }\hspace{0pt}%DIFAUXCMD
presents comparative statics.
\mbox{%DIFAUXCMD
\cref{sec:domar} }\hspace{0pt}%DIFAUXCMD
derives belief-adjusted Domar weights.
\mbox{%DIFAUXCMD
\cref{sec:dynamic} }\hspace{0pt}%DIFAUXCMD
sketches }\DIFaddend a dynamic extension\DIFdelbegin \DIFdel{with adjustment costs, illustrating how temporary sentiment shocks can have persistent effects on network architecture through hysteresis. \mbox{%DIFAUXCMD
\cref{sec:concl} }\hspace{0pt}%DIFAUXCMD
}\DIFdelend \DIFaddbegin \DIFadd{, and \mbox{%DIFAUXCMD
\cref{sec:conclusion} }\hspace{0pt}%DIFAUXCMD
}\DIFaddend concludes.

%==============================================================================
\section{\DIFdelbegin \DIFdel{The }\DIFdelend Model}\label{sec:model}
%==============================================================================

We consider \DIFdelbegin \DIFdel{a static economy populated by }\DIFdelend \DIFaddbegin \DIFadd{an economy with }\DIFaddend $n$ industries\DIFdelbegin \DIFdel{, indexed by $\I = \{1, \ldots, n\}$. The economy is characterized by its production technology and the information structure facing firms. We integrate these elements to define the network formation game.
}%DIFDELCMD < 

%DIFDELCMD < %%%
\subsection{\DIFdel{Technology and Costs}}
%DIFAUXCMD
\addtocounter{subsection}{-1}%DIFAUXCMD
%DIFDELCMD < 

%DIFDELCMD < %%%
\DIFdel{Each firm }\DIFdelend \DIFaddbegin \DIFadd{. Each industry }\DIFaddend $i$ produces a \DIFdelbegin \DIFdel{distinct }\DIFdelend \DIFaddbegin \DIFadd{differentiated }\DIFaddend good using labor \DIFdelbegin \DIFdel{$L_i$ and a set of intermediate inputs . The production process involves two distinct decisions: an }\emph{\DIFdel{ex ante}} %DIFAUXCMD
\DIFdel{choice of technology (the set of suppliers) and an }\emph{\DIFdel{ex post}} %DIFAUXCMD
\DIFdel{choice of factor quantities. }%DIFDELCMD < 

%DIFDELCMD < %%%
\DIFdel{Let $S_i \subseteq \I \setminus \{i\}$ denote the set of suppliers chosen by firm $i$. Given this extensive margin choice, the production function is }\begin{displaymath}
    \DIFdel{Y_i = A_i(\mu) F_i(S_i, L_i, \{X_{ij}\}_{j \in S_i}),
}\end{displaymath}%DIFAUXCMD
\DIFdel{where $A_i(\mu)$ is a productivity shifter strictly increasing in the aggregate state }\DIFdelend \DIFaddbegin \DIFadd{and intermediate inputs purchased from a subset of other industries. Production is subject to an aggregate productivity shock }\DIFaddend $\mu$ \DIFdelbegin \DIFdel{, and $F_i$ is a constant returns to scale (CRS) aggregator. We assume labor is essential to production, ensuring bounded output and well-defined prices.
}%DIFDELCMD < 

%DIFDELCMD < %%%
\DIFdel{Markets are perfectly contestable. Once the state $\mu$ is realized and networks $S = (S_1, \ldots, S_n)$ are formed, firms choose quantities to minimize costs. The resulting unit cost function for firm $i$ is
}\begin{displaymath}\DIFdel{\label{eq:unit_cost}
    K_i(S_i, \mu, P) = \frac{1}{A_i(\mu)} \min_{L_i, \{X_{ij}\}} \left\{ L_i + \sum_{j \in S_i} P_j X_{ij} \;\Bigg|\; F_i(\cdot) = 1 \right\}.
}\end{displaymath}%DIFAUXCMD
\DIFdel{Equilibrium prices $P^*(\mu, S)$ are the fixed point of the system $P_i = K_i(S_i, \mu, P)$. As established in \mbox{%DIFAUXCMD
\citet{acemoglu2020endogenous}}\hspace{0pt}%DIFAUXCMD
, the essentiality of labor guarantees that for any network $S$, a unique positive price vector exists. Crucially, prices are monotone in the network structure: if any firm$j$ expands its supplier set $S_j$ to $S'_j \supset S_j$, the unit costs of firm $j$ fall (weakly), reducing prices for all downstream firms}\DIFdelend \DIFaddbegin \DIFadd{that is not directly observed; instead, each firm receives a private signal correlated with the shock. The model generates strategic complementarities in network formation because a firm's inference about fundamentals is also an inference about other firms' network choices}\DIFaddend .

\subsection{Information and \DIFdelbegin \DIFdel{Beliefs}\DIFdelend \DIFaddbegin \DIFadd{beliefs}\DIFaddend }

\DIFdelbegin \DIFdel{Firms make their extensive margin decision $S_i$ }\emph{\DIFdel{before}} %DIFAUXCMD
\DIFdel{observing the true state $\mu$. Instead, they face a problem of dispersed information. The fundamental state }\DIFdelend \DIFaddbegin \DIFadd{There is an aggregate productivity state }\DIFaddend $\mu \in \R$\DIFdelbegin \DIFdel{is drawn from a prior distribution. Each firm }\DIFdelend \DIFaddbegin \DIFadd{. Firm }\DIFaddend $i$ observes a private signal $s_i \in \R$. \DIFdelbegin %DIFDELCMD < 

%DIFDELCMD < %%%
\DIFdel{To capture the correlation inherent in supply chain signals, we assume the }\DIFdelend \DIFaddbegin \DIFadd{The }\DIFaddend joint distribution of $(\mu, s_1, \ldots, s_n)$ satisfies \DIFdelbegin \emph{\DIFdel{affiliation}}%DIFAUXCMD
\DIFdelend \DIFaddbegin \DIFadd{an affiliation condition that orders beliefs in a useful way}\DIFaddend .

\DIFdelbegin %DIFDELCMD < \begin{assumption}[Affiliated Information]\label{ass:affiliation}
%DIFDELCMD < %%%
\DIFdel{The random variables }\DIFdelend \DIFaddbegin \begin{assumption}[Affiliated information]\label{ass:affiliated}
\DIFadd{The vector }\DIFaddend $(\mu, s_1, \ldots, s_n)$ \DIFdelbegin \DIFdel{are affiliated. That is, their joint probability density function }\DIFdelend \DIFaddbegin \DIFadd{has a joint density }\DIFaddend $f$ \DIFaddbegin \DIFadd{that }\DIFaddend is log-supermodular: \DIFdelbegin \begin{displaymath}
    \DIFdel{f(z \vee z') f(z \wedge z') \geq f(z) f(z'),
}\end{displaymath}%DIFAUXCMD
\DIFdel{for all $z, z'$ in the support,
}\DIFdelend \DIFaddbegin \DIFadd{for all $z, z' \in \R^{n+1}$,
}\[
\DIFadd{f(z \vee z') \, f(z \wedge z') \ge f(z) \, f(z'),
}\]
\DIFaddend where $\vee$ and $\wedge$ denote \DIFdelbegin \DIFdel{component-wise }\DIFdelend \DIFaddbegin \DIFadd{componentwise }\DIFaddend maximum and minimum.
\end{assumption}

Affiliation \DIFdelbegin \DIFdel{implies that signals are positively dependent in a strong sense. A high realization of }\DIFdelend \DIFaddbegin \DIFadd{is a strong form of positive correlation. The Gaussian factor model---where $\mu \sim \N(\bar\mu, \sigma_\mu^2)$ and $s_i = \mu + \varepsilon_i$ with i.i.d.\ noise $\varepsilon_i \sim \N(0, \sigma_\varepsilon^2)$---is a leading example. The posterior mean $\E[\mu \mid s_i] = \frac{\sigma_\mu^2 s_i + \sigma_\varepsilon^2 \bar\mu}{\sigma_\mu^2 + \sigma_\varepsilon^2}$ is increasing in }\DIFaddend $s_i$\DIFdelbegin \DIFdel{leads firm $i$ to update its beliefs in two ways. First, it places higher probability on a high fundamental state $\mu$ (Monotone Likelihood Ratio Property). Second, and strategically more important, it places higher probability on other firmshaving observed high signals }\DIFdelend .
\DIFdelbegin \DIFdel{Specifically, the conditional expectation $\E[g(\s_{-i}) \mid s_i]$ is non-decreasing }\DIFdelend \DIFaddbegin 

\DIFadd{The key implication of affiliation is that beliefs are ordered by first-order stochastic dominance. A higher signal $s_i$ shifts the distribution of $\mu$ upward; it also shifts beliefs about other firms' signals upward. This ordering underlies the comparative statics developed below.
}

\begin{theorem}[Affiliation orders beliefs]\label{thm:mlrp}
\DIFadd{Under \mbox{%DIFAUXCMD
\cref{ass:affiliated}}\hspace{0pt}%DIFAUXCMD
: (i) The conditional density of $\mu$ given $s_i$ satisfies the monotone likelihood ratio property (MLRP). (ii) MLRP implies $\pi(\cdot \mid s_i') \ge_{\mathrm{FOSD}} \pi(\cdot \mid s_i)$ for $s_i' > s_i$. (iii) Beliefs about other signals are ordered: $\E[g(s_{-i}) \mid s_i]$ is increasing }\DIFaddend in $s_i$ for any \DIFdelbegin \DIFdel{non-decreasing }\DIFdelend \DIFaddbegin \DIFadd{increasing }\DIFaddend function $g$.
\DIFaddbegin \end{theorem}
\DIFaddend 

\subsection{\DIFdelbegin \DIFdel{The Network Formation Game}\DIFdelend \DIFaddbegin \DIFadd{Production technology}\DIFaddend }

\DIFdelbegin \DIFdel{Firm }\DIFdelend \DIFaddbegin \DIFadd{Industry }\DIFaddend $i$ chooses \DIFdelbegin \DIFdel{its supplier set $S_i$ to minimize expected total costs, comprising the variable cost of production and a fixed adoption cost $\gamma |S_i|$. A strategy for firm $i$ is a mapping $\sigma_i: \R \to 2^{\I \setminus \{i\}}$ from signals to supplier sets}\DIFdelend \DIFaddbegin \DIFadd{a supplier set $S_i \subseteq \I \setminus \{i\}$, labor $L_i \ge 0$, and intermediate inputs $X_i = (X_{ij})_{j \in S_i}$. Output is
}\[
\DIFadd{Y_i = \theta_i(\mu) \, F_i(S_i, A_i(S_i), L_i, X_i),
}\]
\DIFadd{where the productivity shifter $\theta_i(\mu) = \exp(\varphi_i \mu + \eta_i)$ links firm output to the aggregate state. The parameter $\varphi_i \ge 0$ governs sensitivity; $\eta_i$ is an idiosyncratic component known to all. The term $A_i(S_i)$ captures deterministic productivity gains from the supplier set}\DIFaddend .

\DIFdelbegin \DIFdel{Let $\sigma_{-i}$ denote the strategy profile of all other firms.
Firm }\DIFdelend \DIFaddbegin \begin{assumption}[Technology]\label{ass:technology}
\DIFadd{For each }\DIFaddend $i$ \DIFdelbegin \DIFdel{'s expected cost given signal $s_i$ and }\DIFdelend \DIFaddbegin \DIFadd{and supplier set $S_i$: (i) $F_i$ is continuous and strictly increasing in inputs; (ii) $F_i$ is homogeneous of degree one in $(L_i, X_i)$; (iii) Labor is essential: $F_i(S_i, A_i, 0, X_i) = 0$.
}\end{assumption}

\DIFadd{The Cobb-Douglas specification $F_i = A_i(S_i) L_i^{\alpha_i} \prod_{j \in S_i} X_{ij}^{\beta_{ij}}$ with $\alpha_i + \sum_j \beta_{ij} = 1$ satisfies these conditions. A CES aggregator also works provided inputs are complements ($\sigma \le 1$); with gross substitutes ($\sigma > 1$), labor essentiality fails.
}

\subsection{\DIFadd{Cost minimization and network choice}}

\DIFadd{Following \mbox{%DIFAUXCMD
\citet{acemoglu2020endogenous}}\hspace{0pt}%DIFAUXCMD
, firm behavior reduces to cost minimization. Given a }\DIFaddend supplier set $S_i$ \DIFdelbegin \DIFdel{is:
}\DIFdelend \DIFaddbegin \DIFadd{and input prices $P$, the }\textbf{\DIFadd{unit cost function}} \DIFadd{is
}\begin{equation}\DIFadd{\label{eq:unit_cost}
K_i(S_i, A_i(S_i), P) = \min_{L_i, X_i} \left\{ L_i + \sum_{j \in S_i} P_j X_{ij} : F_i(S_i, A_i, L_i, X_i) = 1 \right\}.
}\end{equation}
\DIFadd{By homogeneity, producing $Y_i$ units costs $Y_i \cdot K_i$. At the extensive margin, firm $i$ chooses a supplier set to minimize total cost:
}\DIFaddend \begin{equation}\DIFdelbegin %DIFDELCMD < \label{eq:objective}
%DIFDELCMD <     %%%
\DIFdel{\mathcal{C}_i(}\DIFdelend \DIFaddbegin \label{eq:technology_choice}
\DIFaddend S_i\DIFdelbegin \DIFdel{, s_i; \sigma_{-i}}\DIFdelend \DIFaddbegin \DIFadd{^*(P}\DIFaddend ) \DIFdelbegin \DIFdel{= \E }%DIFDELCMD < \left[ %%%
\DIFdelend \DIFaddbegin \DIFadd{\in \argmin_{S_i \subseteq \I \setminus \{i\}} }\left\{ \DIFaddend K_i(S_i, \DIFdelbegin \DIFdel{\mu, P^*}\DIFdelend \DIFaddbegin \DIFadd{A_i}\DIFaddend (\DIFdelbegin \DIFdel{\mu, }\DIFdelend S_i\DIFaddbegin \DIFadd{)}\DIFaddend , \DIFdelbegin \DIFdel{\sigma_{-i}(\s_{-i}}\DIFdelend \DIFaddbegin \DIFadd{P}\DIFaddend ) \DIFdelbegin \DIFdel{)) }\DIFdelend + \gamma |S_i| \DIFdelbegin \DIFdel{\;}%DIFDELCMD < \Bigg|%%%
\DIFdel{\; s_i }%DIFDELCMD < \right]%%%
\DIFdel{.
}\DIFdelend \DIFaddbegin \right\}\DIFadd{,
}\DIFaddend \end{equation}
\DIFdelbegin \DIFdel{\mbox{%DIFAUXCMD
\cref{eq:objective} }\hspace{0pt}%DIFAUXCMD
highlights the strategic interaction.
Firm }\DIFdelend \DIFaddbegin \DIFadd{where $\gamma \ge 0$ is a per-link adoption cost. This two-stage structure---choose suppliers, then optimize inputs---means the strategic game is over supplier sets rather than continuous input choices.
}

\subsection{\DIFadd{Payoffs}}

\DIFadd{Given network $S$, state $\mu$, and equilibrium prices $P^*(S, \mu)$, firm }\DIFaddend $i$'s \DIFdelbegin \DIFdel{costs depend on equilibrium prices $P^*$. These prices depend on the supplier choices of firms $j \neq i$, which in turn depend on their private signals $\s_{-i}$ via the strategies $\sigma_{-i}$. Consequently, }\DIFdelend \DIFaddbegin \DIFadd{payoff from supplier set $S_i$ is
}\begin{equation}\DIFadd{\label{eq:payoff}
\Pi_i(S_i, S_{-i}; \mu) = P_i^* Y_i^* - \left( L_i^* + \sum_{j \in S_i} P_j^* X_{ij}^* \right) - \gamma |S_i|,
}\end{equation}
\DIFadd{where $(L_i^*, X_i^*)$ solve \mbox{%DIFAUXCMD
\cref{eq:unit_cost} }\hspace{0pt}%DIFAUXCMD
and $Y_i^* = \theta_i(\mu) F_i(S_i, A_i, L_i^*, X_i^*)$. Under contestability, output price equals a markup over unit cost: $P_i = (1 + \tau_i) \theta_i(\mu)^{-1} K_i$. This implies operating profits are a constant share $\tau_i / (1 + \tau_i)$ of revenue; the network choice affects profits through the adoption cost $\gamma |S_i|$ and through the unit cost $K_i$.
}

\subsection{\DIFadd{Timing and household}}

\DIFadd{A representative household supplies one unit of labor inelastically and consumes the $n$ goods with homothetic preferences. The sequence of events is:
}\begin{enumerate}
\item \DIFadd{Nature draws $(\mu, s_1, \ldots, s_n)$.
}\item \DIFadd{Each }\DIFaddend firm $i$ \DIFdelbegin \DIFdel{must forecast not just fundamentals, but the sentiment of its peers}\DIFdelend \DIFaddbegin \DIFadd{observes $s_i$ and chooses supplier set $S_i$}\DIFaddend .
\DIFaddbegin \item \DIFadd{The state $\mu$ is realized, production takes place, and prices clear markets.
}\end{enumerate}
\DIFadd{The game is a Bayesian game over supplier sets, with the price system determined by equilibrium conditions.
}\DIFaddend 

%==============================================================================
\section{\DIFaddbegin \DIFadd{Competitive }\DIFaddend Equilibrium \DIFaddbegin \DIFadd{and Price }\DIFaddend Characterization}\DIFdelbegin %DIFDELCMD < \label{sec:equilibrium}
%DIFDELCMD < %%%
\DIFdelend \DIFaddbegin \label{sec:equilibrium_characterization}
\DIFaddend %==============================================================================

\DIFdelbegin \DIFdel{We now establish the existence and structure of equilibria. The game defined above presents technical challenges: the action space is discrete and high-dimensional (the lattice of subsets), while the state space is continuous. We overcome these by characterizing the economy as a supermodular Bayesian game}\DIFdelend \DIFaddbegin \DIFadd{This section collects the equilibrium definition and the price characterization used in the strategic analysis. The key object is the equilibrium price mapping $P^*(S,\mu)$. For each network profile $S$ and realized state $\mu$, the mapping returns the equilibrium price vector. Existence and uniqueness allow us to treat supplier choices as inducing a well-defined price system}\DIFaddend .

\subsection{\DIFdelbegin \DIFdel{Strategic Complementarities}\DIFdelend \DIFaddbegin \DIFadd{Competitive equilibrium}\DIFaddend }

\DIFdelbegin \DIFdel{The existence of monotone equilibria rests on two forms of complementarity: technological complementarity in production and inferential complementarity in beliefs. }\DIFdelend \DIFaddbegin \DIFadd{We adopt a reduced-form notion of contestability: prices equal a constant markup over unit costs. Let $\tau_i\ge 0$ denote sector $i$'s markup parameter.
}\DIFaddend 

\DIFdelbegin \DIFdel{We first impose a standard restriction on the cost function to ensure that inputs are complements in adoption. }\DIFdelend \DIFaddbegin \begin{definition}[Competitive equilibrium]\label{def:equilibrium}
\DIFadd{Fix a realized state $\mu$ and a network profile $S=(S_1,\ldots,S_n)$. A competitive equilibrium is a tuple $(P,C,L,X,Y)$ such that:
}\begin{enumerate}[label=(\roman*)]
    \item \DIFadd{(}\emph{\DIFadd{Contestability}}\DIFadd{) For each $i$,
    }\begin{equation}\DIFadd{\label{eq:contestability}
        P_i = (1+\tau_i)\,\theta_i(\mu)^{-1}\,K_i(S_i,A_i(S_i),P).
    }\end{equation}
    \item \DIFadd{(}\emph{\DIFadd{Cost minimization}}\DIFadd{) Given $(S_i,P)$, the choices $(L_i,X_i)$ attain the minimum in \mbox{%DIFAUXCMD
\cref{eq:unit_cost}}\hspace{0pt}%DIFAUXCMD
.
    }\item \DIFadd{(}\emph{\DIFadd{Household optimization}}\DIFadd{) $C$ maximizes household utility given prices and income.
    }\item \DIFadd{(}\emph{\DIFadd{Market clearing}}\DIFadd{) For each $i$, $C_i+\sum_{j}X_{ji}=Y_i$, and $\sum_i L_i=1$.
}\end{enumerate}
\end{definition}
\DIFaddend 

\DIFdelbegin %DIFDELCMD < \begin{assumption}[Cost Submodularity]\label{ass:submod}
%DIFDELCMD < %%%
\DIFdel{The unit cost function $K_i(S_i, \mu, P)$ has decreasing differences in $(S_i, P)$. That is }\DIFdelend \DIFaddbegin \subsection{\DIFadd{Existence and uniqueness of equilibrium prices}}

\DIFadd{Fix $S$ and $\mu$. Taking logs in \mbox{%DIFAUXCMD
\cref{eq:contestability} }\hspace{0pt}%DIFAUXCMD
yields
}\begin{equation}\DIFadd{\label{eq:log_price_system}
p_i = \log(1+\tau_i) - (\varphi_i\mu+\eta_i) + k_i(S_i,a_i(S_i),p),
}\end{equation}
\DIFadd{where $p_i=\log P_i$}\DIFaddend , \DIFdelbegin \DIFdel{the marginal cost reduction from adding a supplier is greater when the prices of other inputs are lower.
    }\DIFdelend \DIFaddbegin \DIFadd{$a_i=\log A_i$, and $k_i=\log K_i$.
}

\begin{proposition}[Existence and uniqueness of prices]\label{prop:price_existence}
\DIFadd{Suppose \mbox{%DIFAUXCMD
\cref{ass:technology} }\hspace{0pt}%DIFAUXCMD
holds. Fix $S$ and $\mu$. Then there exists a unique equilibrium price vector $P^*(S,\mu)$.
}\end{proposition}

\begin{proof}
\DIFadd{Rewriting \mbox{%DIFAUXCMD
\cref{eq:log_price_system} }\hspace{0pt}%DIFAUXCMD
as $\Phi(p) = 0$ where $\Phi_i(p) = p_i - k_i - \log(1+\tau_i) + \varphi_i\mu + \eta_i$, we have $\nabla\Phi = I - \nabla_p k$. Labor essentiality implies row sums of $\nabla_p k$ are strictly below one. By the Gale-Nikaido univalence theorem \mbox{%DIFAUXCMD
\citep{acemoglu2020endogenous}}\hspace{0pt}%DIFAUXCMD
, $\Phi$ is a global homeomorphism, so the system has a unique solution.
}\end{proof}

\subsection{\DIFadd{Monotone price response}}

\DIFadd{Our strategic results use a monotonicity property of the equilibrium price mapping. We now establish this property under natural conditions on the technology.
}

\begin{assumption}[Isotone technology productivity]\label{ass:isotone_A}
\DIFadd{For each firm $i$, the productivity term $A_i(S_i)$ is isotone in $S_i$ under set inclusion: $S_i' \supseteq S_i$ implies $A_i(S_i') \ge A_i(S_i)$.
}\DIFaddend \end{assumption}

\DIFdelbegin \DIFdel{This assumption holdsfor standard CES and Cobb-Douglas aggregators where the elasticity of substitution exceeds unity. It implies that a general decline in the price level increases the incentive for firm }\DIFdelend \DIFaddbegin \begin{lemma}[Monotone price response]\label{lem:monotone_price}
\DIFadd{Suppose \mbox{%DIFAUXCMD
\cref{ass:technology,ass:isotone_A} }\hspace{0pt}%DIFAUXCMD
hold. Fix $\mu$. If $S' \succeq S$ (element-wise set inclusion), then $P^*(S',\mu) \le P^*(S,\mu)$ componentwise.
}\end{lemma}

\begin{proof}
\DIFadd{By \mbox{%DIFAUXCMD
\cref{ass:isotone_A}}\hspace{0pt}%DIFAUXCMD
, $S_i' \supseteq S_i$ implies $A_i(S_i') \ge A_i(S_i)$. A larger supplier set combined with higher productivity reduces unit costs: $K_i(S_i', A_i(S_i'), P) \le K_i(S_i, A_i(S_i), P)$ for any $P$.
}

\DIFadd{Consider the price mapping $T: P \mapsto P'$ where $P'_i = (1+\tau_i) \theta_i(\mu)^{-1} K_i(S_i, A_i(S_i), P)$. Under network $S'$, the mapping $T'$ has $T'_i(P) \le T_i(P)$ for all $P$ and }\DIFaddend $i$\DIFdelbegin \DIFdel{to expand its network.
}\DIFdelend \DIFaddbegin \DIFadd{. Since both $T$ and $T'$ are contractions on $[\underline{P}, \bar{P}]$ (the spectral radius is below 1 by labor essentiality), and $T' \le T$ pointwise, the unique fixed points satisfy $P^{*}(S') \le P^*(S)$ by a standard monotone operator argument.
}\end{proof}
\DIFaddend 

\DIFdelbegin \DIFdel{We can now establish the supermodularity of the game.
Let $\Pi_i = -\mathcal{C}_i$ denote the payoff function.
}\DIFdelend %DIF > ==============================================================================
\DIFaddbegin \section{\DIFadd{Strategic Complementarities}}\label{sec:complementarities}
%DIF > ==============================================================================
\DIFaddend 

\DIFdelbegin %DIFDELCMD < \begin{lemma}[Global Complementarities]\label{lem:complementarities}
%DIFDELCMD < %%%
\DIFdel{Under \mbox{%DIFAUXCMD
\cref{ass:affiliation,ass:submod}}\hspace{0pt}%DIFAUXCMD
:
}%DIFDELCMD < \begin{enumerate}
%DIFDELCMD <     \item %%%
\textbf{\DIFdel{Strategic Complementarity:}} %DIFAUXCMD
\DIFdel{$\Pi_i$ has increasing differences in $(S_i, \sigma_{-i})$. If rivals adopt larger supplier sets (in the set inclusionorder)for any given signal, firm }\DIFdelend \DIFaddbegin \DIFadd{This section shows that network formation is a Bayesian game of strategic complementarities. The argument combines three ingredients: the action space is a lattice, payoffs are supermodular in own actions, and higher signals shift interim beliefs upward about both fundamentals and opponents' actions.
}

\subsection{\DIFadd{Action space as a lattice}}

\DIFadd{Each firm's action is $a_i=(S_i,L_i,X_i)$, where
}\[
\DIFadd{a_i \in \Sset_i \equiv 2^{\I\setminus\{i\}} \times }[\DIFadd{0,\bar L}] \DIFadd{\times }[\DIFadd{0,\bar X}]\DIFadd{^{n-1}.
}\]
\DIFadd{We order actions by
}\[
\DIFadd{(S_i,L_i,X_i)\succeq(S_i',L_i',X_i') \quad\Longleftrightarrow\quad
S_i\supseteq S_i',\ L_i\ge L_i',\ X_i\ge X_i' \text{ componentwise}.
}\]

\begin{lemma}[Action space]\label{lem:lattice}
\DIFadd{Under $\succeq$, $\Sset_i$ is a compact lattice.
}\end{lemma}

\begin{proof}
\DIFadd{The power set $2^{\I\setminus\{i\}}$ is a finite lattice under inclusion with $\vee=\cup$ and $\wedge=\cap$. The intervals $[0,\bar L]$ and $[0,\bar X]^{n-1}$ are compact complete lattices under the usual order. The Cartesian product of lattices is a lattice with componentwise join and meet.
}\end{proof}

\subsection{\DIFadd{Supermodularity}}

\begin{assumption}[Technological complementarity]\label{ass:complementarity}
\DIFadd{The production function $F_i$ exhibits increasing differences in inputs: for $S_i'\supseteq S_i$ and $X'\ge X$,
}\[
\DIFadd{F_i(S_i',A_i(S_i'),L,X')-F_i(S_i,A_i(S_i),L,X')
\ge
F_i(S_i',A_i(S_i'),L,X)-F_i(S_i,A_i(S_i),L,X).
}\]
\end{assumption}

\begin{lemma}[Supermodularity of payoffs]\label{lem:supermod}
\DIFadd{Under \mbox{%DIFAUXCMD
\cref{ass:technology,ass:complementarity}}\hspace{0pt}%DIFAUXCMD
, firm }\DIFaddend $i$\DIFdelbegin \DIFdel{'s incentive to expand $S_i$ increases.
}%DIFDELCMD < \item %%%
\textbf{\DIFdel{Single-Crossing in Type:}} %DIFAUXCMD
\DIFdel{$\Pi_i$ has increasing differences in $(S_i, s_i)$. A higher signal increases the marginal benefit of expanding the supplier set }\DIFdelend \DIFaddbegin \DIFadd{'s payoff is supermodular in its own action $a_i$}\DIFaddend .
\DIFdelbegin %DIFDELCMD < \end{enumerate}
%DIFDELCMD < %%%
\DIFdelend \end{lemma}

\DIFdelbegin %DIFDELCMD < \begin{proof}
%DIFDELCMD < %%%
\DIFdel{Consider Strategic Complementarity. If rivals play a ``larger'' strategy $\sigma'_{-i} \supseteq \sigma_{-i}$, then for any realization of signals, the resulting network is denser. By the monotonicity of inverse M-matrices \mbox{%DIFAUXCMD
\citep{acemoglu2020endogenous}}\hspace{0pt}%DIFAUXCMD
, a denser network lowers equilibrium prices $P^*$.
By \mbox{%DIFAUXCMD
\cref{ass:submod}}\hspace{0pt}%DIFAUXCMD
, lower prices increase the marginal return to adding suppliers.
}\DIFdelend \DIFaddbegin \begin{proof}
\DIFadd{By \mbox{%DIFAUXCMD
\citet{topkis1998supermodularity}}\hspace{0pt}%DIFAUXCMD
, a function on a lattice is supermodular if and only if it has increasing differences in each pair of variables. \mbox{%DIFAUXCMD
\cref{ass:complementarity} }\hspace{0pt}%DIFAUXCMD
provides increasing differences between $(S_i,X_i)$. Positive scalar multiplication preserves supermodularity. Input expenditures and adoption costs are modular (additively separable). Subtracting a modular function preserves supermodularity.
}\end{proof}
\DIFaddend 

\DIFdelbegin \DIFdel{Consider Single-Crossing. Let $\Delta(\mu, P) = K_i(S_i, \cdot) - K_i(S'_i, \cdot)$ be the cost saving from expansion ($S'_i \supset S_i$). This saving is increasing in $\mu$ (higher productivity scales up the firm ) and decreasing in $P$ (lower prices make expansion more valuable). Since $P^*$ }\DIFdelend \DIFaddbegin \subsection{\DIFadd{Price-action single crossing}}

\begin{lemma}[Single crossing in prices]\label{lem:price_sc}
\DIFadd{Under \mbox{%DIFAUXCMD
\cref{ass:isotone_A}}\hspace{0pt}%DIFAUXCMD
, payoffs have increasing differences in $(a_i,a_{-i})$.
}\end{lemma}

\begin{proof}
\DIFadd{If $a_{-i}'\succeq a_{-i}$, then $P^*(a_i,a_{-i}',\mu)\le P^*(a_i,a_{-i},\mu)$ by \mbox{%DIFAUXCMD
\cref{ass:isotone_A}}\hspace{0pt}%DIFAUXCMD
. For $a_i'\succeq a_i$, define the incremental payoff
}\[
\DIFadd{\Delta\Pi(P)=\Pi_i(a_i',P)-\Pi_i(a_i,P).
}\]
\DIFadd{Since input costs enter linearly with a negative sign, $\Delta\Pi(P)$ }\DIFaddend is decreasing in \DIFdelbegin \DIFdel{both $\mu$ and $\s_{-i}$,
the composite gain function
$G(\mu, \s_{-i})$ }\DIFdelend \DIFaddbegin \DIFadd{$P$. Lower prices induced by $a_{-i}'$ therefore raise the gain from choosing $a_i'$ rather than $a_i$.
}\end{proof}

\subsection{\DIFadd{Information single crossing}}

\begin{lemma}[Information single crossing]\label{lem:info_sc}
\DIFadd{Suppose \mbox{%DIFAUXCMD
\cref{ass:affiliated,ass:complementarity} }\hspace{0pt}%DIFAUXCMD
holds and opponents use monotone strategies $\sigma_{-i}$. Then expected payoffs satisfy single crossing in $(a_i,s_i)$: for $a_i'\succeq a_i$ and $s_i'>s_i$,
}\[
\DIFadd{\E\!\left[\Pi_i(a_i',\sigma_{-i}(s_{-i});\mu,P^*)-\Pi_i(a_i,\sigma_{-i}(s_{-i});\mu,P^*)\mid s_i'\right]
\ge
\E\!\left[\cdot\mid s_i\right].
}\]
\end{lemma}

\begin{proof}
\DIFadd{Define the gain function
}\[
\DIFadd{h(\mu,s_{-i})=\Pi_i(a_i',\sigma_{-i}(s_{-i});\mu,P^*)-\Pi_i(a_i,\sigma_{-i}(s_{-i});\mu,P^*).
}\]

\textbf{\DIFadd{Step 1:}} \DIFadd{$h$ }\DIFaddend is increasing in \DIFdelbegin \DIFdel{both arguments. By the properties of affiliated variables \mbox{%DIFAUXCMD
\citep{milgrom1982theory}}\hspace{0pt}%DIFAUXCMD
, the conditional expectation $\E[G(\mu, \s_{-i}) \mid s_i]$ is increasing }\DIFdelend \DIFaddbegin \DIFadd{$\mu$. The shifter $\theta_i(\mu)=\exp(\varphi_i\mu+\eta_i)$ is increasing in $\mu$ when $\varphi_i\ge 0$, and $a_i'\succeq a_i$ weakly increases production possibilities. Hence the incremental benefit from $a_i'$ is increasing in $\mu$.
}

\textbf{\DIFadd{Step 2:}} \DIFadd{$h$ is increasing in $s_{-i}$. Monotone $\sigma_{-i}$ implies $\sigma_{-i}(s_{-i}')\succeq\sigma_{-i}(s_{-i})$ for $s_{-i}'\ge s_{-i}$. By \mbox{%DIFAUXCMD
\cref{ass:isotone_A}}\hspace{0pt}%DIFAUXCMD
, this lowers equilibrium prices. By \mbox{%DIFAUXCMD
\cref{lem:price_sc}}\hspace{0pt}%DIFAUXCMD
, lower prices increase the gain from expansion, so $h$ is increasing in $s_{-i}$.
}

\textbf{\DIFadd{Step 3:}} \DIFadd{Apply \mbox{%DIFAUXCMD
\cref{thm:mlrp,thm:fosd_integration}}\hspace{0pt}%DIFAUXCMD
. Since $h$ is increasing in $(\mu,s_{-i})$ and the conditional distribution of $(\mu,s_{-i})$ given $s_i$ is ordered by FOSD }\DIFaddend in $s_i$\DIFaddbegin \DIFadd{, we have $\E[h(\mu,s_{-i})\mid s_i']\ge \E[h(\mu,s_{-i})\mid s_i]$}\DIFaddend .
\end{proof}

\DIFdelbegin \subsection{\DIFdel{Existence of Monotone Equilibria}}
%DIFAUXCMD
\addtocounter{subsection}{-1}%DIFAUXCMD
\DIFdelend %DIF > ==============================================================================
\DIFaddbegin \section{\DIFadd{Monotone Bayesian Nash Equilibria}}\label{sec:monotone_equilibria}
%DIF > ==============================================================================
\DIFaddend 

\DIFdelbegin \DIFdel{\mbox{%DIFAUXCMD
\cref{lem:complementarities} }\hspace{0pt}%DIFAUXCMD
allows us to apply the machinery of supermodular games. The set of monotone strategies$\Sigma$, endowed with the pointwise inclusion order, forms a complete lattice. The best-response mapping preserves this order. By Tarski's Fixed Point Theorem, we obtain our first main result }\DIFdelend \DIFaddbegin \DIFadd{This section uses lattice methods to establish equilibrium existence in monotone strategies}\DIFaddend . \DIFaddbegin \DIFadd{The result is useful because it gives a disciplined way to describe equilibrium network regimes in a high-dimensional environment.
}\DIFaddend 

\DIFdelbegin %DIFDELCMD < \begin{theorem}[Extremal Monotone Equilibria]%%%
\DIFdelend \DIFaddbegin \begin{theorem}[Extremal monotone equilibria]\DIFaddend \label{thm:existence}
\DIFdelbegin \DIFdel{The network formation game possesses a greatest Bayesian Nash equilibrium $\bar{\sigma}$ and a least Bayesian Nash equilibrium $\underline{\sigma}$. These equilibria are in monotone pure strategies: for every firm $i$, if $s'_i > s_i$, then $\sigma_i(s_i) \subseteq \sigma_i(s'_i)$.
}\DIFdelend \DIFaddbegin \DIFadd{Under \mbox{%DIFAUXCMD
\cref{ass:affiliated,ass:technology,ass:complementarity,ass:isotone_A}}\hspace{0pt}%DIFAUXCMD
, the Bayesian game admits a nonempty complete lattice of monotone Bayesian Nash equilibria. In particular, there exist greatest and least equilibria $\bar\sigma$ and $\underline\sigma$ in the lattice of monotone strategies.
}\DIFaddend \end{theorem}

\DIFdelbegin \DIFdel{This theorem implies that beliefs shape the network structure in a predictable, ordered way.
In the ``greatest '' }\DIFdelend \DIFaddbegin \begin{proof}
\DIFadd{We verify the conditions of \mbox{%DIFAUXCMD
\citet{van2007monotone}}\hspace{0pt}%DIFAUXCMD
.
}

\textbf{\DIFadd{Step 1: Strategy lattice.}} \DIFadd{Let $\Sigma_i$ be the set of isotone functions $\sigma_i:\R\to\Sset_i$. By \mbox{%DIFAUXCMD
\cref{lem:lattice}}\hspace{0pt}%DIFAUXCMD
, $\Sset_i$ is a compact lattice. The set $\Sigma=\prod_i\Sigma_i$ is a complete lattice under pointwise order.
}

\textbf{\DIFadd{Step 2: Supermodularity.}} \DIFadd{By \mbox{%DIFAUXCMD
\cref{lem:supermod}}\hspace{0pt}%DIFAUXCMD
, payoffs are supermodular in $a_i$.
}

\textbf{\DIFadd{Step 3: Increasing differences.}} \DIFadd{By \mbox{%DIFAUXCMD
\cref{lem:price_sc}}\hspace{0pt}%DIFAUXCMD
, payoffs have increasing differences in $(a_i,a_{-i})$.
}

\textbf{\DIFadd{Step 4: Single crossing.}} \DIFadd{By \mbox{%DIFAUXCMD
\cref{lem:info_sc}}\hspace{0pt}%DIFAUXCMD
, expected payoffs satisfy single crossing in $(a_i,s_i)$.
}

\textbf{\DIFadd{Step 5: Monotone best responses.}} \DIFadd{By \mbox{%DIFAUXCMD
\citet{milgrom1994monotone}}\hspace{0pt}%DIFAUXCMD
, the best-response correspondence admits an isotone selection in $(s_i,\sigma_{-i})$.
}

\textbf{\DIFadd{Step 6: Fixed point.}} \DIFadd{The best-response operator $\BR:\Sigma\to\Sigma$ is isotone. By \mbox{%DIFAUXCMD
\cref{thm:tarski}}\hspace{0pt}%DIFAUXCMD
, an isotone map on a complete lattice has a nonempty complete lattice of fixed points.
}\end{proof}

\begin{remark}[Interpretation]
\DIFadd{In the greatest monotone }\DIFaddend equilibrium, firms \DIFdelbegin \DIFdel{coordinate on optimistic beliefs, leading to dense production networks. In the ``least '' equilibrium , pessimistic coordination leads to sparse networks. Both regimesare consistent with the same underlying fundamentals; the selection is driven by the self-fulfilling nature of sentiment.
}\DIFdelend \DIFaddbegin \DIFadd{respond to any signal as aggressively as possible, taking as given that others respond aggressively as well. The least equilibrium is the pessimistic counterpart. This ordering gives a simple language for ``optimistic'' and ``pessimistic'' network regimes.
}\end{remark}
\DIFaddend 

%==============================================================================
\section{\DIFdelbegin \DIFdel{The Sentiment Multiplier}\DIFdelend \DIFaddbegin \DIFadd{Comparative Statics}\DIFaddend }\DIFdelbegin %DIFDELCMD < \label{sec:statics}
%DIFDELCMD < %%%
\DIFdelend \DIFaddbegin \label{sec:comparative}
\DIFaddend %==============================================================================

\DIFdelbegin \DIFdel{We now turn to the central economic mechanism.
How do these strategic complementarities amplify shocks?
}\DIFdelend \DIFaddbegin \DIFadd{This section derives comparative statics for the extremal equilibria. The logic is common across results. We first show that a parameter shift moves interim incentives in a monotone direction for every fixed opponent strategy. This shifts best responses. Since the best-response operator is isotone, the ordering transfers to the extremal fixed points.
}\DIFaddend 

\DIFdelbegin \DIFdel{Consider a ``sentiment shock'': a uniform shift in signals $s_i \to s_i + \delta$ holding the fundamental $\mu$ constant. In a standard model with exogenous networks, this noise would be ignored.
Here, it is not.
}\DIFdelend \DIFaddbegin \subsection{\DIFadd{Adoption costs}}
\DIFaddend 

\DIFdelbegin \DIFdel{We decompose the total network response $dy/ds$ into a direct effect (response to fundamentals $\E[\mu]$)}\DIFdelend \DIFaddbegin \begin{theorem}[Adoption cost reduction]\label{thm:gamma}
\DIFadd{Let $\gamma$ be the per-link adoption cost in \mbox{%DIFAUXCMD
\cref{eq:technology_choice}}\hspace{0pt}%DIFAUXCMD
. The extremal equilibria $\bar\sigma$ }\DIFaddend and \DIFaddbegin \DIFadd{$\underline\sigma$ are antitone in $\gamma$: lower $\gamma$ expands equilibrium supplier sets.
}\end{theorem}

\begin{proof}
\DIFadd{The term $-\gamma|S_i|$ has decreasing differences in $(S_i,\gamma)$. By monotone comparative statics for supermodular games \mbox{%DIFAUXCMD
\citep{topkis1998supermodularity}}\hspace{0pt}%DIFAUXCMD
, extremal fixed points are antitone in $\gamma$.
}\end{proof}

\begin{remark}[Economic content]
\DIFadd{Lower adoption costs raise the return to forming supplier links for every configuration of opponents' networks. Because best responses are increasing, this local change propagates through equilibrium prices and expands the network economy-wide.
}\end{remark}

\subsection{\DIFadd{Belief shifts}}

\begin{theorem}[Optimism and network expansion]\label{thm:beliefs}
\DIFadd{If interim beliefs shift upward in the FOSD sense, the extremal monotone equilibria expand.
}\end{theorem}

\begin{proof}
\DIFadd{An upward FOSD shift in beliefs increases the interim expected gain from expansion for every fixed opponent strategy (by \mbox{%DIFAUXCMD
\cref{thm:fosd_integration} }\hspace{0pt}%DIFAUXCMD
applied to the gain function $h$ in \mbox{%DIFAUXCMD
\cref{lem:info_sc}}\hspace{0pt}%DIFAUXCMD
). This shifts best responses upward. By monotone comparative statics for extremal fixed points, the extremal equilibria increase.
}\end{proof}

\begin{remark}[Direct and strategic channels]
\DIFadd{The belief shift affects incentives through a direct and }\DIFaddend a strategic \DIFdelbegin \DIFdel{effect (response to others' actions $\E[\s_{-i}]$)}\DIFdelend \DIFaddbegin \DIFadd{channel. The direct channel raises $\E[\theta_i(\mu)\mid s_i]$ and therefore the expected return to adopting a larger supplier set}\DIFaddend . The strategic \DIFdelbegin \DIFdel{effect generates a multiplier. }\DIFdelend \DIFaddbegin \DIFadd{channel raises $\E[g(s_{-i})\mid s_i]$ for increasing $g$ and therefore raises expected opponent expansion. Under \mbox{%DIFAUXCMD
\cref{ass:isotone_A}}\hspace{0pt}%DIFAUXCMD
, higher expected opponent expansion lowers expected input prices and further raises the return to expansion.
}\end{remark}
\DIFaddend 

\DIFdelbegin %DIFDELCMD < \begin{proposition}[The Sentiment Multiplier]\label{prop:multiplier}
%DIFDELCMD < %%%
\DIFdel{In the symmetric Gaussian limit, let $\beta \in (0,1)$ represent the technological elasticity of network formation with respect to aggregate prices, and let $\rho \in (0,1)$ represent the correlation of private signals.
The elasticity of network density with respect to a sentiment shock is}\DIFdelend %DIF > ==============================================================================
\DIFaddbegin \section{\DIFadd{Belief-Adjusted Domar Weights}}\label{sec:domar}
%DIF > ==============================================================================

\DIFadd{We now specialize to the }\textbf{\DIFadd{Cobb-Douglas/Gaussian}} \DIFadd{case to obtain explicit formulas for how beliefs enter aggregate productivity. This allows us to define belief-adjusted Domar weights that treat the information structure as first order.
}

\subsection{\DIFadd{Cobb-Douglas production and Gaussian signals}}

\DIFadd{Assume Cobb-Douglas technology:
}\begin{equation}\DIFadd{\label{eq:cobb_douglas}
    Y_i = \theta_i(\mu) L_i^{\alpha_i} \prod_{j \in S_i} X_{ij}^{\beta_{ij}}, \quad \text{with } \alpha_i + \sum_{j \in S_i} \beta_{ij} = 1.
}\end{equation}
\DIFadd{Productivity is log-linear in the common factor:
}\[
\DIFadd{\theta_i(\mu) = \exp(\varphi_i \mu + \eta_i),
}\]
\DIFadd{where $\varphi_i > 0$ measures sector $i$'s exposure to aggregate conditions and $\eta_i$ is an idiosyncratic component.
}

\DIFadd{Signals are Gaussian as in \mbox{%DIFAUXCMD
\cref{ex:gaussian}}\hspace{0pt}%DIFAUXCMD
: $s_i = \mu + \varepsilon_i$ with $\varepsilon_i \sim \mathcal{N}(0, \sigma_\varepsilon^2)$ independent across $i$ and of $\mu$.
}

\subsection{\DIFadd{Signal-conditioned Domar weights}}

\DIFadd{Let the equilibrium mapping from the signal profile $\s = (s_1, \ldots, s_n)$ to allocations be
}\[
\DIFadd{\s \mapsto \big(P(\s), Y(\s), C(\s), S(\s)\big),
}\]
\DIFadd{where $S(\s)$ is the endogenous network and $(P, Y, C)$ are induced prices, outputs, and final demands.
}

\begin{definition}[Signal-conditioned Domar weight]
\DIFadd{The }\textbf{\DIFadd{signal-conditioned Domar weight}} \DIFadd{of sector $i$ is:
}\DIFaddend \begin{equation}\DIFdelbegin \DIFdel{\frac{dy}{ds} }\DIFdelend \DIFaddbegin \label{eq:domar}
    \DIFadd{D_i(\s) \equiv \frac{P_i(\s) Y_i(\s)}{\sum_{k=1}^n P_k(\s) C_k(\s)} }\DIFaddend = \DIFdelbegin \DIFdel{\mathcal{M} \cdot \frac{\partial y}{\partial \E[\mu]}, }%DIFDELCMD < \quad %%%
\DIFdel{\text{where } \mathcal{M} = \frac{1}{1 - \rho \beta}}\DIFdelend \DIFaddbegin \DIFadd{\frac{P_i(\s) Y_i(\s)}{\mathrm{GDP}(\s)}}\DIFaddend .
\end{equation}
\DIFdelbegin %DIFDELCMD < \end{proposition}
%DIFDELCMD < %%%
\DIFdelend \DIFaddbegin \end{definition}
\DIFaddend 

\DIFdelbegin \DIFdel{The term $\mathcal{M}$ is the }\emph{\DIFdel{sentiment multiplier}}%DIFAUXCMD
\DIFdel{. It exceeds unity whenever $\rho > 0$ and inputs are complements ($\beta > 0$). It captures a feedback loop: optimism leads to network expansion, which lowers prices, which validates the optimism}\DIFdelend \DIFaddbegin \DIFadd{This object is a function of signals because signals determine networks and thus prices. It summarizes which sectors are systemically important in the equilibrium induced by the belief state $\s$}\DIFaddend .

\DIFdelbegin \DIFdel{Crucially, the multiplier depends on the information structure. If signals are perfectly idiosyncratic ($\rho \to 0$) , the multiplier disappears ($\mathcal{M} \to 1$).
If signals are highly correlated ($\rho \to 1$), the multiplier approaches the theoretical maximum defined by the technology.
This reveals that correlated information is not merely a statistical nuisance; it is the transmission mechanism for strategic amplification}\DIFdelend \DIFaddbegin \subsection{\DIFadd{Interim Domar weights}}

\DIFadd{From the perspective of agent $i$, who observes only $s_i$, the relevant object is the expected Domar weight.
}

\begin{definition}[Interim Domar weight]
\DIFadd{Agent $i$'s }\textbf{\DIFadd{interim Domar weight}} \DIFadd{for sector $j$ is:
}\begin{equation}
    \DIFadd{D_j^i(s_i) \equiv \E\left[ D_j(\s) \mid s_i \right].
}\end{equation}
\end{definition}

\DIFadd{This is what firm $i$ }\emph{\DIFadd{believes}} \DIFadd{the Domar weight to be, given its information}\DIFaddend . \DIFaddbegin \DIFadd{Under affiliation, these interim expectations satisfy monotonicity:
}\DIFaddend 

\DIFaddbegin \begin{lemma}[Monotonicity of interim Domar weights]
\DIFadd{If the network is monotone in signals (\mbox{%DIFAUXCMD
\cref{thm:existence}}\hspace{0pt}%DIFAUXCMD
) and larger networks increase sector $j$'s output share, then $D_j^i(s_i)$ is non-decreasing in $s_i$.
}\end{lemma}

\subsection{\DIFadd{Belief-adjusted Domar elasticities}}

\DIFadd{The Hulten/Domar logic says that a productivity change in sector $i$ moves aggregate output by that sector's Domar weight. In our setting, we differentiate with respect to the }\emph{\DIFadd{belief state}}\DIFadd{, allowing networks to adjust.
}

\DIFadd{Define the posterior mean belief:
}\[
\DIFadd{\hat{\mu}(\s) \equiv \E[\mu \mid \s], \qquad \hat{\theta}_i(\s) \equiv \exp(\varphi_i \hat{\mu}(\s) + \eta_i).
}\]

\begin{definition}[Belief-adjusted Domar elasticity]
\DIFadd{The }\textbf{\DIFadd{belief-adjusted Domar elasticity}} \DIFadd{of sector $i$ is:
}\begin{equation}
    \DIFadd{\Lambda_i(\s) \equiv \frac{\partial \log \mathrm{GDP}(\s)}{\partial \log \hat{\theta}_i(\s)}.
}\end{equation}
\DIFadd{The }\textbf{\DIFadd{aggregate belief-Domar loading}} \DIFadd{is:
}\begin{equation}\DIFadd{\label{eq:lambda}
    \Lambda(\s) \equiv \frac{\partial \log \mathrm{GDP}(\s)}{\partial \hat{\mu}(\s)} = \sum_{i=1}^n \Lambda_i(\s) \cdot \varphi_i.
}\end{equation}
\end{definition}

\DIFadd{This $\Lambda(\s)$ is the summary statistic that treats beliefs as first order: it captures how a small belief shift about $\mu$ changes aggregate output through }\emph{\DIFadd{both}} \DIFadd{the direct fundamental channel and the strategic network channel.
}

\DIFaddend \subsection{\DIFdelbegin \DIFdel{Information Quality }\DIFdelend \DIFaddbegin \DIFadd{Decomposition: Hulten term }\DIFaddend and \DIFdelbegin \DIFdel{Volatility}\DIFdelend \DIFaddbegin \DIFadd{strategic amplification}\DIFaddend }

\DIFdelbegin \DIFdel{We can now ask how the quality of information affects economic stability. We distinguish between two dimensions of information quality: signal precision (variance of noise)and signal correlation (covariance of noise). }%DIFDELCMD < 

%DIFDELCMD < \begin{theorem}[Information and Volatility]\label{thm:comparative_statics}
%DIFDELCMD < %%%
\DIFdel{In }\DIFdelend \DIFaddbegin \begin{proposition}[Belief-adjusted Domar decomposition]\label{prop:domar_decomp}
\DIFadd{In the Cobb-Douglas/Gaussian economy with interior equilibrium, the aggregate belief-Domar loading decomposes as:
}\begin{equation}\DIFadd{\label{eq:decomposition}
    \Lambda(\s) = \underbrace{\sum_{i=1}^n D_i(\s) \cdot \varphi_i}_{\text{Hulten/Domar term}} \times \underbrace{\frac{1}{1 - \theta(\s)}}_{\text{strategic amplification}},
}\end{equation}
\DIFadd{where the }\textbf{\DIFadd{equilibrium feedback index}} \DIFadd{is
}\begin{equation}\DIFadd{\label{eq:theta}
    \theta(\s) \equiv \rho\big(B(\s)\big) \in (0,1),
}\end{equation}
\DIFadd{with $B(\s)$ }\DIFaddend the \DIFdelbegin \DIFdel{Gaussian signal limit:
}%DIFDELCMD < \begin{enumerate}
\begin{enumerate}%DIFAUXCMD
%DIFDELCMD <     \item %%%
\item%DIFAUXCMD
\textbf{\DIFdel{Precision:}} %DIFAUXCMD
\DIFdel{An increase in signal precision (lower $\sigma_\varepsilon^2$)reduces the conditional volatility of the network structure.
}%DIFDELCMD < \item %%%
\item%DIFAUXCMD
\textbf{\DIFdel{Correlation:}} %DIFAUXCMD
\DIFdel{An increase in the correlation of error terms (higher $\rho$, holding marginal variance constant)increases the conditional volatility of the network structure. }
\end{enumerate}%DIFAUXCMD
%DIFDELCMD < \end{enumerate}
%DIFDELCMD < \end{theorem}
%DIFDELCMD < %%%
\DIFdelend \DIFaddbegin \DIFadd{$n \times n$ matrix of equilibrium input-output shares: $B_{ij}(\s) = \beta_{ij} \cdot \mathbf{1}\{j \in S_i(\s)\}$.
}\end{proposition}
\DIFaddend 

\DIFdelbegin %DIFDELCMD < \begin{proof}[Sketch]
%DIFDELCMD < %%%
\DIFdel{As precision increases, firms place less weight on the prior and more on the realization, but the variance of the ``noise'' component shrinks faster than the strategic response increases. In contrast, increasing correlation raises $\rho$ without improving information quality. This raises the multiplier $\mathcal{M}$ directly, causing the economy to react more violently to non-fundamental shocks.
}\DIFdelend \DIFaddbegin \begin{proof}
\DIFadd{In the Cobb-Douglas case, the cost function is $K_i = \frac{1}{A_i} \prod_{j \in S_i} P_j^{\beta_{ij}}$ (up to a constant involving $\alpha_i$). Taking logs and substituting the contestability condition $p_i = -\log \theta_i(\mu) + k_i$:
}\[
\DIFadd{p_i = -(\varphi_i \mu + \eta_i) + \sum_{j \in S_i} \beta_{ij} p_j + c_i,
}\]
\DIFadd{where $c_i$ collects constants. In matrix form: $p = -\Phi \mu + B(\s) p + c$, where $\Phi = (\varphi_1, \ldots, \varphi_n)'$. Solving: $p = (I - B)^{-1}(c - \Phi \mu)$.
}

\DIFadd{Log-linearizing GDP around the equilibrium:
}\[
\DIFadd{d \log \text{GDP} = \sum_i D_i \cdot d \log Y_i = \sum_i D_i (\varphi_i \, d\mu - dp_i).
}\]
\DIFadd{Substituting $dp = -(I-B)^{-1} \Phi \, d\mu$ and simplifying:
}\[
\DIFadd{\frac{d \log \text{GDP}}{d\mu} = D' \Phi + D' (I-B)^{-1} \Phi = D' (I-B)^{-1} \Phi.
}\]
\DIFadd{The Leontief inverse $(I-B)^{-1} = \sum_{k=0}^\infty B^k$ captures the geometric series of network spillovers. The amplification factor is bounded by $(1 - \rho(B))^{-1}$ where $\rho(B) < 1$ by labor essentiality. Defining $\theta(\s) = \rho(B(\s))$ yields the decomposition.
}\DIFaddend \end{proof}

\DIFdelbegin \DIFdel{This result suggests that opacity in supply chains acts as an amplifier. When information is noisy and errors are correlated (e. g., when firms rely on the same few public indicators), the sentiment multiplier is maximized, leading to fragile supply chains susceptible to sentiment-driven fluctuations.
}\DIFdelend \DIFaddbegin \paragraph{\DIFadd{Interpretation.}} \DIFadd{In optimistic belief states, the endogenous network $S(\s)$ is denser, which increases the nonzero entries in $B(\s)$, pushing $\theta(\s)$ up and making the multiplier larger. When the network is exogenous (fixed $S$), beliefs do not affect $B$ and the amplification factor is constant. Endogenous networks under dispersed information make the amplification state-dependent.
}\DIFaddend 

%==============================================================================
\section{Dynamic Extension}\label{sec:dynamic}
%==============================================================================

\DIFdelbegin \DIFdel{We briefly sketch a dynamic extension to illustrate persistence. Let $t=1, 2, \dots$. Firms face asymmetric adjustment costs:
forming a link costs $\gamma^+$, while severing one costs $\gamma^-$}\DIFdelend \DIFaddbegin \DIFadd{The static analysis describes network formation in a single period. In many applications, supplier links persist because forming or severing links is costly. This section develops a dynamic extension that preserves the monotone structure, following \mbox{%DIFAUXCMD
\citet{van2007monotone}}\hspace{0pt}%DIFAUXCMD
.
}

\subsection{\DIFadd{Dynamic Bayesian game}}

\DIFadd{Time is discrete, $t = 0, 1, 2, \ldots$. Each period, nature draws a productivity shock $\mu_t$ from a stationary distribution. Firms observe private signals $s_{it}$ correlated with $\mu_t$ and with each other's signals (affiliation). The network choice $S_{it}$ is made at the beginning of period $t$ after observing $s_{it}$}\DIFaddend .
\DIFdelbegin \begin{displaymath}
    \DIFdel{\Gamma(S_{it}, S_{i,t-1}) = \gamma^+ |S_{it} \setminus S_{i,t-1}| + \gamma^- |S_{i,t-1} \setminus S_{it}|.
}\end{displaymath}%DIFAUXCMD
\DIFdelend \DIFaddbegin 

\DIFadd{A firm's }\textbf{\DIFadd{state}} \DIFadd{at time $t$ is the pair $(s_{it}, S_{i,t-1})$---the current signal and the inherited network. Firms discount future payoffs at rate $\beta \in (0,1)$. The period payoff is:
}\[
\DIFadd{\Pi_i(S_{it}, S_{-i,t}, \mu_t) - c(S_{it}, S_{i,t-1}),
}\]\DIFaddend 
\DIFdelbegin \DIFdel{In this environment, the state space expands to include the previous network $S_{t-1}$. The strategic complementarity now interacts with hysteresis. }\DIFdelend \DIFaddbegin \DIFadd{where $\Pi_i$ is as in \mbox{%DIFAUXCMD
\cref{eq:payoff} }\hspace{0pt}%DIFAUXCMD
and $c(\cdot)$ is an adjustment cost:
}\[
\DIFadd{c(S_{it}, S_{i,t-1}) = \gamma^+|S_{it}\setminus S_{i,t-1}|+\gamma^-|S_{i,t-1}\setminus S_{it}|.
}\]

\subsection{\DIFadd{Value function and Bellman equation}}

\DIFadd{Given opponents' Markov strategy $\sigma_{-i}$, firm $i$'s }\textbf{\DIFadd{value function}} \DIFadd{satisfies the Bellman equation:
}\begin{equation}\DIFadd{\label{eq:bellman}
V_i(s_{it}, S_{i,t-1}; \sigma_{-i}) = \max_{S_{it}} \left\{ \E\left[\Pi_i(S_{it}, \sigma_{-i}(s_{-i,t}), \mu_t) \mid s_{it}\right] - c(S_{it}, S_{i,t-1}) + \beta \, \E\left[V_i(s_{i,t+1}, S_{it}; \sigma_{-i}) \mid s_{it}\right] \right\}.
}\end{equation}

\DIFadd{The expectation integrates over $(\mu_t, s_{-i,t})$ conditional on $s_{it}$, and over the next-period signal $s_{i,t+1}$ given its realized distribution.
}

\subsection{\DIFadd{Monotone Markov strategies}}

\DIFadd{We restrict attention to }\textbf{\DIFadd{Markov strategies}} \DIFadd{that depend only on the current state $(s_{it}, S_{i,t-1})$, not on the full history. A Markov strategy is }\textbf{\DIFadd{monotone}} \DIFadd{if $\sigma_i(s_{it}, S_{i,t-1})$ is non-decreasing in $s_{it}$ (with respect to set inclusion) for each $S_{i,t-1}$.
}

\DIFadd{The space of monotone Markov strategies forms a complete lattice under the pointwise order.
}

\begin{theorem}[Existence of monotone Markov equilibria]\label{thm:dynamic}
\DIFadd{Under dynamic analogues of \mbox{%DIFAUXCMD
\cref{ass:affiliated,ass:technology,ass:complementarity,ass:isotone_A}}\hspace{0pt}%DIFAUXCMD
, the dynamic game possesses a greatest and a least monotone Markov perfect equilibrium. In these equilibria, network expansion is monotone in the current signal: optimistic firms expand, pessimistic firms contract.
}\end{theorem}

\begin{proof}
\DIFadd{We verify the conditions of \mbox{%DIFAUXCMD
\citet{van2007monotone}}\hspace{0pt}%DIFAUXCMD
. The Bellman operator $T: V \mapsto TV$ defined by \mbox{%DIFAUXCMD
\cref{eq:bellman} }\hspace{0pt}%DIFAUXCMD
is a contraction in the sup-norm with modulus $\beta < 1$, so it has a unique fixed point $V^*$. Given $V^*$, the best-response operator maps monotone strategies to monotone strategies by the single-crossing property (Lemma~\ref{lem:info_sc}). The space of monotone strategies is a complete lattice. By Tarski's fixed point theorem, extremal fixed points exist in the strategy space.
}\end{proof}

\subsection{\DIFadd{Hysteresis and persistence}}

\DIFadd{A key feature of the dynamic model is }\textbf{\DIFadd{belief-network feedback}}\DIFadd{. As firms observe positive signals, they update beliefs upward, expand networks, benefit from lower prices, and enter the next period with a denser inherited network $S_{i,t-1}$. The adjustment cost creates }\textbf{\DIFadd{hysteresis}}\DIFadd{:
}

\begin{remark}[Hysteresis]
\DIFadd{If adjustment costs are asymmetric ($\gamma^+ > \gamma^-$), the economy may exhibit path dependence. }\DIFaddend A temporary negative \DIFdelbegin \DIFdel{sentiment shock can cause firms to sever links}\DIFdelend \DIFaddbegin \DIFadd{shock can push the economy to the sparse equilibrium}\DIFaddend . Because reforming \DIFdelbegin \DIFdel{these }\DIFdelend links is costly ($\gamma^+ > 0$), the economy \DIFdelbegin \DIFdel{may remain trapped in a sparse network equilibrium even after beliefs }\DIFdelend \DIFaddbegin \DIFadd{remains sparse even when signals }\DIFaddend recover, unless a sufficiently \DIFdelbegin \DIFdel{strong }\DIFdelend \DIFaddbegin \DIFadd{large }\DIFaddend positive shock coordinates a \DIFdelbegin \DIFdel{move back }\DIFdelend \DIFaddbegin \DIFadd{return }\DIFaddend to the dense equilibrium.
\DIFaddbegin \end{remark}
\DIFaddend 

%==============================================================================
\section{Conclusion}\DIFdelbegin %DIFDELCMD < \label{sec:concl}
%DIFDELCMD < %%%
\DIFdelend \DIFaddbegin \label{sec:conclusion}
\DIFaddend %==============================================================================

\DIFdelbegin \DIFdel{This paper has argued that the structure of production networks is endogenous to the beliefs of the firms that comprise them. By integrating dispersed , affiliated informationinto a model of costly network formation, we identified a sentiment multiplier that amplifies fundamental shocks. }\DIFdelend \DIFaddbegin \DIFadd{We developed a theory of endogenous production network formation under dispersed information. The production environment follows \mbox{%DIFAUXCMD
\citet{acemoglu2020endogenous} }\hspace{0pt}%DIFAUXCMD
but introduces private affiliated signals about aggregate productivity. Affiliation implies that a firm's optimism about fundamentals is also optimism about others' optimism. Under input complementarities, this inference generates strategic complementarities in supplier adoption decisions.
}\DIFaddend 

\DIFaddbegin \DIFadd{Our main results establish existence of extremal monotone Bayesian Nash equilibria and derive comparative statics with respect to adoption costs and belief shifts. The analysis provides a disciplined sense in which sentiment can be self-reinforcing through network formation: changes in beliefs can reorganize supplier sets and thereby change equilibrium input prices. The belief-adjusted Domar weights show precisely how the information structure enters first order in determining aggregate productivity.
}

\DIFaddend Our results imply that supply chain robustness is not merely a technological question but an informational one. Policies that improve transparency---standardizing data on upstream capacity or aggregate input flows---can reduce the correlation of forecast errors, dampening the strategic amplification of noise.
\DIFdelbegin \DIFdel{Conversely, in opaque environments, rational firms will coordinate on sentiment, generating excess volatility and potential fragility.
}%DIFDELCMD < 

%DIFDELCMD < %%%
\DIFdelend %==============================================================================
%DIF <  REFERENCES
\DIFaddbegin \appendix
\section{\DIFadd{Mathematical Preliminaries}}\label{app:math}
\DIFaddend %==============================================================================
\DIFdelbegin %DIFDELCMD < \bibliographystyle{plainnat}
%DIFDELCMD < \bibliography{references}
%DIFDELCMD < %%%
\DIFdelend 

\DIFdelbegin %DIFDELCMD < \newpage
%DIFDELCMD < \appendix
%DIFDELCMD < %%%
\section{\DIFdel{Omitted Proofs}}%DIFAUXCMD
\addtocounter{section}{-1}%DIFAUXCMD
%DIFDELCMD < \label{app:proofs}
%DIFDELCMD < 

%DIFDELCMD < %%%
\subsection{\DIFdel{Proof of Lemma \ref{lem:complementarities}}}
%DIFAUXCMD
\addtocounter{subsection}{-1}%DIFAUXCMD
\DIFdel{We must show $\Delta(s_i) = \E[ \Pi(S') - \Pi(S) \mid s_i ]$ is increasing in $s_i$.
Define the cost difference function $g(\mu, \s_{-i}) = K(S, \mu, P(\mu, \sigma_{-i}(\s_{-i}))) - K(S', \mu, P(\mu, \sigma_{-i}(\s_{-i})))$. We verify $g$ is increasing in its arguments:
}%DIFDELCMD < 

%DIFDELCMD < %%%
\textbf{\DIFdel{State $\mu$:}} %DIFAUXCMD
\DIFdel{The unit cost $K_i$ is proportional to $1/A_i(\mu)$. The cost saving $g$ scales with $1/A_i$. Assuming convex costs, the benefit of expansion is increasing in productivity. }%DIFDELCMD < 

%DIFDELCMD < %%%
\textbf{\DIFdel{Opponent Signals $\s_{-i}$:}} %DIFAUXCMD
\DIFdel{Since $\sigma_{-i}$ is monotone, higher $\s_{-i}$ implies larger $S_{-i}$. By the P-matrix property of production networks, larger $S_{-i}$ implies lower equilibrium prices $P$. By \mbox{%DIFAUXCMD
\cref{ass:submod}}\hspace{0pt}%DIFAUXCMD
, lower prices increase the cost saving from expansion.
}%DIFDELCMD < 

%DIFDELCMD < %%%
\DIFdel{Thus, $g$ is increasing. Since signals are affiliated, the conditional expectation of an increasing function is increasing \mbox{%DIFAUXCMD
\citep{milgrom1982theory}}\hspace{0pt}%DIFAUXCMD
. Thus, $\Delta(s_i)$ is increasing.
}%DIFDELCMD < 

%DIFDELCMD < %%%
\subsection{\DIFdel{Derivation of the Sentiment Multiplier}}
%DIFAUXCMD
\addtocounter{subsection}{-1}%DIFAUXCMD
\DIFdel{In the symmetric Gaussian limit, let $y$ be }\DIFdelend \DIFaddbegin \DIFadd{This appendix collects the standard results used in }\DIFaddend the \DIFdelbegin \DIFdel{network action.
The best response function is linear: $y_i = \alpha \E[\mu|s_i] + \beta \E[y_{-i}|s_i]$.
}\DIFdelend \DIFaddbegin \DIFadd{main text. Proofs can be found in the cited references.
}\DIFaddend 

\DIFdelbegin \DIFdel{In a symmetric equilibrium, the strategy is linear: $y(s) = c s$. Substituting this into the best response:
}\begin{align*}
\DIFdel{c s }&\DIFdel{= \alpha \rho s + \beta \E[c s_{-i}|s_i] }\\
\DIFdel{c s }&\DIFdel{= \alpha \rho s + \beta c \rho s.
}\end{align*}%DIFAUXCMD
\DIFdel{Solving for $c$ yields $c (1 - \beta \rho) = \alpha \rho$, or $c = \frac{\alpha \rho}{1 - \beta \rho}$.
}\DIFdelend \DIFaddbegin \subsection{\DIFadd{Lattice theory}}
\DIFaddend 

\DIFdelbegin \DIFdel{The term $(1-\beta \rho)^{-1}$ represents the multiplier.
}%DIFDELCMD < 

%DIFDELCMD < %%%
\subsection{\DIFdel{Lattice Theory Preliminaries}}
%DIFAUXCMD
\addtocounter{subsection}{-1}%DIFAUXCMD
%DIFDELCMD < 

%DIFDELCMD < %%%
\DIFdelend \begin{definition}
A \textbf{lattice} is a partially ordered set $(L,\preceq)$ where every pair $x,y$ has a least upper bound $x\vee y$ (join) and greatest lower bound $x\wedge y$ (meet). A lattice is \textbf{complete} if every subset has a join and meet.
\end{definition}

\DIFaddbegin \begin{definition}
\DIFadd{A function $f:L\to\R$ is }\textbf{\DIFadd{supermodular}} \DIFadd{if $f(x\vee y)+f(x\wedge y)\ge f(x)+f(y)$. A function $f:L \times T \to \R$ has }\textbf{\DIFadd{increasing differences}} \DIFadd{in $(x,t)$ if $f(x',t') - f(x,t') \ge f(x',t) - f(x,t)$ for $x' \succeq x$ and $t' \succeq t$.
}\end{definition}

\DIFaddend \begin{theorem}[Tarski's Fixed Point Theorem]\label{thm:tarski}
Let $L$ be a complete lattice and $f:L\to L$ be isotone\DIFdelbegin \DIFdel{(order-preserving). Then the set of fixed points $\mathrm{Fix}(f) = \{x \in L : f(x) = x\}$ }\DIFdelend \DIFaddbegin \DIFadd{. Then $\mathrm{Fix}(f)$ }\DIFaddend is a nonempty complete lattice.
\end{theorem}

\DIFdelbegin %DIFDELCMD < \begin{proof}
%DIFDELCMD < %%%
\DIFdel{Let $A = \{x \in L : x \preceq f(x)\}$. Since $L$ is complete, $A$ has a supremum $\bar{x} = \bigvee A$. For any $x \in A$, we have $x \preceq f(x)$. Since $x \preceq \bar{x}$ and $f$ is isotone, $f(x) \preceq f(\bar{x})$. Thus $x \preceq f(\bar{x})$ for all $x \in A$, so $\bar{x} \preceq f(\bar{x})$. Since $\bar{x} \preceq f(\bar{x})$ and $f$ is isotone, $f(\bar{x}) \preceq f(f(\bar{x}))$. Thus $f(\bar{x}) \in A$, so $f(\bar{x}) \preceq \bar{x}$. Combining, $f(\bar{x}) = \bar{x}$.
}%DIFDELCMD < \end{proof}
%DIFDELCMD < 

%DIFDELCMD < %%%
\DIFdelend \begin{theorem}[Topkis's Monotonicity Theorem]\label{thm:topkis}
\DIFdelbegin \DIFdel{Let $L$ be a lattice and $T$ a partially ordered set. }\DIFdelend If $f:L\times T\to\R$ is supermodular in $x$ and has increasing differences in $(x,t)$, then \DIFdelbegin \DIFdel{$\argmax_{x \in L} f(x,t)$ }\DIFdelend \DIFaddbegin \DIFadd{$\argmax_x f(x,t)$ }\DIFaddend is isotone in $t$\DIFdelbegin \DIFdel{(in the strong set order)}\DIFdelend .
\end{theorem}

\DIFaddbegin \subsection{\DIFadd{Affiliation and stochastic dominance}}

\begin{definition}
\DIFadd{Random variables $Z$ with density $f$ are }\textbf{\DIFadd{affiliated}} \DIFadd{if $f(z \vee z') f(z \wedge z') \ge f(z) f(z')$ for all $z, z'$.
}\end{definition}

\begin{theorem}[Milgrom--Weber]\label{thm:milgrom_weber}
\DIFadd{Affiliation implies MLRP; MLRP implies FOSD.
}\end{theorem}

\begin{theorem}[FOSD Integration]\label{thm:fosd_int}
\DIFadd{If $F' \ge_{\mathrm{FOSD}} F$ and $g$ is increasing, then $\E_{F'}[g] \ge \E_F[g]$.
}\end{theorem}

%DIF > ==============================================================================
\bibliographystyle{plainnat}
\bibliography{references}
%DIF > ==============================================================================
 \DIFaddend\end{document}
