\documentclass[11pt,a4paper]{article}

% ===== Packages =====
\usepackage[utf8]{inputenc}
\usepackage[T1]{fontenc}
\usepackage{mathpazo}           % Palatino for math
\usepackage{palatino}           % Palatino for text
\usepackage{microtype}          % Better typography
\usepackage{amsmath,amssymb,amsthm}
\usepackage{mathtools}
\usepackage{bbm}
\usepackage{enumitem}
\usepackage[margin=1in]{geometry}
\usepackage{setspace}
\usepackage{graphicx}
\usepackage{booktabs}
\usepackage{natbib}
\usepackage{hyperref}
\usepackage{cleveref}

\hypersetup{
  colorlinks=true,
  linkcolor=blue,
  citecolor=blue,
  urlcolor=blue
}

% ===== Theorem Environments =====
\newtheorem{theorem}{Theorem}
\newtheorem{lemma}[theorem]{Lemma}
\newtheorem{proposition}[theorem]{Proposition}
\newtheorem{corollary}[theorem]{Corollary}
\theoremstyle{definition}
\newtheorem{definition}[theorem]{Definition}
\newtheorem{assumption}{Assumption}
\newtheorem{remark}{Remark}
\newtheorem{example}{Example}

% ===== Convenient Macros =====
\newcommand{\E}{\mathbb{E}}
\newcommand{\R}{\mathbb{R}}
\newcommand{\N}{\mathcal{N}}
\newcommand{\1}{\mathbbm{1}}
\newcommand{\I}{\mathcal{I}}
\newcommand{\Sset}{\mathcal{S}}
\newcommand{\Pcal}{\mathcal{P}}
\newcommand{\BR}{\mathrm{BR}}
\DeclareMathOperator*{\argmin}{arg\,min}
\DeclareMathOperator*{\argmax}{arg\,max}

% ===== Document Metadata =====
\title{\textbf{Sentiment and Supply Chains:\\ How Beliefs Shape Production Networks}}
\author{
    Author Name\thanks{Affiliation. Email: author@university.edu}
}
\date{\today}

\begin{document}

\maketitle

\begin{abstract}
\noindent We study production network formation when firms have private, correlated signals about aggregate productivity. Each firm chooses which supplier links to adopt (extensive margin) and input quantities (intensive margin) under general production technologies satisfying homogeneity and labor essentiality. When technologies exhibit input complementarities and signals are affiliated, the induced Bayesian game has strategic complementarities. Using lattice-theoretic methods---without differentiability---we prove existence of extremal monotone Bayesian Nash equilibria where firms with higher signals adopt denser networks. The ``sentiment multiplier'' arises because firms cannot distinguish fundamentals from correlated noise: optimistic signals trigger network expansions that lower input prices and reinforce further expansion. We establish existence via P-matrix theory and comparative statics via monotone methods.
\end{abstract}

\medskip
\noindent\textbf{Keywords:} Production networks, dispersed information, strategic complementarities, supermodular games, P-matrices, affiliation

\medskip
\noindent\textbf{JEL Codes:} D21, D83, L14, L23

\onehalfspacing

%========================================
\section{Introduction}
%========================================

Supply chains are opaque webs of trust. A manufacturer deciding whether to invest in a new supplier relationship rarely observes the precise productivity of that partner, the aggregate state of demand, or upstream capacity constraints. Decisions are made under dispersed information: delivery delays, procurement chatter, small price movements, and local shortages. These signals are naturally correlated across firms because they reflect common macro and sectoral factors.

This paper asks: \emph{how does dispersed, correlated information interact with the endogenous formation of production networks?} Networks are not passive objects that transmit shocks; they are chosen in response to beliefs about shocks. When firms cannot disentangle fundamentals from correlated noise, ``sentiment'' becomes self-reinforcing through network formation.

We develop a model of endogenous production network formation under private information, following the framework of \citet{acemoglu2020endogenous}. Production technologies are general---satisfying homogeneity of degree one and labor essentiality---with CES as a special case. Firms engage in two-stage cost minimization: first choosing input quantities for a given supplier set (intensive margin), then choosing the cost-minimizing technology (extensive margin). Equilibrium prices are determined by contestability (price equals marginal cost) and market clearing.

The key innovation is that firms observe private signals about an aggregate productivity state. Under affiliated information \citep{milgrom1982theory}, a higher signal makes a firm both (i) more optimistic about fundamentals and (ii) more optimistic that others are optimistic. This creates strategic complementarities that amplify belief-driven network dynamics.

\paragraph{Related Literature.} This paper contributes to the production networks literature \citep{long1983real, horvath2000sectoral, gabaix2011granular, acemoglu2012network}, endogenous network formation \citep{oberfield2018theory, acemoglu2020endogenous}, supply chain disruptions \citep{barrot2016input, carvalho2021supply}, uncertainty \citep{bloom2009impact, bloom2018really, jurado2015measuring}, and Bayesian games with strategic complementarities \citep{van2007monotone, milgrom1994monotone}.


%========================================
\section{Information Structure}\label{sec:information}
%========================================

We begin with the information structure because it is foundational to the strategic analysis. Let $\mu \in \mathcal{M} \subseteq \R$ be an aggregate productivity state. Each firm $i \in \I = \{1, \ldots, n\}$ observes a private signal $s_i \in \R$.

\subsection{Affiliation and Stochastic Dominance}

\begin{definition}[Affiliation]\label{def:affiliation}
Random variables $Z = (Z_1, \ldots, Z_m)$ with joint density $f$ on a product lattice are \emph{affiliated} if for all $z, z'$:
\[
f(z \vee z') f(z \wedge z') \geq f(z) f(z'),
\]
where $\vee$ and $\wedge$ denote componentwise max and min.
\end{definition}

Affiliation formalizes ``positive correlation'' and is satisfied by many natural models, including Gaussian common factors and log-concave densities.

\begin{assumption}[Affiliated Information]\label{ass:affiliated}
The vector $(\mu, s_1, \ldots, s_n)$ is affiliated.
\end{assumption}

\begin{theorem}[Milgrom-Weber 1982]\label{thm:mlrp}
Suppose Assumption~\ref{ass:affiliated} holds. Then:
\begin{enumerate}[label=(\roman*)]
    \item The conditional distribution of $\mu$ given $s_i$ satisfies the monotone likelihood ratio property (MLRP) in $s_i$: for $s_i' > s_i$,
    \[
    \frac{f(\mu \mid s_i')}{f(\mu \mid s_i)} \text{ is increasing in } \mu.
    \]
    \item MLRP implies first-order stochastic dominance (FOSD):
    \[
    s_i' > s_i \implies \pi(\cdot \mid s_i') \geq_{\text{FOSD}} \pi(\cdot \mid s_i).
    \]
    \item Beliefs about other signals are also FOSD-ordered:
    \[
    s_i' > s_i \implies \E[g(s_{-i}) \mid s_i'] \geq \E[g(s_{-i}) \mid s_i]
    \]
    for any increasing function $g$.
\end{enumerate}
\end{theorem}

\begin{theorem}[FOSD Integration]\label{thm:fosd_integration}
Let $F' \geq_{\text{FOSD}} F$ be two distributions. Then for any increasing function $g$:
\[
\E_{F'}[g] \geq \E_F[g].
\]
\end{theorem}

These theorems are the workhorses of our strategic analysis: they connect higher signals to more optimistic beliefs about fundamentals \emph{and} about others' signals.

\begin{example}[Gaussian Common Factor]\label{ex:gaussian}
Let $\mu \sim \mathcal{N}(\bar{\mu}, \sigma_\mu^2)$ and $s_i = \mu + \varepsilon_i$ with i.i.d.\ $\varepsilon_i \sim \mathcal{N}(0, \sigma_\varepsilon^2)$. Then $(\mu, s_1, \ldots, s_n)$ are affiliated. The posterior is:
\[
\mu \mid s_i \sim \mathcal{N}\left( \frac{\sigma_\mu^2 s_i + \sigma_\varepsilon^2 \bar{\mu}}{\sigma_\mu^2 + \sigma_\varepsilon^2}, \frac{\sigma_\mu^2 \sigma_\varepsilon^2}{\sigma_\mu^2 + \sigma_\varepsilon^2} \right).
\]
\end{example}


%========================================
\section{Environment}\label{sec:environment}
%========================================

\subsection{Technology}

There are $n$ industries, each producing a distinct good. Industry $i$ chooses:
\begin{itemize}
    \item A \textbf{supplier set} $S_i \subseteq \I \setminus \{i\}$
    \item \textbf{Intermediate inputs} $X_i = (X_{ij})_{j \neq i}$ with $X_{ij} = 0$ for $j \notin S_i$
    \item \textbf{Labor} $L_i \geq 0$
\end{itemize}

Production is given by:
\[
Y_i = \theta_i(\mu) F_i(S_i, A_i(S_i), L_i, X_i),
\]
where $\theta_i(\mu) = e^{\varphi_i \mu + \eta_i}$ with $\varphi_i \geq 0$ (sensitivity to aggregate state) and $\eta_i$ (firm-specific, known to all).

\begin{assumption}[Production Technology]\label{ass:technology}
For each $i$ and $S_i$:
\begin{enumerate}[label=(\roman*)]
    \item $F_i(S_i, A_i, L_i, X_i)$ is continuous, strictly increasing in $(L_i, X_i)$, and strictly increasing in $A_i$.
    \item $F_i$ is homogeneous of degree 1 in $(L_i, X_i)$.
    \item \textbf{(Labor Essentiality)} $F_i(S_i, A_i, 0, X_i) = 0$ for all $X_i$.
\end{enumerate}
\end{assumption}

\begin{remark}
Assumption~\ref{ass:technology} does \emph{not} require differentiability. The CES family
\[
F_i = A_i(S_i) \left[ \gamma_L^{1/\sigma} L_i^{\frac{\sigma-1}{\sigma}} + \sum_{j \in S_i} \alpha_{ij}^{1/\sigma} X_{ij}^{\frac{\sigma-1}{\sigma}} \right]^{\frac{\sigma}{\sigma-1}}
\]
satisfies Assumption~\ref{ass:technology} and is used as a leading example, but our main results hold for general $F_i$.
\end{remark}

\subsection{Cost Minimization}

Following \citet{acemoglu2020endogenous}, we characterize firm behavior via cost minimization, not profit maximization.

\paragraph{Step 1: Intensive Margin.} Given supplier set $S_i$, prices $P$, and productivity $A_i(S_i)$, the \textbf{unit cost function} is:
\begin{equation}\label{eq:unit_cost}
K_i(S_i, A_i(S_i), P) = \min_{L_i, X_i} \left\{ L_i + \sum_{j \in S_i} P_j X_{ij} : F_i(S_i, A_i, L_i, X_i) = 1 \right\}.
\end{equation}
By homogeneity of degree 1, the cost of producing $Y_i$ units is $Y_i \cdot K_i(S_i, A_i, P)$.

\paragraph{Step 2: Extensive Margin.} Given prices $P$, the firm chooses the cost-minimizing technology:
\begin{equation}\label{eq:technology_choice}
S_i^* \in \argmin_{S_i} K_i(S_i, A_i(S_i), P).
\end{equation}

\begin{remark}
The two-step decomposition holds because the problem is separable: conditional on $S_i$, optimal $(L_i, X_i)$ solve the unit cost problem; the technology choice then compares unit costs across supplier sets.
\end{remark}

\subsection{Equilibrium}

\begin{definition}[Equilibrium]\label{def:equilibrium}
Given state $\mu$, an equilibrium is a tuple $(P^*, S^*, C^*, L^*, X^*, Y^*)$ such that:
\begin{enumerate}
    \item \textbf{(Contestability)} $P_i^* = (1 + \mu_i) \theta_i(\mu)^{-1} K_i(S_i^*, A_i(S_i^*), P^*)$ for all $i$.
    \item \textbf{(Cost Minimization)} $L_i^*, X_i^*$ solve \eqref{eq:unit_cost}; $S_i^*$ solves \eqref{eq:technology_choice}.
    \item \textbf{(Consumer Maximization)} $C^*$ maximizes utility given $P^*$ and income.
    \item \textbf{(Market Clearing)} For all $i$: $C_i^* + \sum_j X_{ji}^* = Y_i^*$ and $\sum_i L_i^* = 1$.
\end{enumerate}
\end{definition}


%========================================
\section{Equilibrium Existence}\label{sec:existence}
%========================================

We establish equilibrium existence using P-matrix theory, which does not require differentiability of production functions.

\subsection{P-Matrices and Global Inversion}

\begin{definition}[P-Matrix]\label{def:pmatrix}
A square matrix $B$ is a \textbf{P-matrix} if all its principal minors are positive.
\end{definition}

\begin{theorem}[Hawkins-Simon 1948]\label{thm:hawkins_simon}
A matrix $B = I - A$ with $A \geq 0$ is a P-matrix if and only if $(I-A)^{-1}$ exists and has non-negative entries.
\end{theorem}

\begin{theorem}[Gale-Nikaido 1965]\label{thm:gale_nikaido}
Let $\Phi: \R^n \to \R^n$ be continuously differentiable. If the Jacobian $D\Phi(x)$ is a P-matrix for all $x$, then $\Phi$ is a global homeomorphism.
\end{theorem}

\subsection{Existence of Equilibrium Prices}

Fix a production network $S = (S_1, \ldots, S_n)$ and state $\mu$. Equilibrium prices satisfy:
\[
p_i = \log(1 + \mu_i) - \varphi_i \mu - \eta_i + k_i(S_i, a_i(S_i), p),
\]
where $p_i = \log P_i$, $a_i = \log A_i$, and $k_i = \log K_i$.

\begin{proposition}[Existence and Uniqueness of Prices]\label{prop:price_existence}
Under Assumption~\ref{ass:technology}, for any fixed network $S$ and state $\mu$, there exists a unique equilibrium price vector $P^*(S, \mu)$.
\end{proposition}

\begin{proof}
Define $\Phi(p) = p - k(S, a(S), p) - b$, where $b_i = \log(1 + \mu_i) - \varphi_i \mu - \eta_i$. The Jacobian is $D\Phi = I - J_{k,p}$, where $J_{k,p}$ has entries $\partial k_i / \partial p_j$.

By labor essentiality (Assumption~\ref{ass:technology}(iii)), there exists $\theta < 1$ such that $\sum_j \partial k_i / \partial p_j < \theta$ for all $i$. This ensures $I - J_{k,p}$ has all eigenvalues with positive real parts and is a P-matrix (Theorem~\ref{thm:hawkins_simon}).

By Theorem~\ref{thm:gale_nikaido}, $\Phi$ is a global homeomorphism, so there exists a unique $p^*$ with $\Phi(p^*) = 0$.
\end{proof}


%========================================
\section{Strategic Complementarities}\label{sec:complementarities}
%========================================

We now establish strategic complementarities without assuming differentiability.

\subsection{Action Space as a Lattice}

Each firm's action is $a_i = (S_i, L_i, X_i) \in \Sset_i = 2^{\I \setminus \{i\}} \times [0, \bar{L}] \times [0, \bar{X}]^{n-1}$.

Define the partial order $\succeq$ by:
\[
(S_i, L_i, X_i) \succeq (S_i', L_i', X_i') \iff S_i \supseteq S_i', \, L_i \geq L_i', \, X_i \geq X_i' \text{ componentwise}.
\]

\begin{lemma}[Action Space is a Compact Lattice]\label{lem:lattice}
Under the order $\succeq$, $\Sset_i$ is a compact lattice.
\end{lemma}

\begin{proof}
The power set $2^{\I \setminus \{i\}}$ is a finite lattice under inclusion with $\vee = \cup$, $\wedge = \cap$. The intervals $[0, \bar{L}]$ and $[0, \bar{X}]^{n-1}$ are compact complete lattices under the usual order. The Cartesian product of lattices is a lattice with componentwise operations.
\end{proof}

\subsection{Supermodularity}

\begin{assumption}[Technological Complementarity]\label{ass:complementarity}
The production function $F_i$ exhibits \emph{increasing differences} in inputs: for $S_i' \supseteq S_i$ and $X' \geq X$,
\[
F_i(S_i', A_i(S_i'), L, X') - F_i(S_i, A_i(S_i), L, X') \geq F_i(S_i', A_i(S_i'), L, X) - F_i(S_i, A_i(S_i), L, X).
\]
\end{assumption}

\begin{lemma}[Supermodularity of Payoffs]\label{lem:supermod}
Under Assumptions~\ref{ass:technology} and \ref{ass:complementarity}, the profit function is supermodular in $a_i = (S_i, L_i, X_i)$.
\end{lemma}

\begin{proof}
By \citet{topkis1998supermodularity}, a function on a lattice is supermodular if and only if it has increasing differences in each pair of variables. Assumption~\ref{ass:complementarity} provides increasing differences between $(S_i, X_i)$. Revenue $P_i \theta_i(\mu) F_i(a_i)$ inherits this property (positive scalar multiplication preserves supermodularity). Costs $L_i + \sum_j P_j X_{ij} + \gamma |S_i|$ are modular (additively separable). Subtracting a modular function preserves supermodularity.
\end{proof}

\subsection{Price-Action Single Crossing}

\begin{assumption}[Monotone Price Equilibrium]\label{ass:monotone_price}
The equilibrium price map $P^*: (a, \mu) \mapsto P^*(a, \mu)$ is \emph{antitone} in $a$: if $a \succeq a'$, then $P^*(a, \mu) \leq P^*(a', \mu)$ componentwise.
\end{assumption}

\begin{lemma}[Single Crossing in Prices]\label{lem:price_sc}
Under Assumptions~\ref{ass:technology} and \ref{ass:monotone_price}, payoffs have increasing differences in $(a_i, a_{-i})$.
\end{lemma}

\begin{proof}
If $a_{-i}' \succeq a_{-i}$, then $P^*(a_i, a_{-i}', \mu) \leq P^*(a_i, a_{-i}, \mu)$ by Assumption~\ref{ass:monotone_price}. For $a_i' \succeq a_i$, the incremental profit from expanding is:
\[
\Delta\Pi(P) = \Pi_i(a_i', P) - \Pi_i(a_i, P).
\]
Since costs are linear in $(P_j X_{ij})$ with negative sign, $\Delta\Pi(P)$ is decreasing in $P$. Lower prices from $a_{-i}' \succeq a_{-i}$ increase the gain from expansion.
\end{proof}

\subsection{Information Single Crossing}

\begin{lemma}[Information Single Crossing]\label{lem:info_sc}
Under Assumptions~\ref{ass:affiliated} and \ref{ass:complementarity}, if opponents use monotone strategies $\sigma_{-i}$, then expected payoffs satisfy single crossing in $(a_i, s_i)$: for $a_i' \succeq a_i$ and $s_i' > s_i$,
\[
\E[\Pi_i(a_i', \sigma_{-i}(s_{-i}); \mu, P^*) - \Pi_i(a_i, \sigma_{-i}(s_{-i}); \mu, P^*) \mid s_i'] \geq \E[\cdot \mid s_i].
\]
\end{lemma}

\begin{proof}
Define the gain function $h(\mu, s_{-i}) = \Pi_i(a_i', \sigma_{-i}(s_{-i}); \mu, P^*) - \Pi_i(a_i, \sigma_{-i}(s_{-i}); \mu, P^*)$.

\textbf{Step 1:} $h$ is increasing in $\mu$. Revenue $P_i \theta_i(\mu) [F_i(a_i') - F_i(a_i)]$ is increasing in $\mu$ since $\theta_i(\mu) = e^{\varphi_i \mu + \eta_i}$ with $\varphi_i \geq 0$ and $F_i(a_i') \geq F_i(a_i)$.

\textbf{Step 2:} $h$ is increasing in $s_{-i}$. Monotone $\sigma_{-i}$ implies $\sigma_{-i}(s_{-i}') \succeq \sigma_{-i}(s_{-i})$ for $s_{-i}' \geq s_{-i}$. By Assumption~\ref{ass:monotone_price}, this lowers $P^*$. By Lemma~\ref{lem:price_sc}, lower prices increase $h$.

\textbf{Step 3:} Apply Theorem~\ref{thm:mlrp}. Since $h$ is increasing in $(\mu, s_{-i})$ and beliefs about $(\mu, s_{-i})$ are FOSD-ordered in $s_i$, Theorem~\ref{thm:fosd_integration} implies:
\[
\E[h(\mu, s_{-i}) \mid s_i'] \geq \E[h(\mu, s_{-i}) \mid s_i].
\]
\end{proof}


%========================================
\section{Monotone Equilibria}\label{sec:equilibrium}
%========================================

\begin{theorem}[Existence of Extremal Monotone Equilibria]\label{thm:existence}
Under Assumptions~\ref{ass:affiliated}, \ref{ass:technology}, \ref{ass:complementarity}, and \ref{ass:monotone_price}, the Bayesian game admits a nonempty complete lattice of monotone Bayesian Nash equilibria. In particular, there exist greatest and least equilibria $\bar{\sigma}$ and $\underline{\sigma}$ in the lattice of monotone strategies.
\end{theorem}

\begin{proof}
We verify the conditions of \citet{van2007monotone}.

\textbf{Step 1: Strategy Lattice.} Let $\Sigma_i$ be the set of monotone (isotone) functions $\sigma_i: \R \to \Sset_i$. By Lemma~\ref{lem:lattice}, $\Sset_i$ is a compact lattice. The set $\Sigma = \prod_i \Sigma_i$ is a complete lattice under pointwise order.

\textbf{Step 2: Supermodularity.} By Lemma~\ref{lem:supermod}, payoffs are supermodular in $a_i$.

\textbf{Step 3: Increasing Differences.} By Lemma~\ref{lem:price_sc}, payoffs have increasing differences in $(a_i, a_{-i})$.

\textbf{Step 4: Single Crossing.} By Lemma~\ref{lem:info_sc}, payoffs satisfy single crossing in $(a_i, s_i)$.

\textbf{Step 5: Monotone Best Responses.} By \citet{milgrom1994monotone}, under supermodularity and single crossing, the best-response correspondence admits an isotone selection in $(s_i, \sigma_{-i})$.

\textbf{Step 6: Fixed Point.} The aggregate best-response map $\BR: \Sigma \to \Sigma$ is isotone. By \citet{tarski1955lattice}, an isotone map on a complete lattice has a nonempty complete lattice of fixed points.
\end{proof}


%========================================
\section{Comparative Statics}\label{sec:comparative}
%========================================

\begin{theorem}[Adoption Cost Reduction]\label{thm:gamma}
Let $\gamma$ be the per-link adoption cost. The extremal equilibria $\bar{\sigma}$ and $\underline{\sigma}$ are antitone in $\gamma$: lower $\gamma$ expands equilibrium networks.
\end{theorem}

\begin{proof}
The term $-\gamma |S_i|$ has decreasing differences in $(S_i, \gamma)$. By monotone comparative statics for supermodular games \citep{topkis1998supermodularity}, extremal fixed points are antitone in $\gamma$.
\end{proof}

\begin{theorem}[Information Multiplier]\label{thm:beliefs}
If interim beliefs shift upward in the FOSD sense, the extremal equilibria expand.
\end{theorem}

\begin{proof}
More optimistic beliefs increase the expected gain from expansion at every opponent strategy (by FOSD integration applied to the gain function $h$). This shifts best responses upward. By monotone comparative statics, extremal fixed points increase.
\end{proof}


%========================================
\section{Dynamic Extension}\label{sec:dynamic}
%========================================

Consider a dynamic setting with link adjustment costs:
\[
\Gamma(S_{it}, S_{i,t-1}) = \gamma^+ |S_{it} \setminus S_{i,t-1}| + \gamma^- |S_{i,t-1} \setminus S_{it}|.
\]

\begin{theorem}[Monotone Markov Equilibria and Hysteresis]\label{thm:dynamic}
Under dynamic analogues of the assumptions, there exist Markov perfect equilibria in monotone strategies. Starting from a denser network $A_0$, the equilibrium path remains denser. Temporary shocks have persistent effects when $\gamma^+, \gamma^- > 0$.
\end{theorem}


%========================================
\section{Conclusion}\label{sec:conclusion}
%========================================

We have developed a rigorous theory of production network formation under dispersed information. Using general production technologies satisfying homogeneity and labor essentiality (not just CES), we established equilibrium existence via P-matrix theory and strategic complementarities via lattice methods---all without differentiability. The affiliated information structure generates an ``information multiplier'': correlated signals trigger self-reinforcing network expansions or contractions.


%========================================
\appendix
\section{Mathematical Preliminaries}\label{app:math}
%========================================

\subsection{Lattice Theory}

\begin{definition}
A \textbf{lattice} is a partially ordered set $(L, \preceq)$ where every pair $x, y$ has a least upper bound $x \vee y$ (join) and greatest lower bound $x \wedge y$ (meet). A lattice is \textbf{complete} if every subset has a join and meet.
\end{definition}

\begin{theorem}[Tarski 1955]\label{thm:tarski}
Let $L$ be a complete lattice and $f: L \to L$ be isotone. Then the set of fixed points of $f$ is a nonempty complete lattice.
\end{theorem}

\begin{definition}
A function $f: L \to \R$ on a lattice is \textbf{supermodular} if for all $x, y \in L$:
\[
f(x \vee y) + f(x \wedge y) \geq f(x) + f(y).
\]
\end{definition}

\begin{theorem}[Topkis 1998]\label{thm:topkis}
If $f: L \times T \to \R$ is supermodular in $x$ and has increasing differences in $(x, t)$, then $\argmax_x f(x, t)$ is isotone in $t$.
\end{theorem}

\subsection{P-Matrix Theory}

\begin{proof}[Proof of Theorem~\ref{thm:hawkins_simon}]
The Hawkins-Simon conditions state that $I - A$ is a P-matrix iff all leading principal minors of $I - A$ are positive. This is equivalent to $(I - A)^{-1} = \sum_{k=0}^\infty A^k$ converging with non-negative entries, which holds iff the spectral radius $\rho(A) < 1$. For $A \geq 0$, this is equivalent to having a productive Leontief system.
\end{proof}

\subsection{Affiliation Theory}

\begin{proof}[Proof of Theorem~\ref{thm:mlrp}]
The affiliation inequality $f(z \vee z') f(z \wedge z') \geq f(z) f(z')$ is equivalent to log-supermodularity of $f$. Integrating over $z_{-i}$, the conditional density $f(\mu \mid s_i)$ inherits log-supermodularity in $(\mu, s_i)$, which is the MLRP. MLRP implies FOSD by standard arguments (integrate the likelihood ratio).
\end{proof}


%========================================
\bibliographystyle{aer}
\bibliography{references}
%========================================

\end{document}
