\documentclass[11pt,a4paper]{article}

% ===== Packages =====
\usepackage[utf8]{inputenc}
\usepackage[T1]{fontenc}
\usepackage{mathpazo}           % Palatino for math
\usepackage{palatino}           % Palatino for text
\usepackage{microtype}          % Better typography
\usepackage{amsmath,amssymb,amsthm}
\usepackage{mathtools}
\usepackage{bbm}
\usepackage{enumitem}
\usepackage[margin=1in]{geometry}
\usepackage{setspace}
\usepackage{graphicx}
\usepackage{booktabs}
\usepackage{natbib}
\usepackage{hyperref}
\usepackage{cleveref}

\hypersetup{
  colorlinks=true,
  linkcolor=blue,
  citecolor=blue,
  urlcolor=blue
}

% ===== Theorem Environments =====
\newtheorem{theorem}{Theorem}
\newtheorem{lemma}[theorem]{Lemma}
\newtheorem{proposition}[theorem]{Proposition}
\newtheorem{corollary}[theorem]{Corollary}
\theoremstyle{definition}
\newtheorem{definition}[theorem]{Definition}
\newtheorem{assumption}{Assumption}
\newtheorem{remark}{Remark}
\newtheorem{example}{Example}

% ===== Convenient Macros =====
\newcommand{\E}{\mathbb{E}}
\newcommand{\R}{\mathbb{R}}
\newcommand{\N}{\mathcal{N}}
\newcommand{\1}{\mathbbm{1}}
\newcommand{\I}{\mathcal{I}}
\newcommand{\Sset}{\mathcal{S}}
\newcommand{\Pcal}{\mathcal{P}}
\newcommand{\BR}{\mathrm{BR}}
\newcommand{\s}{\mathbf{s}}
\DeclareMathOperator*{\argmin}{arg\,min}
\DeclareMathOperator*{\argmax}{arg\,max}

% ===== Document Metadata =====
\title{\textbf{Sentiment and Supply Chains:\\ How Beliefs Shape Production Networks}}
\author{
    Author Name\thanks{Affiliation. Email: author@university.edu}
}
\date{\today}

\begin{document}


\maketitle

\begin{abstract}
\noindent We study production network formation when firms have private, correlated signals about aggregate productivity. Each firm chooses supplier links (the extensive margin) and input quantities (the intensive margin) under general production technologies satisfying homogeneity and labor essentiality. When technologies exhibit input complementarities and signals are affiliated, the induced Bayesian game features strategic complementarities. Using lattice-theoretic methods, we prove existence of extremal monotone Bayesian Nash equilibria in which firms with higher signals adopt weakly larger supplier sets. We then derive comparative statics that keep beliefs at the center of the analysis: adoption costs shift networks in the standard direction, while first-order shifts in interim beliefs generate an additional strategic channel operating through expectations about others' expansion and the resulting equilibrium price system.
\end{abstract}

\medskip
\noindent\textbf{Keywords:} Production networks, dispersed information, strategic complementarities, supermodular games, P-matrices, affiliation

\medskip
\noindent\textbf{JEL Codes:} D21, D83, D85, L14, L23

\onehalfspacing

%==============================================================================
\section{Introduction}\label{sec:introduction}
%==============================================================================

Supplier relationships are formed under uncertainty. Establishing a new input relationship typically requires search, contracting, and relationship-specific investments. These choices are rarely made with complete information about aggregate conditions or about upstream capacity. Instead, firms rely on partial and noisy indicators---delivery delays, procurement quotes, and local shortages. Because these indicators reflect common macroeconomic and sectoral forces, different firms' signals are correlated.

This paper studies how dispersed and correlated information shapes the endogenous formation of production networks. In our setting, networks are not passive objects that merely transmit shocks. They are equilibrium outcomes of firms' technology choices. When firms cannot disentangle fundamentals from correlated noise, changes in sentiment can reorganize the network itself, and the resulting change in input prices feeds back into further technology adoption.

We build on the endogenous network framework of \citet{acemoglu2020endogenous}. Firms operate constant-returns technologies with labor essentiality. Given a supplier set, a firm chooses labor and intermediate inputs to minimize unit costs; it then chooses the supplier set that minimizes unit costs at prevailing prices, taking into account per-link adoption costs. We depart from the complete-information benchmark by assuming that the aggregate productivity state is unobserved. Each firm instead observes a private signal about the state. Signals are \emph{affiliated} in the sense of \citet{milgrom1982theory}, so a higher signal makes a firm more optimistic about fundamentals and, crucially, more optimistic about other firms' signals.

Affiliation introduces a strategic channel that is distinct from the technological channel emphasized in complete-information models. With input complementarities, a firm's gain from expanding its supplier set is higher when other firms expand because expansion lowers equilibrium input prices. Under affiliation, a firm's signal is also informative about others' expansion. The interaction of these forces generates strategic complementarities in the Bayesian game of network formation.

Our analysis delivers two sets of results. First, we show that the network formation problem is a supermodular Bayesian game under general conditions on production technology and on the information structure. We then apply lattice methods to prove existence of extremal monotone Bayesian Nash equilibria. In these equilibria, firms with higher signals adopt weakly larger supplier sets. Second, we use the same monotone structure to derive comparative statics. Lower adoption costs expand equilibrium networks in the standard sense. More importantly for our setting, a first-order shift toward more optimistic interim beliefs expands equilibrium networks through both a direct channel (higher expected fundamentals) and a strategic channel (higher expected expansion by others and therefore lower expected input prices).

\paragraph{Related literature.}
This paper contributes to the production networks literature \citep{long1983real, horvath2000sectoral, gabaix2011granular, acemoglu2012network} and to models of endogenous network formation \citep{oberfield2018theory, acemoglu2020endogenous}. On supply-chain disruptions and propagation, see \citet{barrot2016input, carvalho2021supply}. On uncertainty and dispersed information, see \citet{bloom2009impact, bloom2018really, jurado2015measuring}. Methodologically, we rely on the theory of supermodular games \citep{topkis1998supermodularity, milgrom1994monotone} and on existence results for monotone equilibria in Bayesian games \citep{van2007monotone}.

\paragraph{Roadmap.}
\cref{sec:information} introduces the information structure.
\cref{sec:environment} defines the production environment.
\cref{sec:equilibrium_characterization} collects the equilibrium definition and characterizes equilibrium prices conditional on a network.
\cref{sec:complementarities} establishes strategic complementarities.
\cref{sec:monotone_equilibria} proves existence of extremal monotone equilibria.
\cref{sec:comparative} presents comparative statics.
\cref{sec:domar} derives belief-adjusted Domar weights.
\cref{sec:dynamic} sketches a dynamic extension, and \cref{sec:conclusion} concludes.

%==============================================================================
\section{Model}\label{sec:model}
%==============================================================================

We consider an economy with $n$ industries. Each industry $i$ produces a differentiated good using labor and intermediate inputs purchased from a subset of other industries. Production is subject to an aggregate productivity shock $\mu$ that is not directly observed; instead, each firm receives a private signal correlated with the shock. The model generates strategic complementarities in network formation because a firm's inference about fundamentals is also an inference about other firms' network choices.

\subsection{Information and beliefs}

There is an aggregate productivity state $\mu \in \R$. Firm $i$ observes a private signal $s_i \in \R$. The joint distribution of $(\mu, s_1, \ldots, s_n)$ satisfies an affiliation condition that orders beliefs in a useful way.

\begin{assumption}[Affiliated information]\label{ass:affiliated}
The vector $(\mu, s_1, \ldots, s_n)$ has a joint density $f$ that is log-supermodular: for all $z, z' \in \R^{n+1}$,
\[
f(z \vee z') \, f(z \wedge z') \ge f(z) \, f(z'),
\]
where $\vee$ and $\wedge$ denote componentwise maximum and minimum.
\end{assumption}

Affiliation is a strong form of positive correlation. The Gaussian factor model---where $\mu \sim \N(\bar\mu, \sigma_\mu^2)$ and $s_i = \mu + \varepsilon_i$ with i.i.d.\ noise $\varepsilon_i \sim \N(0, \sigma_\varepsilon^2)$---is a leading example. The posterior mean $\E[\mu \mid s_i] = \frac{\sigma_\mu^2 s_i + \sigma_\varepsilon^2 \bar\mu}{\sigma_\mu^2 + \sigma_\varepsilon^2}$ is increasing in $s_i$.

The key implication of affiliation is that beliefs are ordered by first-order stochastic dominance. A higher signal $s_i$ shifts the distribution of $\mu$ upward; it also shifts beliefs about other firms' signals upward. This ordering underlies the comparative statics developed below.

\begin{theorem}[Affiliation orders beliefs]\label{thm:mlrp}
Under \cref{ass:affiliated}: (i) The conditional density of $\mu$ given $s_i$ satisfies the monotone likelihood ratio property (MLRP). (ii) MLRP implies $\pi(\cdot \mid s_i') \ge_{\mathrm{FOSD}} \pi(\cdot \mid s_i)$ for $s_i' > s_i$. (iii) Beliefs about other signals are ordered: $\E[g(s_{-i}) \mid s_i]$ is increasing in $s_i$ for any increasing function $g$.
\end{theorem}

\subsection{Production technology}

Industry $i$ chooses a supplier set $S_i \subseteq \I \setminus \{i\}$, labor $L_i \ge 0$, and intermediate inputs $X_i = (X_{ij})_{j \in S_i}$. Output is
\[
Y_i = \theta_i(\mu) \, F_i(S_i, A_i(S_i), L_i, X_i),
\]
where the productivity shifter $\theta_i(\mu) = \exp(\varphi_i \mu + \eta_i)$ links firm output to the aggregate state. The parameter $\varphi_i \ge 0$ governs sensitivity; $\eta_i$ is an idiosyncratic component known to all. The term $A_i(S_i)$ captures deterministic productivity gains from the supplier set.

\begin{assumption}[Technology]\label{ass:technology}
For each $i$ and supplier set $S_i$: (i) $F_i$ is continuous and strictly increasing in inputs; (ii) $F_i$ is homogeneous of degree one in $(L_i, X_i)$; (iii) Labor is essential: $F_i(S_i, A_i, 0, X_i) = 0$.
\end{assumption}

The Cobb-Douglas specification $F_i = A_i(S_i) L_i^{\alpha_i} \prod_{j \in S_i} X_{ij}^{\beta_{ij}}$ with $\alpha_i + \sum_j \beta_{ij} = 1$ satisfies these conditions. A CES aggregator also works provided inputs are complements ($\sigma \le 1$); with gross substitutes ($\sigma > 1$), labor essentiality fails.

\subsection{Cost minimization and network choice}

Following \citet{acemoglu2020endogenous}, firm behavior reduces to cost minimization. Given a supplier set $S_i$ and input prices $P$, the \textbf{unit cost function} is
\begin{equation}\label{eq:unit_cost}
K_i(S_i, A_i(S_i), P) = \min_{L_i, X_i} \left\{ L_i + \sum_{j \in S_i} P_j X_{ij} : F_i(S_i, A_i, L_i, X_i) = 1 \right\}.
\end{equation}
By homogeneity, producing $Y_i$ units costs $Y_i \cdot K_i$. At the extensive margin, firm $i$ chooses a supplier set to minimize total cost:
\begin{equation}\label{eq:technology_choice}
S_i^*(P) \in \argmin_{S_i \subseteq \I \setminus \{i\}} \left\{ K_i(S_i, A_i(S_i), P) + \gamma |S_i| \right\},
\end{equation}
where $\gamma \ge 0$ is a per-link adoption cost. This two-stage structure---choose suppliers, then optimize inputs---means the strategic game is over supplier sets rather than continuous input choices.

\subsection{Payoffs}

Given network $S$, state $\mu$, and equilibrium prices $P^*(S, \mu)$, firm $i$'s payoff from supplier set $S_i$ is
\begin{equation}\label{eq:payoff}
\Pi_i(S_i, S_{-i}; \mu) = P_i^* Y_i^* - \left( L_i^* + \sum_{j \in S_i} P_j^* X_{ij}^* \right) - \gamma |S_i|,
\end{equation}
where $(L_i^*, X_i^*)$ solve \cref{eq:unit_cost} and $Y_i^* = \theta_i(\mu) F_i(S_i, A_i, L_i^*, X_i^*)$. Under contestability, output price equals a markup over unit cost: $P_i = (1 + \tau_i) \theta_i(\mu)^{-1} K_i$. This implies operating profits are a constant share $\tau_i / (1 + \tau_i)$ of revenue; the network choice affects profits through the adoption cost $\gamma |S_i|$ and through the unit cost $K_i$.

\subsection{Timing and household}

A representative household supplies one unit of labor inelastically and consumes the $n$ goods with homothetic preferences. The sequence of events is:
\begin{enumerate}
\item Nature draws $(\mu, s_1, \ldots, s_n)$.
\item Each firm $i$ observes $s_i$ and chooses supplier set $S_i$.
\item The state $\mu$ is realized, production takes place, and prices clear markets.
\end{enumerate}
The game is a Bayesian game over supplier sets, with the price system determined by equilibrium conditions.

%==============================================================================
\section{Competitive Equilibrium and Price Characterization}\label{sec:equilibrium_characterization}
%==============================================================================

This section collects the equilibrium definition and the price characterization used in the strategic analysis. The key object is the equilibrium price mapping $P^*(S,\mu)$. For each network profile $S$ and realized state $\mu$, the mapping returns the equilibrium price vector. Existence and uniqueness allow us to treat supplier choices as inducing a well-defined price system.

\subsection{Competitive equilibrium}

We adopt a reduced-form notion of contestability: prices equal a constant markup over unit costs. Let $\tau_i\ge 0$ denote sector $i$'s markup parameter.

\begin{definition}[Competitive equilibrium]\label{def:equilibrium}
Fix a realized state $\mu$ and a network profile $S=(S_1,\ldots,S_n)$. A competitive equilibrium is a tuple $(P,C,L,X,Y)$ such that:
\begin{enumerate}[label=(\roman*)]
    \item (\emph{Contestability}) For each $i$,
    \begin{equation}\label{eq:contestability}
        P_i = (1+\tau_i)\,\theta_i(\mu)^{-1}\,K_i(S_i,A_i(S_i),P).
    \end{equation}
    \item (\emph{Cost minimization}) Given $(S_i,P)$, the choices $(L_i,X_i)$ attain the minimum in \cref{eq:unit_cost}.
    \item (\emph{Household optimization}) $C$ maximizes household utility given prices and income.
    \item (\emph{Market clearing}) For each $i$, $C_i+\sum_{j}X_{ji}=Y_i$, and $\sum_i L_i=1$.
\end{enumerate}
\end{definition}

\subsection{Existence and uniqueness of equilibrium prices}

Fix $S$ and $\mu$. Taking logs in \cref{eq:contestability} yields
\begin{equation}\label{eq:log_price_system}
p_i = \log(1+\tau_i) - (\varphi_i\mu+\eta_i) + k_i(S_i,a_i(S_i),p),
\end{equation}
where $p_i=\log P_i$, $a_i=\log A_i$, and $k_i=\log K_i$.

\begin{definition}[P-matrix]\label{def:pmatrix}
A square matrix $B$ is a \textbf{P-matrix} if all its principal minors are positive.
\end{definition}

\begin{theorem}[Hawkins--Simon]\label{thm:hawkins_simon}
A matrix $B=I-A$ with $A\ge 0$ is a P-matrix if and only if $B^{-1}$ exists and has nonnegative entries.
\end{theorem}

\begin{theorem}[Gale--Nikaido]\label{thm:gale_nikaido}
Let $\Phi:\R^n\to\R^n$ be continuously differentiable. If the Jacobian $D\Phi(x)$ is a P-matrix for all $x$, then $\Phi$ is a global homeomorphism.
\end{theorem}

\begin{proposition}[Existence and uniqueness of prices]\label{prop:price_existence}
Suppose \cref{ass:technology} holds and that $p\mapsto k(S,a(S),p)$ is continuously differentiable. Fix $S$ and $\mu$. Then \cref{eq:log_price_system} has a unique solution $p^*(S,\mu)$, and therefore a unique equilibrium price vector $P^*(S,\mu)$.
\end{proposition}

\begin{proof}
Define $\Phi(p)=p-k(S,a(S),p)-b$, where $b_i=\log(1+\tau_i)-(\varphi_i\mu+\eta_i)$. Then $p$ solves \cref{eq:log_price_system} if and only if $\Phi(p)=0$. The Jacobian is $D\Phi(p)=I-J_{k,p}(p)$, where $J_{k,p}(p)$ has entries $\partial k_i/\partial p_j$.

In differentiable environments, $\partial k_i/\partial p_j$ coincides with a cost share. Labor essentiality implies that the share of intermediate inputs in unit costs is strictly below one. Thus there exists $\kappa\in(0,1)$ such that for all $i$ and all $p$,
\[
\sum_{j=1}^n \frac{\partial k_i}{\partial p_j}(p) \le \kappa.
\]
Since $J_{k,p}(p)\ge 0$, the bound implies that $I-J_{k,p}(p)$ is a P-matrix by \cref{thm:hawkins_simon}. Hence $D\Phi(p)$ is a P-matrix for all $p$, and \cref{thm:gale_nikaido} implies that $\Phi$ is a global homeomorphism. Therefore $\Phi$ has a unique zero.
\end{proof}

\subsection{Monotone price response}

Our strategic results use a monotonicity property of the equilibrium price mapping. We now establish this property under natural conditions on the technology.

\begin{assumption}[Isotone technology productivity]\label{ass:isotone_A}
For each firm $i$, the productivity term $A_i(S_i)$ is isotone in $S_i$ under set inclusion: $S_i' \supseteq S_i$ implies $A_i(S_i') \ge A_i(S_i)$.
\end{assumption}

\begin{lemma}[Monotone price response]\label{lem:monotone_price}
Suppose \cref{ass:technology,ass:isotone_A} hold. Fix $\mu$. If $S' \succeq S$ (element-wise set inclusion), then $P^*(S',\mu) \le P^*(S,\mu)$ componentwise.
\end{lemma}

\begin{proof}
By \cref{ass:isotone_A}, $S_i' \supseteq S_i$ implies $A_i(S_i') \ge A_i(S_i)$. A larger supplier set combined with higher productivity reduces unit costs: $K_i(S_i', A_i(S_i'), P) \le K_i(S_i, A_i(S_i), P)$ for any $P$.

Consider the price mapping $T: P \mapsto P'$ where $P'_i = (1+\tau_i) \theta_i(\mu)^{-1} K_i(S_i, A_i(S_i), P)$. Under network $S'$, the mapping $T'$ has $T'_i(P) \le T_i(P)$ for all $P$ and $i$. Since both $T$ and $T'$ are contractions on $[\underline{P}, \bar{P}]$ (the spectral radius is below 1 by labor essentiality), and $T' \le T$ pointwise, the unique fixed points satisfy $P^{*}(S') \le P^*(S)$ by a standard monotone operator argument.
\end{proof}

%==============================================================================
\section{Strategic Complementarities}\label{sec:complementarities}
%==============================================================================

This section shows that network formation is a Bayesian game of strategic complementarities. The argument combines three ingredients: the action space is a lattice, payoffs are supermodular in own actions, and higher signals shift interim beliefs upward about both fundamentals and opponents' actions.

\subsection{Action space as a lattice}

Each firm's action is $a_i=(S_i,L_i,X_i)$, where
\[
a_i \in \Sset_i \equiv 2^{\I\setminus\{i\}} \times [0,\bar L] \times [0,\bar X]^{n-1}.
\]
We order actions by
\[
(S_i,L_i,X_i)\succeq(S_i',L_i',X_i') \quad\Longleftrightarrow\quad
S_i\supseteq S_i',\ L_i\ge L_i',\ X_i\ge X_i' \text{ componentwise}.
\]

\begin{lemma}[Action space]\label{lem:lattice}
Under $\succeq$, $\Sset_i$ is a compact lattice.
\end{lemma}

\begin{proof}
The power set $2^{\I\setminus\{i\}}$ is a finite lattice under inclusion with $\vee=\cup$ and $\wedge=\cap$. The intervals $[0,\bar L]$ and $[0,\bar X]^{n-1}$ are compact complete lattices under the usual order. The Cartesian product of lattices is a lattice with componentwise join and meet.
\end{proof}

\subsection{Supermodularity}

\begin{assumption}[Technological complementarity]\label{ass:complementarity}
The production function $F_i$ exhibits increasing differences in inputs: for $S_i'\supseteq S_i$ and $X'\ge X$,
\[
F_i(S_i',A_i(S_i'),L,X')-F_i(S_i,A_i(S_i),L,X')
\ge
F_i(S_i',A_i(S_i'),L,X)-F_i(S_i,A_i(S_i),L,X).
\]
\end{assumption}

\begin{lemma}[Supermodularity of payoffs]\label{lem:supermod}
Under \cref{ass:technology,ass:complementarity}, firm $i$'s payoff is supermodular in its own action $a_i$.
\end{lemma}

\begin{proof}
By \citet{topkis1998supermodularity}, a function on a lattice is supermodular if and only if it has increasing differences in each pair of variables. \cref{ass:complementarity} provides increasing differences between $(S_i,X_i)$. Positive scalar multiplication preserves supermodularity. Input expenditures and adoption costs are modular (additively separable). Subtracting a modular function preserves supermodularity.
\end{proof}

\subsection{Price-action single crossing}

\begin{lemma}[Single crossing in prices]\label{lem:price_sc}
Under \cref{ass:isotone_A}, payoffs have increasing differences in $(a_i,a_{-i})$.
\end{lemma}

\begin{proof}
If $a_{-i}'\succeq a_{-i}$, then $P^*(a_i,a_{-i}',\mu)\le P^*(a_i,a_{-i},\mu)$ by \cref{ass:isotone_A}. For $a_i'\succeq a_i$, define the incremental payoff
\[
\Delta\Pi(P)=\Pi_i(a_i',P)-\Pi_i(a_i,P).
\]
Since input costs enter linearly with a negative sign, $\Delta\Pi(P)$ is decreasing in $P$. Lower prices induced by $a_{-i}'$ therefore raise the gain from choosing $a_i'$ rather than $a_i$.
\end{proof}

\subsection{Information single crossing}

\begin{lemma}[Information single crossing]\label{lem:info_sc}
Suppose \cref{ass:affiliated,ass:complementarity} holds and opponents use monotone strategies $\sigma_{-i}$. Then expected payoffs satisfy single crossing in $(a_i,s_i)$: for $a_i'\succeq a_i$ and $s_i'>s_i$,
\[
\E\!\left[\Pi_i(a_i',\sigma_{-i}(s_{-i});\mu,P^*)-\Pi_i(a_i,\sigma_{-i}(s_{-i});\mu,P^*)\mid s_i'\right]
\ge
\E\!\left[\cdot\mid s_i\right].
\]
\end{lemma}

\begin{proof}
Define the gain function
\[
h(\mu,s_{-i})=\Pi_i(a_i',\sigma_{-i}(s_{-i});\mu,P^*)-\Pi_i(a_i,\sigma_{-i}(s_{-i});\mu,P^*).
\]

\textbf{Step 1:} $h$ is increasing in $\mu$. The shifter $\theta_i(\mu)=\exp(\varphi_i\mu+\eta_i)$ is increasing in $\mu$ when $\varphi_i\ge 0$, and $a_i'\succeq a_i$ weakly increases production possibilities. Hence the incremental benefit from $a_i'$ is increasing in $\mu$.

\textbf{Step 2:} $h$ is increasing in $s_{-i}$. Monotone $\sigma_{-i}$ implies $\sigma_{-i}(s_{-i}')\succeq\sigma_{-i}(s_{-i})$ for $s_{-i}'\ge s_{-i}$. By \cref{ass:isotone_A}, this lowers equilibrium prices. By \cref{lem:price_sc}, lower prices increase the gain from expansion, so $h$ is increasing in $s_{-i}$.

\textbf{Step 3:} Apply \cref{thm:mlrp,thm:fosd_integration}. Since $h$ is increasing in $(\mu,s_{-i})$ and the conditional distribution of $(\mu,s_{-i})$ given $s_i$ is ordered by FOSD in $s_i$, we have $\E[h(\mu,s_{-i})\mid s_i']\ge \E[h(\mu,s_{-i})\mid s_i]$.
\end{proof}

%==============================================================================
\section{Monotone Bayesian Nash Equilibria}\label{sec:monotone_equilibria}
%==============================================================================

This section uses lattice methods to establish equilibrium existence in monotone strategies. The result is useful because it gives a disciplined way to describe equilibrium network regimes in a high-dimensional environment.

\begin{theorem}[Extremal monotone equilibria]\label{thm:existence}
Under \cref{ass:affiliated,ass:technology,ass:complementarity,ass:isotone_A}, the Bayesian game admits a nonempty complete lattice of monotone Bayesian Nash equilibria. In particular, there exist greatest and least equilibria $\bar\sigma$ and $\underline\sigma$ in the lattice of monotone strategies.
\end{theorem}

\begin{proof}
We verify the conditions of \citet{van2007monotone}.

\textbf{Step 1: Strategy lattice.} Let $\Sigma_i$ be the set of isotone functions $\sigma_i:\R\to\Sset_i$. By \cref{lem:lattice}, $\Sset_i$ is a compact lattice. The set $\Sigma=\prod_i\Sigma_i$ is a complete lattice under pointwise order.

\textbf{Step 2: Supermodularity.} By \cref{lem:supermod}, payoffs are supermodular in $a_i$.

\textbf{Step 3: Increasing differences.} By \cref{lem:price_sc}, payoffs have increasing differences in $(a_i,a_{-i})$.

\textbf{Step 4: Single crossing.} By \cref{lem:info_sc}, expected payoffs satisfy single crossing in $(a_i,s_i)$.

\textbf{Step 5: Monotone best responses.} By \citet{milgrom1994monotone}, the best-response correspondence admits an isotone selection in $(s_i,\sigma_{-i})$.

\textbf{Step 6: Fixed point.} The best-response operator $\BR:\Sigma\to\Sigma$ is isotone. By \cref{thm:tarski}, an isotone map on a complete lattice has a nonempty complete lattice of fixed points.
\end{proof}

\begin{remark}[Interpretation]
In the greatest monotone equilibrium, firms respond to any signal as aggressively as possible, taking as given that others respond aggressively as well. The least equilibrium is the pessimistic counterpart. This ordering gives a simple language for ``optimistic'' and ``pessimistic'' network regimes.
\end{remark}

%==============================================================================
\section{Comparative Statics}\label{sec:comparative}
%==============================================================================

This section derives comparative statics for the extremal equilibria. The logic is common across results. We first show that a parameter shift moves interim incentives in a monotone direction for every fixed opponent strategy. This shifts best responses. Since the best-response operator is isotone, the ordering transfers to the extremal fixed points.

\subsection{Adoption costs}

\begin{theorem}[Adoption cost reduction]\label{thm:gamma}
Let $\gamma$ be the per-link adoption cost in \cref{eq:technology_choice}. The extremal equilibria $\bar\sigma$ and $\underline\sigma$ are antitone in $\gamma$: lower $\gamma$ expands equilibrium supplier sets.
\end{theorem}

\begin{proof}
The term $-\gamma|S_i|$ has decreasing differences in $(S_i,\gamma)$. By monotone comparative statics for supermodular games \citep{topkis1998supermodularity}, extremal fixed points are antitone in $\gamma$.
\end{proof}

\begin{remark}[Economic content]
Lower adoption costs raise the return to forming supplier links for every configuration of opponents' networks. Because best responses are increasing, this local change propagates through equilibrium prices and expands the network economy-wide.
\end{remark}

\subsection{Belief shifts}

\begin{theorem}[Optimism and network expansion]\label{thm:beliefs}
If interim beliefs shift upward in the FOSD sense, the extremal monotone equilibria expand.
\end{theorem}

\begin{proof}
An upward FOSD shift in beliefs increases the interim expected gain from expansion for every fixed opponent strategy (by \cref{thm:fosd_integration} applied to the gain function $h$ in \cref{lem:info_sc}). This shifts best responses upward. By monotone comparative statics for extremal fixed points, the extremal equilibria increase.
\end{proof}

\begin{remark}[Direct and strategic channels]
The belief shift affects incentives through a direct and a strategic channel. The direct channel raises $\E[\theta_i(\mu)\mid s_i]$ and therefore the expected return to adopting a larger supplier set. The strategic channel raises $\E[g(s_{-i})\mid s_i]$ for increasing $g$ and therefore raises expected opponent expansion. Under \cref{ass:isotone_A}, higher expected opponent expansion lowers expected input prices and further raises the return to expansion.
\end{remark}

%==============================================================================
\section{Belief-Adjusted Domar Weights}\label{sec:domar}
%==============================================================================

We now specialize to the \textbf{Cobb-Douglas/Gaussian} case to obtain explicit formulas for how beliefs enter aggregate productivity. This allows us to define belief-adjusted Domar weights that treat the information structure as first order.

\subsection{Cobb-Douglas production and Gaussian signals}

Assume Cobb-Douglas technology:
\begin{equation}\label{eq:cobb_douglas}
    Y_i = \theta_i(\mu) L_i^{\alpha_i} \prod_{j \in S_i} X_{ij}^{\beta_{ij}}, \quad \text{with } \alpha_i + \sum_{j \in S_i} \beta_{ij} = 1.
\end{equation}
Productivity is log-linear in the common factor:
\[
\theta_i(\mu) = \exp(\varphi_i \mu + \eta_i),
\]
where $\varphi_i > 0$ measures sector $i$'s exposure to aggregate conditions and $\eta_i$ is an idiosyncratic component.

Signals are Gaussian as in \cref{ex:gaussian}: $s_i = \mu + \varepsilon_i$ with $\varepsilon_i \sim \mathcal{N}(0, \sigma_\varepsilon^2)$ independent across $i$ and of $\mu$.

\subsection{Signal-conditioned Domar weights}

Let the equilibrium mapping from the signal profile $\s = (s_1, \ldots, s_n)$ to allocations be
\[
\s \mapsto \big(P(\s), Y(\s), C(\s), S(\s)\big),
\]
where $S(\s)$ is the endogenous network and $(P, Y, C)$ are induced prices, outputs, and final demands.

\begin{definition}[Signal-conditioned Domar weight]
The \textbf{signal-conditioned Domar weight} of sector $i$ is:
\begin{equation}\label{eq:domar}
    D_i(\s) \equiv \frac{P_i(\s) Y_i(\s)}{\sum_{k=1}^n P_k(\s) C_k(\s)} = \frac{P_i(\s) Y_i(\s)}{\mathrm{GDP}(\s)}.
\end{equation}
\end{definition}

This object is a function of signals because signals determine networks and thus prices. It summarizes which sectors are systemically important in the equilibrium induced by the belief state $\s$.

\subsection{Interim Domar weights}

From the perspective of agent $i$, who observes only $s_i$, the relevant object is the expected Domar weight.

\begin{definition}[Interim Domar weight]
Agent $i$'s \textbf{interim Domar weight} for sector $j$ is:
\begin{equation}
    D_j^i(s_i) \equiv \E\left[ D_j(\s) \mid s_i \right].
\end{equation}
\end{definition}

This is what firm $i$ \emph{believes} the Domar weight to be, given its information. Under affiliation, these interim expectations satisfy monotonicity:

\begin{lemma}[Monotonicity of interim Domar weights]
If the network is monotone in signals (\cref{thm:existence}) and larger networks increase sector $j$'s output share, then $D_j^i(s_i)$ is non-decreasing in $s_i$.
\end{lemma}

\subsection{Belief-adjusted Domar elasticities}

The Hulten/Domar logic says that a productivity change in sector $i$ moves aggregate output by that sector's Domar weight. In our setting, we differentiate with respect to the \emph{belief state}, allowing networks to adjust.

Define the posterior mean belief:
\[
\hat{\mu}(\s) \equiv \E[\mu \mid \s], \qquad \hat{\theta}_i(\s) \equiv \exp(\varphi_i \hat{\mu}(\s) + \eta_i).
\]

\begin{definition}[Belief-adjusted Domar elasticity]
The \textbf{belief-adjusted Domar elasticity} of sector $i$ is:
\begin{equation}
    \Lambda_i(\s) \equiv \frac{\partial \log \mathrm{GDP}(\s)}{\partial \log \hat{\theta}_i(\s)}.
\end{equation}
The \textbf{aggregate belief-Domar loading} is:
\begin{equation}\label{eq:lambda}
    \Lambda(\s) \equiv \frac{\partial \log \mathrm{GDP}(\s)}{\partial \hat{\mu}(\s)} = \sum_{i=1}^n \Lambda_i(\s) \cdot \varphi_i.
\end{equation}
\end{definition}

This $\Lambda(\s)$ is the summary statistic that treats beliefs as first order: it captures how a small belief shift about $\mu$ changes aggregate output through \emph{both} the direct fundamental channel and the strategic network channel.

\subsection{Decomposition: Hulten term and strategic amplification}

\begin{proposition}[Belief-adjusted Domar decomposition]\label{prop:domar_decomp}
In the Cobb-Douglas/Gaussian economy with interior equilibrium, the aggregate belief-Domar loading decomposes as:
\begin{equation}\label{eq:decomposition}
    \Lambda(\s) = \underbrace{\sum_{i=1}^n D_i(\s) \cdot \varphi_i}_{\text{Hulten/Domar term}} \times \underbrace{\frac{1}{1 - \theta(\s)}}_{\text{strategic amplification}},
\end{equation}
where the \textbf{equilibrium feedback index} is
\begin{equation}\label{eq:theta}
    \theta(\s) \equiv \rho\big(B(\s)\big) \in (0,1),
\end{equation}
with $B(\s)$ the $n \times n$ matrix of equilibrium input-output shares: $B_{ij}(\s) = \beta_{ij} \cdot \mathbf{1}\{j \in S_i(\s)\}$.
\end{proposition}

\begin{proof}
In the Cobb-Douglas case, the cost function is $K_i = \frac{1}{A_i} \prod_{j \in S_i} P_j^{\beta_{ij}}$ (up to a constant involving $\alpha_i$). Taking logs and substituting the contestability condition $p_i = -\log \theta_i(\mu) + k_i$:
\[
p_i = -(\varphi_i \mu + \eta_i) + \sum_{j \in S_i} \beta_{ij} p_j + c_i,
\]
where $c_i$ collects constants. In matrix form: $p = -\Phi \mu + B(\s) p + c$, where $\Phi = (\varphi_1, \ldots, \varphi_n)'$. Solving: $p = (I - B)^{-1}(c - \Phi \mu)$.

Log-linearizing GDP around the equilibrium:
\[
d \log \text{GDP} = \sum_i D_i \cdot d \log Y_i = \sum_i D_i (\varphi_i \, d\mu - dp_i).
\]
Substituting $dp = -(I-B)^{-1} \Phi \, d\mu$ and simplifying:
\[
\frac{d \log \text{GDP}}{d\mu} = D' \Phi + D' (I-B)^{-1} \Phi = D' (I-B)^{-1} \Phi.
\]
The Leontief inverse $(I-B)^{-1} = \sum_{k=0}^\infty B^k$ captures the geometric series of network spillovers. The amplification factor is bounded by $(1 - \rho(B))^{-1}$ where $\rho(B) < 1$ by labor essentiality. Defining $\theta(\s) = \rho(B(\s))$ yields the decomposition.
\end{proof}

\paragraph{Interpretation.} In optimistic belief states, the endogenous network $S(\s)$ is denser, which increases the nonzero entries in $B(\s)$, pushing $\theta(\s)$ up and making the multiplier larger. When the network is exogenous (fixed $S$), beliefs do not affect $B$ and the amplification factor is constant. Endogenous networks under dispersed information make the amplification state-dependent.

%==============================================================================
\section{Dynamic Extension}\label{sec:dynamic}
%==============================================================================

The static analysis describes network formation in a single period. In many applications, supplier links persist because forming or severing links is costly. This section develops a dynamic extension that preserves the monotone structure, following \citet{van2007monotone}.

\subsection{Dynamic Bayesian game}

Time is discrete, $t = 0, 1, 2, \ldots$. Each period, nature draws a productivity shock $\mu_t$ from a stationary distribution. Firms observe private signals $s_{it}$ correlated with $\mu_t$ and with each other's signals (affiliation). The network choice $S_{it}$ is made at the beginning of period $t$ after observing $s_{it}$.

A firm's \textbf{state} at time $t$ is the pair $(s_{it}, S_{i,t-1})$---the current signal and the inherited network. Firms discount future payoffs at rate $\beta \in (0,1)$. The period payoff is:
\[
\Pi_i(S_{it}, S_{-i,t}, \mu_t) - c(S_{it}, S_{i,t-1}),
\]
where $\Pi_i$ is as in \cref{eq:payoff} and $c(\cdot)$ is an adjustment cost:
\[
c(S_{it}, S_{i,t-1}) = \gamma^+|S_{it}\setminus S_{i,t-1}|+\gamma^-|S_{i,t-1}\setminus S_{it}|.
\]

\subsection{Value function and Bellman equation}

Given opponents' Markov strategy $\sigma_{-i}$, firm $i$'s \textbf{value function} satisfies the Bellman equation:
\begin{equation}\label{eq:bellman}
V_i(s_{it}, S_{i,t-1}; \sigma_{-i}) = \max_{S_{it}} \left\{ \E\left[\Pi_i(S_{it}, \sigma_{-i}(s_{-i,t}), \mu_t) \mid s_{it}\right] - c(S_{it}, S_{i,t-1}) + \beta \, \E\left[V_i(s_{i,t+1}, S_{it}; \sigma_{-i}) \mid s_{it}\right] \right\}.
\end{equation}

The expectation integrates over $(\mu_t, s_{-i,t})$ conditional on $s_{it}$, and over the next-period signal $s_{i,t+1}$ given its realized distribution.

\subsection{Monotone Markov strategies}

We restrict attention to \textbf{Markov strategies} that depend only on the current state $(s_{it}, S_{i,t-1})$, not on the full history. A Markov strategy is \textbf{monotone} if $\sigma_i(s_{it}, S_{i,t-1})$ is non-decreasing in $s_{it}$ (with respect to set inclusion) for each $S_{i,t-1}$.

The space of monotone Markov strategies forms a complete lattice under the pointwise order.

\begin{theorem}[Existence of monotone Markov equilibria]\label{thm:dynamic}
Under dynamic analogues of \cref{ass:affiliated,ass:technology,ass:complementarity,ass:isotone_A}, the dynamic game possesses a greatest and a least monotone Markov perfect equilibrium. In these equilibria, network expansion is monotone in the current signal: optimistic firms expand, pessimistic firms contract.
\end{theorem}

\begin{proof}
We verify the conditions of \citet{van2007monotone}. The Bellman operator $T: V \mapsto TV$ defined by \cref{eq:bellman} is a contraction in the sup-norm with modulus $\beta < 1$, so it has a unique fixed point $V^*$. Given $V^*$, the best-response operator maps monotone strategies to monotone strategies by the single-crossing property (Lemma~\ref{lem:info_sc}). The space of monotone strategies is a complete lattice. By Tarski's fixed point theorem, extremal fixed points exist in the strategy space.
\end{proof}

\subsection{Hysteresis and persistence}

A key feature of the dynamic model is \textbf{belief-network feedback}. As firms observe positive signals, they update beliefs upward, expand networks, benefit from lower prices, and enter the next period with a denser inherited network $S_{i,t-1}$. The adjustment cost creates \textbf{hysteresis}:

\begin{remark}[Hysteresis]
If adjustment costs are asymmetric ($\gamma^+ > \gamma^-$), the economy may exhibit path dependence. A temporary negative shock can push the economy to the sparse equilibrium. Because reforming links is costly ($\gamma^+ > 0$), the economy remains sparse even when signals recover, unless a sufficiently large positive shock coordinates a return to the dense equilibrium.
\end{remark}

%==============================================================================
\section{Conclusion}\label{sec:conclusion}
%==============================================================================

We developed a theory of endogenous production network formation under dispersed information. The production environment follows \citet{acemoglu2020endogenous} but introduces private affiliated signals about aggregate productivity. Affiliation implies that a firm's optimism about fundamentals is also optimism about others' optimism. Under input complementarities, this inference generates strategic complementarities in supplier adoption decisions.

Our main results establish existence of extremal monotone Bayesian Nash equilibria and derive comparative statics with respect to adoption costs and belief shifts. The analysis provides a disciplined sense in which sentiment can be self-reinforcing through network formation: changes in beliefs can reorganize supplier sets and thereby change equilibrium input prices. The belief-adjusted Domar weights show precisely how the information structure enters first order in determining aggregate productivity.

Our results imply that supply chain robustness is not merely a technological question but an informational one. Policies that improve transparency---standardizing data on upstream capacity or aggregate input flows---can reduce the correlation of forecast errors, dampening the strategic amplification of noise.

%==============================================================================
\appendix
\section{Mathematical Preliminaries}\label{app:math}
%==============================================================================

\subsection{Lattice theory}

\begin{definition}
A \textbf{lattice} is a partially ordered set $(L,\preceq)$ where every pair $x,y$ has a least upper bound $x\vee y$ (join) and greatest lower bound $x\wedge y$ (meet). A lattice is \textbf{complete} if every subset has a join and meet.
\end{definition}

\begin{theorem}[Tarski's Fixed Point Theorem]\label{thm:tarski}
Let $L$ be a complete lattice and $f:L\to L$ be isotone (order-preserving). Then the set of fixed points $\mathrm{Fix}(f) = \{x \in L : f(x) = x\}$ is a nonempty complete lattice.
\end{theorem}

\begin{proof}
\textbf{Step 1: Existence of a greatest fixed point.}
Let $A = \{x \in L : x \preceq f(x)\}$. Since $L$ is complete, $A$ has a supremum $\bar{x} = \bigvee A$. We show $\bar{x}$ is a fixed point.

For any $x \in A$, we have $x \preceq f(x)$. Since $x \preceq \bar{x}$ and $f$ is isotone, $f(x) \preceq f(\bar{x})$. Thus $x \preceq f(\bar{x})$ for all $x \in A$, so $\bar{x} \preceq f(\bar{x})$.

Since $\bar{x} \preceq f(\bar{x})$ and $f$ is isotone, $f(\bar{x}) \preceq f(f(\bar{x}))$. Thus $f(\bar{x}) \in A$, so $f(\bar{x}) \preceq \bar{x}$.

Combining, $f(\bar{x}) = \bar{x}$.

\textbf{Step 2: Existence of a least fixed point.}
Define $B = \{x \in L : f(x) \preceq x\}$ and $\underline{x} = \bigwedge B$. By a symmetric argument, $\underline{x}$ is a fixed point.

\textbf{Step 3: The set of fixed points is a complete lattice.}
For any subset $S \subseteq \mathrm{Fix}(f)$, define $A_S = \{x \in L : x \succeq \bigvee S \text{ and } f(x) \preceq x\}$. This set is nonempty (since the greatest fixed point $\bar{x} \in A_S$) and its infimum is a fixed point that serves as the join of $S$ in $\mathrm{Fix}(f)$. Meets are constructed dually.
\end{proof}

\begin{definition}
A function $f:L\to\R$ on a lattice is \textbf{supermodular} if for all $x,y\in L$,
\[
f(x\vee y)+f(x\wedge y)\ge f(x)+f(y).
\]
A function $f:L \times T \to \R$ has \textbf{increasing differences} in $(x,t)$ if for all $x' \succeq x$ and $t' \succeq t$,
\[
f(x',t') - f(x,t') \ge f(x',t) - f(x,t).
\]
\end{definition}

\begin{theorem}[Topkis's Monotonicity Theorem]\label{thm:topkis}
Let $L$ be a lattice and $T$ a partially ordered set. If $f:L\times T\to\R$ is supermodular in $x$ and has increasing differences in $(x,t)$, then $\argmax_{x \in L} f(x,t)$ is isotone in $t$ (in the strong set order).
\end{theorem}

\begin{proof}
\textbf{Step 1: Strong set order.}
For sets $A, B \subseteq L$, we say $A \preceq_s B$ (strong set order) if for all $a \in A$ and $b \in B$, we have $a \wedge b \in A$ and $a \vee b \in B$.

\textbf{Step 2: Key inequality.}
Let $t' \succeq t$ and suppose $x^* \in \argmax_x f(x,t)$ and $x^{**} \in \argmax_x f(x,t')$. We must show $x^* \wedge x^{**} \in \argmax_x f(x,t)$ and $x^* \vee x^{**} \in \argmax_x f(x,t')$.

By optimality: $f(x^*,t) \ge f(x^* \wedge x^{**},t)$ and $f(x^{**},t') \ge f(x^* \vee x^{**},t')$.

By supermodularity: $f(x^* \vee x^{**},t) + f(x^* \wedge x^{**},t) \ge f(x^*,t) + f(x^{**},t)$.

By increasing differences: $f(x^* \vee x^{**},t') - f(x^{**},t') \ge f(x^* \vee x^{**},t) - f(x^{**},t)$.

Combining these inequalities:
\begin{align*}
0 &\ge f(x^* \vee x^{**},t') - f(x^{**},t') & \text{(optimality of } x^{**}) \\
&\ge f(x^* \vee x^{**},t) - f(x^{**},t) & \text{(increasing differences)} \\
&\ge f(x^*,t) - f(x^* \wedge x^{**},t) & \text{(supermodularity)} \\
&\ge 0 & \text{(optimality of } x^*).
\end{align*}
All inequalities are equalities, so $x^* \wedge x^{**} \in \argmax_x f(x,t)$ and $x^* \vee x^{**} \in \argmax_x f(x,t')$.
\end{proof}

\subsection{P-matrix theory}

\begin{definition}
A square matrix $B \in \R^{n \times n}$ is a \textbf{P-matrix} if all its principal minors are positive.
\end{definition}

\begin{theorem}[Hawkins--Simon]\label{thm:hawkins_simon_proof}
Let $A \ge 0$ be a nonnegative matrix. The following are equivalent:
\begin{enumerate}[label=(\roman*)]
    \item $I-A$ is a P-matrix.
    \item $(I-A)^{-1}$ exists and $(I-A)^{-1} \ge 0$.
    \item The spectral radius $\rho(A) < 1$.
\end{enumerate}
\end{theorem}

\begin{proof}
\textbf{(iii) $\Rightarrow$ (ii):}
If $\rho(A) < 1$, the Neumann series $\sum_{k=0}^\infty A^k$ converges. Since $A \ge 0$, each term is nonnegative, so
\[
(I-A)^{-1} = \sum_{k=0}^\infty A^k \ge 0.
\]

\textbf{(ii) $\Rightarrow$ (i):}
For any principal submatrix $A_J$ (corresponding to index set $J$), we have $(I_J - A_J)^{-1} \ge 0$ (the Schur complement structure preserves nonnegativity). A matrix with nonnegative inverse has positive determinant, so all principal minors of $I-A$ are positive.

\textbf{(i) $\Rightarrow$ (iii):}
Suppose $\rho(A) \ge 1$. By the Perron-Frobenius theorem, $A$ has a nonnegative eigenvector $v \ge 0$ with eigenvalue $\lambda = \rho(A) \ge 1$. Then $(I-A)v = (1-\lambda)v$. If $\lambda = 1$, $I-A$ is singular. If $\lambda > 1$, then $(I-A)v = (1-\lambda)v < 0$, contradicting the requirement that $(I-A)^{-1} \ge 0$ maps nonnegative vectors to nonnegative vectors.
\end{proof}

\begin{theorem}[Gale--Nikaido Univalence]\label{thm:gale_nikaido_proof}
Let $\Phi: \R^n \to \R^n$ be continuously differentiable. If the Jacobian $D\Phi(x)$ is a P-matrix for all $x \in \R^n$, then $\Phi$ is a global homeomorphism.
\end{theorem}

\begin{proof}[Proof sketch]
\textbf{Step 1: Local injectivity.}
A P-matrix is nonsingular (all principal minors positive implies full rank). By the inverse function theorem, $\Phi$ is a local diffeomorphism at every point.

\textbf{Step 2: Global injectivity.}
The key insight is that P-matrices are closed under positive diagonal scaling and satisfy a specific ``inwardness'' property. If $\Phi(x) = \Phi(y)$ for $x \ne y$, consider the path $\gamma(t) = \Phi(x + t(y-x))$. The P-matrix property forces $\gamma$ to move in all coordinates simultaneously, preventing $\gamma(0) = \gamma(1)$ unless $x = y$.

\textbf{Step 3: Surjectivity.}
Local injectivity plus the P-matrix property implies that $\|\Phi(x)\| \to \infty$ as $\|x\| \to \infty$ (properness). By Hadamard's theorem, a proper local homeomorphism is a global homeomorphism.
\end{proof}

\subsection{Affiliation and stochastic dominance}

\begin{definition}
Random variables $Z = (Z_1, \ldots, Z_m)$ with joint density $f$ are \textbf{affiliated} if $f$ is log-supermodular:
\[
f(z \vee z') f(z \wedge z') \ge f(z) f(z') \quad \text{for all } z, z'.
\]
\end{definition}

\begin{theorem}[Affiliation implies MLRP]\label{thm:affiliation_mlrp}
If $(Z_1, Z_2)$ are affiliated, then the conditional density $f(z_1 \mid z_2)$ satisfies the monotone likelihood ratio property in $z_2$.
\end{theorem}

\begin{proof}
The conditional density is $f(z_1 \mid z_2) = f(z_1, z_2) / f_{Z_2}(z_2)$. For $z_2' > z_2$ and $z_1' > z_1$:
\begin{align*}
\frac{f(z_1' \mid z_2')}{f(z_1 \mid z_2')} \cdot \frac{f(z_1 \mid z_2)}{f(z_1' \mid z_2)}
&= \frac{f(z_1', z_2') \cdot f(z_1, z_2)}{f(z_1, z_2') \cdot f(z_1', z_2)}.
\end{align*}
Define $z = (z_1, z_2)$ and $z' = (z_1', z_2')$. Then $z \vee z' = (z_1', z_2')$ and $z \wedge z' = (z_1, z_2)$. By log-supermodularity:
\[
f(z_1', z_2') \cdot f(z_1, z_2) \ge f(z_1, z_2') \cdot f(z_1', z_2).
\]
Thus the ratio is $\ge 1$, which is the MLRP condition.
\end{proof}

\begin{theorem}[MLRP implies FOSD]\label{thm:mlrp_fosd}
If $f(\cdot \mid s')$ dominates $f(\cdot \mid s)$ in the likelihood ratio order for $s' > s$, then $F(\cdot \mid s') \ge_{\mathrm{FOSD}} F(\cdot \mid s)$.
\end{theorem}

\begin{proof}
MLRP means $f(\mu \mid s') / f(\mu \mid s)$ is increasing in $\mu$. Define $\Lambda(\bar{\mu}) = \int_{-\infty}^{\bar{\mu}} f(\mu \mid s') d\mu / \int_{-\infty}^{\bar{\mu}} f(\mu \mid s) d\mu$. 

Since the likelihood ratio is increasing, $\Lambda(\bar{\mu})$ is increasing in $\bar{\mu}$. We have $\Lambda(-\infty) = 0$ and $\Lambda(\infty) = 1$ (both CDFs integrate to 1).

For FOSD, we need $F(\bar{\mu} \mid s') \le F(\bar{\mu} \mid s)$, i.e., $\int_{-\infty}^{\bar{\mu}} f(\mu \mid s') d\mu \le \int_{-\infty}^{\bar{\mu}} f(\mu \mid s) d\mu$.

Equivalently: $\int_{-\infty}^{\bar{\mu}} [f(\mu \mid s') - f(\mu \mid s)] d\mu \le 0$.

The integrands start negative (for small $\mu$, the ratio is below its average of 1) and end positive, crossing zero once. The integral is the area under the difference, which is nonpositive up to every point $\bar{\mu}$.
\end{proof}

\begin{theorem}[FOSD Integration]\label{thm:fosd_int}
If $F' \ge_{\mathrm{FOSD}} F$ and $g$ is increasing, then $\E_{F'}[g(X)] \ge \E_F[g(X)]$.
\end{theorem}

\begin{proof}
By integration by parts:
\[
\E_F[g(X)] = \int_{-\infty}^\infty g(x) dF(x) = g(\infty) - \int_{-\infty}^\infty g'(x) F(x) dx.
\]
Since $F'(x) \le F(x)$ for all $x$ (FOSD) and $g'(x) \ge 0$ (increasing):
\[
\E_{F'}[g(X)] = g(\infty) - \int_{-\infty}^\infty g'(x) F'(x) dx \ge g(\infty) - \int_{-\infty}^\infty g'(x) F(x) dx = \E_F[g(X)].
\]
\end{proof}

\subsection{Proof of price existence (Proposition~\ref{prop:price_existence})}

\begin{proof}[Full proof]
Define the mapping $\Phi: \R^n \to \R^n$ by
\[
\Phi_i(p) = p_i - k_i(S_i, a_i(S_i), p) - b_i,
\]
where $b_i = \log(1+\tau_i) - (\varphi_i \mu + \eta_i)$. Equilibrium prices solve $\Phi(p) = 0$.

\textbf{Step 1: Jacobian structure.}
The Jacobian is $D\Phi(p) = I - J_{k,p}(p)$, where $[J_{k,p}]_{ij} = \partial k_i / \partial p_j$.

By Shepard's lemma, in the differentiable case:
\[
\frac{\partial k_i}{\partial p_j}(p) = \frac{P_j X_{ij}^*(P)}{K_i(S_i, A_i, P)} = \text{cost share of input } j \text{ in sector } i.
\]

\textbf{Step 2: Row sum bound.}
Labor essentiality implies that labor has a positive cost share in every sector. Thus:
\[
\sum_{j=1}^n \frac{\partial k_i}{\partial p_j}(p) = \text{(total intermediate cost share)} < 1.
\]
Let $\kappa = \sup_{i,p} \sum_j \partial k_i / \partial p_j < 1$.

\textbf{Step 3: P-matrix verification.}
Since $J_{k,p} \ge 0$ (cost shares are nonnegative) and row sums are bounded by $\kappa < 1$, we have $\rho(J_{k,p}) \le \kappa < 1$. By \cref{thm:hawkins_simon_proof}, $I - J_{k,p}$ is a P-matrix.

\textbf{Step 4: Global homeomorphism.}
By \cref{thm:gale_nikaido_proof}, $\Phi$ is a global homeomorphism. Therefore $\Phi(p) = 0$ has a unique solution.
\end{proof}

%==============================================================================
\bibliographystyle{plainnat}
\bibliography{references}
%==============================================================================
\end{document}
