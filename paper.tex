\documentclass[12pt,a4paper]{article}

%==============================================================================
% PACKAGES AND CONFIGURATION
%==============================================================================
\usepackage[utf8]{inputenc}
\usepackage[T1]{fontenc}
\usepackage{mathpazo}           % Palatino: A standard, serious font for econ theory
\usepackage{microtype}          % Improved typography
\usepackage{amsmath,amssymb,amsthm,mathtools}
\usepackage{bbm}
\usepackage{enumitem}
\usepackage[margin=1.25in]{geometry} % Generous margins for readability/notes
\usepackage{setspace}
\usepackage{graphicx}
\usepackage{booktabs}
\usepackage[authoryear]{natbib}
\usepackage{titlesec}
\usepackage{xcolor}
\usepackage[colorlinks=true, linkcolor=DarkBlue, citecolor=DarkBlue, urlcolor=DarkBlue]{hyperref}

% Define Econometrica-style colors
\definecolor{DarkBlue}{rgb}{0.0, 0.0, 0.55}

% Formatting section headers to be sober and academic
\titleformat{\section}{\large\bfseries\scshape}{\thesection.}{1em}{}
\titleformat{\subsection}{\bfseries\itshape}{\thesubsection.}{1em}{}
\titlespacing*{\section}{0pt}{3.5ex plus 1ex minus .2ex}{2.3ex plus .2ex}

% Spacing configuration for ~25-30 pages
\onehalfspacing
\setlength{\parskip}{0.6em}
\setlength{\parindent}{1.5em}

%==============================================================================
% THEOREM ENVIRONMENTS
%==============================================================================
\newtheorem{theorem}{Theorem}
\newtheorem{lemma}{Lemma}
\newtheorem{proposition}{Proposition}
\newtheorem{corollary}{Corollary}
\newtheorem{conjecture}{Conjecture}

\theoremstyle{definition}
\newtheorem{definition}{Definition}
\newtheorem{assumption}{Assumption}
\newtheorem{example}{Example}
\newtheorem{condition}{Condition}

\theoremstyle{remark}
\newtheorem{remark}{Remark}

%==============================================================================
% MACROS & NOTATION
%==============================================================================
\newcommand{\E}{\mathbb{E}}
\newcommand{\R}{\mathbb{R}}
\newcommand{\N}{\mathcal{N}}
\newcommand{\1}{\mathbbm{1}}
\newcommand{\I}{\mathcal{I}}         % Set of industries
\newcommand{\Sset}{\mathcal{S}}      % Signal space
\newcommand{\Lcal}{\mathcal{L}}      % Lattice of networks
\newcommand{\Fcal}{\mathcal{F}}      % Filtration/Sigma-algebra
\newcommand{\Mcal}{\mathcal{M}}      % Multiplier matrix
\newcommand{\Acal}{\mathcal{A}}
\newcommand{\BR}{\mathrm{BR}}        % Best Response
\newcommand{\dmu}{\mathrm{d}\mu}
\newcommand{\ve}{\varepsilon}
\newcommand{\s}{\mathbf{s}}          % Signal vector
\newcommand{\p}{\mathbf{p}}          % Price vector
\newcommand{\muvec}{\boldsymbol{\mu}}
\DeclareMathOperator*{\argmin}{arg\,min}
\DeclareMathOperator*{\argmax}{arg\,max}
\DeclareMathOperator{\Cov}{Cov}
\DeclareMathOperator{\Var}{Var}

%==============================================================================
% TITLE PAGE
%==============================================================================
\title{\textbf{Sentiment and Supply Chains:\\ Endogenous Production Networks under Uncertainty}\thanks{We thank Daron Acemoglu and seminar participants for detailed comments. This paper generalizes the framework of \citet{acemoglu2020endogenous} to environments with dispersed information. All errors are our own.}}

\author{
    \textsc{Author Name}\thanks{Department of Economics, University Name. Email: author@university.edu}
}

\date{\today}

\begin{document}

\maketitle

\begin{abstract}
\noindent We study the formation of production networks when firms possess dispersed, affiliated information about aggregate productivity. Firms make extensive margin decisions (choosing supplier sets) and intensive margin decisions (input quantities) under uncertainty. We characterize the economy as a supermodular Bayesian game. Using lattice-theoretic methods, we prove the existence of extremal monotone Bayesian Nash equilibria where firms with optimistic private signals adopt denser sets of inputs. We identify a ``belief multiplier'' mechanism: because signals are affiliated, a firm's optimism rationally raises its expectation of others' optimism, leading to coordinated network expansions that lower equilibrium prices and validate the initial beliefs. We decompose network volatility into fundamental and strategic components, showing that higher signal correlation amplifies volatility while higher signal precision dampens it. The model suggests that opacity in supply chains acts as an amplifier of aggregate shocks.

\bigskip
\noindent\textbf{Keywords:} Production networks, dispersed information, strategic complementarities, supermodular games, affiliation, belief multiplier.

\noindent\textbf{JEL Codes:} D21, D83, D85, L14, E32.
\end{abstract}

\thispagestyle{empty}
\newpage
\setcounter{page}{1}

%==============================================================================
\section{Introduction}
%==============================================================================

Modern supply chains are characterized by two salient features: complexity and opacity. A manufacturer deciding whether to invest in a new specialized supplier relationship or adopt a capital-intensive logistics technology rarely observes the precise productivity of that partner, the aggregate state of demand, or upstream capacity constraints. Instead, decisions are made on the basis of dispersed, noisy signals: procurement delays, industry chatter, small price movements, and local order books. These signals are naturally correlated across firms because they reflect common macroeconomic and sectoral factors.

This paper asks: \emph{How does dispersed, correlated information interact with the endogenous formation of production networks?} Networks are not passive objects that merely transmit shocks; they are constructed by agents acting on beliefs. When firms cannot perfectly disentangle fundamentals from correlated noise, ``sentiment'' becomes a driver of real economic structure.

We develop a model of endogenous production network formation under private information. We build on the framework of \citet{acemoglu2020endogenous}, where firms choose technology sets to minimize unit costs. We depart from the complete information benchmark by assuming that aggregate productivity is unobserved. Instead, firms receive private signals that are \emph{affiliated} in the sense of \citet{milgrom1982theory}. This information structure introduces a distinct strategic channel. In standard production network models, complementarities arise solely from technology: a decrease in supplier prices encourages downstream expansion. In our environment, complementarities also arise from inference.

Our analysis yields three main theoretical contributions.

First, we establish the existence of \textbf{monotone Bayesian Nash equilibria} in the game of network formation. The space of possible production networks is high-dimensional and discrete. However, we show that under general assumptions on production technology (homogeneity of degree one and labor essentiality) and information (affiliation), the induced game is supermodular. Using lattice-theoretic methods---specifically Tarski's fixed point theorem applied to the interim strategy space---we prove that firms with more optimistic private signals monotonically expand their supplier sets. This result implies the existence of distinct ``optimistic'' (high density) and ``pessimistic'' (low density) regimes for the same underlying fundamentals, driven by the coordination of beliefs.

Second, we identify a \textbf{belief multiplier}. In standard Keynesian models, multipliers arise from demand externalities. Here, the multiplier is supply-side and structural. When firms become optimistic, they form denser networks. This expansion lowers unit costs and equilibrium prices throughout the economy, making further expansion profitable for others. The amplification arises through the interaction of affiliated beliefs and strategic complementarities: optimism begets optimism.

Third, we provide comparative statics with respect to the \textbf{information structure}. We show that increasing the precision of private signals reduces network volatility, as firms place less weight on the correlated prior and more on the idiosyncratic realization. Conversely, increasing the correlation of error terms across firms (without changing their marginal variance) increases network volatility. This result suggests that information transparency---reducing the correlation of forecast errors---is a distinct policy tool for stabilizing supply chains.

\paragraph{Related Literature.}
This paper bridges the gap between the production networks literature and the literature on global games and dispersed information. The foundational insight that network structure matters for aggregate volatility dates to \citet{long1983real} and was formalized by \citet{horvath2000sectoral}, \citet{dupor1999aggregation}, and \citet{gabaix2011granular}. \citet{acemoglu2012network} showed that heavy-tailed degree distributions can generate aggregate fluctuations from idiosyncratic shocks, overturning the law of large numbers. Empirical work by \citet{atalay2011network}, \citet{atalay2017sectoral}, and \citet{foerster2011sectoral} quantifies the role of sectoral linkages.

The theoretical framework for production networks was developed by \citet{hulten1978growth}, extended by \citet{jones2011intermediate} and \citet{baqaee2019macroeconomic}, \citet{baqaee2018cascading}, and \citet{baqaee2019productivity}. On endogenous network formation, \citet{oberfield2018theory} and \citet{acemoglu2020endogenous} provide key foundations, while \citet{liu2019industrial} and \citet{bigio2020distortions} study policy implications. Recent work on supply chain disruptions includes \citet{barrot2016input}, \citet{boehm2019input}, \citet{carvalho2021supply}, and \citet{acemoglu2020firms}. On firm heterogeneity in networks, see \citet{bernard2022origins} and \citet{boehm2020misallocation}.

We build directly on \citet{acemoglu2020endogenous}, extending their complete-information analysis to a Bayesian setting. We differ from \citet{kopytov2024endogenous}, who also study uncertainty in networks. Their mechanism relies on risk aversion (second-moment effects): firms diversify to insure against variance. In contrast, our agents are risk-neutral cost minimizers; our results are driven by strategic complementarities in \emph{beliefs} (first-moment effects). We show that even without risk aversion, the coordination motive inherent in supply chains creates fragility.

Our information-theoretic approach connects to the uncertainty literature: \citet{bloom2009impact}, \citet{bloom2014fluctuations}, \citet{bloom2018really}, \citet{baker2016measuring}, and \citet{jurado2015measuring} on uncertainty shocks; \citet{fajgelbaum2017uncertainty} on uncertainty traps; \citet{nieuwerburgh2006learning} on learning in business cycles. On information in networks, \citet{elliott2022supply} studies fragility from strategic link formation, while \citet{herskovic2018networks} examines asset pricing implications. The methodology builds on \citet{topkis1998supermodularity}, \citet{milgrom1994monotone}, and the Bayesian games literature of \citet{van2007monotone} and \citet{morris2002social}.

The remainder of the paper is organized as follows. Section \ref{sec:info} defines the information structure. Section \ref{sec:env} describes the production environment. Section \ref{sec:beliefs} characterizes the belief hierarchy. Section \ref{sec:strategic} establishes strategic complementarities. Section \ref{sec:equilibrium} proves the existence of monotone equilibria. Section \ref{sec:multiplier} formalizes the belief multiplier. Section \ref{sec:statics} presents comparative statics. Section \ref{sec:dynamic} sketches a dynamic extension. Section \ref{sec:concl} concludes.

%==============================================================================
\section{Information Structure}\label{sec:info}
%==============================================================================

We begin by defining the probabilistic environment, which is foundational to the strategic analysis. Consider an economy with $n$ firms indexed by $\I = \{1, \ldots, n\}$. The fundamental state of the economy is described by a random variable $\mu \in \Mcal \subseteq \R$, representing aggregate productivity. This state is unobserved.

Each firm $i$ observes a private signal $s_i \in \Sset_i \subseteq \R$. Let $\s = (s_1, \ldots, s_n)$ denote the profile of signals. The joint distribution of $(\mu, \s)$ is governed by a cumulative distribution function $F(\mu, \s)$ with a strictly positive density $f(\mu, \s)$ with respect to a product measure.

\subsection{Affiliation and Stochastic Dominance}

To capture the idea that signals are correlated reflections of the same underlying reality, we assume the joint distribution satisfies \emph{affiliation}, a strong form of positive dependence introduced by \citet{milgrom1982theory}.

\begin{definition}[Affiliation]\label{def:affiliation}
The random variables $Z = (\mu, s_1, \ldots, s_n)$ are \emph{affiliated} if their joint density $f$ is log-supermodular. That is, for all $z, z'$ in the support of $f$:
\begin{equation}
    f(z \vee z') f(z \wedge z') \geq f(z) f(z'),
\end{equation}
where $\vee$ and $\wedge$ denote the component-wise maximum and minimum, respectively.
\end{definition}

Affiliation implies that a high realization of one variable makes high realizations of the other variables more likely in the sense of the Monotone Likelihood Ratio Property (MLRP).

\begin{assumption}[Affiliated Information]\label{ass:affiliated}
The vector $(\mu, s_1, \ldots, s_n)$ is affiliated.
\end{assumption}

This assumption allows us to order beliefs and higher-order beliefs unambiguously. We invoke the following key properties from \citet{milgrom1982theory}.

\begin{theorem}[Properties of Affiliated Beliefs]\label{thm:milgrom}
Under Assumption \ref{ass:affiliated}:
\begin{enumerate}[label=(\roman*)]
    \item \textbf{MLRP:} The conditional density $f(\mu \mid s_i)$ satisfies the Monotone Likelihood Ratio Property in $s_i$.
    \item \textbf{Stochastic Dominance:} If $s_i' > s_i$, then the posterior distribution of $\mu$ given $s_i'$ dominates the distribution given $s_i$ in the first-order sense (FOSD).
    \item \textbf{Ordered Beliefs about Others:} The conditional distribution of the vector of others' signals $\s_{-i}$ given $s_i$ is increasing in $s_i$ in the multivariate FOSD sense. Specifically, for any non-decreasing function $g(\s_{-i})$, the expectation $\E[g(\s_{-i}) \mid s_i]$ is non-decreasing in $s_i$.
\end{enumerate}
\end{theorem}

Property (iii) is the engine of our strategic analysis. It implies that an optimistic firm (observing high $s_i$) rationally expects other firms to be optimistic as well.

\subsection{Leading Example: Gaussian Common Factor}
Throughout the paper, we use the Gaussian structure to provide closed-form intuition.

\begin{example}[Gaussian Signals]\label{ex:gaussian}
Let $\mu \sim \N(\mu_0, \sigma_\mu^2)$. Each firm observes:
\[
    s_i = \mu + \varepsilon_i, \quad \varepsilon_i \sim \N(0, \sigma_\varepsilon^2),
\]
where $\varepsilon_i$ are i.i.d. across firms and independent of $\mu$. The joint distribution is multivariate normal with covariance $\Cov(s_i, s_j) = \sigma_\mu^2 \ge 0$, which implies affiliation.

The posterior expectation of $\mu$ is linear:
\begin{equation}\label{eq:posterior_mu}
    \E[\mu \mid s_i] = (1-\rho) \mu_0 + \rho s_i, \quad \text{where } \rho = \frac{\sigma_\mu^2}{\sigma_\mu^2 + \sigma_\varepsilon^2}.
\end{equation}
Crucially, firm $i$'s expectation of firm $j$'s signal is:
\begin{equation}\label{eq:posterior_others}
    \E[s_j \mid s_i] = \E[\E[s_j \mid \mu] \mid s_i] = \E[\mu \mid s_i] = (1-\rho) \mu_0 + \rho s_i.
\end{equation}
This explicit linearity allows us to quantify the strength of strategic feedback using the parameter $\rho$.
\end{example}

%==============================================================================
\section{Production Environment}\label{sec:env}
%==============================================================================

We adopt the production framework of \citet{acemoglu2020endogenous}, adapted to an incomplete information setting. There are $n$ sectors. Each sector $i$ produces a distinct good using labor $L_i$ and intermediate inputs.

\subsection{Technology and Costs}

Firm $i$ makes an extensive margin choice: it selects a set of suppliers $S_i \subseteq \I \setminus \{i\}$. Given $S_i$, the production function is:
\begin{equation}
    Y_i = \theta_i(\mu) F_i(S_i, L_i, \{X_{ij}\}_{j \in S_i}),
\end{equation}
where $\theta_i(\mu)$ is a productivity shifter strictly increasing in $\mu$, and $F_i$ is the aggregator.

\begin{assumption}[Technology]\label{ass:tech}
For all $i$ and $S_i$:
\begin{enumerate}[label=(\roman*)]
    \item $F_i$ is continuous, concave, strictly increasing, and homogeneous of degree one (CRS) in inputs.
    \item Labor is essential: $F_i(S_i, 0, \cdot) = 0$.
    \item \textbf{Technological Monotonicity:} For any price vector $P$, the unit cost achievable with a larger supplier set $S_i' \supset S_i$ is weakly lower than with $S_i$.
\end{enumerate}
\end{assumption}

Firms operate in competitive markets. Given a state $\mu$, a network $S = (S_1, \ldots, S_n)$, and a price vector $P$, the unit cost for firm $i$ is derived from cost minimization:
\begin{equation}
    K_i(S_i, \mu, P) = \frac{1}{\theta_i(\mu)} \min_{L_i, \{X_{ij}\}} \left\{ L_i + \sum_{j \in S_i} P_j X_{ij} \mid F_i(\cdot) = 1 \right\}.
\end{equation}
We normalize the wage $w=1$.

\subsection{Market Clearing and Equilibrium Prices}

In the production stage (after $\mu$ is realized and $S$ is fixed), prices must equal unit costs. The equilibrium price vector $P^*(\mu, S)$ is the fixed point of:
\begin{equation}\label{eq:price_system}
    P_i = K_i(S_i, \mu, P) \quad \forall i \in \I.
\end{equation}

\begin{proposition}[Existence and Uniqueness of Prices]\label{prop:prices}
Under Assumption \ref{ass:tech}, for any $\mu$ and network $S$, there exists a unique strictly positive price vector $P^*(\mu, S)$ solving \eqref{eq:price_system}. Furthermore, $P^*$ is decreasing in $\mu$ and non-increasing in $S$ (under the inclusion order).
\end{proposition}
\begin{proof}
See \citet{acemoglu2020endogenous}. The proof relies on the fact that the Jacobian $I - \frac{\partial K}{\partial P}$ is an M-matrix (a class of P-matrices) due to the essentiality of labor, guaranteeing global univalence via the Gale-Nikaido theorem.
\end{proof}

\begin{corollary}[Uniqueness of Allocations]\label{cor:allocations}
Given the unique equilibrium price vector $P^*(\mu, S)$, the allocations $(X^*, Y^*, C^*, L^*)$ are uniquely determined.
\end{corollary}
\begin{proof}
Given $P^*$ and the technology choice $S_i$, the factor demands $L_i^*$ and $X_i^*$ are uniquely determined by the strictly convex cost minimization problem \eqref{eq:unit_cost}. The output $Y_i^*$ follows from market clearing: $Y_i^* = F_i(S_i, L_i^*, X_i^*)$. Aggregate labor supply pins down the scale via $\sum_i L_i^* = 1$. Consumer demands $C^*$ are uniquely determined by utility maximization given $P^*$ and income. By the same Gale-Nikaido argument, this system has a unique solution.
\end{proof}

\subsection{The Network Formation Game}

The extensive margin decision is made \emph{ex ante}. Firm $i$ observes $s_i$ and chooses $S_i$ to minimize expected unit costs. The strategy of firm $i$ is a mapping $\sigma_i: \Sset_i \to 2^{\I \setminus \{i\}}$. Let $\sigma_{-i}$ denote the strategies of opponents.

Firm $i$'s objective is to minimize the expected cost function:
\begin{equation}\label{eq:objective}
    \mathcal{C}_i(S_i, s_i; \sigma_{-i}) = \E \left[ K_i(S_i, \mu, P^*(\mu, S_i, \sigma_{-i}(\s_{-i}))) \mid s_i \right].
\end{equation}
This formulation highlights the strategic interaction: firm $i$'s cost depends on $P^*$, which depends on $S_{-i}$, which depends on $\s_{-i}$ via $\sigma_{-i}$. Thus, firm $i$ must forecast the signals and actions of other firms.

%==============================================================================
\section{Expectations and the Belief Hierarchy}\label{sec:beliefs}
%==============================================================================

In a complete information setting, firm $i$ observes $\mu$ and can perfectly anticipate $S_{-i}$. Here, firm $i$ faces a \emph{hierarchical inference} problem.
\begin{enumerate}
    \item \textbf{First-order belief:} What is $\mu$?
    \item \textbf{Second-order belief:} What do others believe about $\mu$? (This determines their $S_{-i}$).
\end{enumerate}

\begin{lemma}[Monotonicity of Expectations]\label{lem:expectations}
    Let $h(\mu, \s_{-i})$ be a function that is non-decreasing in $\mu$ and in $\s_{-i}$ (component-wise). Then the conditional expectation function
    \begin{equation}
        H(s_i) = \E[h(\mu, \s_{-i}) \mid s_i]
    \end{equation}
    is non-decreasing in $s_i$.
\end{lemma}
\begin{proof}
    This follows directly from Theorem \ref{thm:milgrom}. Affiliation implies that the conditional distribution of the vector $(\mu, \s_{-i})$ given $s_i$ is stochastically increasing in $s_i$. The expectation of an increasing function with respect to a stochastically increasing distribution is increasing.
\end{proof}

This lemma is the mathematical engine of the belief multiplier. It implies that if equilibrium prices are lower when fundamentals are good ($\mu$ high) and when peers are optimistic ($\s_{-i}$ high), then a firm observing a high $s_i$ will rationally expect lower prices.

%==============================================================================
\section{Strategic Complementarities}\label{sec:strategic}
%==============================================================================

To prove the existence of equilibria, we characterize the economy as a supermodular game. This requires defining a lattice structure on strategies and showing the objective function satisfies increasing differences.

\subsection{Lattice Structure}
The set of possible supplier combinations for firm $i$ is $\Lcal_i = 2^{\I \setminus \{i\}}$. Ordered by set inclusion $\subseteq$, $\Lcal_i$ is a complete lattice.
A strategy $\sigma_i$ is \textbf{monotone} if $s_i' > s_i \implies \sigma_i(s_i) \subseteq \sigma_i(s_i')$. The space of monotone strategies $\Sigma_i$ is also a complete lattice under the pointwise order.

\subsection{Payoff Supermodularity}
Let $\Pi_i = -\mathcal{C}_i$ be the payoff (negative cost). We require two conditions:
\begin{enumerate}
    \item \textbf{Strategic Complementarity:} $\Pi_i$ has increasing differences in $(S_i, \sigma_{-i})$.
    \item \textbf{Single-Crossing in Type:} $\Pi_i$ has increasing differences in $(S_i, s_i)$.
\end{enumerate}

We assume the cost function exhibits technological complementarity.

\begin{assumption}[Cost Submodularity]\label{ass:submod}
The unit cost function $K_i(S_i, \mu, P)$ has decreasing differences in $(S_i, P)$. That is, the marginal cost reduction from adding a supplier is larger when input prices $P$ are lower.
\end{assumption}
This assumption holds for Cobb-Douglas production functions and, more generally, whenever adding a supplier yields greater cost savings in environments where input prices are lower. Intuitively, lower prices encourage larger input usage, amplifying the benefit of access to additional suppliers.

\begin{lemma}[Strategic Complementarity]\label{lem:strat_comp}
    Under Assumption \ref{ass:submod}, the payoff $\Pi_i$ has increasing differences in $(S_i, \sigma_{-i})$. That is, if rivals play a ``larger'' strategy $\sigma_{-i}' \succeq \sigma_{-i}$, firm $i$'s incentive to expand $S_i$ increases.
\end{lemma}
\begin{proof}
    If $\sigma_{-i}' \succeq \sigma_{-i}$, then for any realization of $\s_{-i}$, the network $S_{-i}' \supseteq S_{-i}$. By the properties of P-matrices in production networks, larger networks imply lower equilibrium prices $P^*$. By Assumption \ref{ass:submod}, lower prices increase the marginal benefit of expanding $S_i$. Thus, the expected benefit of expansion is higher under $\sigma_{-i}'$.
\end{proof}

\begin{lemma}[Information Single-Crossing]\label{lem:info_sc}
    Under Assumptions \ref{ass:affiliated} and \ref{ass:submod}, if rivals play monotone strategies, then $\Pi_i$ has increasing differences in $(S_i, s_i)$.
\end{lemma}
\begin{proof}
    Let $\Delta(S_i, S_i') = \Pi_i(S_i') - \Pi_i(S_i)$ for $S_i' \supset S_i$. This is the expected cost saving from expansion.
    The realized cost saving depends on $\mu$ and $P$.
    \begin{enumerate}
        \item Higher $\mu$ lowers unit costs directly (via $\theta_i$), scaling up the absolute savings.
        \item Higher $\s_{-i}$ leads to larger $S_{-i}$ (by monotonicity of rivals) and thus lower $P$. Lower $P$ increases savings (Assumption \ref{ass:submod}).
    \end{enumerate}
    Thus, the integrand (cost saving) is increasing in $(\mu, \s_{-i})$.
    By Lemma \ref{lem:expectations} (hierarchical inference), the expected value of this increasing function is increasing in $s_i$.
\end{proof}

%==============================================================================
\section{Monotone Equilibria}\label{sec:equilibrium}
%==============================================================================

We now state the main existence result.

\begin{theorem}[Existence of Extremal Monotone Equilibria]\label{thm:existence}
    The network formation game has a greatest Bayesian Nash Equilibrium $\bar{\sigma}$ and a least Bayesian Nash Equilibrium $\underline{\sigma}$. These equilibria are in monotone pure strategies: for every firm $i$, $\sigma_i(s_i)$ is non-decreasing in $s_i$ with respect to set inclusion.
\end{theorem}

\begin{proof}
    The proof applies Tarski's Fixed Point Theorem to the lattice of monotone strategies.
    \begin{enumerate}
        \item The strategy space $\Sigma = \prod \Sigma_i$ is a complete lattice.
        \item Define the best-response mapping $\Psi: \Sigma \to \Sigma$ where $\Psi_i(\sigma_{-i}) = \argmax_{\tau} \Pi_i(\tau, \sigma_{-i})$.
        \item By Lemma \ref{lem:info_sc}, the objective satisfies single-crossing in type, so the optimal strategy is monotone. Thus $\Psi$ maps $\Sigma$ to $\Sigma$.
        \item By Lemma \ref{lem:strat_comp}, the objective satisfies increasing differences in strategies. By Topkis's Monotonicity Theorem, $\Psi$ is an isotone (order-preserving) map.
        \item By Tarski's Theorem, the set of fixed points of an isotone map on a complete lattice is a non-empty complete lattice.
    \end{enumerate}
\end{proof}

\paragraph{Discussion.}
The existence of extremal equilibria implies potential multiplicity. $\bar{\sigma}$ represents an ``optimistic regime'' where firms coordinate on dense networks, justified by low prices. $\underline{\sigma}$ is a ``pessimistic regime.'' In both regimes, however, network density is strictly increasing in sentiment.

%==============================================================================
\section{The Belief Multiplier}\label{sec:multiplier}
%==============================================================================

The key mechanism in our model is a \emph{belief multiplier}: optimistic beliefs propagate through the network, generating amplified responses in equilibrium network density. Unlike standard multipliers derived from continuous elasticities, our multiplier operates through the discrete, combinatorial structure of network formation.

\begin{proposition}[Belief Multiplier: Monotonicity]\label{prop:multiplier}
Consider two information structures $\mathcal{I}$ and $\mathcal{I}'$ such that the induced posteriors satisfy $\pi'(\cdot \mid s_i) \geq_{\text{FOSD}} \pi(\cdot \mid s_i)$ for all $s_i$ (more optimistic beliefs). Let $\bar{\sigma}$ and $\bar{\sigma}'$ denote the greatest monotone equilibria under $\mathcal{I}$ and $\mathcal{I}'$ respectively. Then:
\[
\bar{\sigma}'(s_i) \supseteq \bar{\sigma}(s_i) \quad \text{for all } s_i.
\]
That is, more optimistic beliefs lead to (weakly) denser equilibrium networks at every signal realization.
\end{proposition}

\begin{proof}
The proof proceeds by monotone comparative statics on the best-response correspondence.

\textbf{Step 1:} Under more optimistic beliefs $\mathcal{I}'$, for any fixed opponent strategy $\sigma_{-i}$, firm $i$'s expected cost of expansion decreases. This follows from FOSD: if $\mu$ is expected to be higher, then $\theta_i(\mu)$ is higher and unit costs $K_i$ are lower. Moreover, if others' signals are expected to be higher (via affiliation), expected prices $\E[P^* \mid s_i]$ are lower.

\textbf{Step 2:} By the single-crossing property (Lemma~\ref{lem:info_sc}), the best-response correspondence $\BR_i(\sigma_{-i}; \mathcal{I}')$ is pointwise greater than $\BR_i(\sigma_{-i}; \mathcal{I})$.

\textbf{Step 3:} The greatest equilibrium is the limit of iterated best responses starting from the maximal strategy. Since the best-response operator shifts upward under $\mathcal{I}'$, the fixed point $\bar{\sigma}'$ is pointwise greater than $\bar{\sigma}$.
\end{proof}

This proposition captures the \emph{belief multiplier} without assuming differentiability or continuous actions. The amplification arises because:
\begin{enumerate}
    \item \textbf{Direct channel:} A higher signal $s_i$ raises $\E[\mu \mid s_i]$, directly increasing the expected benefit of expansion.
    \item \textbf{Strategic channel:} A higher signal $s_i$ raises $\E[s_j \mid s_i]$ for $j \neq i$ (by affiliation). This leads firm $i$ to expect that others will expand, lowering expected prices, further increasing the benefit of expansion.
\end{enumerate}

The combinatorial nature of network formation means this multiplier operates through \emph{set inclusion}: the equilibrium supplier set $S_i^*(s_i)$ expands discretely as beliefs become more optimistic. The strategic channel reinforces the direct channel because the lattice of supplier sets is closed under union---if both firm $i$ and its peers expand, the intersection benefits compound

%==============================================================================
\section{Comparative Statics}\label{sec:statics}
%==============================================================================

We examine how the quality of information affects the real economy.

\begin{theorem}[Signal Precision and Volatility]\label{thm:precision}
    Consider the Gaussian environment. An increase in signal precision $\tau_\varepsilon = 1/\sigma_\varepsilon^2$ (holding $\tau_\mu$ fixed) reduces the ex-post volatility of the network structure conditional on fundamentals.
\end{theorem}
\emph{Intuition:} As $\tau_\varepsilon \to \infty$, $\rho \to 1$, which increases the multiplier $\Mcal$. However, the variance of the ``noise'' component of the signal $\varepsilon$ goes to zero faster than the multiplier increases. Firms track fundamentals more closely and react less to pure noise.

\begin{theorem}[Correlation and Instability]\label{thm:correlation}
    Holding the marginal variance of signals constant, an increase in the correlation between error terms $\varepsilon_i$ increases aggregate network volatility.
\end{theorem}
\begin{proof}
    (Sketch) An increase in error correlation increases the strength of higher-order beliefs: when $\rho$ is higher, a firm's signal is more informative about others' signals. By the belief multiplier (Proposition~\ref{prop:multiplier}), this amplifies the strategic channel. A correlated noise shock is magnified because firms expect others to have observed similar signals.
\end{proof}

This result suggests that \textbf{opacity} (low precision) combined with \textbf{correlated information sources} (e.g., reliance on a few public indicators or rumors) creates the most volatile supply chains.

%==============================================================================
\section{Dynamic Extension}\label{sec:dynamic}
%==============================================================================

We briefly sketch a dynamic extension to illustrate persistence. Let $t=1, 2, \dots$. Firms face asymmetric adjustment costs:
\begin{equation}
    \Gamma(S_{it}, S_{i,t-1}) = \gamma^+ |S_{it} \setminus S_{i,t-1}| + \gamma^- |S_{i,t-1} \setminus S_{it}|,
\end{equation}
where $\gamma^+$ is the cost of forming a link and $\gamma^-$ is the cost of severing one.

\begin{proposition}[Hysteresis]\label{prop:hysteresis}
    In the dynamic game, a temporary negative sentiment shock ($s_t$ low) can cause a permanent simplification of the production network. If the economy shifts to the sparse equilibrium, the cost $\gamma^+$ prevents a return to the dense equilibrium even when beliefs recover, unless a coordinated positive shock occurs.
\end{proposition}

%==============================================================================
\section{Conclusion}\label{sec:concl}
%==============================================================================

This paper has integrated dispersed information into the theory of endogenous production networks. We showed that the formation of supply chains is driven by a \textbf{belief multiplier} arising from the interaction of affiliated beliefs and strategic complementarities.

Our results imply that supply chain volatility is not merely a reflection of fundamental TFP shocks but is endogenous to the information structure. Policies that improve transparency---such as standardized reporting of supply chain stress or public data on aggregate input flows---can reduce the correlation of belief errors and dampen the belief multiplier. Conversely, reliance on opaque, correlated signals exacerbates instability.

%==============================================================================
% REFERENCES
%==============================================================================
\bibliographystyle{aer}
\bibliography{references}

%==============================================================================
% APPENDIX
%==============================================================================
\newpage
\appendix
\section{Appendix: Omitted Proofs}

\subsection{Lattice Theory Preliminaries}
We utilize the following definitions and theorems.

\begin{definition}[Complete Lattice]
A partially ordered set $(L, \preceq)$ is a \emph{complete lattice} if every subset $S \subseteq L$ has both a supremum (least upper bound) $\bigvee S$ and an infimum (greatest lower bound) $\bigwedge S$.
\end{definition}

\begin{theorem}[Tarski's Fixed Point Theorem]
Let $L$ be a complete lattice and $f: L \to L$ be an isotone (order-preserving) map. Then the set of fixed points of $f$ is a non-empty complete lattice. In particular, there exist greatest and least fixed points.
\end{theorem}

Our strategy space $\Sigma_i$ is the set of monotone functions from $\R$ to the power set $2^{\I \setminus \{i\}}$. The power set is a complete lattice under inclusion. The function space is a complete lattice under the pointwise order.

\begin{definition}[Supermodularity]
A function $f: L \to \R$ on a lattice is \emph{supermodular} if for all $x, y \in L$:
\[
f(x \vee y) + f(x \wedge y) \geq f(x) + f(y).
\]
\end{definition}

\begin{definition}[Increasing Differences]
A function $f: L \times T \to \R$ has \emph{increasing differences} in $(x, t)$ if for all $x' \succeq x$ and $t' \succeq t$:
\[
f(x', t') - f(x, t') \geq f(x', t) - f(x, t).
\]
\end{definition}

\begin{theorem}[Topkis's Monotonicity Theorem]
If $f: L \times T \to \R$ is supermodular in $x$ and has increasing differences in $(x, t)$, then $\argmax_x f(x, t)$ is isotone in $t$.
\end{theorem}

\subsection{Proof of Lemma \ref{lem:info_sc} (Single-Crossing)}
We must show $\Delta(s_i) = \E[ \Pi(S') - \Pi(S) \mid s_i ]$ is increasing in $s_i$.
Let $g(\mu, \s_{-i}) = K(S, \mu, P(\mu, \sigma_{-i}(\s_{-i}))) - K(S', \mu, P(\mu, \sigma_{-i}(\s_{-i})))$.
We need to show $g$ is increasing in its arguments.

\textbf{Step 1: Monotonicity in $\mu$.}
The unit cost is $K_i \propto 1/\theta_i(\mu)$. Since $\theta_i(\mu)$ is increasing in $\mu$, higher $\mu$ lowers unit costs. The cost difference $K(S) - K(S')$ scales with $1/\theta_i$. Assuming costs are convex in $S$ (diminishing returns), the \emph{benefit} of expansion (negative cost difference) scales positively with productivity.

\textbf{Step 2: Monotonicity in $\s_{-i}$.}
Since $\sigma_{-i}$ is monotone, higher $\s_{-i}$ implies larger $S_{-i}$. By the P-matrix property (Proposition \ref{prop:prices}), larger $S_{-i}$ implies lower $P$. By Assumption \ref{ass:submod} (decreasing differences in $(S, P)$), lower $P$ increases the benefit of expansion.

\textbf{Step 3: Applying Lemma \ref{lem:expectations}.}
Since $g$ is increasing in $(\mu, \s_{-i})$, by Lemma \ref{lem:expectations}, $\Delta(s_i) = \E[g(\mu, \s_{-i}) \mid s_i]$ is increasing in $s_i$. \qed

\subsection{Remark: Linear Approximation in Gaussian Case}
In the Gaussian limit with continuous actions, one can approximate the belief multiplier using a linear best-response. Let $y$ denote network density and consider:
\[
y_i = \alpha \E[\mu|s_i] + \beta \E[y_{-i}|s_i].
\]
In a symmetric equilibrium $y(s) = c s$, substituting Gaussian posteriors yields $c = \frac{\alpha \rho}{1 - \beta \rho}$. The term $(1-\beta \rho)^{-1}$ captures the amplification from iterated expectations. However, this continuous approximation abstracts from the discrete, combinatorial nature of the supplier set $S_i$. The monotone comparative statics in Proposition~\ref{prop:multiplier} provides a more general characterization that does not require differentiability or continuous actions. \qed

\end{document}
