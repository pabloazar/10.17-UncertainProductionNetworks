%2multibyte Version: 5.50.0.2960 CodePage: 65001
%              Scientific Word   Wrap/Unwrap  Version 2.5             %
%              Scientific Word   Wrap/Unwrap  Version 3.0             %
% If you are separating the files in this message by hand, you will   %
% need to identify the file type and place it in the appropriate      %
% directory.  The possible types are: Document, DocAssoc, Other,      %
% Macro, Style, Graphic, PastedPict, and PlotK_i(S_i,a_i,p)Pict. Extract files      %
% tagged as Document, DocAssoc, or Other into your TeX source file    %
% directory.  Macro files go into your TeX macros directory. Style    real
% files are used by Scientific Word and do not need to be extr`acted.  %
% Graphic, PastedPict, and PlotPict files should be placed in a       %
% graphics directory.                                                 %
% Graphic files need to be converted from the text format (this is    %
% done for e-mail compatability) to the original 8-bit binary format. %
% Files included:                                                     %
% "/document/RulingCoalition0910da.tex", Document, 124889, 9/10/2006, 17:25:25, ""%
% "/document/Presentation1/Slide2.JPG", ImportPict, 17649, 7/1/2006, 23:52:18, ""%
% "/document/Presentation1/Slide1.JPG", ImportPict, 20655, 7/1/2006, 23:52:28, ""%
% "/document/RC4/Figure1.jpg", ImportPict, 4162, 7/1/2006, 23:27:50, ""%
% "/document/RC4/Figure3.jpg", ImportPict, 4327, 7/1/2006, 23:26:49, ""%
% "/document/RC4/Figure6.jpg", ImportPict, 4518, 7/1/2006, 23:24:17, ""%
% "/document/ExampleMarkovian.wmf", ImportPict, 38674, 8/19/2006, 7:30:13, ""%
% "/document/ExampleSWDEnotMTHPE.wmf", ImportPict, 5764, 9/4/2006, 12:05:31, ""%
%%%%%%%%%%%%%% Start /document/RulingCoalition0910da.tex %%%%%%%%%%%%%%
%\usepackage{ENPackage}
%\usepackage{color}
%\usetikzlibrary{calc}
%\usepackage{adjustbox}
%\linespread{1.6}
%\topmargin=-1.5cm
%\oddsidemargin=0cm
%\textheight=23.1cm
%\textwidth=16.6cm


\documentclass[11pt]{article}
%%%%%%%%%%%%%%%%%%%%%%%%%%%%%%%%%%%%%%%%%%%%%%%%%%%%%%%%%%%%%%%%%%%%%%%%%%%%%%%%%%%%%%%%%%%%%%%%%%%%%%%%%%%%%%%%%%%%%%%%%%%%%%%%%%%%%%%%%%%%%%%%%%%%%%%%%%%%%%%%%%%%%%%%%%%%%%%%%%%%%%%%%%%%%%%%%%%%%%%%%%%%%%%%%%%%%%%%%%%%%%%%%%%%%%%%%%%%%%%%%%%%%%%%%%%%
\usepackage{amsmath}
\usepackage{amsfonts}
\usepackage{amssymb}
\usepackage{graphicx}
\usepackage{url}
\usepackage{tikz}
\usepackage{booktabs}
\usepackage{caption}
\usepackage{subcaption}
\usepackage{multirow}
\usepackage{wrapfig}
\usepackage[margin=0.8in]{geometry}
\usepackage{float}
\usepackage[export]{adjustbox}
\usepackage{setspace}
\usepackage{wrapfig,lipsum,booktabs}
\setcounter{MaxMatrixCols}{10}
%TCIDATA{OutputFilter=LATEX.DLL}
%TCIDATA{Version=5.50.0.2960}
%TCIDATA{Codepage=65001}
%TCIDATA{<META NAME="SaveForMode" CONTENT="1">}
%TCIDATA{BibliographyScheme=Manual}
%TCIDATA{Created=Saturday, June 03, 2006 17:55:33}
%TCIDATA{LastRevised=Wednesday, May 29, 2019 16:55:36}
%TCIDATA{<META NAME="GraphicsSave" CONTENT="32">}
%TCIDATA{<META NAME="DocumentShell" CONTENT="Standard LaTeX\Blank - Standard LaTeX Article">}
%TCIDATA{Language=American English}
%TCIDATA{CSTFile=LaTeX article (bright).cst}
%TCIDATA{ComputeDefs=
%$\chi \left( s\right) $
%$\theta $
%$b_{R}$
%$\beta $
%$b_{P}$
%$\gamma _{M}$
%$\gamma _{P}$
%$A_{P}$
%$A_{\infty }^{x}$
%}


\usetikzlibrary{calc}
\renewcommand{\hat}{\widehat}
\DeclareMathOperator{\uls}{uniform\, lim\, sup}
\DeclareMathOperator{\uli}{uniform\, lim\, inf}
\newcommand{\fhead}[1]{\begin{large} \bfseries #1  \end{large}}
\newcommand{\thead}[1]{\multicolumn{4}{c}{\begin{large} \bfseries #1  \end{large}}}
\newcommand{\theadprime}[1]{\multicolumn{3}{c}{\begin{large} \bfseries #1  \end{large}}}
\newcommand{\theadprimeprime}[1]{\multicolumn{6}{c}{\begin{large} \bfseries #1  \end{large}}}
\newcommand{\pnote}[1]{\textcolor{red}{(Pablo's note: #1)}}
\renewcommand{\hat}{\widehat}
\renewcommand{\bar}{\overline}
\newtheorem{theorem}{Theorem}
\newtheorem{acknowledgement}{Acknowledgement}
\newtheorem{algorithm}{Algorithm}
\newtheorem{assumption}{Assumption}
\newtheorem{axiom}{Axiom}
\newtheorem{case}{Case}
\newtheorem{claim}{Claim}
\newtheorem{conclusion}{Conclusion}
\newtheorem{condition}{Condition}
\newtheorem{condition*}{Condition}
\newtheorem{conjecture}{Conjecture}
\newtheorem{corollary}{Corollary}
\newtheorem{criterion}{Criterion}
\newtheorem{definition}{Definition}
\newtheorem{lemma}{Lemma}
\newtheorem{problem}{Problem}
\newtheorem{proposition}{Proposition}
\newtheorem{proposition*}{Proposition}
\newtheorem{solution}{Solution}
\newtheorem{summary}{Summary}
\newtheorem{example}{Example}
\newtheorem{exercise}{Exercise}
\newtheorem{notation}{Notation}
\newtheorem{remark}{Remark}
\newenvironment{theorembis}[1]
 {\renewcommand{\thetheorem}{\ref{#1}$'$}   \addtocounter{theorem}{-1}   \begin{theorem}}
  {\end{theorem}}
\newenvironment{assumptionbis}[1]
 {\renewcommand{\theassumption}{\ref{#1}$'$}   \addtocounter{assumption}{-1}   \begin{assumption}}
  {\end{assumption}}
\newenvironment{proof}[1][Proof]{\noindent\textbf{#1.} }{\ \rule{0.5em}{0.5em}}
\setcounter{secnumdepth}{2}
\pagestyle{plain}
\setcounter{figure}{0}
\sloppy
\renewcommand{\pmb}[1]{#1}
\input{tcilatex}
\renewcommand{\tilde}{\widetilde}

\begin{document}

\author{\textbf{Daron Acemoglu} \\
%EndAName
\textbf{MIT} \and \textbf{Pablo D. Azar} \\
%EndAName
\textbf{MIT}}
\title{\textbf{Endogenous Production Networks\thanks{%
We thank Nikhil Agarwal, Isaiah Andrews, Kevin Barefoot, Matthew Calby, Ozan
Candogan, Chad Jones, Chelsea Nelson, Sarah Osborne, Christina Patterson,
Alireza Tahbaz-Salehi, Ricky Stewart, Gianluca Violante, Dan Waldinger, four
anonymous referees, and participants in seminars at the AEA 2018, the
Becker-Friedman Institute, Bilkent University, Boston University and MIT for
suggestions. We gratefully acknowledge financial support from the Bradley
Foundation, the Sloan Foundation, the Stanley and Rhoda Fischer Fellowship
and the Becker-Friedman Institute.}}}
\date{August 2019.}
\maketitle

\begin{center}
\textbf{Abstract}
\end{center}

We develop a tractable model of endogenous production networks. Each one of
a number of products can be produced by combining labor and an endogenous
subset of the other products as inputs. Different combinations of inputs
generate (prespecified) levels of productivity and various distortions may
affect costs and prices. We establish the existence and uniqueness of an
equilibrium and provide comparative static results on how prices and
endogenous technology/input choices (and thus the production network)
respond to changes in parameters. These results show that improvements in
technology (or reductions in distortions) spread throughout the economy via
input-output linkages and reduce all prices, and under reasonable
restrictions on the menu of production technologies, also lead to a denser
production network. Using a dynamic version of the model we establish that
the endogenous evolution of the production network could be a powerful force
towards sustained economic growth. At the root of this result is the fact
that the arrival of a few new products expands the set of technological
possibilities of all existing industries by a large amount --- that is, if
there are $n$ products, the arrival of one more new product increases the
combinations of inputs that each existing product can use from $2^{n-1}$ to $%
2^{n}$, thus enabling significantly more pronounced cost reductions from
choice of input combinations. These cost reductions then spread to other
industries via lower input prices and incentivize them to also adopt
additional inputs.

\textbf{Keywords:} economic growth, economic networks, input-output
linkages, network formation organization of production, production network,
productivity.

\textbf{JEL Classification:} C67, O41, L23, E10, E23.

%TCIMACRO{\TeXButton{EndSmall}{\end{small}}}%
%BeginExpansion
%\end{small}%
%EndExpansion

\bigskip

\thispagestyle{empty}\bigskip \bigskip \bigskip

\newpage

\bigskip

\setcounter{page}{1}\baselineskip0.64cm

\section{Introduction}

Many goods and services are produced with more complex supply chains today
than in the past. For example, agricultural production now uses satellites
to evaluate crop yields, sensors, GPS devices and other electronics for
automatic navigation and specialized computer software and hardware as well
as sensors to test soil quality.\footnote{%
See Mulla and Khosla (2017).} Automotive manufacturing has undergone an even
deeper transformation. The first commercial car designed by Karl Benz in
1885 had a body made of wood and steel. Modern car bodies are instead made
up of aluminum alloys and carbon fibers.\footnote{%
\url{https://www.oxfordsurfaces.com/resource/evolution-of-materials-in-the-car-industry/}%
} At the same time, carburetors have been replaced by electronic fuel
injectors, traditional exhaust systems have been transformed with catalytic
converters, and a range of electronic components, sensors, computer software
and chemicals and hydraulics have been added to improve aerodynamic
efficiency, steering, driving and safety. The range of inputs used for
design, assembly, welding and painting have also expanded, most importantly
with the introduction of various different types of robots and dedicated
machinery. The production of telecommunication equipment has undergone major
changes too. For example, cables are no longer made from iron and steel or
from copper, but from optic fiber; their insulation material now uses pvc
and polyethylene instead of cotton, lead and copper.\footnote{%
\url{https://www.copper.org/applications/telecomm/consumer/evolution.html}.}
The resulting improvements in cable quality in turn impacted other
industries where it is used as input, most notably telephony, television and
internet services. %
%As another example, telecommunication cables were originally made with iron or steel. These materials were strong, but not good conductors. When a technique was developed to produce strong copper wire in 1877, the nascent telephone industry seized on the new development and used mostly copper wire to connect the telephone network. The cables were insulated with cotton, paraffin or rubber, and then wrapped in lead.   Further improvements in materials led to cables being insulated with plastic, the development of coaxial cables (with two conducting cables separated by a teflon or nylon insulator)) and more recently the substitution of copper wires for optic fiber wires.
The change in production supply chains is not limited just to the
introduction of new materials and electronic hardware and software. Many
more industries now use management consulting and other business services in
their production process.

Even though the aforementioned changes take place at the level of
disaggregated inputs, similar trends are visible at a more aggregated level
as well. Figure 1 uses the harmonized input-output tables of the US economy
for 61 summary industries from the Bureau of Economic Analysis (BEA) to show
that the range of inputs used by the typical US industry has expanded over
the last several decades. The average number of suppliers across the 61
industries in 1963 was less than 50. By 1996 this had increased to more than
57. There are similar differences in input-output linkages across countries
and regions.\footnote{%
For example, in OECD data the input-output matrices of more advanced
economies such as the US or Germany are denser than those of developing
economies such as Mexico or Argentina (%
\url{http://www.oecd.org/trade/input-outputtables.htm}). Relatedly, Boehm
and Oberfield (2018) document large differences in firm supply chains across
different states of India.}

What explains the different structures of input usage over time and across
countries? Do these differences contribute to productivity and growth
differences across these economies? In this paper, we take a first step
towards answering these questions by developing a tractable framework with
endogenous input-output linkages. %
\begin{figure}[tbph]
\centering
\includegraphics{../ReplicationFiles/Figures/Figure1.jpeg}

{\textbf{\ Figure
1: Evolution of US input-output linkages. }Average degree (number of
suppliers) from the US summary input-output tables, 1963-1996. We use the
harmonized commodity-by-commodity matrix, which is available for 61 industries during
this period.}
\end{figure}

In our model, each one of $n$ industries decides which subset of the other $%
n-1$ industries (products) to use as input suppliers, and then how much of
each one of these inputs to purchase. Each different input combination leads
to a different constant returns to scale production function. We suppose in
addition that costs in each industry are affected by \textquotedblleft
distortions\textquotedblright , which could result from taxes, contracting
frictions or markups. The market structure in our economy is
\textquotedblleft contestable\textquotedblright\ --- meaning that several
firms have access to the same menu of technologies. This ensures that in
equilibrium, each industry chooses the cost-minimizing quantities of inputs
and sets its price equal to this minimal unit cost, and by the same
reasoning, also chooses the cost-minimizing technology. In equilibrium, the
choice of inputs and technology in each industry is cost-minimizing, and
price is proportional to marginal cost.\footnote{%
Throughout, we use the terms \textquotedblleft technology
choice\textquotedblright , \textquotedblleft set of
inputs\textquotedblright\ and \textquotedblleft input
combination\textquotedblright\ interchangeably. We also use input-output
structure (linkages) and production network interchangeably.}

Our first major results establish the existence and generic uniqueness of an
equilibrium and explore its efficiency properties. The equilibrium has an
intuitive structure, which we exploit to establish several comparative
static results. First, when a product adopts additional inputs, this reduces
not just its price but all prices in the economy (relative to the wage) ---
because this product is now a cheaper input to all other industries. Second,
under a reasonable restriction on the menu of technologies, we establish
that a change in technology that makes the adoption of additional inputs
more productive for one industry --- or an exogenous reduction in
distortions --- reduces all prices in the economy and via this channel
induces an expansion in the set of input suppliers for all industries.
Third, we also show that comparative statics are potentially
\textquotedblleft discontinuous\textquotedblright : a small change for a
single industry can cause a large change in GDP and/or trigger a major shift
in the production structure of many industries.

We then extend this model to a dynamic setup. We assume that a new product
arrives at each date and can be adopted as an input by other sectors (but
generates only limited utility benefits so that economic growth will not
result from love-for-variety). Our main result establishes that when firms
choose the cost-minimizing combination of inputs, the economy achieves
sustained growth in the long run. Intuitively, with $n$ products in the
economy, each industry has a choice between $2^{n-1}$ combinations of
inputs. With one more product added to the mix, the number of feasible input
combinations increases to $2^{n}$ for each one of the $n$ existing products.
The choice of the best technique from this (significantly) expanded set of
options leads to nontrivial cost reductions. Crucially, however, economic
growth is not just driven by the cost reductions enjoyed by the product
making the choice, but also by the induced cost reductions that this
generates for other industries through input-output linkages. We show that
if the distribution of log productivity of different combinations has
sufficiently thick tails (e.g., exponential or Gumbel), this gradually
expanded set of options for production techniques generates exponential
growth.\footnote{%
If log productivity has an exponential distribution, then the level of
productivity has a Pareto distribution, and if log productivity has a Gumbel
distribution, then the level of productivity has a Frechet distribution.}
This growth result is robust to a variety of modifications in the
environment such as allowing only a subset of industries to choose their
inputs, introducing restrictions on the set of allowable input combinations,
varying or endogenizing the rate at which new products are introduced, and
incorporating \textquotedblleft creative destruction\textquotedblright\ (new
products replacing old ones).\footnote{%
Notably, the origins of growth in our economy are different than those
emphasized in the previous literature. First, the nature of growth in our
model is connected to but different from the idea of recombinant growth in
Weitzman (1999), as well as the related ideas in Auerswald, Kauffman, Lobo
and Shell (2000) and Ghiglino (2012). In particular, in contrast to the
recombinant growth notion, in our model ideas are not generated by
combining, or searching within the set of all existing ideas; rather, a
small trickle of new products significantly expands the input combinations
that existing products can use, and this then spreads to the rest of the
economy by reducing costs for others. Second, as already hinted at, growth
is not driven by the combination of expanded products and love-for-variety
(as would be the case in Romer, 1990, or Grossman and Helpman, 1992). Third,
it is not a consequence of proportionately more products or innovations
arriving over time (as is the case in Romer, 1990, Jones, 1995, Eaton and
Kortum, 2001, or Klette and Kortum, 2004). Fourth, it is not driven by
proportional improvements in the productivity of all industries as in
quality-ladder models (as in Aghion and Howitt, 1992, or Grossman and
Helpman, 1991). Finally, it is also not due to thick-tailed productivity
draws continuously improving technology and spreading in the economy via a
diffusion process (as in Akcigit, Celik and Greenwood, 2016, Lucas, 2009,
Lucas and Moll, 2014, Perla and Tonetti, 2014). Crucially, even though the
distribution of productivity across different input mixes is thick-tailed in
our economy, this by itself does not lead to sustained growth. It is the
conjunction of the significant increase in the number of options of input
combinations and the endogenous change in the attractiveness of input
combinations following changes in prices that underpin growth.}

We further use the tractable special case of our model with Cobb-Douglas
production technologies and Gumbel-distributed log productivity terms to
derive an explicit logistic equation linking the likelihood of an industry
being used as a supplier to another industry to the vector of prices. Using
this equation we show that while the distribution of \textquotedblleft
indegrees\textquotedblright\ (the number of suppliers per product) has only
limited inequality or asymmetry across sectors, the distribution of
\textquotedblleft outdegrees\textquotedblright\ (the number of customers of
each industry) is much more unequal. This prediction is consistent with the
structure of US input-output tables. Moreover, under an additional
assumption on the distribution of sectoral shares, we show that the
distribution of outdegrees in our model is Pareto --- a pattern that also
matches the stylized facts documented in Acemoglu et al. (2012).

Finally, we make a first attempt to investigate the contribution of changing
input combinations to productivity growth. We use data from the US economy
from 1987-2007 in order to estimate the contribution of changes in supplier
sets to industry productivity. Our estimates suggest that new input
combinations may account for between 40\% and 64\% of average industry TFP
growth. Naturally, this exercise should be interpreted with caution, since
it relies on the simplified structure of our model, and the observed
relationship between new input combinations and industry productivity growth
could be driven by other omitted factors. Nevertheless, this illustrative
exercise highlights that the contribution of new input combinations to
productivity growth could be quite important and should be studied more
systematically in the future.

Our paper relates to a growing literature on the role of input-output
linkages in macroeconomics, including Long and Plosser (1983), Ciccone
(2002), Gabaix (2011), Jones (2011), Acemoglu, Carvalho, Ozdaglar and
Tahbaz-Salehi (2012), Acemoglu, Ozdaglar and Tahbaz-Salehi (2017), Bartleme
and Gorodnichenko (2015), Biglio and La'o (2016), Baqaee (2017), Fadinger,
Ghiglino and Teteryatnikova (2016), Liu (2017), Baqaee and Farhi (2017), and
Caliendo, Parro and Tsyvinski (2017). Some of these papers, including the
last three, allow for non-Cobb-Douglas technologies and thus endogenize the
intensity with which different inputs are used (and thus the different
entries of the input-output matrix). However, they do not investigate which
combinations of inputs will be used --- that is, the extensive margin of the
input-output matrix --- which is our focus in this paper. Our analysis
demonstrates that both the mathematical structure of this problem and its
economic implications are very different from intensive margin decisions. In
particular, our results on comparative statics of the equilibrium production
network, on endogenous growth from input combinations and on cross-sectional
implications have no counterpart in this literature.\footnote{%
More recently, Biglio and La'o (2016), Liu (2017), Caliendo, Parro and
Tsyvinski (2017) and Baqaee and Farhi (2018) analyze the implications of
distortions in input-output economies, but do not focus on endogenous
input-output linkages. Nor do they provide the general characterization,
existence, uniqueness and comparative static results we present.}

A few recent papers on endogenous input-output linkages are more closely
related to our work. Carvalho and Voigtlander (2015) construct a model in
which producers search for new inputs and confront some of the implications
of this model with the US input-output tables. Atalay, Hortacsu, Roberts and
Syverson (2011) study the choice of suppliers at the firm level. More
closely related to our paper are the important prior work by Oberfield
(2017), independent contemporaneous work by Taschereau-Dumouchel (2017) and
more recent work by Boehm and Oberfield (2018). Oberfield constructs an
elegant model of the endogenous evolution of the input-output architecture,
but with two notable differences from our work. First, at a technical level,
Oberfield considers a matching model, which does not lead to the type of
general equilibrium characterization and comparative static results we
present below. Second and more importantly, for tractability reasons
Oberfield restricts attention to a situation in which each good can only use
a single supplier and as a result cannot study the questions that make up
our main focus --- how the technology choice of an industry affects the
structure of input-output linkages for the entire economy, equilibrium
complementarities and sustained long-run growth. Taschereau-Dumouchel (2017)
studies the formation of a production network in the context of business
cycle dynamics. Focusing on the social planner's problem, he investigates
whether the formation and the response to shocks of equilibrium networks
exacerbate economic volatility. Subsequent work by Boehm and Oberfield
(2018) constructs a firm-level model of input choices and estimates it on
microdata. Finally, Gualdi and Mandel (2017) consider an agent-based model
where firms combine new inputs following some simple rules on input adoption
and imitation, and show via simulations that their setup can generate
long-run growth. Though the mechanism that leads to economic growth in this
paper is related to ours, their framework neither contains the
characterization and comparative static results we present nor clarifies the
source of sustained growth and its limitations.

Our paper is also related to the literature on sourcing decisions in
international trade, for example, Chaney (2014), Antras and Chor (2013),
Eaton, Kortum and Kramarz (2014), Antras et al. (2017), Lim (2017) and
Tintelnot et al. (2017). None of these papers study the endogenous
determination of the production network or the implications we focus on,
such as the comparative statics of the production network, endogenous growth
and cross-sectional regularities.

The rest of the paper is organized as follows. The next section introduces
our basic model. Equilibrium existence, uniqueness and efficiency are
studied in Section \ref{section equilibrium}. Section \ref{section
comparative statics} presents our main comparative static results. Section %
\ref{section dynamic} presents our growth model and shows how sustained
economic growth can emerge in this setup. Section \ref{section
cross-sectional implications} derives the cross-sectional implications of
our model. Section \ref{section counterfactual} presents a preliminary
investigation of the contribution of changing input-output combinations to
US productivity growth. Section \ref{section conclusion} concludes, while
the Appendices contain the proofs of the results stated in the text as well
as some additional theoretical, empirical and quantitative results.

\paragraph{Notation}

For any pair of $m$-dimensional vectors $\alpha ,\beta \in \mathbb{R}^{m}$,
we write $\alpha \geq \beta $ if and only if $\alpha _{i}\geq \beta _{i}$
for every $i\in \{1,...,m\}$, and $\alpha >\beta $ if $\alpha \geq \beta $
and there exists at least one $i$ such that $\alpha _{i}>\beta _{i}$. For
any two functions $f,g:D\rightarrow \mathbb{R}^{m}$, we write $f\geq g$ if $%
f(x)\geq g(x)$ for all $x\in D$. If $\alpha \in \mathbb{R}^{n\times m}$ is a
matrix, we denote the row vector $\{\alpha _{ij}\}_{j=1}^{m}$ by $\alpha
_{i} $. Unless specified otherwise, we will use lowercase variables to
denote logarithms of the corresponding uppercase variables. For example if $%
P=(P_{1},...,P_{n})\in \mathbb{R}_{>0}^{n}$ is a vector of prices, then $%
p=(p_{1},...,p_{n})=(\log P_{1},...,\log P_{n})$ will denote the vector of
log prices.

\section{Model\label{section model}}

In this section we introduce our static model, which features endogenous
input choice subject to distortions. We analyze this model in the next two
sections and then generalize it to a dynamic setting in Section \ref{section
dynamic}.

\subsection{Production Technology, Market Structure and Preferences}

There is a set $\mathcal{N}=\{1,...,n\}$ of industries, each producing a
single good, denoted by $Y_{i}$ for industry $i$. Throughout, we assume that
each industry is \emph{contestable} in the sense that\ a large number of
firms have access to the same production technology and can enter any sector
without any entry barriers. This will ensure that equilibrium profits are
always equal to zero. When this will cause no confusion, we work with a
representative firm for each industry, and use industry $i$, product $i$ and
firm $i$ (for $i\in \mathcal{N}$) interchangeably.

The production technology for industry $i$ is%
\begin{equation*}
Y_{i}=F_{i}(S_{i},A_{i}(S_{i}),L_{i},X_{i}).
\end{equation*}%
Here $L_{i}$ is the amount of labor used,$\ S_{i}\subset \{1,...,n\}$
denotes the set of (endogenous) suppliers, $X_{i}=\{X_{ij}\}_{j\in S_{i}}$
is the vector of intermediate goods, and $A_{i}(S_{i})$ designates the
productivity of the technology generated by the use of inputs in the set $%
S_{i}$ (and for now we do not need to specify its dimension).\footnote{%
Clearly, $L_{i},X_{i}$ and $Y_{i}$ as well as consumption $C_{i}$ are
nonnegative for all $i$, but we leave this restriction implicit throughout
to simplify the notation.} We assume throughout that $F_{i}$ does not depend
on $X_{ij}$ for $j\notin S_{i}$. The dependence of the technology of
production on the set of inputs is the crucial feature of our model and
captures the possibility that by combining a richer set of inputs an
industry may achieve greater productivity. Motivated by this aspect of input
choice, we interchangeably refer to the choice of $S_{i}$ as \emph{%
technology choice }or\emph{\ choice of input suppliers.}\footnote{\label%
{footnote hierarchical}In this section, we assume for notational convenience
that any combination of inputs is admissible. We discuss this issue further
in Section \ref{section dynamic}, where we generalize the setup so that some
input classes are \textquotedblleft essential\textquotedblright\ for certain
sectors (e.g., precision tools need to use at least some metals). We also
assume that each industry can use its own output as an input, which is only
relevant for our quantitative exercise below (since the diagonal elements of
the US input-output matrix are nonzero).} We impose the following assumption
on this production technology:

\begin{assumption}
\label{assumption 1}For each $i=1,2,\ldots ,n$, $%
F_{i}(S_{i},A_{i}(S_{i}),L_{i},X_{i})$ is strictly quasi-concave, exhibits
constant returns to scale in $(L_{i},X_{i})$, and is increasing and
continuous in $A_{i}(S_{i})$, $L_{i}$ and $X_{i}$, and strictly increasing
in $A_{i}(S_{i})$ when $L_{i}>0$ and $X_{i}>0$. Moreover, labor is an
essential factor of production in the sense that $F_{i}(0,\cdot ,\cdot
,\cdot )=0$.
\end{assumption}

Constant returns to scale on the production side is natural. The strict
quasi-concavity of the production function ensures that input demands given
technology are uniquely determined, while the feature that output is
increasing in the productivity parameters enables us to identify
\textquotedblleft better technology\textquotedblright\ with greater $%
A_{i}(S_{i})$. Finally, the assumption that labor is essential rules out the
extreme possibility that labor can be made redundant by some combination of
existing inputs and ensure that the output level of each industry will
always be finite.

We model the consumer side via a representative household whose preferences
are given by
\begin{equation}
u(C_{1},...,C_{n}),  \label{consumer utility}
\end{equation}%
and impose the following minimal conditions:

\begin{assumption}
\label{assumption 1 second part}$u(C_{1},\ldots ,C_{n})$ is continuous,
differentiable, increasing and strictly quasi-concave, and all goods are
normal.\footnote{%
The assumptions that $u$ is differentiable and all goods are normal are used
only in the proof of Lemma \ref{lem:fixedPoint} and can be relaxed, though
at the expense of significant additional complications.}
\end{assumption}

The representative household has one unit of labor endowment, which it
supplies inelastically, and receives the profits, if any, from all
industries. Throughout, we choose the wage as the numeraire,%
\begin{equation*}
W=1.
\end{equation*}

We also introduce distortions, which could result from taxes, regulations,
contracting frictions, credit market imperfections or markups. Specifically,
we assume that industry $i$ is subject to a distortion equal to $\mu
_{i}\geq 0$, modeled as an effective ad valorem tax at the rate $\mu _{i}$.
This implies that, due to the distortions, the marginal cost of industry $i$%
\ is multiplied by $1+\mu _{i}$. Clearly, when $\mu _{i}=0$ for all $i$, we
have a fully competitive/contestable economy. Depending on their source,
distortions may be pure waste or generate revenues for either firms or the
government to be rebated back to the representative household. We assume
that a fraction $\lambda _{i}$ of the revenues generated by distortions from
industry $i$ are distributed back to the representative household and the
rest are waste. So $\lambda _{i}=0$ for all $i$ corresponds to the case
where distortions are pure waste, while $\lambda _{i}=1$ captures the case
in which all distortions generate tax revenues or profits for firms.%
\footnote{%
This modeling of distortions is similar to that in Biglio and La'o (2016),
Liu (2017), Caliendo, Parro and Tsyvinski (2017) and Baqaee and Farhi
(2018), though these papers differ in whether they assume distortions are
pure waste or generate revenues. Our formulation nests these various
possibilities.
\par
A straightforward generalization of our formulation is to assume that
distortions are customer or \textquotedblleft edge\textquotedblright\
specific as well (i.e., equal to $\mu _{ij}$ for inputs supplied to industry
$j$). This might be because of customer-specific\ taxes or because of
contracting frictions that apply between some customer-supplier pairs (see
Boehm and Oberfield, 2018, for a model of such frictions). For notational
simplicity, we focus on the case where there is a single industry-specific
distortion, $\mu _{i}$, even though all of our results readily generalize to
an environment in which distortions depend on the destination industry,
i.e., take the form $\mu _{ij}$ (and in this case we could also set $\mu
_{ii}=0$ so that an industry's purchase of its own output is not subject to
distortions or markups).} Denoting the price of good $i$ (inclusive of
distortions and markups if any) by $P_{i}$, the budget constraint of the
representative household can then be written as%
\begin{equation}
\sum_{i=1}^{n}P_{i}C_{i}\leq 1+\sum_{i=1}^{n}\Lambda _{i},
\label{budget constraint}
\end{equation}%
where $\Lambda _{i}=\lambda _{i}\frac{\mu _{i}}{1+\mu _{i}}P_{i}Y_{i}$
denotes the revenue from distortions in industry $i$ rebated back to the
representative household.

\subsection{Cost Minimization}

The contestable market structure implies that price in each industry will be
equal to effective marginal cost (inclusive of the distortions). We first
derive this marginal cost by considering the cost minimization problem in
each industry. Let us break this cost minimization problem down into two
parts: first, choose $X_{i}$ and $L_{i}$ taking $S_{i}$ as given, and then
choose the set of suppliers or \textquotedblleft
technology\textquotedblright , $S_{i}$. The first step determines the \emph{%
unit cost function} $K_{i}(S_{i},A_{i}(S_{i}),P)$ as%
\begin{gather}
K_{i}(S_{i},A_{i}(S_{i}),P)=\min_{X_{i},L_{i}}L_{i}+\sum_{j\in
S_{i}}P_{j}X_{ij}  \label{firm optimization} \\
\text{ subject to }F_{i}(S_{i},A_{i}(S_{i}),L_{i},X_{i})=1.  \notag
\end{gather}%
The unit cost function is conditioned on the set of inputs, $S_{i}$, because
this determines which prices matter for costs, and also captures the
dependence of the technology of production and thus the cost function on the
set of inputs beyond the productivity shifter $A_{i}(S_{i})$. In addition,
because $F_{i}$ is strictly increasing and continuous in $A_{i}$, the unit
cost function $K_{i}(S_{i},A_{i},P)$ is strictly decreasing and continuous
in $A_{i}$.

The second step of cost minimization is the choice of technology/input
suppliers to minimize this unit cost function for each $i=1,2,\ldots ,n$,
that is,%
\begin{equation}
S_{i}^{\ast }\in \arg \min_{S_{i}}K_{i}(S_{i},A_{i}(S_{i}),P)\text{.}
\label{technology choice}
\end{equation}

Given this cost function and distortion $\mu _{i}$, the equilibrium price of
industry $i$ is given by $P_{i}^{\ast }=(1+\mu _{i})K_{i}(S_{i}^{\ast
},A_{i}(S_{i}^{\ast }),P)$, and then the amount rebated to the
representative household is%
\begin{equation}
\Lambda _{i}^{\ast }=\lambda _{i}\frac{\mu _{i}}{1+\mu _{i}}P_{i}^{\ast
}Y_{i}^{\ast },  \label{profits}
\end{equation}%
where $Y_{i}^{\ast }$ denotes its output.

\subsection{Equilibrium}

\begin{definition}
\label{definition of equilibrium}\textbf{(Definition of equilibrium) }An
equilibrium is a tuple $(P^{\ast },S^{\ast },C^{\ast },L^{\ast },X^{\ast
},Y^{\ast })$ such that

\begin{enumerate}
\item \textbf{(Contestability) }For each $i=1,2,\ldots ,n$,%
\begin{equation}
P_{i}^{\ast }=(1+\mu _{i})K_{i}(S_{i}^{\ast },A_{i}(S_{i}^{\ast }),P^{\ast
}).  \label{price equals marginal cost}
\end{equation}

\item \textbf{(Consumer maximization) }The consumption vector $C^{\ast }$
maximizes (\ref{consumer utility}) subject to (\ref{budget constraint})
given prices $P^{\ast }$, where $\Lambda _{i}^{\ast }$ is determined by
equation (\ref{profits}).

\item \textbf{(Cost minimization) }For each $i=1,2,\ldots ,n$, factor
demands $L_{i}^{\ast }$ and $X_{i}^{\ast }$ are a solution to (\ref{firm
optimization}), and the technology choice $S_{i}^{\ast }$ is a solution to (%
\ref{technology choice}) given the price vector $P^{\ast }$.

\item \textbf{(Market clearing) }For each $i=1,2,\ldots ,n$,
\begin{equation}
C_{i}^{\ast }+\sum_{j=1}^{n}X_{ji}^{\ast }=(1-(1-\lambda _{i})\frac{\mu _{i}%
}{1+\mu _{i}})Y_{i}^{\ast }\text{,  }Y_{i}^{\ast }=F_{i}(S_{i}^{\ast
},A_{i}^{\ast }(S_{i}^{\ast }),L_{i}^{\ast },X_{i}^{\ast })\text{  and  }%
\sum_{j=1}^{n}L_{j}^{\ast }=1.  \label{market clearing}
\end{equation}
\end{enumerate}
\end{definition}

The first condition imposes the major implication of contestability ---
price is equal to marginal cost (inclusive of distortions) --- while the
last three conditions are standard.

Several observations are useful. First, as already noted above, when $\mu
_{i}=0$ for all industries, our economy is fully competitive. Second, ours
can be viewed as a generalization of Samuelson's (1954) \textquotedblleft
no-substitution economy\textquotedblright\ (to an environment with
endogenous production networks and distortions) where prices are determined
entirely on the production side, without reference to consumer preferences,
as condition (\ref{price equals marginal cost}) in Definition \ref%
{definition of equilibrium} makes it clear.\footnote{\label{footnote
Samuelson}This is straightforward to see when $\mu _{i}=0$ for all
industries, but is true more generally as we show next.
\par
Note also that Samuelson's notion of equilibrium is similar to ours, but
imposes an additional condition requiring that the level of consumption of
the first good is maximized given the level of consumption of the remaining
goods in the economy. Our analysis in the next section shows that
Samuelson's additional condition is redundant because the equilibrium price
vector is always unique.} This property is a consequence of the contestable
market structure we have assumed. Third, the market clearing condition for
industry $i$ incorporates the fact that only a $\lambda _{i}$ fraction of
revenues $\frac{\mu _{i}}{1+\mu _{i}}P_{i}Y_{i}$ from distortions that are
rebated back to the representative household, and therefore $(1-\lambda _{i})%
\frac{\mu _{i}}{1+\mu _{i}}$ units of output are lost due to frictions.
Fourth, the labor market clearing condition could have been dropped by
Walras's law, but we wrote it as part of market clearing for emphasis.
Finally, the vector of equilibrium technology choices $S^{\ast }$ describes
a \emph{network} --- the equilibrium production network --- since it
specifies the set of suppliers (technologies used) for each industry.

\subsection{Cobb-Douglas Production Functions with Hicks-Neutral Technology}

The simplest example of production technologies that satisfy part 1 of
Assumption \ref{assumption 1} is the family of Cobb-Douglas production
functions with Hicks-neutral technology, given by%
\begin{equation*}
F_{i}(S_{i},A_{i}(S_{i}),L_{i},X_{i})=\frac{1}{(1-\sum_{j\in S_{i}}\alpha
_{ij})^{1-\sum_{j\in S_{i}}\alpha _{ij}}\prod_{j\in S_{i}}\alpha
_{ij}^{\alpha _{ij}}}A_{i}(S_{i})L_{i}^{1-\sum_{j\in S_{i}}\alpha
_{ij}}\prod_{j\in S_{i}}X_{ij}^{\alpha _{ij}}.
\end{equation*}%
For each $i=1,2,\ldots ,n$, $A_{i}(S_{i})$ is a scalar representing
Hicks-neutral productivity, and $S_{i}$ indexes the dependence of the
technology on both $A_{i}(S_{i})$ and the $\alpha _{ij}$'s.\footnote{%
The \textquotedblleft entropy\textquotedblright -like denominator is
included in this production function as a normalization, in particular to
simplify the unit cost function derived next. Whether this normalization is
present or not makes no difference in our static model. It is not important
in the dynamic model either, since it grows at a linear (subexponential)
rate as $n\rightarrow \infty $ and thus does not affect the asymptotic
growth rate of the economy.} We show in Lemma \ref{lemma CD} in Appendix B
that the corresponding unit cost function for industry $i$\ is
\begin{equation}
K_{i}(S_{i},A_{i}(S_{i}),P_{i})=\frac{1}{A_{i}(S_{i})}\prod_{j\in
S_{i}}P_{j}^{\alpha _{ij}}.  \label{unit cost for CD}
\end{equation}%
This cost function illustrates the tradeoff that a firm faces when it
chooses the set $S_{i}$ to minimize costs. There might be sets where $%
\prod_{j\in S_{i}}P_{j}^{\alpha _{ij}}$ is low, but $A_{i}(S_{i})$ is high,
and sets where $\prod_{j\in S_{i}}P_{j}^{\alpha _{ij}}$ is high, but $%
A_{i}(S_{i})$ is low. The firm will choose a set of suppliers by balancing
this tradeoff between high productivity and low prices (or vice versa).

Cobb-Douglas production functions enable us to obtain a closed-form solution
for equilibrium prices. Let us denote logs with lower case; that is, $%
p_{i}=\log P_{i}$ and $a_{i}=\log A_{i}.$ We can then write the log unit
cost as a function of log productivities and log prices,%
\begin{equation*}
k_{i}(S_{i},a_{i}(S_{i}),p)=-a_{i}(S_{i})+\sum_{j\in S_{i}}\alpha _{ij}p_{j}.
\end{equation*}%
Using (\ref{price equals marginal cost}), equilibrium log prices are%
\begin{equation}
p_{i}^{\ast }=\log (1+\mu _{i})+\sum\limits_{j\in S_{i}}(\alpha
_{ij}p_{j}^{\ast })-a_{i}.  \label{eq:logcost}
\end{equation}

Equation (\ref{eq:logcost}) admits a closed-form solution for prices. Let $%
\alpha (S)\in \mathbb{R}^{n\times n}$ be a matrix with
\begin{equation*}
\alpha _{ij}(S)=%
\begin{cases}
\alpha _{ij}\text{ if }j\in S_{i} \\
0\text{ otherwise}%
\end{cases}%
\end{equation*}%
Then given equilibrium technology choices represented by $S^{\ast }$, log
prices satisfy%
\begin{eqnarray}
p^{\ast } &=&-(I-\alpha (S^{\ast }))^{-1}(a(S^{\ast })-\log (1+\mu ))  \notag
\\
&=&-\mathcal{L}(S^{\ast })(a(S^{\ast })-m),  \label{Leontief equation}
\end{eqnarray}%
where $a(S^{\ast })=(a_{1}(S^{\ast }),\ldots ,a_{n}(S^{\ast }))^{\prime }$
is the column vector of equilibrium log productivities, $m=(\log (1+\mu
_{1}),\ldots ,\log (1+\mu _{t}))^{\prime }$ is the vector of log distortions
and the second line defines the \emph{Leontief inverse},
\begin{equation*}
\mathcal{L}(S^{\ast })=(I-\alpha (S^{\ast }))^{-1},
\end{equation*}%
which will play an important role whenever we work with Cobb-Douglas
production functions. Equation (\ref{Leontief equation}) verifies, in the
special case of Cobb-Douglas technology, our previous claim that prices are
determined without reference to preferences.

%\subsection{Example: CES families}
%
%Let $\sigma > 0, \sigma \neq 1$, and consider the production function
%\begin{gather}
%F_{i}(L_{i},X_i,S_{i},A_i)= A_i ((1-\sum_{j \in S_i} \alpha_{ij})^{\frac{1}{\sigma}}L_{i}^{\frac{\sigma-1}{\sigma}}+  \sum_{j \in S_i} \alpha_{ij}^{\frac{1}{\sigma}} X_{ij}^{\frac{\sigma-1}{\sigma}})^{\frac{\sigma}{\sigma-1}}.
%\end{gather}
%
%The corresponding cost function is
%\begin{gather}
%K_i(S_i,A_i,P) = \frac{1}{A_i}((1-\sum_{j \in S_i} \alpha_{ij}) + \sum_{j \in S_i} \alpha_{ij} P_j^{1-\sigma})^{\frac{1}{1-\sigma}}.
%\end{gather}
%It is convenient to raise both sides to the $1-\sigma$ power to obtain the following equation which is linear in $P_1^{1-\sigma},...,P_n^{1-\sigma}$
%\begin{gather}
%\label{transformed cost}
%K_i^{1-\sigma}(S_i,A_i,P) = \frac{1}{A_i^{1-\sigma}}  ((1-\sum_{j \in S_i} \alpha_{ij}) + \sum_{j \in S_i} \alpha_{ij} P_j^{1-\sigma}).
%\end{gather}
%Let   $D$ be a diagonal matrix such that $D_{ii} = \frac{1}{A_i^{1-\sigma}}$, $b =(1-\sum_{j \in S_1} \alpha_{1j}),...,(1-\sum_{j \in S_n} \alpha_{nj}))'$  and    $P^{1-\sigma} = (P_1^{1-\sigma},...,P_n^{1-\sigma})'$. We can solve for prices using the linear system of equations
%\begin{gather}
%(I - D \alpha(S)) P^{1-\sigma} = D b.
%\end{gather}

\section{Equilibrium Characterization\label{section equilibrium}}

In this section, we first establish the existence of an equilibrium in our
static economy and then prove that this equilibrium is generically unique,
and finally study its efficiency properties. Existence and uniqueness of
equilibrium are challenging because each industry has a high-dimensional
\textquotedblleft nonconvex\textquotedblright\ technology choice.
Nevertheless, we can establish both properties using lattice theoretic ideas
and exploiting the fact that the equilibrium will feature a form of
monotonicity\ whereby equilibrium prices of all industries always decline
with the adoption of additional (cost-minimizing) technologies by any
industry.

\subsection{Existence of Equilibrium}

We start with a lemma that will be useful in proving both existence and
uniqueness of equilibrium. The proof of this lemma, like all other proofs
(unless otherwise indicated), is presented in Appendix A.

\begin{lemma}
\label{lem:fixedPoint}Suppose Assumptions \ref{assumption 1} and \ref%
{assumption 1 second part} hold. Then given an exogenous network $S_{i}$, $%
P^{\ast }>0$ is an equilibrium price vector if and only if (\ref{price
equals marginal cost}) holds for each $i=1,2,\ldots ,n$.
\end{lemma}

The \textquotedblleft only if\textquotedblright\ part of this lemma is a
direct implication of the definition of an equilibrium, while the
\textquotedblleft if\textquotedblright\ part is more substantive and shows
that with exogenous networks, any vector of prices equal to unit costs is
part of an equilibrium. An important implication of this lemma, which is
established as part of its proof, is that given an equilibrium vector of
prices, $P^{\ast }$, there is a unique vector of sectoral outputs, $Y^{\ast
} $, consumption levels, $C^{\ast }$, intermediate input levels, $X^{\ast }$%
, and labor demands, $L^{\ast }$. Using this result, we establish the
existence of an equilibrium.

\begin{theorem}
\label{existence theorem}\textbf{(Existence)} Suppose Assumptions \ref%
{assumption 1} and \ref{assumption 1 second part} hold. Then an equilibrium $%
(P^{\ast },S^{\ast },C^{\ast },L^{\ast },X^{\ast },Y^{\ast })$ exists.
\end{theorem}

\subsection{Uniqueness of Equilibrium}

In this subsection, we establish the uniqueness of equilibrium prices and
generic uniqueness of equilibrium technology choices. To establish
genericity, we need to consider variations in exogenous parameters. Given
the uniqueness of equilibrium prices, it is sufficient to focus on a subset
of the exogenous parameters corresponding to the shifters of the production
technology, $\{A_{i}(S_{i})$ $\}_{i=1}^{n}$. Let us take each $A_{i}(S_{i})$
to be represented by an $\ell $-dimensional vector, so that $%
A_{i}=(A_{i}(\varnothing ),A_{i}(\{1\}),...,A_{i}(\{1,...,n\}))_{i=1}^{n}$
is also a vector in $\mathbb{R}^{\ell \times 2^{n-1}}$, and $%
A=(A_{1},...,A_{n})$ is a vector in $\mathbb{R}^{n\times \ell \times
2^{n-1}} $. We define generic uniqueness in terms of the Lebesgue measure on
the parameters $A\in \mathbb{R}^{n\times \ell \times 2^{n-1}}$.

\begin{definition}
\textbf{(Genericity) }The equilibrium network is generically unique if the
set
\begin{equation*}
\mathcal{A}=\{A:\text{ There exist at least two distinct equilibrium
networks }S^{\ast },S^{\ast \ast }\}
\end{equation*}%
has Lebesgue measure zero in $\mathbb{R}^{n\times \ell \times 2^{n-1}}$.
\end{definition}

\begin{theorem}
\label{thm:unique}\textbf{(Uniqueness) }Suppose Assumptions \ref{assumption
1} and \ref{assumption 1 second part} hold. Then the equilibrium price
vector $P^{\ast }$ is uniquely determined, and the equilibrium network $%
S^{\ast }$ and quantities $C^{\ast }$, $L^{\ast }$, $X^{\ast }$ and $Y^{\ast
}$ are generically unique.
\end{theorem}

We demonstrate as part of the proof of Theorem \ref{existence theorem} that
the set of equilibrium prices forms a lattice, which implies that for any
two distinct vectors of equilibrium prices, there exists a minimal vector of
equilibrium prices. We then show that this is not possible, establishing
uniqueness of equilibrium prices and thus quantities. Non-uniqueness of the
equilibrium network can only arise if two choices of input combinations
yield exactly the same unit cost for an industry, which is a non-generic
possibility, proving the generic uniqueness of the equilibrium network and
equilibrium quantities.

\subsection{Efficiency}

The next theorem characterizes the efficiency properties of our equilibrium.
To simplify this result, we impose differentiability for production
technologies as well (differentiability of utility was imposed in Assumption %
\ref{assumption 1 second part}). We say that $(\varnothing ,...,\varnothing
) $ is a Pareto efficient production network if the Pareto efficient
allocation involves no input-output linkages between industries.\footnote{%
Clearly, since the economy is inhabited by a representative household,
Pareto efficiency is equivalent to the maximization of the representative
household's utility.}

\begin{theorem}
\label{theorem efficiency}\textbf{(Efficiency) }Suppose Assumptions \ref%
{assumption 1} and \ref{assumption 1 second part} hold. Suppose also that
the production function $F_{i}$ is differentiable for each $i=1,2,\ldots ,n$.

\begin{enumerate}
\item If $\mu _{i}=0$ for all $i=1,2,\ldots ,n$ so that all distortions are
equal to zero, then the equilibrium is Pareto efficient.

\item If $\mu _{i}=\mu _{0}>0$ and $\lambda_{i} = 1$ for all $i=1,2,\ldots
,n $ and $(\varnothing ,...,\varnothing )$ is the unique Pareto efficient
production network, then the equilibrium is Pareto efficient.

\item If $\mu _{i}=\mu _{0}>0$ and $\lambda_{i} = 1$ for all $i=1,2,\ldots
,n $ and $(\varnothing ,...,\varnothing )$ is not a Pareto efficient
production network, then the equilibrium is not Pareto efficient.

\item If there exist $i$ and $i^{\prime }$ such that $\mu _{i}>0$ and $\mu
_{i}\neq \mu _{i^{\prime }}$ or there exists $i$ such that $(1-\lambda
_{i})\mu _{i}>0$, then the equilibrium is not Pareto efficient.
\end{enumerate}
\end{theorem}

The first part of this theorem proves that when the equilibrium is
competitive (with zero distortions), it is also Pareto efficient.\footnote{%
Note that even in this case we could not establish the existence of
equilibrium by appealing to the Second Welfare Theorem because the choice
over sets of inputs makes ours a nonconvex economy.} The second part shows
that if the (Pareto) efficient production network involves no input-output
linkages and all distortions are equal and rebated fully to the
representative household, then the equilibrium is again efficient. This is
because relative consumption levels are not distorted, and given the
inelastic supply of labor, the economy with equal distortions replicates the
allocation with zero distortions. The third and fourth parts show that
excepting these cases, the equilibrium is inefficient. In the third part, we
again focus on the case with constant distortions, but now the efficient
production network involves linkages. In this case the inefficiency is a
consequence of the impact of distortions on the choice between labor and
non-labor inputs. Finally, in the fourth part the inefficiency applies
regardless of whether or not the efficient production network is empty
because distortions are unequal across sectors and distort relative
consumption choices, or because distortions generate waste.

It is also worth noting that distortions generate additional inefficiencies
via their impact on the equilibrium production network. The next example
illustrates this point, focusing on an economy with constant distortions
across sectors.

\begin{example}
\textbf{(Distortions and inefficient technology choice) }Consider an economy
with two industries. The representative household's utility function is $%
U(C_{1},C_{2})=\log (C_{1})+\log (C_{2})$. Each industry has a Cobb-Douglas
production function with parameters $\alpha _{12}=\alpha _{21}=\frac{1}{2}$.
Moreover, to simplify the example we assume that industries cannot use their
own output as input. Distortions are constant and set to $\mu _{1}=\mu
_{2}=\mu _{0}>0$, and we also assume $\lambda _{1}=\lambda _{2}=1$. Suppose $%
a_{1}(\varnothing )=a_{2}(\varnothing )=0$, $a_{1}(\{2\})=a_{2}(\{1\})=1$.
Then log prices are given by
\begin{equation*}
p_{1}=\log (1+\mu _{0})-a_{1}(S_{1})+I_{2\in S_{1}}\frac{1}{2}p_{2}
\end{equation*}%
\begin{equation*}
p_{2}=\log (1+\mu _{0})-a_{2}(S_{2})+I_{1\in S_{2}}\frac{1}{2}p_{1}
\end{equation*}%
where $I_{j\in S_{i}}$ is an indicator function for $j$ belonging to $S_{i}$.

If $\log (1+\mu _{0})<2$, then the unique equilibrium involves $S_{1}=\{2\}$
and $S_{2}=\{1\}$. In this case $p_{1}=p_{2}=2(\log (1+\mu _{0})-1)$ and log
unit costs are $k_{1}=k_{2}=-2+\log (1+\mu _{0})$. Now if a firm in either
industry deviates and chooses $S_{i}=\varnothing $, then its log unit costs
would be $0$, which would increase its costs since $\log (1+\mu _{0})<2$.
Substituting these prices and resulting revenues from distortions into the
representative household's budget constraint, equilibrium consumption levels
are $C_{1}=C_{2}=\frac{1}{2}\frac{e^{2}}{1+\mu _{0}}>\frac{1}{2}$.

However, if distortions are higher, in particular, $\log (1+\mu _{0})>2$,
then the unique equilibrium involves $S_{1}=S_{2}=\varnothing $, $%
p_{1}=p_{2}=\log (1+\mu _{0})$ and $k_{1}=k_{2}=0$. Now a deviation to
adopting the input of the other industry would lead to a log unit cost of $%
-2+\log (1+\mu _{0})>0$ since in this case $\log (1+\mu _{0})>2$.
Equilibrium consumption levels can then be computed as $C_{1}=C_{2}=\frac{1}{%
2}$. This verifies that the equilibrium now has a lower GDP and lower
welfare for the representative household because distortions have impacted
the equilibrium production network as well.
\end{example}

\section{Comparative Statics\label{section comparative statics}}

%
%In this section, we prove a. The first result applies to all CES functions when the technology function $A_i(S_i)$ is a constant $A_i \in \mathbb{R}$ which does not depend on $S_i$.  The second result applies when the production function is  Cobb-Douglas and the technology function $A_i(S_i)$ is log submodular in $S_i$. In both cases, we show that when technology increases, prices weakly decrease and the network $S$ weakly increases.
%

%\paragraph{CES} Let $A_i(S_i)$ be equal to a constant $A_i$ which is independent of $S_i$. We can prove the following single-crossing property when the production function belongs to the CES family.
%
%\begin{proposition}
%\label{single crossing CES}
%Let $\sigma \neq 1$ and let $A_i(S_i)$ be equal to a constant $A_i$ independent of $S_i$. Let $P',P$ be price vectors such that $P_j^{'} < P_j$. Let $S_i' \supset S_i$. Then the function $K_i(S_i,A_i,P)$ satisfies the single-crossing property
%\begin{gather}
%K_i^(S_i',A_i,P) - K_i(S_i,A_i,P) \leq 0 \implies K_i(S_i',A_i,P') - K_i(S_i,A_i,P') \leq 0.
%\end{gather}

%%In addition, if the production function is Cobb-Douglas, then the function $\log K_i(S_i,A_i,P)$ satisfies the single-crossing property
%%\begin{gather}
%%\log K_i(S_i',A_i,P) - \log K_i(S_i,A_i,P) \leq 0 \implies \log K_i(S_i',A_i,P') -\log K_i(S_i,A_i,P') \leq 0.
%%\end{gather}
%\end{proposition}
%
%\begin{proof}
%We prove this for two separate cases. The first case is when $\sigma < 1$, so that $P_i' < P_i$ implies $P_i^{'1-\sigma}  < P_i^{1-\sigma}$. In this case, we have
%\begin{gather}
%K_i(S_i',A_i(S_i'),P) - K_i(S_i,A_i(S_i),P) \leq 0  \implies \notag \\
%K_i^{1-\sigma}(S_i',A_i(S_i'),P) - K_i^{1-\sigma}(S_i,A_i(S_i),P) \leq 0  \implies \notag \\
%\frac{1}{A_i^{1-\sigma}}  ((1-\sum_{j \in S_i'} \alpha_{ij})^{\sigma} - (1-\sum_{j \in S_i} \alpha_{ij})^{\sigma}) + \frac{1}{A_i^{1-\sigma}} \sum_{j \in S_i'-S_i} \alpha_{ij}^{\sigma} P_j^{1-\sigma}  \leq 0 \implies \notag \\
%\frac{1}{A_i^{1-\sigma}}  ((1-\sum_{j \in S_i'} \alpha_{ij})^{\sigma} - (1-\sum_{j \in S_i} \alpha_{ij})^{\sigma}) + \frac{1}{A_i^{1-\sigma}} \sum_{j \in S_i'-S_i} \alpha_{ij}^{\sigma} P_j'^{1-\sigma}  \leq 0 \implies \notag \\
%K_i^{1-\sigma}(S_i',A_i(S_i'),P') - K_i^{1-\sigma}(S_i,A_i(S_i),P') \leq 0 \implies \notag \\
%K_i(S_i',A_i(S_i'),P) - K_i(S_i,A_i(S_i),P) \leq 0
%\end{gather}
%
%The second case is when $\sigma > 1$, so that $P_i' < P_i$ implies $P_i^{'1-\sigma} > P_i^{1-\sigma}$. In this case, we have
%\begin{gather}
%K_i(S_i',A_i(S_i'),P) - K_i(S_i,A_i(S_i),P) \leq 0  \implies \notag \\
%K_i^{1-\sigma}(S_i',A_i(S_i'),P) - K_i^{1-\sigma}(S_i,A_i(S_i),P) \geq 0  \implies \notag \\
%\frac{1}{A_i^{1-\sigma}}  ((1-\sum_{j \in S_i'} \alpha_{ij})^{\sigma} - (1-\sum_{j \in S_i} \alpha_{ij})^{\sigma}) + \frac{1}{A_i^{1-\sigma}}  \sum_{j \in S_i'-S_i} \alpha_{ij}^{\sigma} P_j^{1-\sigma}  \geq 0 \implies \notag \\
%\frac{1}{A_i^{1-\sigma}}  ((1-\sum_{j \in S_i'} \alpha_{ij})^{\sigma} - (1-\sum_{j \in S_i} \alpha_{ij})^{\sigma}) + \frac{1}{A_i^{1-\sigma}} \sum_{j \in S_i'-S_i} \alpha_{ij}^{\sigma} P_j'^{1-\sigma}  \geq 0 \implies \notag \\
%K_i^{1-\sigma}(S_i',A_i(S_i'),P') - K_i^{1-\sigma}(S_i,A_i(S_i),P') \geq 0 \implies \notag \\
%K_i(S_i',A_i(S_i'),P) - K_i(S_i,A_i(S_i),P) \leq 0
%\end{gather}
%
%\end{proof}\\

In this section, we present our main comparative statics results. We first
establish that when any industry's technology improves or distortions
declines, all prices (weakly) decrease. We next prove that if cost functions
satisfy a simple single-crossing condition, then an improvement in
technology will make the equilibrium network (weakly) expand. We then show
that this single-crossing condition is satisfied when (1) production
functions are supermodular; (2) production functions are Cobb-Douglas with
Hicks-neutral technologies; or (3) they have a constant elasticity of
substitution (CES) with input-specific productivity terms. We also show
that, because of the endogeneity of input choices, comparative statics can
be \textquotedblleft discontinuous\textquotedblright , in the sense that
small changes in parameters or distortions can lead to large changes in GDP
and/or the equilibrium production network.

We should note at the outset that all of our comparative statics work
through two complementary channels. The first is a \emph{direct effect}; say
$A_{i}(S_{i})$ increases, then because it has access to better technology,
industry $i$ reduces its unit cost. The second is an \emph{indirect effect},
generated because industry $i$'s technology improvements are transmitted to
other industries via price changes. If industry $i$'s price is lower, its
customers will have lower unit costs, and then their customers will have
lower unit costs as well and so on. Furthermore, because the production
network is endogenous, when industry $i$'s price decreases, other industries
are more likely to adopt it as a supplier, decreasing their own costs, which
in turn makes them more likely to be adopted as suppliers to other
industries.

\subsection{Comparative Statics for Prices}

We first show that any improvement in technologies or reduction in
distortions --- in the sense of a shift in the vector of technologies from $%
A $ to $A^{\prime }\geq A$ or in the vector of distortions from $\mu $ to $%
\mu ^{\prime }\leq \mu $ --- leads to lower prices for all products.

\begin{theorem}
\label{cs1}\textbf{(Comparative statics of prices) }Suppose Assumptions \ref%
{assumption 1} and \ref{assumption 1 second part} hold. Consider a shift in
technology from $A$ to $A^{\prime }\geq A$ and/or a decline in distortions
from $\mu $ to $\mu ^{\prime }\leq \mu $, and let $P^{\ast }$ and $P^{\ast
\ast }$ be the respective equilibrium price vectors. Then $P^{\ast \ast
}\leq P^{\ast }$.
\end{theorem}

Intuitively, an improvement in technology (or reduction in distortions)
reduces the costs and thus the prices of affected industries. But since the
outputs of these industries are used as inputs for the production of other
goods in the economy, the prices of all goods (weakly) decline as a result.
Notably, no further assumptions are necessary for this result.

\subsection{Comparative Statics for Technology Choices}

In contrast to prices, the comparative statics for technology choices
(equilibrium network) need additional assumptions. This is for two reasons.
First, to encourage (or not to discourage) the adoption of an additional
product $j$ as an input for industry $i$, we need the \textquotedblleft
marginal return to adopting $j$\textquotedblright\ to increase, but an
improvement in the technology for industry $i$ ($A^{\prime }\geq A$ as in
Theorem \ref{cs1}) does not ensure this. Second, we need to rule out the
possibility that the reduction in prices following from the adoption of an
additional input by an industry discourages the adoption of additional
inputs. The next two definitions introduce the conditions we need to ensure
these two features.

The first one defines a \emph{positive technology shock}, which embeds the
notion that a shift in technology not only improves the level of
productivity of different input combinations but also the marginal return
from adopting additional input combinations. It also imposes a
quasi-submodularity condition, which implies that additional inputs do not
directly reduce the productivity from the adoption of yet further inputs. We
define the last requirement directly using the unit cost function --- rather
than the production functions --- for convenience.

\begin{definition}
\textbf{(Positive technology shock) }A change from $A$ to $A^{\prime }$ is a
\emph{positive technology shock }if

\begin{enumerate}
\item \textbf{(higher level) }$A^{\prime }\geq A$;

\item \textbf{(quasi-submodularity)} for each $i=1,2,\ldots ,n$ and for all $%
P$, $K_{i}(S_{i},A_{i}(S_{i}),P)$ is quasi-submodular in $%
(S_{i},A_{i}(S_{i}))$.\footnote{%
Or more explicitly, for every $S_{i},T_{i},A_{i},P$ we have $%
K_{i}(S_{i},A_{i}(S_{i}),P)\leq K_{i}(S_{i}\cap T_{i},A_{i}(S_{i}\cap
T_{i}),P)\implies K_{i}(S_{i}\cup T_{i},A_{i}(S_{i}\cup T_{i}),P)\leq
K_{i}(T_{i},A_{i}(T_{i}),P)$ and $K_{i}(S_{i},A_{i}(S_{i}),P)<K_{i}(S_{i}%
\cap T_{i},A_{i}(S_{i}\cap T_{i}),P)\implies K_{i}(S_{i}\cup
T_{i},A_{i}(S_{i}\cup T_{i}),P)<K_{i}(T_{i},A_{i}(T_{i}),P).$}
\end{enumerate}
\end{definition}

The quasi-submodularity condition implies that when $A$ increases to $%
A^{\prime }$, there are higher marginal returns to adopting a larger set of
technologies, as we show in the next lemma.

\begin{lemma}
\label{lemma higher marginal returns}Suppose that for each $i=1,2,\ldots ,n$%
, $K_{i}(S_{i},A_{i}(S_{i}),P)$ is quasi-supermodular in $%
(S_{i},A_{i}(S_{i}))$. Then for each $i=1,2,\ldots ,n$ and for all $P$ and
for all $S_{i}\subset S_{i}^{\prime }$, we have%
\begin{equation*}
K_{i}(S_{i}^{\prime },A_{i}(S_{i}^{\prime
}),P)-K_{i}(S_{i},A_{i}(S_{i}),P)\leq 0\implies K_{i}(S_{i}^{\prime
},A_{i}^{\prime }(S_{i}^{\prime }),P)-K_{i}(S_{i},A_{i}^{\prime
}(S_{i}),P)\leq 0.
\end{equation*}
\end{lemma}

Quasi-submodularity ensures that, holding prices constant, an improvement in
technology from $A$ to $A^{\prime }$ encourages the adoption of a larger set
of inputs. But as highlighted in Theorem \ref{cs1}, an improvement in
technology also leads to lower prices. The next definition introduces the
requirement that the return to additional technology adoption does not
diminish as prices decline. This is a reasonable restriction (since lower
prices mean that the cost of buying inputs associated with the new
technology is also lower), even though it is by no means automatic. We show
later in this section that several common production functions satisfy this
restriction.

\begin{definition}
\textbf{(Technology-price single-crossing condition) }For each $i=1,2,\ldots
,n$, the unit cost function $K_{i}(S_{i},A_{i}(S_{i}),P)$ satisfies the
\emph{technology-price single-crossing condition }in the sense that for all
sets of inputs $S_{i},S_{i}^{\prime }$ with $S_{i}\subset S_{i}^{\prime }$
and all prices vectors $P^{\prime },P$ with $P_{-i}^{\prime }\leq P_{-i}$,
we have%
\begin{equation*}
K_{i}(S_{i}^{\prime },A_{i}(S_{i}^{\prime
}),P)-K_{i}(S_{i},A_{i}(S_{i}),P)\leq 0\implies K_{i}(S_{i}^{\prime
},A_{i}(S_{i}^{\prime }),P^{\prime })-K_{i}(S_{i},A_{i}(S_{i}),P^{\prime
})\leq 0.
\end{equation*}
\end{definition}

Note that in contrast to the quasi-submodularity condition, this
single-crossing condition is a joint restriction on how the unit cost
function changes when both the set of inputs and prices are modified.

The next theorem is our main comparative static result and proves that under
the technology-price single crossing condition, a positive technology shock
or a reduction in distortions encourages technology adoption by all
industries.

\begin{theorem}
\label{theorem network increasing}\textbf{(Comparative statics of the
production network)} Suppose Assumptions \ref{assumption 1} and \ref%
{assumption 1 second part} and the technology-price single-crossing
condition hold. Then a positive technology shock or a decrease in
distortions (weakly) increases the equilibrium network from $S^{\ast }$ to $%
S^{\ast \ast }$.
\end{theorem}

By definition a positive technology shock creates direct incentives for the
adoption of additional inputs. This implies that, all else equal,
\textquotedblleft affected\textquotedblright\ industries (weakly) increase
their sets of suppliers. This then creates a series of indirect effects,
because the use of better technology reduces their prices. The
technology-price single-crossing condition implies that facing lower prices,
other industries will also be induced to (weakly) expand their sets of
suppliers. The logic for the effects of distortions is similar: lower
distortions reduce prices and under the technology-price single-crossing
condition this encourages an expansion of the set of suppliers for other
industries.

The technology-price single-crossing condition is not always satisfied as we
show in Example \ref{example counter example} in Appendix B. Nevertheless,
it is satisfied for several common families of production technologies. The
proofs of the next three propositions are provided in Appendix B.

\begin{proposition}
\label{prop:supermodular}Suppose $F_{i}(L_{i},X_{i},A_{i}(S_{i}),S_{i})$ is
supermodular in all its arguments. Then the unit cost function $%
K_{i}(S_{i},A_{i}(S_{i}),P)$ satisfies the technology-price single-crossing
condition.
\end{proposition}

%\paragraph{Specification 2: Hicks-Neutral Cobb-Douglas functions}
Even more important in many applications with input-output linkages is the
family of Cobb-Douglas production functions. The next proposition shows that
Cobb-Douglas production functions with Hicks-neutral technology satisfy the
technology-price single crossing condition.

\begin{proposition}
\label{prop:CD}Suppose $F_{i}(S_{i},A_{i}(S_{i}),L_{i},X_{i})$ is in the
Hicks-neutral Cobb-Douglas family. Then the unit cost function $%
K_{i}(S_{i},A_{i}(S_{i}),P)$ satisfies the technology-price single-crossing
condition.
\end{proposition}

The previous two propositions established the technology-price
single-crossing condition when the productivity of an industry, and thus its
unit cost function, depends on the set of inputs, $S_{i}$. Our next example
is more restrictive in this regard in that we consider \textquotedblleft
input-specific\textquotedblright\ productivities, meaning that each input
has a specific productivity (for the sector in question) which applies
regardless of which other inputs are being used. We then show that when
production functions are CES with input-specific productivities, the
single-crossing property is again satisfied.

\begin{proposition}
\label{cs ces}Suppose $F_{i}(S_{i},A_{i}(S_{i}),L_{i},X_{i})$ is a CES
function with input-specific productivities, i.e.,
\begin{equation}
(\sum_{j\in S_{i}}\alpha _{ij} (A_{ij}X_{ij})^{\frac{%
\sigma -1}{\sigma }}+(1-\sum_{j\in S_{i}}\alpha _{ij})
L_{i}^{\frac{\sigma -1}{\sigma }})^{\frac{\sigma }{\sigma -1}}
\end{equation}%
with $\sigma \neq 1$. Then the unit cost function $%
K_{i}(S_{i},A_{i}(S_{i}),P)$ satisfies the technology-price single-crossing
condition. \label{main CES}
\end{proposition}

\subsection{Discontinuous Effects}

One novel feature of our economy is that, because of the changes in the
production network, small changes in technology or parameters can lead to
discontinuous effects. In this subsection, we first illustrate the
possibility of discontinuous changes in GDP and then show how there can also
be discontinuous network effects in the sense that small changes in
productivity lead to a large change in the equilibrium production network.
These examples further illustrate that while these discontinuous responses
partly reflect the discreteness of the choices over the set of suppliers in
our model, they are crucially a consequence of the interdependent nature of
technology adoption decisions --- the adoption of a productive technology
reduces an industry's unit cost of production and makes it more attractive
as an input supplier to other industries.

\begin{example}
\textbf{(Discontinuous GDP)} Consider an economy with two industries, both
of which have Cobb-Douglas production functions, with respective parameters $%
\alpha _{12}=\alpha _{21}=\frac{1}{2}$, and assume that industries cannot
use their own output as input. The representative household has a utility
function $U(C_{1},C_{2})=\log (C_{1})+\log (C_{2})$. Suppose $\mu _{1}=\mu
_{2}=\mu _{0}>0$, $\lambda _{1}=\lambda _{2}=1$, $a_{1}(\varnothing
)=a_{2}(\varnothing )=\epsilon >0$, $a_{1}(\{2\})=a_{2}(\{1\})=\frac{1}{2}%
\log (1+\mu _{0})$. Therefore, log prices satisfy%
\begin{gather*}
p_{1}=\log (1+\mu _{0})-a_{i}(S_{i})+I_{2\in S_{1}}\frac{1}{2}p_{2} \\
p_{2}=\log (1+\mu _{0})-a_{i}(S_{i})+I_{1\in S_{2}}\frac{1}{2}p_{1}.
\end{gather*}

The unique equilibrium then involves $S_{1}=S_{2}=\varnothing $ and $%
p_{1}=p_{2}=\log (1+\mu _{0})-\epsilon $. To see this, note that marginal
costs for both industries are $-\epsilon $. If a firm in industry 1 deviates
and chooses $S_{1}=\{2\}$, then its log marginal cost would be $-\frac{1}{2}%
\epsilon >-\epsilon $. Analogously, a deviation to $S_{2}=\{1\}$ in industry
2 will also raise costs. In this equilibrium, industry $i$'s revenues from
distortions are $\frac{\mu _{0}}{1+\mu _{0}}P_{i}C_{i}$ and equilibrium
consumption levels are $C_{1}^{I}=C_{2}^{I}=\frac{e^{\epsilon }}{2}$. Thus
as $\epsilon \rightarrow 0^{+}$, $C_{1}^{I}=C_{2}^{I}\rightarrow \frac{1}{2}$%
.

Now consider a change in technology to $a_{1}(\{2\})=a_{2}(\{1\})=\frac{1}{2}%
\log (1+\mu _{0})+\epsilon $. For $\epsilon $ small, this is a small change
in technology. Following this change in technology, the unique equilibrium
changes to $S_{1}=\{2\}$ and $S_{2}=\{1\}$, and $p_{1}=p_{2}=\log (1+\mu
_{0})-2\epsilon $. Now industry $i$'s revenues from distortions are $\Lambda
_{i}=\frac{\mu _{0}}{1+\mu _{0}}P_{i}(C_{i}+X_{-i,i})$ where $X_{-i,i}$ is
industry $-i$'s use of industry $i$'s inputs, and equilibrium consumption
levels can be computed as $C_{1}^{II}=C_{2}^{II}=\frac{e^{2\epsilon }}{2}%
+\mu _{0}X_{12}=\frac{e^{2\epsilon }}{2}(1+\mu _{0})>\frac{e^{\epsilon }}{2}$%
. As $\epsilon \rightarrow 0^{+}$, $C_{1}^{II}=C_{2}^{II}\rightarrow \frac{1%
}{2}+\frac{\mu _{0}}{2}>\frac{1}{2}$, and so consumption levels and GDP
change discontinuously following an infinitesimal change in $\epsilon $.
\end{example}

This example shows how an infinitesimal change in productivity can have a
first-order impact on GDP when the equilibrium production network changes in
response and there are distortions (or markups). Intuitively, an industry
can add new suppliers following an increase in productivity. Since it was
previously minimizing its costs, this change can only have a small impact on
its profits. But when the industry adds a new supplier, its purchases from
this new supplier go from zero to a positive amount. Because, in the
presence of distortions/markups, prices are not equal to marginal cost, this
change can have a nontrivial impact on the supplier's profits, which are
partially rebated to the representative household. Underscoring the central
role of the endogeneity of the production network in this result, we show in
Theorem \ref{proposition continuity no network} in Appendix B that when the
production network is exogenous, there are no discontinuous effects on GDP,
with or without distortions.

In Appendix D we illustrate this discontinuous effect for an economy
calibrated to the 2007 US input-output data (for 391 sectors) with
distortions at the two-digit level given by De Loecker, Eeckhout and Unger's
(2018) markup estimates.\footnote{\label{footnote data}For this exercise, we
exclude from the 2007 input-output tables the government sector as well as
two sectors with zero labor income, privately-owned residential property and
the sector made up of custom duties. Throughout, GDP refers to US GDP after
these sectors are excluded (the sum of the value added of the remaining
sectors).} We then consider a 1\% increase in the TFP of the detailed
industries in the two-digit computer and electronic product manufacturing
sector (NAICS sector 334, which accounts for just 1.98\% of GDP). Because
the increase in the productivity of these industries makes them more likely
to be adopted as inputs to other industries, we find that the equilibrium
production network changes significantly (288 new edges are added to the
input-output matrix) and real GDP increases by 0.72\%. Of this increase,
0.13 percentage points are accounted for by the rise in value added in
computer and electronic product manufacturing and the remaining 0.59
percentage points come from the induced expansion of other sectors.

We then repeat the same exercise for two economies, one with Cobb-Douglas and
the other with CES sectoral production functions, calibrated to the same US
data, but at the level of 84 three-digit industries. More importantly, for
these more aggregated economies we do not allow any extensive margin changes
in the production network. So with Cobb-Douglas technologies the
input-output matrix remains unchanged while with CES technologies the
entries in the input-output matrix change following changes in prices, but
no new links are added to it. We find that the same 1\%
TFP increase in computer and electronic product manufacturing leads to a
0.04\% increase in GDP  with Cobb-Douglas technologies.\footnote{%
Because of markups, the magnitudes of the effects for the aggregated
economies (without extensive margin change in the input-output matrix) are a
little larger than the impact implied by Hulten's Theorem for an economy
without distortions (Hulten, 1978). In particular, the Domar weight of the
computer and electronic product manufacturing sector is 2.68\%, so Hulten's
Theorem implies that, without distortions/markups, a 1\% increase in TFP should have
increased GDP by about $0.01\times 2.68\%\approx 0.03\%$, compared to the $%
0.04\%$ increase we find in the aggregated Cobb-Douglas economy.} With CES technologies the same 1\% TFP increase in electronics leads to a $0.01\%$ increase in  GDP when the elasticity of substitution is $\sigma=2$, and a $0.09\%$ increase in GDP when $\sigma = \frac{1}{2}$.  This
illustrates both the potentially discontinuous impact of shocks in the
presence of an endogenous production network and the fact that behavior in
an economy with endogenous input-output linkages cannot generally be
replicated with a more aggregated economy without endogenous linkages.

The next example shows that it is not just GDP but the equilibrium
production network that can respond discontinuously to a small change in
technology or parameters.

\begin{figure}[H]
\centering
\begin{adjustbox}{minipage=\linewidth,scale=0.75}
\begin{subfigure}[b]{0.4\textwidth}
\centering
\includegraphics[width=\textwidth]{TikzPicture1.pdf}
\caption[Network2]%
{{\small Initial network}}
\label{fig:mean and std of net14}
\end{subfigure}
\hfill
\begin{subfigure}[b]{0.4\textwidth}
\centering
\includegraphics[width=\textwidth]{TikzPicture2.pdf}
\caption[]%
{{\small Network immediately after shock to industry 1's productivity}}
\label{fig:mean and std of net24}
\end{subfigure}
\vskip\baselineskip
\begin{subfigure}[b]{0.4\textwidth}
\centering
\includegraphics[width=\textwidth]{TikzPicture3.pdf}
\caption[]%
{{\small Network after all industries choose industry 1 as a supplier  }}
\label{fig:mean and std of net34}
\end{subfigure}
\hfill
\begin{subfigure}[b]{0.4\textwidth}
\centering
\includegraphics[width=\textwidth]{TikzPicture4.pdf}
\caption[]%
{{\small Network after the new equilibrium is reached }}
\label{fig:mean and std of net44}
\end{subfigure}
\end{adjustbox}
\caption*{\textbf{Figure 2: Evolution of the input-output network in Example
3.} This figure shows how the input-output network in Example 3 changes
after a shock to industry 1's productivity. Immediately after the shock,
industry 1 will adopt all other industries as suppliers. This leads to a
drop in industry 1's price. This drop in price is large enough that all
other industries adopt industry 1 as a supplier. This leads to a cascade
effect of declining prices, until all industries adopt each other as
suppliers. }
\end{figure}

\begin{example}
\textbf{(Discontinuous network effects) }Consider an economy with $n$
industries. Each industry has a Hicks-neutral Cobb-Douglas production
function. Given a log price vector $p$, each industry $i$ chooses a set of
suppliers $S_{i}$ which minimizes the log unit cost function $%
k_{i}(S_{i},a_{i}(S_{i}),p)=-a_{i}(S_{i})+\sum_{j\in S_{i}}\alpha _{ij}p_{j}$%
. The economy's initial log productivity function is $a_{i}(\varnothing )=0$
and $a_{i}(S_{i})=-\epsilon $ for all $i$ and all $S_{i}\neq \varnothing $,
where $\epsilon >0$ can be taken to be arbitrarily small. To simplify the
example, we set distortions equal to zero, $\mu _{i}=0$ for all $%
i=1,2,\ldots ,n$ and assume that own output cannot be used as input. It is
then straightforward to verify that the unique equilibrium network is empty,
i.e., $(S_{1},...,S_{n})=(\varnothing ,...,\varnothing )$, and the
equilibrium log price vector is $p=0$.

Consider next a change in technology increasing industry 1's log
productivity to $a_{1}^{\prime }(\{1,...,n\}\backslash \{1\})=\kappa
\epsilon $, and $a_{1}^{\prime }(S_{1})=a_{1}(S_{1})$ for all $S_{1}\neq
\{1,...,n\}\backslash \{1\}$. (There is no change for other industries, $%
a_{i}^{\prime }(S_{i})=a_{i}(S_{i})$ for all $i\neq 1$ and all $S_{i}$). We
take $\kappa >\max_{i\neq 1}\frac{1}{\alpha _{i1}}$. The following
tattonement process converges to a new equilibrium, represented by a network
$S^{\prime }$ and log prices $p^{\prime
}=-(I-\alpha(S^{\prime}))a(S^{\prime})$.

In round 0 of the tattonement process, we take the initial price vector $p=0$
as given and allow each industry to minimize costs using their new
technology $a^{\prime }$. Therefore, in this round all industries except
industry 1 choose $S_{i}=\varnothing $, while industry 1 sets $%
S_{1}=\{1,...,n\}\backslash \{1\}$ and achieves log unit cost of $%
\min_{S_{1}}k(S_{1},a_{1}^{\prime }(S_{1}),p)=-a_{1}(S_{1})=-\kappa \epsilon
$. Consequently, at the end of round 0, the network is $S^{0}=(\{1,...,n\}-%
\{1\},\varnothing ,...,\varnothing )$ and the log price vector is $%
p^{0}=(-\kappa \epsilon ,0,...,0)$.

In round 1, we now impose the log price vector of $p^{0}$, which resulted
from round 0. Now industry 1 still chooses $S_{1}=\{1,...,n\}\backslash
\{1\} $. All other industries will choose industry 1 as a supplier since $%
\epsilon -\alpha _{i1}\kappa \epsilon <0$ (which follows since we have
assumed that $\kappa >\max_{i}\frac{1}{\alpha _{i1}}$). Thus at the end of
round 1, we have $S^{1}=(\{1,...,n\}\backslash \{1\},\{1\},...,\{1\})$ and a
log price vector of $p^{1}$ such that $p_{i}^{1}<0$ for all $i$.

In round 2 of the tattonement process, we take the log price vector to be $%
p^{1}$, all of whose elements are negative. This ensures that for each
industry $i$, the cost-minimizing set of suppliers is now $S_{i}=\{1,...,n\}$%
, leading to a vector of log prices of $p^{2}\leq p^{1}<0$.

In round $t\geq 3$, the network of technologies has converged, i.e., $%
S^{\prime }=S^{3}=S^{2}$, and only prices are updated with%
\begin{equation*}
p_{i}^{t}=-a_{i}(S_{i}^{\prime })+\sum_{j\in S_{i}^{\prime }}\alpha
_{ij}p_{j}^{t-1}.
\end{equation*}%
This process converges to $p^{\prime }$. Thus in this example, a small
change in productivity shifts the equilibrium technology choices from the
empty to the complete network.
\end{example}

%%\begin{figure}
%\centering
%\includegraphics[scale=0.4]{../Figures/Example1.png}
%\caption{The network in example 1, with 4 firms. Firm $i$ can choose between using only labor, or using its immediate predecessor $i-1$ as a supplier.}
%\end{figure}

\section{Growth with Endogenous Production Networks\label{section dynamic}}

We now extend our baseline model to a dynamic framework and show how our
approach isolates a new economic force --- productivity growth from new
input combinations --- that can generate sustained economic growth. For this
purpose, we start with the special case of our baseline model with
Hicks-neutral Cobb-Douglas functions. The key economic force propagating
sustained growth can be understood as follows: If there are $t$ products in
the economy, then each industry $i$ has access to $t-1$ possible suppliers
and $2^{t-1}$ ways of combining these suppliers. Even with a trickle of new
products, there is thus a significant expansion of the set of inputs
available to each industry and selecting the most beneficial combination to
achieve a high $A_{i}(S_{i})$ and/or low prices for inputs $j\in S_{i}$ can
generate significant cost reductions.

In the baseline version of our dynamic model, one new industry arrives in
each period, and all firms have the option of updating their technology by
combining the new industry's product with any other subset of products. We
ensure that growth is not driven because of expanding product variety by
imposing that new products have limited benefits to consumers. But they may
significantly reduce the production cost of existing products due to a \emph{%
selection effect} --- each industry can now choose its production technology
from the exponentially greater number of options made possible by the
arrival of one more input. We then show that this selection effect can
generate sustained growth.

After establishing this result, we show that it generalizes to an economy
with non-Cobb-Douglas production functions, to an environment in which there
are various constraints on technology choices, to a setting in which the
arrival of new products may be faster or slower than our benchmark and to an
economy in which new products emerge from new input combinations. In the
final subsection we consider a more substantive generalization where new
products replace some of the old ones as either inputs or in consumption.

\subsection{Model}

There are countably infinite time periods indexed by $t\in \{1,2,3,...\}$.
At each time $t$, a new product arrives. Products are indexed by the time at
which they arrive, so that the product arriving at time $t$ is referred to
as product $t$. We index all endogenous variables with time, for example
writing $P_{i}(t)$ for the equilibrium price of product $i$ at time $t$.
Analogously, we denote the values of $L_{i},Y_{i},X_{ij},C_{i},$ and $S_{i}$
at time $t$ by $L_{i}(t),Y_{i}(t),X_{ij}(t),C_{i}(t),$ and $S_{i}(t)$. The
equilibrium wage rate at each $t$ is set as the numeraire, i.e.,%
\begin{equation*}
W(t)=1\text{ for all }t.
\end{equation*}

\paragraph{Production Technology}

At time $t$, each industry $i\in \{1,...,t\}$ has access to a collection of
production technologies indexed by the set of suppliers $S_{i}(t)\subset
\{1,...,t\}$. Instead of Assumption \ref{assumption 1}, we now impose:

\begin{assumptionbis}{assumption 1}
Production functions are in the Cobb-Douglas family with Hicks-neutral
technologies. That is, for industry $i$ at time $t$, we have%
\begin{equation*}
Y_{i}(t)=\frac{A_{i}(S_{i}(t))}{(1-\sum_{j\in S_{i}(t)}\alpha
_{ij})^{1-\sum_{j\in S_{i}(t)}\alpha _{ij}}\prod_{j\in S_{i}(t)}\alpha
_{ij}^{\alpha _{ij}}}L_{i}(t)^{1-\sum_{j\in S_{i}(t)}\alpha
_{ij}}\prod_{j\in S_{i}(t)}(X_{ij}(t))^{\alpha _{ij}}.
\end{equation*}
\label{assumption1prime}
\end{assumptionbis}

As in our baseline model, adopting (or dropping) new suppliers is costless.
This implies that at each point in time, regardless of the way that the
households trade off current and future consumption, firms will adopt the
cost-minimizing combination of inputs.

We continue to assume the same contestable market structure with distortions
as in our static model, and also suppose that distortions are constant over
time and we continue to denote them by the vector $\mu $. We also rule out
the possibility that the distortions for new goods are unbounded. That is:

\begin{assumption}
\label{assumption markups} There exists $\mu _{0}<\infty $ such that $\sup
\{\mu _{t}\}_{t=1}^{\infty }\leq \mu _{0}$.
\end{assumption}

\paragraph{Preferences}

On the preference side, we replace Assumption \ref{assumption 1 second part}
with

\begin{assumptionbis}{assumption 1 second part}
Time-$t$ preferences of the representative household take a Cobb-Douglas form,%
\begin{equation}
u(C_{1}(t),...,C_{t}(t),\beta )=\left[ \prod_{i=1}^{t}\left( \frac{\beta _{i}%
}{\sum_{i=1}^{t}\beta _{i}}\right) ^{-\beta
_{i}}\prod_{i=1}^{t}C_{i}(t)^{\beta _{i}}\right] ^{\frac{1}{%
\sum_{i=1}^{t}\beta _{i}}},  \label{eq:growth_utility}
\end{equation}%
where the vector $\beta $ satisfies $\beta _{t}\geq 0$ for all $t$ and $%
\sum_{t=1}^{\infty }\beta _{t}=1$. \label{assumption2prime}
\end{assumptionbis}

The first term in square brackets is included as convenient a normalization.
The overall utility of the representative household is given by a discounted
sum of its time-$t$ preferences.

Crucially, this specification implies that $\lim_{t\rightarrow \infty }\beta
_{t}=0$.\footnote{%
In fact, we can set $\beta _{t}=0$ for all $t$ after some $T^{\ast }<\infty $
with no change in any of our results.} This feature highlights the reason
why we have adopted Cobb-Douglas preferences --- to construct a simple
measure of GDP/utility (see (\ref{real GDP}) below), and to clarify that
direct utility gains from the addition of new products are minimal.
Intuitively, this feature can be justified as follows: we can imagine that
the representative household's core necessities are met by goods introduced
relatively early in the development process (e.g., hot food, clothing and
entertainment), while goods introduced later (such as microwave ovens,
automated textile technologies and streaming) could be useful for more
efficiently meeting these necessities, but will not directly increase
consumer utility by a large amount. We discuss how this assumption can be
relaxed, allowing new goods to replace old goods, later in this section.

%Logarithmic preferences imply that the indirect utility function of the
%representative household at time $t$ is%
%\begin{equation*}
%V(P(t))=\sum_{i=1}^{t}\beta _{i}\log \frac{\beta _{i}}{P_{i}(t)}%
%=\sum_{i=1}^{t}\beta _{i}\log \beta _{i}-\sum_{i=1}^{t}\beta _{i}\log
%P_{i}(t).
%\end{equation*}

At time $t$, nominal GDP is given by $Y^{N}(t)=%
\sum_{i=1}^{t}P_{i}(t)C_{i}(t)=1+\sum_{i=1}^{t}\lambda _{i}\frac{\mu _{i}}{%
1+\mu _{i}}P_{i}(t)Y_{i}(t)$ (which includes labor income and income rebated
from taxes and profits). Real GDP, which is also equal to the representative
household's utility $u(C_{1}(t),...,C_{t}(t),\beta )$, is obtained by
deflating nominal GDP by the ideal price index derived from the
representative household's utility maximization problem.\footnote{%
Utility-maximizing consumption levels for the representative household are $%
C_{i}^{\ast }(t)=\frac{\beta _{i}}{\sum_{i=1}^{t}\beta _{i}}\frac{Y^{N}(t)}{%
P_{i}(t)}$. Substituting these into (\ref{eq:growth_utility}), we obtain $%
U(C_{1}^{\ast }(t),...,C_{t}^{\ast }(t),\beta )=\left[ \prod_{i=1}^{t}(\frac{%
\beta _{i}}{\sum_{i=1}^{t}\beta _{i}})^{-\beta _{i}}\prod_{i=1}^{t}(\frac{%
\beta _{i}}{\sum_{i=1}^{t}\beta _{i}}\frac{Y^{N}(t)}{P_{i}(t)})^{\beta _{i}}%
\right] ^{\frac{1}{\sum_{i=1}^{t}\beta _{i}}}=\frac{Y^{N}(t)}{%
\prod_{i=1}^{t}P_{i}(t)^{\frac{\beta _{i}}{\sum_{i=1}^{t}\beta _{i}}}}.$}
Specifically, real GDP is%
\begin{equation}
Y(t)=\frac{Y^{N}(t)}{\prod_{i=1}^{t}P_{i}(t)^{\frac{\beta _{i}}{%
\sum_{i=1}^{t}\beta _{j}}}}.  \label{real GDP}
\end{equation}%
We define the asymptotic growth rate of real GDP as:\footnote{%
An alternative definition of the asymptotic growth rate would have been $%
\lim_{t\rightarrow \infty }\Delta Y(t)$. When this limit exists, it is
straightforward to see that $g^{\ast }=\lim_{t\rightarrow \infty }\Delta
Y(t) $. However, this limit may fail to exist, even though $g^{\ast }$ is
well defined (e.g., because $\Delta Y(t)$ fluctuates between high and low
values even asymptotically). Our definition thus avoids these inessential
complications.}%
\begin{equation*}
g^{\ast }=\lim_{t\rightarrow \infty }\left( \frac{\log Y(t)}{t}\right) .
\end{equation*}

The next lemma shows that asymptotically the growth rate of real GDP is a
simple function of changes in prices, highlighting that asymptotic growth in
this economy is a result of declines in production costs and prices.

\begin{lemma}
\label{lemma growth rate}\textbf{(Asymptotic growth) }The asymptotic growth
rate of real GDP is%
\begin{equation*}
g^{\ast }=\lim_{t\rightarrow \infty }\left( -\frac{\pi (t)}{t}\right) ,
\end{equation*}%
where $\pi (t)=\frac{\beta _{i}}{\sum_{i=1}^{t}\beta _{i}}p_{i}(t)$ is the
(log) ideal price index at time $t$.
\end{lemma}

From (\ref{Leontief equation}) the (log) ideal price index can be written as%
\begin{equation}
\pi (t)=-\beta (t)^{\prime }\mathcal{L}(t)a(S(t)-m(t)),
\label{asymptotic price equation}
\end{equation}%
where $\beta (t)=(\frac{\beta _{i}}{\sum_{i=1}^{t}\beta _{i}},\ldots ,\frac{%
\beta _{t}}{\sum_{i=1}^{t}\beta _{i}})^{\prime }$ is the vector of
consumption shares at time $t$, $%
a(S(t))=(a_{1}(S_{1}(t)),...,a_{t}(S_{t}(t)))^{\prime }$ is the vector of
log productivity terms, $\mathcal{L}(S(t))$ is the Leontief inverse matrix
when the input-output network is given by $S(t)$, and $m(t)=(\log (1+\mu
_{1}),\ldots ,\log (1+\mu _{t}))^{\prime }$ is the vector of log distortions.

We also impose two additional assumptions in our dynamic analysis.

\begin{assumption}
\label{assumption logproductivity negative} For a fixed $t$ and $i\in
\{1,...,t\}$, the log productivity vector $a_{i}(t)=\{a_{i}(S_{i},t)%
\}_{S_{i}\subset \{1,...,t\}}$ is drawn from a distribution $\Phi _{i}(t)$.
Furthermore, there exists a constant $D>0$ such that, if $\{a_{i}(t)\}_{t\in
\mathbb{N}}$ is a sequence of log productivity vectors for industry $i$,
then
\begin{equation*}
\lim_{t\rightarrow \infty }\max_{S_{i}\subset \{1,...,t\}}\frac{%
a_{i}(S_{i},t)}{t}=D\text{ almost surely .}
\end{equation*}
\end{assumption}

Assumption \ref{assumption logproductivity negative} rules out log
productivity distributions that have either too thin or too thick tails.
Note that this assumption does not require the draws of log productivities
to be identical or independent, thus allowing for both correlation between
different productivity realizations and the possibility that the
productivity of certain inputs in all or some industries are higher than
others. We show in Propositions \ref{proposition gumbel} and \ref%
{proposition exponential distribution} in Appendix B that when the $%
a_{i}(S_{i},t)$'s are independent draws from Gumbel or exponential
distributions, Assumption \ref{assumption logproductivity negative} is
satisfied.\footnote{\label{footnote correlation}In Appendix B we also
explore how different types of non-identical distributions can be
parameterized. We show in Proposition \ref{proposition correlated} that one
tractable example that satisfies Assumption \ref{assumption logproductivity
negative} is $a_{i}(S_{i}(t))=\sum_{j\in S_{i}(t)}\tilde{a}_{j}$ with each $%
\tilde{a}_{j}$ drawn independently from $\{-1,1\}$ with equal probabilities.
Then $Cov(a_{i}(S),a_{i}(S^{\prime }))=\sum_{j\in S_{i}\cap S_{i}^{\prime
}}Var(a_{j})=|S_{j}\cap S_{j}^{\prime }|$. Other correlated productivity
structures are described in the next section.} In contrast, finite and
normal distributions do not satisfy this assumption because their tails
decrease at a faster rate than exponential. This assumption is not satisfied
for the Pareto or Frechet distributions either, this time because their
tails decrease at a slower rate than exponential.

\begin{assumption}
\label{assumption leontief}

\begin{enumerate}
\item There exists $\theta <1$ such that $\sum_{j=1}^{\infty }\alpha
_{ij}\leq \theta $ for all $i\in \mathbb{N}$.

\item Furthermore, for every $\epsilon >0$, there exists a constant $T$ such
that for all $i\in \mathbb{N}$, $\sum_{j=T}^{\infty }\alpha _{ij}\leq
\epsilon $.
\end{enumerate}
\end{assumption}

The first part of Assumption \ref{assumption leontief} imposes that the
matrix norm $\Vert \alpha (S(t))\Vert _{\infty }=\max_{i}\sum_{j}|\mathcal{%
\alpha }_{ij}(S(t))|$ is uniformly bounded for all $t$, and implies that $%
\Vert \mathcal{L}(S(t))\Vert _{\infty }\leq \sum_{\ell =0}^{\infty }\Vert
\alpha ^{\ell }(S(t))\Vert _{\infty }\leq \frac{1}{1-\theta }$. This bound
is a dynamic analogue of our requirement in Assumption \ref{assumption 1}
that labor is an essential factor of production. Without this assumption,
the share of labor in each industry could asymptote to zero. The second part
of Assumption \ref{assumption leontief} states that goods invented
relatively early on do not just have larger consumption shares (as imposed
in Assumption \ref{assumption1prime}), but they also make up the more
important inputs in the sense that the sum of the cost shares of inputs
arriving after some time $T$ are uniformly bounded. This property is used in
the proof of the main result of this section, Theorem \ref{proposition
growth}, and enables us to prove that the upper and lower bounds for the
asymptotic growth rate of the economy with endogenous production networks
are the same; it is relaxed later in this section.

Even though more recent goods are assumed not to make up a large fraction of
input costs, they can have important productivity consequences. The role of
GPS technology in smartphones illustrates this possibility. GPS components
are not essential for smartphones and account for a very small fraction of
costs. For example, the GPS component for iPhone 3G, the first iPhone
featuring this technology, cost \$3.60 or about 2\% of the overall cost of a
smartphone. But by enabling real-time location information to be used for
service delivery and tracking, GPS technology greatly increased the
usefulness of smartphones as an input to other industries such as
ridesharing, trucking and parcel services.\footnote{%
See
\url{
https://www.edn.com/electronics-news/4326563/iPhone-3G-has-173-BOM-iSuppli-estimates}%
.}

\subsection{Sustained Growth}

The main result of this section is that when firms can select their set of
suppliers from all available combinations, the economy will (almost surely)
achieve sustained economic growth.

\begin{theorem}
\label{proposition growth}\textbf{(Growth) }Suppose that Assumptions \ref%
{assumption1prime}, \ref{assumption2prime}, \ref{assumption markups}, \ref%
{assumption logproductivity negative} and \ref{assumption leontief} hold,
and let $D>0$ be as defined in Assumption \ref{assumption logproductivity
negative}. Each industry chooses its set of suppliers $S_{i}^{\ast
}(t)\subset \{1,\ldots ,t\}$. Then for each $i=1,2,\ldots ,t$, the
equilibrium log price vector $p^{\ast }(t)$ satisfies%
\begin{equation*}
\lim_{t\rightarrow \infty }-\frac{p_{i}^{\ast }(t)}{t\sum_{j=1}^{t}\mathcal{L%
}_{ij}}=D>0\text{ almost surely},
\end{equation*}%
and thus%
\begin{equation*}
g^{\ast }=D\sum_{i,j=1}^{\infty }\beta _{i}\mathcal{L}_{ij}>0\text{ almost
surely.}
\end{equation*}
\end{theorem}

When firms can choose their input suppliers in an unrestricted fashion, the
economy (almost surely) achieves sustained growth. The selection of inputs
--- the fact that out of the many new input combination options presented to
them each industry chooses the cost-minimizing combination --- is at the
root of this sustained growth result. This can also be seen from the
following: if we restricted the choice of input combinations (for example,
by allowing firms to choose at any point only between their current input
combination and a randomly chosen alternative set), there would be no
sustained growth. This is shown in Theorem \ref{theorem no growth without
endogenous choice} in Appendix B.

One useful intuition for Theorem \ref{proposition growth} can be obtained as
follows. Suppose that industries choose their input suppliers to maximize
productivity (i.e., industry $i$ choses $S_{i}\in \arg \max_{S}a_{i}(S)$),
then under Assumption \ref{assumption logproductivity negative} the
productivity of each industry $i$, $a_{i}(S)$, would have an asymptotic
growth rate of $D$. From (\ref{asymptotic price equation}), this yields
asymptotic growth at the rate $\beta ^{\prime }\mathcal{L}D$. Though simple
and useful, this intuition is limited as it does not clarify the fundamental
force leading to asymptotic growth. In equilibrium, an industry does not
maximize productivity, but minimizes its costs. All the same, we show that,
under Assumption \ref{assumption leontief}, its asymptotic costs cannot be
much lower or much greater than the cost of a firm maximizing productivity.
The bulk of the proof of the theorem focuses on establishing this step.

Theorem \ref{proposition growth} also illustrates the direct and indirect
effects that the arrival of new technologies has on prices. The direct
effect is that as each industry $i$ faces an expanded set of possible input
combinations, its cost and thus equilibrium price declines. The indirect
effect comes from the fact that, as industry $i$'s price declines,
industries that use this industry's output as input will also benefit
because their costs will decrease. In particular, recall that $-p_{i}^{\ast
}(t)=\sum_{j=1}^{t}\mathcal{L}_{ij}(S(t))(a_{j}(S_{j}(t))-\log (1+\mu _{j}))$%
.

We can measure the direct effect by counterfactually setting the prices of
all intermediate inputs for industry $i$ to $P_{j}(t)=1$ (for $j\neq i$) so
that cost reductions of industries adopting new technologies do not benefit
their customers. In this case, we would have that the log price of all
intermediate inputs is zero, and the log cost of producing good $i$ becomes $%
k_{i}^{\ast }(t)=-a_{i}(S_{i}(t))$. Industry $i$ would then choose $S_{i}$
to maximize $a_{i}(S_{i})$ and consumers would face the price $p_{i}=\log
(1+\mu _{i})-\max_{S_{i}}a_{i}(S_{i})$. The log GDP level would then be $%
\sum_{i=1}^{t}\beta _{i}(\max_{S_{i}^{\prime }}a_{i}(S_{i}^{\prime })-\log
(1+\mu _{i}))$, capturing just the direct effect. The indirect effect is the
difference between this quantity and the (negative) price index, $%
\sum_{i,j=1}^{t}\beta _{i}\mathcal{L}_{ij}(S)(a_{j}(S_{j})-\log (1+\mu
_{j})) $, which includes cost reductions in other sectors working through
the Leontief inverse matrix $\mathcal{L}(S)$.

\subsection{Generalizations}

In this subsection, we show how several of the assumptions used so far can
be relaxed without affecting the main conclusion about new input
combinations generating sustained growth. We start with three corollaries
that generalize certain aspects of our environment and clarify the economic
forces that generate sustained growth in our model. Because they are minor
variations on the proof of Theorem \ref{proposition growth}, the proofs of
these corollaries are omitted.

The first corollary shows that it is sufficient for a subset of industries
to be able to choose their suppliers in an unconstrained manner.

%\begin{corollary}
%Suppose that there exists a finite, nonempty set $\mathcal{S}$ of industries
%for which Assumptions \ref{assumption1prime}, \ref{assumption

%logproductivity negative} and \ref{assumption leontief} hold and that can
%choose their sets of suppliers $S_{i}^{\ast }(t)\subset \varnothing \cup
%\{1,\ldots ,t\}$. The remaining industries can either not
%choose their suppliers or have productivity terms that satisfy $%
%\lim_{n\rightarrow \infty }\frac{\max_{i\in \{1,...,n\}}Z_{i}}{\log _{2}n}=0$%
%. Then%
%\begin{equation*}
%g^{\ast }=D\sum_{i,j=1}^{\infty }\beta _{i}\mathcal{L}_{ij}>0\text{ almost
%surely.}
%\end{equation*}
%\end{corollary}

\begin{corollary}
\label{corollary1} Suppose that there exists a finite, nonempty set $%
\mathcal{S}$ of industries for which Assumptions \ref{assumption1prime}, \ref%
{assumption2prime}, \ref{assumption markups} \ref{assumption logproductivity
negative} and \ref{assumption leontief} hold and that can choose their sets
of suppliers $S_{i}^{\ast }(t)\subset \{1,\ldots ,t\}$. The remaining
industries cannot choose their suppliers. Then%
\begin{equation*}
g^{\ast }=D\sum_{i=1}^{\infty }\sum_{j\in \mathcal{S}}\beta _{i}\mathcal{L}%
_{ij}>0\text{ almost surely.}
\end{equation*}
\end{corollary}

%This corollary thus shows that sustained (exponential) growth does not
%require the conditions of Theorem \ref{proposition growth} for all
%industries. Though the growth rate in this corollary has exactly the same
%form of that in Theorem \ref{proposition growth}, it will be generally quite
%different (smaller than) that rate, because the equilibrium Leontief
%inverse, $\mathcal{L}$, will be different (in fact, element-wise smaller).
Note that the growth rate has a similar expression to that in Theorem \ref%
{proposition growth}, but only considers the sub-block of the Leontief
inverse corresponding to industries in the set $\mathcal{S}$, since growth
is driven by these industries. Though productivity grows in other industries
as well because they use the inputs in $\mathcal{S}$, this growth is slower
and asymptotically dominated by the growth of the industries in $\mathcal{S}$%
.

A restrictive feature of Theorem \ref{proposition growth} is that only one
product arrives at each point in time. The next corollary shows that
sustained growth still emerges when the number of products existing at time $%
t$ can be an arbitrary function of $t$.

\begin{corollary}
\label{corollary2} Suppose that Assumptions \ref{assumption1prime}, \ref%
{assumption2prime}, \ref{assumption markups} and \ref{assumption leontief}
hold. Suppose also that the number of existing products at time $t$ is $n(t)$%
, and a modified version of Assumption \ref{assumption logproductivity
negative} holds where the distribution $\Phi _{i}(t)$ from which $a_{i}(t)$
are drawn satisfies $\lim_{t\rightarrow \infty }\max_{S\subset
\{1,...,n(t)\}}\frac{a_{i}(S(t))}{t}=D$ almost surely, then
\begin{equation*}
g^{\ast }=D\sum_{i,j=1}^{\infty }\beta _{i}\mathcal{L}_{ij}>0\text{ almost
surely.}
\end{equation*}
\end{corollary}

When $n(t)$ grows asymptotically faster (slower) than $t$, then the maximum
productivity term $\max_{S\subset \{1,...,n(t)\}}a_{i}(S)\}$ must grow at a
rate slower (faster) than $t$ to ensure sustained (exponential) growth.%
\footnote{%
For example, when $n(t)=t^{k}$ (with $k>1$), the maximum productivity term
needs to grow more slowly with $t$. This can be achieved, for instance, if $%
a_{i}(S_{i})=b_{i}(S_{i})^{\frac{1}{k}}$, where the $b_{i}(S_{i})$'s are
identically and independently distributed draws from a Gumbel distribution
with parameter $\sigma $. This implies $\lim_{t\rightarrow \infty
}\max_{S\subset \{1,...,t^{k}\}}\frac{b_{i}(S)}{t^{k}}=\sigma \log 2$ almost
surely, and thus $\lim_{t\rightarrow \infty }\max_{S\subset \{1,...,t^{k}\}}%
\frac{a_{i}(S)}{t}=(\sigma \log 2)^{\frac{1}{k}}$ almost surely, as required
in Corollary \ref{corollary2}.}

Finally, the next corollary relaxes both Assumption \ref{assumption
logproductivity negative} and the second part of Assumption \ref{assumption
leontief}, and shows that even though in this case we cannot be sure that
there exists a constant asymptotic growth rate, growth is uniformly bounded
between two constant rates, ensuring that the economy will still exhibit
sustained economic growth.

\begin{corollary}
\label{corollary bounds}Suppose that Assumptions \ref{assumption1prime}, \ref%
{assumption2prime}, \ref{assumption markups} and the first part of
Assumption \ref{assumption leontief} hold. Suppose also that a modified
version of Assumption \ref{assumption logproductivity negative} holds, where
$\lim \inf \max_{S_{i}\subset \{1,...,t\}}\frac{a_{i}(S_{i},t)}{t}=D_{1}>0$
and $\lim \sup \max_{S_{i}\subset \{1,...,t\}}\frac{a_{i}(S_{i},t)}{t}%
=D_{2}>0$ almost surely. Each industry again chooses its set of suppliers $%
S_{i}^{\ast }(t)\subset \{1,\ldots ,t\}$. Then%
\begin{equation*}
D_{1}\sum_{i,j=1}^{\infty }\beta _{i}\mathcal{L}_{ij}\leq g^{\ast }\leq
\frac{D_{2}}{1-\theta }\sum_{i,j=1}^{\infty }\beta _{i}\mathcal{L}_{ij}\text{
almost surely.}
\end{equation*}
\end{corollary}

Our next generalization in this subsection shows how the assumption of
Cobb-Douglas technologies can be relaxed. Specifically, we prove the
possibility of sustained growth with general constant returns to scale
production functions and Hicks-neutral technologies.\footnote{%
We explore the implications of Harrod-neutral technologies in Appendix B.}
For this result, recall that $k_{i}$ is the log unit cost function of
industry $i$ and no longer takes a Cobb-Douglas form.

\begin{theorem}
\label{theorem hicks neutral growth}\textbf{(Growth with general
technologies)} Suppose that all production functions are continuously
differentiable and feature Hicks-neutral technologies in the sense that for
each industry $i=1,2,\ldots ,t$, there exists a continuously differentiable
function $\overline{k}_{i}(S_{i},p)$ such that the log unit cost function
satisfies $k_{i}(S_{i},a_{i}(S_{i}),p)=-a_{i}(S_{i})+\overline{k}%
_{i}(S_{i},p)$. Suppose also that Assumptions \ref{assumption 1 second part}%
, \ref{assumption markups} and \ref{assumption logproductivity negative}
hold, and that there exists $\theta <1$ such that for all $i\in \{1,...,t\}$
and all $t\in \mathbb{N}$, $\sum_{j=1}^{t}\frac{d\log k_{i}}{d\log p_{j}}%
\leq \theta $. Then for each $i$ the equilibrium log price satisfies
\begin{equation*}
D\leq \lim \inf_{t}-\frac{p_{i}^{\ast }(t)}{t}\leq \lim \sup_{t}-\frac{%
p_{i}^{\ast }(t)}{t}\leq \frac{D}{1-\theta }.
\end{equation*}%
If in addition Assumption \ref{assumption2prime} holds, then the equilibrium
growth rate at time $t$, $g^{\ast }(t)=-\frac{1}{t}\sum_{i=1}^{t}\beta
_{i}(t)p_{i}^{\ast }(t)$, satisfies%
\begin{equation*}
D\leq \lim \inf_{t}g^{\ast }(t)\leq \lim \sup_{t}g^{\ast }(t)\leq \frac{D}{%
1-\theta }.
\end{equation*}
\end{theorem}

This theorem shows that Cobb-Douglas production functions are not essential
for our main growth result. Even though we cannot be sure that the economy
converges to a constant growth rate without this assumption, as in Corollary %
\ref{corollary bounds} there exist lower and upper bounds for the asymptotic
growth rate that are constant and are in terms of the same Leontief inverse
expression as in Theorem \ref{proposition growth}.

Yet another important generalization relaxes the assumption that new
products arrive exogenously. Though there are many different ways in which
endogenous creation of new products can be introduced in this framework, one
interesting and novel avenue is to explore whether the combination of new
inputs can lead to new products. Our assumption so far is that as an
industry adopts additional inputs, this can reduce its costs but does not
change the functionality or nature of the good being produced. In practice,
new inputs may not just reduce costs but also transform a product's use in
consumption or as an input significantly, transforming it into a new good.
For example, combining sensors, lidar, new hardware and advanced software
into cars can create a new type of good, autonomous vehicles. One way in
which this can be modeled in our setup is as follows. Suppose that there are
no new products arriving exogenously, but existing ones can be combined with
each other to create additional products. Suppose, in particular, that when
there are $n(t)$ goods at time $t$, society can generate $z(n(t))$ new
products. Because of limits on society's ability to undertake such
combinations at a point in time, we assume that the function $z$ is bounded
above by $\bar{z}<\infty $ (this is similar to what Weitzman, 1998, assumes
in his model of recombinant growth). This implies that asymptotically
society will generate $\bar{z}$ products per period, and thus a slight
variant of Theorem \ref{proposition growth} applies and generates a growth
rate of $g^{\ast }=\bar{z}D\sum_{i,j=1}^{\infty }\beta _{i}\mathcal{L}_{ij}$%
; notably, in this case there is no exogenous arrival of new products.

\subsection{Growth with Essential Inputs and when New Products Replace Old
Ones}

Our formulation so far imposes two assumptions that are not realistic.
First, it ignores the possibility that certain input classes may be
essential for the production of some types of goods. For example, some
metals need to be used for precision tools or agricultural products for food
manufacturing. Second, it does not allow for new inputs, or new input
combinations, to replace old ones, which is an important feature of some of
the examples of new input combinations we discussed in the Introduction
(e.g., electronic fuel injectors replacing carburetors). In this subsection,
we generalize our framework to accommodate both possibilities.

We first introduce a variant of our setup in which there may exist a set of
essential inputs for each industry. Specifically, suppose that there are $%
K<\infty $ categories. At each time $t$, one new good in each category
arrives, so the total number of goods after $t$ time periods is $tK$. The
categories partition the space of goods into $K$ sets $V_{1}(t),...,V_{K}(t)$
so that $\cup _{k=1}^{K}V_{k}(t)=\{1,...,tK\}.$ For each industry $i$, there
is a set $R_{i}\subset \{1,...,K\}$ of essential categories that the
industry needs to produce. Finally, each category $k$ has its own
productivity $A_{i,k}(S_{i,k})$ that depends on the subset $S_{i,k}\subset
V_{i,k}(t)$ of inputs from category $k$. This implies that industry $i$'s
production function now takes the form%
\begin{equation*}
Y_{i}=L_{i}^{1-\sum_{k\in R_{i}}\sum_{j\in S_{i,k}}\alpha
_{ij}}\prod_{k=1}^{K}A_{i,k}(S_{i,k})\prod_{j\in S_{i,k}}X_{ij}^{\alpha
_{ij}},
\end{equation*}%
where $\,S_{i,k^{\prime }}\neq \varnothing $ for each $k^{\prime }\in R_{i}$%
. The next result is a generalization of Theorem \ref{proposition growth} in
the presence of such restrictions on permissible input combinations.

\begin{theorem}
\label{theorem hierarchical}\textbf{(Growth with essential inputs) }Suppose
that Assumptions \ref{assumption1prime}, \ref{assumption2prime}, \ref%
{assumption markups} and \ref{assumption leontief} hold. Suppose also that a
modified version of Assumption \ref{assumption logproductivity negative}
holds where $\lim_{t\rightarrow \infty }\max_{S_{i,k}\subset V_{k(t)}}\frac{%
a_{i,k}(S_{i,k}(t))}{t}=D_{k}>0$ almost surely for each $k\in R_{i}$. Then
for each $i=1,2,\ldots ,t$, the equilibrium log price vector $p^{\ast }(t)$
satisfies%
\begin{equation*}
\lim_{t\rightarrow \infty }-\frac{p_{i}^{\ast }(t)}{t\sum_{k=1}^{K}\sum_{j%
\in V_{k}(t)}\mathcal{L}_{ij}D_{k}}=1>0\text{ almost surely},
\end{equation*}%
and thus%
\begin{equation*}
g^{\ast }=\sum_{i,j=1}^{\infty }\sum_{k=1}^{K}\sum_{j\in V_{k}}\beta _{i}%
\mathcal{L}_{ij}D_{k}>0\text{ almost surely}.
\end{equation*}
\end{theorem}

This result thus shows that various restrictions on combinations of inputs
can be imposed without impacting our main result concerning sustained growth.

The generalization introduced in this subsection also allows us to relax
some aspects of Assumptions \ref{assumption2prime} and \ref{assumption
leontief}. In particular, recall that the first of these imposes that the
consumption shares of new products satisfy $\lim_{t\rightarrow \infty }\beta
_{t}=0$, while the second implies that the cost shares of new inputs are
small. These assumptions therefore rule out a natural type of
\textquotedblleft creative destruction\textquotedblright\ where\ new
products replace older ones in either consumption or production or in both.
To incorporate this possibility, let us again partition the set of goods
into $K$ categories, $V_{1}(t),...,V_{K}(t)$ with $\cup
_{k=1}^{K}V_{k}(t)=\{1,...,tK\}$, but now with the crucial difference that
goods in the same category are more strongly substitutable than in Theorem %
\ref{theorem hierarchical} where at least one --- and possibly many --- of
the goods in the same category are used in production. Instead, we now
assume that in the categories $k=K^{\prime },\ldots ,K$ (where $K^{\prime
}>1 $) the production process uses \emph{only one} good as input from the
same category, while in the first $K^{\prime }$ categories there are no such
restrictions. This implies that goods in categories $V_{K^{\prime
}}(t),...,V_{K}(t)$ are competing against each other in consumption or in
the supply chain of an industry, and when a new one is introduced, it
replaces the previously used good from that category. We continue to impose
Assumptions \ref{assumption2prime} and \ref{assumption leontief} to the
first $K^{\prime }$ categories. This, in particular, implies that for $%
k=1,\ldots ,K^{\prime }$, $\lim_{j,t\rightarrow \infty ,j\in V_{k}(t)}\beta
_{j}=0$ and $\sum_{j,t\geq T,j\in V_{k}(t)}\alpha _{ij}\leq \epsilon $ for
all $i$. But these assumptions are now relaxed for the remaining $%
K-K^{\prime }$ categories. Instead, for those categories we have the
following: for any $j\in V_{k}(t)$ with $k\geq K^{\prime }$, $\beta
_{t}=\beta ^{k}$ and $\sum_{k>1,j\in V_{k}(t)}\alpha _{ij}\leq \bar{\theta}%
<\theta $ (where $\theta <1$ as specified in Assumption \ref{assumption
leontief}). The second part of this assumption implies that the cost share
of new inputs can be large (because they are replacing other inputs that, on
average, have large cost shares). Intuitively, this structure will ensure
that there are new combinations of inputs in the first $K^{\prime }$
categories, while the new inputs introduced in the remaining categories
generate a type of creative destruction, with new inputs or consumption
goods replacing old ones. Under these assumptions, we can establish the
following theorem.\footnote{%
We omit the proof of this theorem, since it is a small variation of the
proof of Theorem \ref{proposition growth}.}

\begin{theorem}
\label{theorem creative destruction}\textbf{(Growth with creative
destruction) }Suppose that there are $K$ categories of inputs $%
V_{1}(t),...,V_{K}(t)$ with $\cup _{k=1}^{K}V_{k}(t)=\{1,...,tK\}$. Suppose
that Assumptions \ref{assumption1prime} and \ref{assumption markups} hold,
and that Assumptions \ref{assumption2prime} and \ref{assumption leontief}
hold for the first $K^{\prime }$ categories (where $K^{\prime }\geq 1$)
while for the remaining $K-K^{\prime }$ categories, we have: for all $j\in
V_{k}(t)$ and $k\geq K^{\prime }$, $\beta _{t}=\beta ^{k}$ and $%
\sum_{k>1,j\in V_{k}(t)}\alpha _{ij}\leq \bar{\theta}<\theta $ (where $%
\theta <1$ as specified in Assumption \ref{assumption leontief}). In
addition, suppose the following version of Assumption \ref{assumption
logproductivity negative} holds for $k=1,\ldots ,K^{\prime }$: $%
\lim_{t\rightarrow \infty }\max_{S_{i,1}\subset V_{k}(t)}\frac{%
a_{i,1}(S_{i,k}(t))}{t}=D_{k}>0$ almost surely. Then%
\begin{equation*}
g^{\ast }=\sum_{i,j=1}^{\infty }\sum_{k=1}^{K^{\prime }}\sum_{j\in
V_{k}}\beta _{i}\mathcal{L}_{ij}D_{k}>0\text{ almost surely}.
\end{equation*}
\end{theorem}

This theorem thus establishes that sustained growth is possible in an
environment in which new inputs replace old ones (or new consumption goods
replace old ones) and can thus have significant shares in the budget of
consumers or an industry's value added (or costs). It also implies that new
input combinations may be associated with smaller intermediate shares in
value added.\footnote{%
A noteworthy observation is that in Theorem \ref{proposition growth}
industries will on average tend to add suppliers, though these will not
affect the intermediate share much after time $T$ (because of Assumption \ref%
{assumption leontief}). Here, instead, new input combinations may reduce the
intermediate share of value added.} Note also that even though asymptotic
growth is driven by the first $K^{\prime }$ categories, the introduction of
new inputs replacing old ones in the other categories also adds to
productivity growth both at the industry and the aggregate level.

\section{Cross-Sectional Implications\label{section cross-sectional
implications}}

In this section, we develop the cross-sectional implications of our model of
endogenous production networks. Our focus will be on a static economy with
large $n$, which will enable us to draw on some of the results developed in
the previous section.\footnote{%
Our cross-sectional results can also be developed in the context of a
growing economy as in the previous section. We focus on the static economy
for simplicity.} Throughout this section we impose Assumptions \ref%
{assumption1prime} and {\ref{assumption2prime}}, ensuring that all
production functions and preferences are Cobb-Douglas. We also impose a
variant of Assumption \ref{assumption logproductivity negative}, which
allows for log productivities to be correlated draws from a Gumbel
distribution. Under these assumptions, we first establish a closed-form
characterization of the probability of industry $j$ to be adopted as a
supplier to industry $i$. We then prove the main result of this section,
showing that under a stronger version of Assumption \ref{assumption leontief}
on the shape of the $\alpha _{ij}$ parameters, the distribution of indegrees
is concentrated (thus exhibiting limited inequality), while the distribution
of outdegrees is much more unequal. In other words, industries tend to be
similar in terms of how many inputs they use, but they are very different in
terms of how many other industries they supply. This contrast is in line
with the patterns visible from the US input-output tables (e.g., Acemoglu et
al., 2012).

\subsection{Closed-form Expressions for Edge Probabilities}

In the rest of this section, we work under the following modified version of
Assumption \ref{assumption logproductivity negative}.

\begin{assumptionbis}{assumption logproductivity negative}
Productivities are given by $a_{i}(S_{i})=\sum_{j\in S_{i}}b_{j}+\epsilon
(S_{i})$, where $\epsilon (S_{i})$ is an (independent) draw from a Gumbel
distribution with cdf $\Phi (x;\sigma )=e^{-e^{-x/\sigma }}$ for each $%
S_{i}\subset \{1,2,\ldots \}$. \label{assumption logit}
\end{assumptionbis}


This assumption allows the productivity of a set of inputs to depend on\ the
\textquotedblleft average\textquotedblright\ productivity of the inputs as
well as a random term drawn from a Gumbel distribution.\footnote{%
We show in Proposition \ref{proposition gumbel3} in Appendix B that if the $%
b_{j}$'s are independent random variables that satisfy $\Pr [b_{j}>-\sigma
\log 2]>\rho $ for some $\rho >0$, then Corollary \ref{corollary bounds}
applies and implies that there is sustained growth in the long run.} For our
analysis in this section, it is convenient to assume that the $b_{j}$'s are
given (so that we can condition on them without introducing additional
notation). Under Assumption \ref{assumption logit}, we can compute a
closed-form (generalized) logit expression for the probability that an edge $%
(i,j)$ is present in the production network.

\begin{lemma}
\label{lemma edge probabilities}\textbf{(Conditional edge probabilities) }%
Suppose that Assumptions \ref{assumption1prime} and \ref{assumption logit}
hold. Then:

\begin{enumerate}
\item Conditional on the price vector $P$, the probability of industry $j$
choosing $S_{i}$ as its set of suppliers is%
\begin{equation*}
\Pr (S_{i}|P)=\frac{e^{\sum_{j\in S_{i}}\frac{b_j -\alpha _{ij} p_{j}}{\sigma%
}}}{\sum_{S_{i}^{\prime }} e^{\sum_{j\in S^{\prime}_{i}}\frac{b_j -\alpha
_{ij} p_{j}}{\sigma}}}=\frac{\prod_{j\in S_{i}}e^{\frac{b_j}{\sigma}}
P_{j}^{-\frac{\alpha _{ij}}{\sigma }}}{Z_{i}}.
\end{equation*}

\item Conditional on the price vector $P$, the probability that industry $j$
is a supplier to industry $i$ is%
\begin{equation*}
\Pr (j\in S_{i}|P)=\frac{e^{\frac{b_j}{\sigma}} P_{j}^{-\frac{\alpha _{ij}}{%
\sigma }}}{1+e^{\frac{b_j}{\sigma}}P_{j}^{-\frac{\alpha _{ij}}{\sigma }}}.
\end{equation*}
\end{enumerate}
\end{lemma}

Lemma \ref{lemma edge probabilities} also yields a simple (generalized)
logit equation for the expected outdegree --- or number of customers --- of
industry $j$:%
\begin{equation}
\sum_{i\in \mathcal{N}}\Pr (j\in S_{i}|P)=\sum_{i\in \mathcal{N}}\frac{e^{%
\frac{b_{j}}{\sigma }}P_{j}^{-\frac{\alpha _{ij}}{\sigma }}}{1+e^{\frac{b_{j}%
}{\sigma }}P_{j}^{-\frac{\alpha _{ij}}{\sigma }}},
\label{equation outdegree}
\end{equation}%
which we will use in the rest of the section.

\subsection{The Distribution of Indegrees and Outdegrees in Large Networks}

We now proceed to characterize the the distribution of indegrees and
outdegrees of large networks. Let $\{\mathcal{E}(n)\}_{n=1}^{\infty }$ be a
sequence of economies where $\mathcal{E}(n)$ has $n$ industries, and let $%
S(n)$ be the equilibrium network in economy $\mathcal{E}(n)$. Let $\mathcal{I%
}_{i}(n)=\frac{1}{n}\sum_{j=1}^{n}\alpha _{ij}(S(n))$ be the (normalized)
indegree of industry $i$ in economy $\mathcal{E}(n)$ (meaning that it is
normalized by the number of industries in the economy, $n$), and let $%
\mathcal{I}(n)=\{\mathcal{I}_{i}(n)\}_{i=1}^{n}$ be the sequence of
(normalized) indegrees. Analogously, let $\mathcal{O}_{j}(n)=\frac{1}{n}%
\sum_{i=1}^{n}\alpha _{ij}(S(n))$ be the (normalized) outdegree of industry $%
j$ and let $\mathcal{O}(n)=\{\mathcal{O}_{j}(n)\}_{j=1}^{n}$ be the sequence
of (normalized) outdegrees.\footnote{%
To simplify the terminology, we refer to $\mathcal{I}(n)$ and $\mathcal{O}%
(n) $ as sequences of indegrees and outdegrees, rather than normalized
indegrees and normalized outdegrees. Clearly, indegrees and outdegrees can
be obtained by multiplying $\mathcal{I}(n)$ and $\mathcal{O}(n)$ by $n$.}
Both $\mathcal{I}(n)$ and $\mathcal{O}(n)$ are random variables over $%
\mathbb{R}^{n}$, where randomness comes from the fact that $%
\{a_{i}(S_{i})\}_{i,S_{i}}$ is a sequence of random variables. Furthermore,
for every $i=1,2,\ldots ,n$, we have $\mathcal{I}_{i}(n),\mathcal{O}(n)\leq
1 $, so $\mathcal{I}(n)$ and $\mathcal{O}(n)$ can be interpreted as elements
of $\ell ^{\infty }$ (with $\mathcal{I}_{i}(n)=\mathcal{O}_{i}(n)=0$ for all
$i>n$).

The main result in this section, established in Theorem \ref{theorem 8}, is
that the distribution of indegrees $\mathcal{I}(n)$ converges uniformly to
the sequence $(0,0,0,...)\in \ell ^{\infty }$ almost surely, while the
limsup and liminf of the sequence $\mathcal{O}(n)$ of outdegrees converge to
non-degenerate distributions over $\ell ^{\infty }$, which together imply
that $\mathcal{O}(n)$ cannot converge to a non-degenerate distribution. To
prove convergence in the first part of the theorem, we introduce the
following strengthening of Assumption \ref{assumption leontief}.

\begin{assumptionbis}{assumption leontief}
Suppose that Assumption \ref{assumption leontief} holds. In addition, for every
industry $j$, the limit $\lim_{n\rightarrow \infty }\frac{1}{n}%
\sum_{i=1}^{n}\alpha _{ij}$ of average exogenous outdegrees always exists. %
\label{assumption further Leontief}
\end{assumptionbis}

In what follows, we use the notation $\alpha _{j}=\lim_{n\rightarrow \infty }%
\frac{1}{n}\sum_{i=1}^{n}\alpha _{ij}$ and $\alpha =\{\alpha _{j}\}_{j\in
\mathbb{N}}$.

\begin{theorem}
\label{theorem 8}\textbf{(Indegrees and outdegrees)} Suppose Assumptions \ref%
{assumption1prime}, \ref{assumption logit}, and \ref{assumption further
Leontief} hold. Then:

\begin{enumerate}
\item $\mathcal{I}(n)$ converges uniformly and almost surely to a degenerate
distribution at $0\in \ell^{\infty}$.

\item $\bar{\mathcal{O}}=\lim \sup_{n\rightarrow \infty }\mathcal{O}(n)$ is
a non-degenerate distribution and $\bar{\mathcal{O}}_{j}\leq \alpha _{j}$
for all $j$. %and the convergence to the limit superior is uniform.

\item $\underline{\mathcal{O}}=\lim \inf_{n\rightarrow \infty }\mathcal{O}%
(n) $ is a non-degenerate distribution and $\underline{\mathcal{O}}_{j}\geq
\alpha _{j}\frac{e^{b_{j}}}{1+e^{b_{j}}}$ for all $j$.
%and the convergence to the limit inferior is uniform.
\end{enumerate}
\end{theorem}

Theorem \ref{theorem 8} establishes that the distribution of outdegrees will
be much more unequal than the distribution of indegrees. This is consistent
with the properties of the US input-output tables, for example, as
documented in Acemoglu et al. (2012), who show that the distribution of
outdegrees has an approximate power law distribution (or Pareto tail). The
next result is a direct corollary of this theorem and shows that if the
distribution of $\alpha _{ij}$'s can be approximated by a power law
distribution, then so can the distribution of outdegrees. For this result,
we utilize a simplified version of the definition of power law distribution
used in Acemoglu et al. (2012).

\begin{corollary}
\label{corollary pareto} Suppose in addition that $\alpha _{ij}$'s have a
power law distribution in the sense that $\alpha _{j}$'s in Assumption \ref%
{assumption further Leontief} satisfy $\alpha _{j}=Bj^{-\delta }h(j)$, where
$\delta >1$, $h(j)$ is a function satisfying $\lim_{x\rightarrow \infty
}h(x)x^{\nu }=\infty $ and $\lim_{x\rightarrow \infty }h(x)x^{-\nu }=0$ for
all $\nu >0$, and $B>0$ is such that $\sum_{j=1}^{n}\alpha _{j}<1$. Then $%
\mathcal{O}_{j} $ has a power law distribution. In particular,%
\begin{equation*}
Bj^{-\delta }h(j)\frac{e^{b_{j}}}{1+e^{b_{j}}}\leq \underline{\mathcal{O}}%
_{j}\leq \bar{\mathcal{O}_{j}}\leq Bj^{-\delta }h(j).
\end{equation*}
\end{corollary}

\section{The Contribution of New Input Combinations to Growth\label{section
counterfactual}}

In this section, we take a first step to estimate the contribution from new
input combinations to industry productivity growth.

\subsection{Data Description}

The NBER-CES manufacturing database provides data on value added, employment
and capital stock for 459 manufacturing industries (identified by their 1987
SIC code) for the 1958-2011 period. This dataset also includes estimates of
total factor productivity (TFP). We combine these with data for 36
non-manufacturing industries from the Bureau of Economic Analysis (BEA) for
1987-2016. The BEA also provides detailed input-output tables every five
years during the 1972-2007 period. Our main sample centers on 1987-2007 when
we have all three of these data sources available. We use the harmonized
input-output tables from Acemoglu, Autor and Patterson (2017), aggregated to
1987 SIC codes for manufacturing industries, and NIPA categories for
non-manufacturing industries. We merge these data with estimates of TFP
growth from the NBER-CES database and the BEA. The resulting dataset
contains 452 manufacturing industries and 36 non-manufacturing industry for
the years 1987, 1992, 1997, 2002 and 2007.\footnote{%
Figure 1 in the Introduction uses BEA's harmonized summary tables for 1963-1997, which
are available for 61 industries. Though summary tables are also available
for 1948-1962 and 1997-2016, in Figure 1 we focused on 1963-1997\ to avoid
the changes in industry definitions in 1963 and 1997. Also in Appendix D we
use the BEA 2007 input-output tables directly rather than the harmonized
tables (since harmonization changes the sparsity of the input-output matrix,
which is important for our exercise in Appendix D).}

\subsection{Estimating Industry Productivity Growth From New Input
Combinations}

In this subsection, we develop an illustrative estimate of the contribution
of new input combinations to industry productivity and thus aggregate TFP
growth. We use the structure outlined in Theorem \ref{proposition growth},
which links productivity gains to new input combinations.

One complication is that changes in input combinations result from two
distinct sources --- reductions in prices of existing inputs encouraging
their adoption and new inputs providing new input combinations with
significantly higher productivity. We are interested in the latter type of
change, which is the source of sustained growth in Theorem \ref{proposition
growth}. Though in the data it is impossible to distinguish precisely
between these two sources of changes in input combinations, we can do so
approximately based on the following observation. Price-induced changes will
typically involve the addition of one or a few inputs in a given time
period. In contrast, when a sector adopts a \textquotedblleft truly new
input\textquotedblright\ --- that is, an input that was previously not
available to it --- this will be associated with a large rearrangement of
its input structure (if input productivities were identically and
independently distributed, we would expect the sector in question to change,
on average, half of its inputs). Motivated by this reasoning, we focus on
\textquotedblleft large\textquotedblright\ changes in input combinations.
More specifically, for every industry $i$ and time $t$, we first compute the
Jaccard distance of sets of suppliers between $t$ and $t-1$,%
\begin{equation*}
J_{i}(t)=\frac{|S_{i}(t)\cup S_{i}(t-1)|-|S_{i}(t)\cap S_{i}(t-1)|}{%
|S_{i}(t)\cup S_{i}(t-1)|},
\end{equation*}%
which is a simple measure of the relative change in the number of suppliers.%
\footnote{%
Relative to the Hamming distance, $|S_{i}(t)\cup S_{i}(t-1)|-|S_{i}(t)\cap
S_{i}(t-1)|$, this measure does not give greater weight to industries that
have a larger number of suppliers.} We then code a dummy for this measure
being above the $20^{th}$ percentile of its distribution in that year across
all industries. This dummy, denoted by $J_{i,20}(t)$, is a proxy for
significant change in input structure (and in Appendix C we show similar
results with different definitions). Using this proxy, we estimate the
regression model%
\begin{equation}
\Delta a_{i}(t)=\gamma \Delta J_{i,20}(t)+\nu _{i}+\eta (t)+\epsilon _{i}(t)
\label{regression equation}
\end{equation}%
on our five-yearly panel between 1987 and 2007 with 488 industries. Here $%
\Delta a_{i}(t)$ is the five-year change in (log) TFP; $\eta (t)$ denotes a
full set of time effects, capturing any common component to industry
productivity growth; $\nu _{i}$ denotes a full set of industry dummies,
which allow for industry-specific linear trends in productivity capturing
the influence of other factors leading to differential productivity growth
across industries; and finally,$\ \epsilon _{i}(t)$ is an error term
representing all other influences. Intuitively, this regression estimates
the extent to which industries undergoing significant changes in their input
structure are experiencing more rapid TFP growth. %

\begin{table}[h]
\label{table 2}\centering%
\input{../ReplicationFiles/Tables/big_table_tfp.tex}
\caption*{\textbf{Table 1: New input combinations and TFP. }The table presents OLS estimates of the regression equation $\Delta \log
TFP_{i}(t)=\beta J_{i,20}(t)+\gamma _{i}+\nu (t)+\epsilon _{i}(t)$
using a dataset of five-year stacked-differences for 488 industries between
1987 and 2007. $J_{i,20}(t)$ is a dummy indicating the Jaccard distance
between the sets of inputs $S_{i}(t)$ and $S_{i}(t-1)$ being above the $%
20^{th}$ percentile of its distribution in that year. Column 1 only includes
period dummies. Column 2 adds industry-specific linear trends, the $\gamma
_{i}$'s. Column 3 adds lagged change in log TFP, $\Delta \log TFP_{i}(t-1)$.
Panel A is for the entire sample. Panel B focuses on manufacturing
industries and Panel C excludes computer industries (those within the three-digit SIC industries 357 and 367 SIC 357 or
367). Standard errors that are robust against arbitrary heterosedasticity
and serial correlation at the level of industry are reported in parentheses.}
\end{table}

The regression results are reported in Table 1. Panel A is for all
industries, while Panel B focuses on the manufacturing sector. Panel C drops
computers and related sectors (three-digit SIC codes 357 and 367), which
have experienced the fastest productivity growth during this time period;
this is most likely for reasons that are unrelated to our mechanism and thus
we would like to ensure that our results are not driven by the computer
sector. The first column in all three panels includes only time period
dummies (and thus no industry-specific linear trends). The second column
adds industry-specific linear trends, which take out any systematic
differences in productivity growth across industries that are likely to be
unrelated to our mechanism. The third column also adds the lagged industry
TFP growth, $\Delta a_{i}(t-1)$, to capture any dynamics in sectoral TFP.
Throughout, the standard errors are robust against arbitrary
heteroscedasticity and serial correlation at the level of industries.

In all columns, we estimate a positive and statistically significant
association between our dummy for significant change in input combinations
and industry productivity growth. For example, the parameter estimate in
column 1 Panel A is 0.018 (standard error = 0.007). It becomes a little
larger when we include linear trends by industry and lagged TFP on the
right-hand side.

We next use the coefficient estimates from Table 1 to get an illustrative
estimate of the contribution of new input combinations to productivity
growth. Namely, we compute counterfactual industry productivity growth
driven entirely by $J_{i,20}(t)$ in (\ref{regression equation}). These
(counterfactual) productivity gains from new input combinations are reported
at the bottom of each panel and are quite sizable. The estimate from column
1 in Panel A, for instance, implies that without the productivity gains from
(significant) new input combinations, average productivity growth would have
been lower by 0.42 percentage points or by about 40\% (relative to the
annualized average industry TFP growth of 1.05\% over this time period). The
estimates in column 2, which incorporate differential trends in
productivity growth across industries, are broadly similar (e.g., a 0.48 
percentage point contribution from new input combinations in Panel A). When we allow for
serially-correlated TFP dynamics by controlling for lagged TFP on the right-hand side, 
the estimates and implied magnitudes are significantly larger 
and suggest an even more important role for new input combinations.

In Appendix C, we report several more robustness checks. First, we show
results using dummy variables with 10\% and 30\% cutoffs, $J_{i,10}(t)$ and $%
J_{i,30}(t)$, with very similar results. We also present weighted
regressions using value added of an industry in 1987 as weights. Finally, we
report an analogous specification using data only from 1997-2007 period in
order to remove any effect arising from the transition form SIC to NAICS
codes in 1997. The results are again broadly similar, though less precise in
some specifications in Panel C.

Overall, this exercise suggests that productivity gains from new input
combinations could be quite large. Nevertheless, our estimates should be
read as illustrative for at least two reasons. First, they rely on the
structure of our model, which is simplified in many dimensions. Second, the
coefficient estimates in Table 1 may be upwardly biased and thus exaggerate
the contribution of new input combinations to productivity growth if sectors
that increase their productivity for other reasons nonetheless end up
increasing the range of inputs they use (for example, exogenous innovations
may encourage the use of new inputs, even if these are not crucial for the
productivity growth that these innovations bring). A more systematic
analysis of the implications and productivity contributions of changing
input-output linkages is beyond the scope of the current paper and is an
area for future work.

\section{Conclusion\label{section conclusion}}

How production is organized differs markedly between countries and over
time. For example, the input-output linkages of the US economy have become
denser over the last 50 years and richer and more productive countries
appear to have denser production networks. We develop a tractable model of
endogenous production networks to provide a conceptual framework for
understanding these patterns and how differences in distortions or
technologies can translate into variation in production networks.

In our model, each product can be produced by combining labor and an
endogenous subset of the other products as inputs. Combinations of inputs
generate different constant returns to scale production functions with
(prespecified) levels of productivity. There may also be distortions
affecting different industries due to taxes, regulations, contracting
frictions or markups. Using this setup, we establish the existence and
uniqueness of an equilibrium with an endogenous production network, and
explore its efficiency properties. We then use our framework to clarify
several new economic tradeoffs and comparative statics that arise in the
context of endogenous production networks. Namely:

\begin{itemize}
\item when a product adopts additional inputs to minimize its costs, this
not only reduces its price, but (weakly) reduces all prices in the economy.
This \textquotedblleft complementarity\textquotedblright\ is a consequence
of the fact that this product has now become a cheaper input to all other
industries;

\item under a reasonable assumption that ensures that lower prices do not
discourage technology adoption, a change in technology that makes the
adoption of additional inputs more productive for one industry --- or a
reduction in distortions in one industry --- expands technology sets for all
industries. This second dimension of complementarity is a new feature of
environments with endogenous production networks;

\item the technology comparative statics mentioned in the previous bullet
point are potentially \textquotedblleft discontinuous\textquotedblright\ in
the sense that a small change for a single industry can cause large changes
in GDP or trigger a chain reaction, leading to major shifts in the
production structure of many industries.
\end{itemize}

The second part of the paper uses a dynamic version of our framework to
study the growth implications of endogenous production networks. Our main
result from this analysis is that the selection of input suppliers and the
indirect effects that this creates on the equilibrium structure of the
production network emerge as powerful forces for sustained economic growth.
The origin of sustained growth in our model is related to, but different
from, Weitzman's (1998) idea of recombinant growth. When a new product
arrives, it becomes a potential input for all existing products, and
significantly expands the number of input combinations (production
techniques) available to other industries. Namely, when there are $n$
products, the arrival of one more new product increases the combinations of
inputs that each existing product can use from $2^{n-1}$ to $2^{n}$, and
thus enables nontrivial cost reductions from the choice of optimal
technology combinations. A first impetus for growth comes from this expanded
set of techniques to which firms have access. Growth in our economy is not
driven by this first impetus alone, however. It is undergirded by the fact
that the adoption of a new production technique reduces the price of the
relevant product, encouraging other industries to adopt this product as an
additional input and change their production techniques.

We view our paper as a first step in the analysis of endogenous formation of
production networks. First, our analysis was greatly simplified by the
contestability assumption and the \textquotedblleft no
substitution\textquotedblright\ property that this implied. The study of
firm-level production networks will necessitate a framework that
incorporates a more realistic market structure, relationship-specific
investments and bargaining between firms and their potential suppliers.
Second, incorporating stochastic elements and failures of input suppliers or
customers on productivity and sourcing decisions is another interesting and
challenging area for future research. Third, another topic for future
research is a more in-depth structural exercise estimating the parameters
regulating the endogenous evolution of production networks, which can then
be used for counterfactual analysis to understand over-time and
cross-country differences in the organization of production. Beyond
firm-level datasets with information on flows of intermediate goods,
detailed bilateral international trade flows would be another empirical
domain where a similar approach could be developed. Finally, our empirical
exercise on the contribution of the changing input-output structure to
economic growth was illustrative. Further exploration of the theoretical and
empirical linkages between the evolution of the production network and
long-run economic growth is another promising area.

\section*{Appendix A: Omitted Proofs From the Text}

\setcounter{equation}{0} \setcounter{lemma}{0} \setcounter{proposition}{0} %
\renewcommand{\theequation}{A\arabic{equation}} \renewcommand{\thelemma}{A%
\arabic{lemma}} \renewcommand{\theproposition}{A\arabic{proposition}}

\begin{proof}[Proof of Lemma \protect\ref{lem:fixedPoint}]
First, suppose that $P^{\ast }$ is a vector of equilibrium prices. Then from
the contestability condition of Definition \ref{definition of equilibrium}, (%
\ref{price equals marginal cost}), we have $P_{i}^{\ast }=(1+\mu
_{i})K_{i}(S_{i},A_{i},P^{\ast })$ for each $i=1,2,\ldots ,n$.

To prove the other direction, suppose that $P_{i}^{\ast }=(1+\mu
_{i})K_{i}(S_{i},A_{i},P^{\ast })$ for each $i=1,2,\ldots ,n$. We show that $%
P^{\ast }$ supports a unique equilibrium.

Let $X_{i}^{\ast }$ and $L_{i}^{\ast }$ be the solutions to the cost
minimization problem of a representative firm in industry $i$, (\ref{firm
optimization}). For given $P^{\ast }$, let $x_{ij}^{\ast }$ denote the units
of good $j$ used for producing one unit of good $i$. Similarly let $%
l_{i}^{\ast }$ be the unit labor requirement of good $i$. In view of
Assumption \ref{assumption 1}, these objects are uniquely defined (because
of strict quasi-concavity) and are independent of the equilibrium output of
this industy, $Y_{i}^{\ast }$ (because of constant returns to scale), but
depend on the price of vector $P^{\ast }$. Clearly, $X_{ij}^{\ast
}=x_{ij}^{\ast }Y_{i}^{\ast }$ and $L_{i}^{\ast }=l_{i}^{\ast }Y_{i}^{\ast }$%
.

Let $Y^{N}=1+\sum_{i=1}^{n}\lambda _{i}\frac{\mu _{i}}{1+\mu _{i}}%
P_{i}^{\ast }Y_{i}^{\ast }$ denote the income of the representative
household and $C_{j}^{\ast }=C_{j}^{\ast }(Y^{N},P^{\ast })$ be its optimal
consumption of good $j$ at prices $P^{\ast }$ and income $Y^{N}$.

The market clearing condition $C_{j}^{\ast }+\sum_{i=1}^{n}X_{ij}^{\ast
}=Y_{j}^{\ast }$ then implies%
\begin{equation*}
C_{j}^{\ast }+\sum_{i=1}^{n}x_{ij}^{\ast }Y_{i}^{\ast }=Y_{j}^{\ast }.
\end{equation*}%
Multiplying this equation by $P_{j}^{\ast }$, we obtain
\begin{equation*}
P_{j}^{\ast }C_{j}^{\ast }+\sum_{i=1}^{n}P_{j}^{\ast }x_{ij}^{\ast
}Y_{i}^{\ast }=P_{j}^{\ast }Y_{j}^{\ast },
\end{equation*}%
or
\begin{equation*}
\hat{C}_{j}+\sum_{i=1}^{n}\frac{P_{j}^{\ast }x_{ij}^{\ast }}{P_{i}^{\ast }}%
\hat{Y}_{i}^{\ast }=\hat{Y}_{j}^{\ast },
\end{equation*}%
where \textquotedblleft \symbol{94}\textquotedblright\ denotes a nominal
variable. Let $\bar{X}$ be a matrix whose $(i,j)^{th}$ component is $\bar{X}%
_{ij}=\frac{P_{j}^{\ast }x_{ij}^{\ast }}{P_{i}^{\ast }}$. Combining these
equations across industries, the vector of nominal outputs, $\hat{Y}$, is a
solution to the following fixed point equation
\begin{equation}
\hat{Y}=\Phi (\hat{Y})=\hat{C}(\hat{Y},P)+\bar{X}^{\prime }\hat{Y},
\label{fixed point equation}
\end{equation}%
where the second equality defines the matrix function $\Phi (\hat{Y})$. We
next prove that, given $P^{\ast }$, (\ref{fixed point equation}) \ has a
unique fixed point, from which all equilibrium quantities can be derived
uniquely.

First note that since the utility function is differentiable (cfr.
Assumption \ref{assumption 1 second part}), $C(\hat{Y},P)$ and thus $\Phi (%
\hat{Y})$ is differentiable. Denote the Jacobian of $\Phi (\hat{Y})$ by $J$
and its $(i,j)$ entry by $J_{i,j}=\frac{\partial {\hat{C}_{i}}}{\partial
\hat{Y}_{j}}+\frac{P_{i}^{\ast }x_{ji}}{P_{j}^{\ast }}\geq 0$ (since all
goods are normal from Assumption \ref{assumption 1 second part}). We now
show that $\Vert J\Vert _{1}=\max_{j}\sum_{i=1}^{n}J_{i,j}<1$. To see this,
note that the representative household's budget constraint implies $%
\sum_{i=1}^{n}\hat{C}_{i}=1+\sum_{i=1}^{n}\lambda_i \frac{\mu _{i}}{1+\mu
_{i}}\hat{Y}_{i}$. Differentiating this expression with respect to $\hat{Y}%
_{j}$, we obtain
\begin{equation}
\sum_{i=1}^{n}\frac{\partial \hat{C}_{i}}{\partial \hat{Y}_{j}}=\lambda_j
\frac{\mu _{j}}{1+\mu _{j}}.  \label{first relationship}
\end{equation}%
Then rearranging (\ref{price equals marginal cost}) and using the fact that
labor is essential from Assumption \ref{assumption 1}, we have
\begin{equation}
\sum_{i=1}^{n}\frac{P_{i}^{\ast }x_{ji}^{\ast }}{P_{j}^{\ast }}<\frac{%
K_{j}(S_{j},A_{j}(S_{j}),P^{\ast })}{P_{j}^{\ast }}=\frac{1}{1+\mu _{j}}.
\label{second lemma relationship}
\end{equation}%
Adding up (\ref{first relationship}) and (\ref{second lemma relationship}),
we obtain%
\begin{equation*}
\sum_{i=1}^{n}J_{i,j}=\sum_{i=1}^{n}\frac{\partial {\hat{C}_{i}}}{\partial
\hat{Y}_{j}}+\frac{P_{i}^{\ast }x_{ji}^{\ast }}{P_{j}^{\ast }}<\lambda_j
\frac{\mu _{j}}{1+\mu _{j}}+\frac{1}{1+\mu _{j}} \leq 1\text{ for all }j%
\text{.}
\end{equation*}%
Since this holds for all columns $j$ of matrix $J$, $\Vert J\Vert _{1}<1$.
Using the definition of matrix norm, we have that, for any $\hat{Y},\hat{Y}%
^{\prime }$,
\begin{equation}
\Vert \Phi (\hat{Y})-\Phi (\hat{Y}^{\prime })\Vert _{1}\leq \Vert J\Vert
_{1}\Vert \hat{Y}-\hat{Y}^{\prime }\Vert _{1}.
\label{key lemma relationship}
\end{equation}%
Since $\Vert J\Vert _{1} < 1$, (\ref{key lemma relationship}) implies that $%
\Phi (\hat{Y})$ is a contraction, and thus given price vector $P^{\ast }$,
there exists a unique fixed point $\hat{Y}^{\ast }$ of $\Phi $. Furthermore,
all equilibrium quantities can be determined from this fixed point as: $%
Y_{i}^{\ast }=\frac{\hat{Y}_{i}(P^{\ast })}{P_{i}^{\ast }}$, $C_{i}^{\ast }=%
\frac{\hat{C}_{i}(\hat{Y}(P^{\ast }),P^{\ast })}{P_{i}^{\ast }} $, $%
X_{ij}^{\ast }=x_{ij}^{\ast }Y_{i}^{\ast }$ and $L_{i}^{\ast }=l_{i}^{\ast
}Y_{i}^{\ast }$ for all $i=1,2,\ldots ,n$. This completes the proof that
given a production network $S$, a price vector $P^{\ast }$ that satisfies (%
\ref{price equals marginal cost}) uniquely define an equilibrium.
\end{proof}
\newline

\begin{proof}[Proof of Theorem \protect\ref{existence theorem}]
Let $\kappa (P)=((1+\mu_1)
\min_{S_{1}}K_{1}(S_{1},A_{1}(S_{1}),P),...,(1+\mu_n)
\min_{S_{n}}K_{n}(S_{n},A_{n}(S_{n}),P))$. We first show that $\kappa $ has
a fixed point, and then show that this corresponds to an equilibrium. To do
this, we prove the following lemma as an intermediate step.

\begin{lemma}
\label{lemma aux1} Let $\mathbb{L}=\{P\geq 0:P_{i}=(1+\mu_i)
\min_{S_{i}}K_{i}(S_{i},A_{i}(S_{i}),P)\}$. Then $\mathbb{L}$ is a non-empty
complete lattice with respect to the operations $P\wedge Q=(\min
(P_{1},Q_{1}),...,\min (P_{n},Q_{n})),P\vee Q=(\max (P_{1},Q_{1}),...,\max
(P_{n},Q_{n})).$
\end{lemma}

\begin{proof}[Proof of Lemma \protect\ref{lemma aux1}]
Let $\mathbb{O}=\{(x_{1},...,x_{n}):x_{i}\geq 0\}$, and then by definition $%
\kappa :\mathbb{O}\rightarrow \mathbb{O}$. We will first show that there is
a subset $\tilde{\mathbb{O}}\subset \mathbb{O}$ which is a complete lattice
with respect to $\wedge $ and $\vee $, and then establish that that $\kappa
(P)$ is increasing in $P$ and maps $\tilde{\mathbb{O}}$ to $\tilde{\mathbb{O}%
}$. The result that $\mathbb{L}$ is a complete lattice follows from these
two steps.

To establish the first step, note that for any $i$, we can produce good $i$
using only labor and incur a cost $\overline{P_{i}}=(1+\mu_i)
K_{i}(\varnothing ,A_{i}(\varnothing ),\{P_{j}\}_{j\in \varnothing })$ that
does not depend on the price vector $P$. Thus, we have $\kappa (P)\leq (%
\overline{P_1},...,\overline{P_n})$ for all price vectors $P$. Since labor
is essential in production, we have $(1+\mu_i) K_{i}(S_{i},A_{i}(S_{i}),0)>0$
for every set $S_{i}$. Define $\underline{P_{i}}= \kappa_i(0) = (1+\mu_i)
\min_{S_{i}}K_{i}(S_{i},A_{i}(S_{i}),0)$. Since $K_{i} $ is increasing in
price, we have $\kappa (P)\geq \kappa(0) = (\underline{P_{1}},...,\underline{%
P_{n}})$ for every price vector $P$. Then $\tilde{\mathbb{O}}=\times
_{i=1}^{n}[\underline{P_{i}},\overline{P_{i}}]$ is a complete lattice, and $%
\kappa $ maps $\tilde{\mathbb{O}}$ to $\tilde{\mathbb{O}}$.

The second step is immediate from the definition of $\kappa (P)$. If $%
P^{\prime }\leq P$, then for any $i$ and $S_{i}$, we have $(1+\mu_i)
K_{i}(S_{i},A_{i}(S_{i}),P^{\prime })\leq (1+\mu_i)
K_{i}(S_{i},A_{i}(S_{i}),P)$. Taking minima on both sides, we get $%
(1+\mu_i)\min_{S_{i}}K_{i}(S_{i},A_{i}(S_{i}),P)\leq (1+\mu_i)
\min_{S_{i}}K_{i}(S_{i},A_{i}(S_{i}),P^{\prime })$, so $\kappa (P^{\prime
})\leq \kappa (P)$. We conclude from Tarski's fixed point theorem that $%
\mathbb{L}$ is a non-empty complete lattice.
\end{proof}
\newline

Since $\mathbb{L}$ is a non-empty complete lattice, $\kappa $ has a fixed
point, and in fact, a smallest fixed point. Take this smallest fixed point,
which simultaneously satisfies $P_{i}^{\ast }=(1+\mu _{i})K_{i}(S_{i}^{\ast
},A_{i}(S_{i}^{\ast }),P^{\ast })$ and $S_{i}^{\ast }\in \arg
\min_{S_{i}}(1+\mu _{i})K_{i}(S_{i},A_{i}(S_{i}),P^{\ast })$. That is, given
$P^{\ast }$, technology choice $S_{i}^{\ast }$ is optimal, and given $%
S^{\ast }$, firms minimize costs. Then with the same argument as in Lemma %
\ref{lem:fixedPoint}, there exist unique equilibrium quantities $X^{\ast
},L^{\ast }$ and $C^{\ast }$, and thus $(P^{\ast },S^{\ast },C^{\ast
},L^{\ast },X^{\ast },Y^{\ast })$ is an equilibrium.
\end{proof}
\newline

\begin{proof}[Proof of Theorem \protect\ref{thm:unique}]
Let $P^{\ast }$ be the minimal element of lattice $\mathbb{L}$ defined in
the proof of Theorem \ref{existence theorem}, which is of course an
equilibrium price vector. If $P^{\ast \ast }$ is another equilibrium price
vector, it must be contained in $\mathbb{L}$ and therefore satisfy $P^{\ast
\ast }>P^{\ast }$. We now derive a contradiction to $P^{\ast \ast }>P^{\ast
} $.

First, note that for each $i=1,2,\ldots ,n$, the unit cost function $%
K_{i}(S_{i},A_{i}(S_{i}),P)$ is concave in prices given $S_{i}$. Since the
minimum of a collection of concave functions is concave, $\kappa
_{i}(P)=(1+\mu _{i})\min_{S_{i}}K_{i}(S_{i},A_{i}(S_{i}),P)$ is also concave.

Then, let $\nu \in (0,1)$ be such that $\nu P^{\ast \ast }\leq P^{\ast }$,
with at least some $r=1,2,\ldots ,n$ such that $\nu P_{r}^{\ast \ast
}=P_{r}^{\ast }$. We have%
\begin{eqnarray*}
\kappa _{r}(P^{\ast })-P_{r}^{\ast } &\geq &\kappa _{r}(\nu P^{\ast \ast
})-\nu P_{r}^{\ast \ast } \\
&\geq &(1-\nu )\kappa _{r}(0)+\nu \kappa _{r}(P^{\ast \ast })-\nu
P_{r}^{\ast \ast } \\
&\geq &(1-\nu )\kappa _{r}(0) \\
&>&0,
\end{eqnarray*}%
where the first line follows because $\kappa _{r}$ is nondecreasing, $\nu
P^{\ast \ast }\leq P^{\ast }$, and $\nu P_{r}^{\ast \ast }=P_{r}^{\ast }$.
The second line follows from the concavity of $\kappa _{r}$. The third line
simply uses the fact that $P^{\ast \ast }$ is a fixed point, i.e., $\kappa
_{r}(P^{\ast \ast })=P_{r}^{\ast \ast }$. Finally, the last inequality
follows because labor is essential by Assumption \ref{assumption 1}, which
implies $\kappa _{r}(0)>0$. But this contradicts the hypothesis that $%
P^{\ast }$ is a fixed point. This contradiction establishes the uniqueness
of equilibrium prices, and then the uniqueness of equilibrium allocations
follows from Lemma \ref{lem:fixedPoint}.\footnote{%
This part of the proof builds on Kennan's (2001) proof of uniqueness of
fixed point for a concave function.}

To prove that the equilibrium network is generically unique, let $S^{\ast
}\neq S^{\ast \ast }$ be two arbitrary networks and let $\mathcal{A}%
(S,S^{\ast \ast })=\{A:S^{\ast }$ and $S^{\ast \ast }\text{ are both
equilibrium networks}\}$. Note that we can write $\mathcal{A}$ as the
countable union $\cup _{S^{\ast },S^{\ast \ast }}\mathcal{A}(S^{\ast
},S^{\ast \ast })$. Thus, if we prove that $\mathcal{A}(S^{\ast },S^{\ast
\ast })$ has measure zero, then we can conclude that $\mathcal{A}$ has
measure zero. Define%
\begin{equation*}
\Delta _{i}(S^{\ast },S^{\ast \ast },A)=(1+\mu _{i})K_{i}(S_{i}^{\ast
},A_{i}(S_{i}^{\ast }),P^{\ast })-(1+\mu _{i})K_{i}(S_{i}^{\ast \ast
},A_{i}(S_{i}^{\ast \ast }),P^{\ast }),
\end{equation*}%
and note that for all parameters $A\in \mathcal{A}(S^{\ast },S^{\ast \ast })$
and each $i\in \{1,...,n\}$, we have $\Delta _{i}(S^{\ast },S^{\ast \ast
},A)=0$.

Because $S^{\ast }\neq S^{\ast \ast }$, there is at least one industry $i$
such that $S_{i}^{\ast }\neq S_{i}^{\ast \ast }$. Recall also that the cost
function $K_{i}(S_{i},A_{i}(S_{i}),P)$ is continuous and strictly decreasing
in $A_{i}(S_{i})\in \mathbb{R}^{\ell }$. Let $A_{i,-S_{i}^{\ast
}}=\{A_{i}(S_{i})\}_{S_{i}\neq S_{i}^{\ast }}$ be the vector of all
technology terms for sets different than $S_{i}^{\ast }$ and let $%
A_{i,-1}(S_{i}^{\ast })=\{A_{i,2}(S_{i}^{\ast }),...,A_{i,\ell }(S_{i}^{\ast
})\}$ be the vector of all components of $A_{i}(S_{i}^{\ast })$ \emph{except
for} the first component $A_{i,1}(S_{i}^{\ast })$. If we keep $%
A_{i,-S_{i}^{\ast }}$ and $A_{i,-1}(S_{i}^{\ast })$ constant, then $\Delta
_{i}(S_{i}^{\ast },S_{i}^{\ast \ast },A)$ is a continuous and strictly
decreasing function of one real variable $A_{i,1}(S_{i}^{\ast })$. This
implies that, for any fixed $A_{i,-S_{i}^{\ast }},A_{i,-1}(S_{i}^{\ast })$,
there exists a unique value of $A_{i,1}(S_{i}^{\ast })$ that satisfies $%
\Delta _{i}(S_{i}^{\ast },S_{i}^{\ast \ast },A)=0$. Hence, $\mathcal{A}%
(S^{\ast },S^{\ast \ast })=\{A:\Delta _{i}(S_{i}^{\ast },S_{i}^{\ast \ast
},A)=0$ for each $i\}$ has measure zero in\ $\mathbb{R}^{n\times \ell \times
2^{n-1}}$, which implies that the equilibrium network is generically unique.
When the equilibrium network is unique so are equilibrium quantities, $%
C^{\ast }$, $L^{\ast }$, $X^{\ast }$ and $Y^{\ast }$.
\end{proof}
\newline

\begin{proof}[Proof of Theorem \protect\ref{theorem efficiency}]
First, for given a given production network $S$, the Pareto efficient
allocation is a solution to the following program:%
\begin{eqnarray}
U(S) &=&\max_{C,X,L}u(C_{1},\ldots ,C_{n})
\label{welfare maximization problem} \\
&&\text{subject to}  \notag \\
&&\sum_{i=1}^{n}L_{i}\leq 1  \notag \\
&&\sum_{i=1}^{n}X_{ij}+C_{j}\leq F_{j}(S_{j},A(S_{j}),L_{j},X_{j})\text{ for
}j=1,\ldots ,n.  \notag
\end{eqnarray}%
This is a concave, differentiable maximization problem (with an nonempty
interior of the constraint set), so the Karush-Kuhn-Tucker (KKT) Theorem
applies (e.g., Bertsekas, Nedic and Ozdaglar, 2003), and implies that $%
(C^{E},X^{E},L^{E})$ is a solution if and only if there exists a vector of
multipliers $(\chi _{0},\chi _{1},\ldots ,\chi _{n})>0$ such that
\begin{equation}
\frac{\partial u_{j}/\partial C_{j}}{\partial u_{i}/\partial C_{i}}=\frac{%
\chi _{j}}{\chi _{i}}\text{ for any }i,j\text{,}
\label{efficient first-order}
\end{equation}%
and also when $S_{i}\neq \varnothing $,%
\begin{equation}
\frac{\partial F_{i}}{\partial X_{ij}}=\frac{\chi _{j}}{\chi _{i}}\text{ for
any }i\text{ and }j\in S_{i},  \label{efficient first-order second}
\end{equation}%
and%
\begin{equation}
\frac{\partial F_{i}}{\partial L_{i}}=\frac{\chi _{0}}{\chi _{i}}.
\label{efficient first-order third}
\end{equation}%
Finally, any Pareto efficient production network satisfies $S^{E}\in \arg
\max_{S}U(S)$, where $U(S)$ is defined by (\ref{welfare maximization problem}%
).

Note also that (\ref{efficient first-order second}) and (\ref{efficient
first-order third}) are the necessary and sufficient first-order conditions
for cost minimization taking the Lagrange multipliers as prices, and thus
using the definition of the unit cost function in (\ref{firm optimization})
and also choosing the same numeraire as in our equilibrium analysis, which
implies $\chi _{0}=1$, we have%
\begin{equation}
\chi _{j}=K_{j}(S_{j}^{E},A_{j}(S_{j}^{E}),\chi
)=L_{j}^{E}+\sum_{i=1}^{n}\chi _{i}X_{ji}^{E},  \label{efficient unit cost}
\end{equation}%
where $\chi =(\chi _{1},\ldots ,\chi _{n})$.

The equilibrium $(P^{\ast },S^{\ast },C^{\ast },L^{\ast },X^{\ast },Y^{\ast
})$, on the other hand, satisfies%
\begin{equation}
\frac{\partial u_{j}/\partial C_{j}}{\partial u_{i}/\partial C_{i}}=\frac{%
P_{j}^{\ast }}{P_{i}^{\ast }}\text{ for any }i,j\text{,}
\label{equilibrium first-order}
\end{equation}%
and also when $S_{i}\neq \varnothing $,%
\begin{equation}
\frac{\partial F_{i}}{\partial X_{ij}}=\frac{P_{j}^{\ast }}{P_{i}^{\ast }}%
\text{ for any }i\text{ and }j\in S_{i},
\label{efficient first-order second}
\end{equation}%
and%
\begin{equation}
\frac{\partial F_{i}}{\partial L_{i}}=\frac{1}{P_{i}^{\ast }}.
\label{efficient first-order third}
\end{equation}%
and (\ref{price equals marginal cost}) for any $j$ with $K_{j}(S_{j}^{\ast
},A_{j}(S_{j}^{\ast }),P^{\ast })$ as given in (\ref{firm optimization}). In
addition, we have the market clearing constraint given by part four of
Definition \ref{definition of equilibrium}. We now prove the claims in the
theorem.

\begin{enumerate}
\item Suppose that $\mu _{i}=0$ for all $i=1,2,\ldots ,n$ and take the
equilibrium production network $S^{\ast }$ as given. Set $\chi _{0}=1$ and $%
\chi _{j}=P_{j}^{\ast }=K_{j}(S_{j}^{\ast },A_{j}(S_{j}^{\ast }),P^{\ast })$%
. This verifies that given $S^{\ast }$, the equilibrium allocation in this
case is Pareto efficient. Moreover, since $S_{j}^{\ast }\in \arg
\min_{S_{j}} $ $K_{j}(S_{j},A_{j}(S_{j}),P^{\ast })$, $S^{\ast }$ also
maximizes $U(S)$, and thus the equilibrium with no distortions is Pareto
efficient.

\item Suppose that $S^{E}=(\varnothing ,\ldots ,\varnothing )$, $\mu
_{1}=...=\mu _{n}=\mu _{0}>0$ and $\lambda _{1}=\ldots =\lambda _{n}=1$.
Since there are no input-output linkages and production functions exhibit
constant returns to scale from Assumption \ref{assumption 1}, they are all
linear in labor, and thus we no longer have conditions (\ref{efficient
first-order second}) and (\ref{efficient first-order third}) in this case.
Since $\lambda _{i}=1$ for all $i$, the market clearing condition in
Definition \ref{definition of equilibrium} coincides with the resource
constrained in (\ref{welfare maximization problem}). Next note that the unit
cost of industry $i$ both in the efficient allocation and in equilibrium is $%
K_{i}=B_{i}$ for some $B_{i}>0$. This implies that equilibrium prices are $%
P_{i}^{\ast }=(1+\mu _{0})B_{i}$ for each $i$. Therefore, in this case any
allocation that satisfies (\ref{efficient first-order}) also satisfies (\ref%
{equilibrium first-order}), and the KKT Theorem implies that the equilibrium
is Pareto efficient.

\item Now $(\varnothing ,\ldots ,\varnothing )$ is no longer Pareto
efficient. If $S^{\ast }=(\varnothing ,\ldots ,\varnothing )$, the
equilibrium is necessarily Pareto inefficient. Hence suppose that $S^{\ast
}=S^{E}\neq (\varnothing ,\ldots ,\varnothing )$, and without loss of any
generality, suppose $S_{i}^{\ast }\neq \varnothing $. We again have $\mu
_{1}=\mu _{2}=...=\mu _{n}=\mu _{0}>0$. Suppose to obtain a contradiction
that given $S^{\ast }$, the equilibrium allocation is Pareto efficient. From
(\ref{price equals marginal cost}) for industry $j$, $P_{j}^{\ast }=(1+\mu
_{0})K_{j}(S_{j}^{\ast },A_{j}(S_{j}^{\ast }),P^{\ast })$. To satisfy (\ref%
{efficient first-order}) and (\ref{equilibrium first-order}), we need $\chi
_{i}=\psi P_{i}^{\ast }$ for some constant $\psi >0$. From (\ref{efficient
first-order third}) Pareto efficiency then requires%
\begin{equation*}
\frac{1}{\frac{P_{i}^{\ast }}{1+\mu _{0}}}=\frac{\partial F_{i}}{L_{i}}%
|_{L_{i}=L^{\ast }}=\frac{\partial F_{i}}{\partial L_{i}}|_{L_{i}=L^{E}}=%
\frac{1}{\chi _{i}}.
\end{equation*}%
This implies $\chi _{i}=\frac{P_{i}^{\ast }}{1+\mu _{0}}$ and thus $\psi =%
\frac{1}{1+\mu _{0}}$. Next recall that $\chi
_{i}=K_{i}(S_{i}^{E},A_{i}(S_{i}^{E}),\chi )=\sum_{j\in S_{i}^{E}}\chi
_{j}X_{ij}^{E}+L^{E}$, which implies%
\begin{equation*}
K_{i}(S_{i}^{\ast },A_{i}(S_{i}^{\ast }),\psi P^{\ast })=\psi P_{i}^{\ast }=%
\frac{1}{1+\mu _{0}}P_{i}^{\ast }=K_{i}(S_{i}^{\ast },A_{i}(S_{i}^{\ast
}),P^{\ast }),
\end{equation*}%
and yields a contradiction since $\psi P^{\ast }<P^{\ast }$ and the cost
function is strictly increasing in the price vector (view of the fact that $%
S_{i}\neq \varnothing $).

%Since $W=1$ (as the
%numeraire), the cost minimization in (\ref{firm optimization}) implies%
%\begin{equation*}
%\frac{\partial F_{1}/\partial X_{12}}{\partial F_{1}/\partial L_{1}}%
%=P_{2}^{\ast }=(1+\mu _{0})K_{2}(S_{2}^{\ast },A_{2}(S_{2}^{\ast }),P^{\ast
%}).
%\end{equation*}%
%But since in the Pareto efficient allocation,
%\begin{equation*}
%\frac{\partial F_{1}/\partial X_{12}}{\partial F_{1}/\partial L_{1}}=\chi
%_{2}=K_{2}(S_{2}^{E},A_{2}(S_{2}^{E}),\chi ),
%\end{equation*}%
%and by hypothesis $S^{E}=S^{\ast }$ and $\chi =P^{\ast }$, we arrive at a
%contradiction and establish Pareto inefficiency in this case.

\item Suppose first that there exist $i,i^{\prime }$ such that $\mu _{i}\neq
\mu _{i}^{\prime }$, and suppose again that $S^{E}=S^{\ast }$ (otherwise we
are done). To simultaneously satisfy (\ref{efficient first-order third}) and
(\ref{equilibrium first-order}), we need $\frac{P_{i}^{\ast }}{P_{i^{\prime
}}^{\ast }}=\frac{\chi _{i}}{\chi _{i^{\prime }}}$. But from (\ref{efficient
unit cost}), we have%
\begin{equation*}
\frac{\chi _{i}}{\chi _{i^{\prime }}}=\frac{%
K_{i}(S_{i}^{E},A_{i}(S_{i}^{E}),\chi )}{K_{i^{\prime }}(S_{i^{\prime
}}^{E},A_{i^{\prime }}(S_{i^{\prime }}^{E}),\chi )},
\end{equation*}%
and from (\ref{price equals marginal cost}),
\begin{equation*}
\frac{P_{i}^{\ast }}{P_{i^{\prime }}}=\frac{(1+\mu _{i})K_{i}(S_{i}^{\ast
},A_{i}(S_{i}^{\ast }),P^{\ast })}{(1+\mu _{i^{\prime }})K_{i^{\prime
}}(S_{i^{\prime }}^{\ast },A_{i^{\prime }}(S_{i^{\prime }}^{\ast }),P^{\ast
})}.
\end{equation*}%
Since the hypothesis $\mu _{i}\neq \mu _{i}^{\prime }$, $S^{E}=S^{\ast }$
and again $\chi =\psi P^{\ast }$ (for $\psi >0$), the previous two
expressions yield a contradiction and imply that there exists no vector of
multipliers that can satisfy the KKT Theorem in the equilibrium allocation,
establishing Pareto inefficiency.

Finally, suppose there exists an industry $i$ such that $(1-\lambda _{i})\mu
_{i}>0$, then inefficiency follows from a simple contradiction argument.
Suppose the equilibrium were inefficient. Then
\begin{equation*}
Y_{i}^{E}=C_{i}^{E}+\sum_{j=1}^{n}X_{ji}^{E}=C_{i}^{\ast
}+\sum_{j=1}^{n}X_{ji}^{\ast }=(1-(1-\lambda _{i})\frac{\mu _{i}}{1+\mu _{i}}%
)Y_{i}^{\ast }<Y_{i}^{E},
\end{equation*}%
where the third equality uses the market clearing condition from Definition %
\ref{definition of equilibrium}. This contradiction completes the proof of
the theorem.
\end{enumerate}
\end{proof}

\begin{proof}[Proof of Lemma \protect\ref{lemma higher marginal returns}]
Let $i=1,2,\ldots ,n$, and let $S_{i}^{\prime }\supset S_{i},A_{i}^{\prime
}\geq A_{i}$. Let $\mathcal{X}=(S_{i},A_{i}^{\prime }),\mathcal{Y}%
=(S_{i}^{\prime },A_{i})$ and use the product lattice ordering so that $%
\mathcal{X}\vee \mathcal{Y}=(S_{i}^{\prime },A_{i}^{\prime }),\mathcal{X}%
\wedge \mathcal{Y}=(S_{i},A_{i})$. Suppose that $K_{i}(S_{i}^{\prime
},A_{i}(S_{i}^{\prime }),P)-K_{i}(S_{i},A_{i}(S_{i}),P)\leq 0$. In our
lattice notation, this can be written as $K_{i}(\mathcal{Y})\leq K_{i}(%
\mathcal{X}\wedge \mathcal{Y})$. The quasi-submodularity of $K_{i}$ implies
that $K_{i}(\mathcal{X}\vee \mathcal{Y})\leq K_{i}(\mathcal{X})$ which is
the same as writing $K_{i}(S_{i}^{\prime },A_{i}^{\prime }(S_{i}^{\prime
}),P)-K_{i}(S_{i},A_{i}^{\prime }(S_{i}),P)\leq 0.$ Thus, we conclude that
\begin{equation*}
K_{i}(S_{i}^{\prime },A_{i}(S_{i}^{\prime
}),P)-K_{i}(S_{i},A_{i}(S_{i}),P)\leq 0\implies K_{i}(S_{i}^{\prime
},A_{i}^{\prime }(S_{i}^{\prime }),P)-K_{i}(S_{i},A_{i}^{\prime
}(S_{i}),P)\leq 0.
\end{equation*}
\end{proof}
\newline

\begin{proof}[Proof of Theorem \protect\ref{cs1}]
Let $P^{0}=P^{\ast }$ and $S^{0}=S^{\ast }$ be the initial vector of
equilibrium prices and equilibrium network. Note that $P^{0}$ satisfies the
fixed point conditions $P_{i}^{0}=(1+\mu
_{i})\min_{S_{i}}K_{i}(S_{i},A(S_{i}),P^{0})\text{ for all }i$. Suppose that
$A_{i}(\cdot )$ increases to $A_{i}^{\prime }(\cdot )$, and define $P^{1}$
so that $P_{i}^{1}=(1+\mu _{i})\min_{S_{i}}K_{i}(S_{i},A_{i}^{\prime
}(S_{i}),P^{0}).$ Since $K_{i}$ is decreasing in $A_{i}$, we have $%
P_{i}^{1}=(1+\mu _{i})\min_{S_{i}}K_{i}(S_{i},A_{i}^{\prime
}(S_{i}),P^{0})\leq (1+\mu
_{i})\min_{S_{i}}K_{i}(S_{i},A_{i}(S_{i}),P^{0})=P_{i}^{0}$, establishing
that $P^{1}\leq P^{0}$.

As in the proof of Theorem \ref{existence theorem}, define $\kappa_i
(P)=(1+\mu_i)\min_{S_{i}}K_{i}(S_{i},A_{i}^{\prime }(S_{i}),P)$. The
equilibrium price $P^{\ast \ast }$ under the new productivity function $%
A_{i}^{\prime }$ is the minimal fixed point of $\kappa$. For $t\geq 1$,
define $P^{t}=\kappa (P^{t-1})$ and note that, since $\kappa $ is increasing
in $P$ and $P^{1}\leq P^{0}$, we have $\lim_{t\rightarrow \infty }P^{t} \leq
P^{1} \leq P^{0} = P^{\ast}$. Furthermore, since $\kappa$ is continuous, $%
\lim_{t\rightarrow \infty }P^{t}$ is a fixed point of $\kappa$. Since $%
P^{\ast \ast}$ is the minimal fixed point, we must have $P^{\ast \ast} \leq
\lim_{t\rightarrow \infty }P^{t} \leq P^0 = P^{\ast}$.
\end{proof}
\newline

\begin{proof}[Proof of Theorem \protect\ref{theorem network increasing}]
Let $S^{0}=S^{\ast }$ be the initial equilibrium network. Note that $S^{0}$
satisfies the fixed point conditions $S_{i}^{0}=\arg \min_{S_{i}}(1+\mu
_{i})K_{i}(S_{i},A(S_{i}),P^{\ast })\text{ for all }i$. Suppose that the
shift from $A_{i}(\cdot )$ to $A_{i}^{\prime }(\cdot )$ is a positive shock,
and define $S^{1}$ such that $S_{i}^{1}\in \arg \min_{S_{i}}(1+\mu
_{i})K_{i}(S_{i},A_{i}^{\prime }(S_{i}),P^{\ast }).$ Using the definition of
positive technology shock, we can apply Theorem 4 in Milgrom and Shannon
(1994) to infer that $S_{i}^{0}\subset S_{i}^{1}$.

As in the proof of Theorem \ref{existence theorem}, define $\kappa
(P)=(1+\mu_i) \min_{S_{i}}K_{i}(S_{i},A_{i}^{\prime }(S_{i}),P)$. Let $P^0 =
P^{\ast}$ and define $P^{t}=\kappa (P^{t-1})$ for $t \geq 1$. From the proof
of Theorem \ref{cs1}, we know that $P^{t}$ is a decreasing sequence with $%
P^{\ast \ast} \leq \lim_{t\rightarrow \infty }P^{t} \leq P^{\ast}$. Since $%
P^{\ast \ast} \leq P^{\ast}$, we apply once more Theorem 4 of Milgrom and
Shannon (1994) to obtain $S_{i}^{\ast \ast }=\arg \min_{S_{i}}
(1+\mu_i)K_{i}(S_{i},A_{i}^{\prime }(S_{i}),P^{\ast \ast })\supset \arg
\min_{S_{i}} (1+\mu_i)K_{i}(S_{i},A_{i}^{\prime
}(S_{i}),P^{\ast})=S_{i}^{1}\supset S_{i}^{0}=S_{i}^{\ast }$. We conclude
that $S^{\ast }\subset S^{\ast \ast }$.
\end{proof}
\newline

\begin{proof}[Proof of Lemma \protect\ref{lemma growth rate}]
Rewrite real GDP as $Y(t)=\frac{Y^{N}(t)}{e^{\pi (t)}}$, where $%
Y^{N(t)}=\sum_{i=1}^{t}P_{i}C_{i}=1+\sum_{i=1}^{t}\lambda_i \frac{\mu _{i}}{%
1+\mu _{i}}P_{i}Y_{i}$ . We next show that $\log Y^{N}(t)=$ $o(t)$, which
then implies that $\lim_{t\rightarrow \infty }\frac{\log Y(t)}{t}%
=-\lim_{t\rightarrow \infty }\frac{ \pi (t)}{t}$ as claimed.

Define the components of the production and cost functions that do not
depend on $A_{i}(S_{i}(t))$ as%
\begin{eqnarray*}
\overline{F}_{i}(X_{i}(t),L_{i}(t),S_{i}(t)) &=&\frac{1}{(1-\sum_{j\in
S_{i}(t)}\alpha _{ij})^{1-\sum_{j\in S_{i}(t)}\alpha _{ij}}\prod_{j\in
S_{i}(t)}\alpha _{ij}^{\alpha _{ij}}}L_{i}(t)^{1-\sum_{j\in S_{i}(t)}\alpha
_{ij}}\prod_{j\in S_{i}(t)}X_{ij}(t)^{\alpha _{ij}} \\
\overline{K}_{i}(S_{i}(t),P(t)) &=&\prod_{j\in S_{i}(t)}P_{j}(t)^{\alpha
_{ij}}.
\end{eqnarray*}%
Therefore, $P_{i}(t)Y_{i}(t)=(1+\mu _{i})\overline{F}%
_{i}(X_{i}(t),L_{i}(t),S_{i}(t))\overline{K}_{i}(S_{i}(t),P(t))$ which does
not depend on $A_{i}(S_{i}(t))$.

Let us then define%
\begin{equation*}
\overline{P}_{i}=(1+\mu _{i})\overline{K}_{i}(\varnothing ,\cdot ).
\end{equation*}%
By the definition of $\overline{K}$, we have that $\overline{K}%
_{i}(\varnothing ,\cdot )=(1+\mu _{i})\prod_{j\in \varnothing }P_{j}^{\alpha
_{ij}}=1+\mu _{i}$. We showed in Theorem \ref{existence theorem} that $%
P(t)\leq \overline{P}$. Furthermore, by Assumption \ref{assumption markups},
we have $\sup_{i\in \mathbb{N}}\overline{P_{i}}=\sup_{i\in \mathbb{N}}(1+\mu
_{i})\leq (1+\mu _{0})$ for some $\mu _{0}\geq 0$. Thus, $P_{i}(t)\leq 1+\mu
_{0}$ for all $t$ and all $i\leq t$.

Next let $\bar{Y}_{i}(S_{i}(t))$ be defined by the following system of
equations:%
\begin{eqnarray*}
\bar{Y}_{i}(S_{i}(t)) &=&\bar{F}_{i}(\{X_{ij}=\bar{Y}_{j}(S_{j})\}_{j\in
S_{i}(t)},L_{i}=1,S_{i}(t)) \\
&=&B_{i}(t)\prod_{j\in S_{i}(t)}\bar{Y}_{j}(S_{j}(t))^{\alpha _{ij}},
\end{eqnarray*}%
where $B_{i}(t)=\frac{1}{(1-\sum_{j\in S_{i}(t)}\alpha _{ij})^{1-\sum_{j\in
S_{i}(t)}\alpha _{ij}}\prod_{j\in S_{i}(t)}\alpha _{ij}^{\alpha _{ij}}}$.
Clearly, the vector $\bar{Y}(S(t))=(\bar{Y}_{1}(S_{1}(t)),\ldots ,\bar{Y}%
_{t}S_{t}(t))$ is an upper bound on the vector of sectoral outputs, and thus
$\bar{Y}^{N}(t)=1+\sum_{i=1}^{t}\lambda _{i}\frac{\mu _{i}}{1+\mu _{i}}\bar{P%
}_{i}\bar{Y}_{i}(S_{i}(t))$ is an upper bound on nominal GDP, $Y^{N}(t)$.
Next taking logarithms, we have%
\begin{equation*}
\bar{y}(S(t))=\alpha (S(t))\bar{y}(S(t))+b(t),
\end{equation*}%
where $\alpha (S(t))$ is the input-output matrix for the production network $%
S(t)$, $\bar{y}(S(t))=(\log \bar{Y}_{1}(S(t)),\ldots ,\log \bar{Y}%
_{t}(S(t)))^{\prime }$ and $b(t)=(\log B_{1}(t),\ldots ,\log
B_{t}(t))^{\prime }$. Thus%
\begin{equation*}
\bar{y}(S(t))=[I-\alpha (S(t))]^{-1}b(t).
\end{equation*}%
In view of Assumption \ref{assumption leontief}, the norm of the matrix $%
[I-\alpha (S(t))]^{-1}$ is less than $\frac{1}{1-\theta }$, and thus for all
$S(t)$, we have%
\begin{equation*}
\bar{y}(S(t))\leq \frac{1}{1-\theta }b(t)\text{ for all }S(t).
\end{equation*}%
Moreover, $b_{i}(t)=-\sum_{j\in S_{i}(t)}\alpha _{ij}\log \alpha
_{ij}-(1-\sum_{j\in S_{i}(t)}\alpha _{ij})\log (1-\sum_{j\in S_{i}(t)}\alpha
_{ij})$ can be interpreted as the entropy of a discrete random variable over
$\{0,1,...,|S_{i}|\}$ that is equal to $\alpha _{ij}$ with probability $%
\alpha _{ij}$ and equal to 0 with probability $1-\sum_{j\in S_{i}}\alpha
_{ij}$. The maximum possible entropy of this random variable is $\log
(|S_{i}|+1)\leq \log (t)$, obtained when $\alpha _{ij}=\frac{1}{|S_{i}|+1}$.
Thus $\bar{y}_{i}(S(t))\leq \frac{1}{1-\theta }\log (t)$, and hence $\bar{Y}%
_{i}(S(t))\leq t^{\frac{1}{1-\theta }}$ for all $S(t)$. Then $%
Y^{N}(t)=1+\sum_{i=1}^{t}\lambda _{i}\frac{\mu _{i}}{1+\mu _{i}}%
P_{i}Y_{i}\leq 1+\sum_{i=1}^{t}\lambda _{i}\frac{\mu _{i}}{1+\mu _{i}}\bar{P}%
_{i}\bar{Y}_{i}\leq 1+t(1+\mu _{0})t^{\frac{1}{1-\theta }}=1+(1+\mu _{0})t^{%
\frac{1}{1-\theta }+1}$. Taking logarithms, we obtain
\begin{equation*}
\log ({Y}^{N}(t))\leq \log (1+(1+\mu _{0})t^{\frac{1}{1-\theta }+1}),
\end{equation*}%
and hence%
\begin{equation*}
\lim_{t\rightarrow \infty }\frac{\log ({Y}^{N}(t))}{t}\leq
\lim_{t\rightarrow \infty }\frac{\log (1+(1+\mu _{0})t^{\frac{1}{1-\theta }%
+1})}{t}=0.
\end{equation*}%
This establishes that $\log Y^{N}(t)=o(t)$, and completes the proof.
\end{proof}
\newline

\begin{proof}[Proof of Theorem \protect\ref{proposition growth}]
Let $\epsilon >0$ and $T(\epsilon )$ be such that for all $i\in \mathbb{N}$,
$\sum_{j=T(\epsilon )}^{\infty }\alpha _{ij}\leq \epsilon $. Recall that $%
\alpha $ is the entire matrix of input-output elasticities, while $\alpha
(S) $ is the observed matrix of input-output elasticities when the
input-output network is given by $S$. Assumption \ref{assumption leontief}
tells us that if $S_{i}\supset \{1,...,T(\epsilon )\}$ for all $i$, we will
have $\sum_{j=1}^{t}\alpha _{ij}(S)\geq \sum_{j=1}^{t}\alpha _{ij}-\epsilon $%
.

We next make use of the following lemma:

\begin{lemma}
\label{lemma leontief} Let $\alpha $ and $\beta $ be non-negative matrices $%
n\times n$ matrices. Let $A=(I-\alpha )^{-1}$ and $B=(I-\beta )^{-1}$. If

\begin{itemize}
\item $\Vert \alpha \Vert_{\infty} \leq \theta , \Vert\beta\Vert_{\infty}
\leq \theta$ for some $\theta < 1$, and

\item $\sum_{j=1}^{n}\beta _{ij}\geq (\sum_{j=1}^{n}\alpha _{ij})-\epsilon $
for every row $i$,
\end{itemize}

then $\sum_{j=1}^{n}B_{ij}\geq (\sum_{j=1}^{n}A_{ij})-\frac{1}{(1-\theta
)^{2}}\epsilon $ for every row $i$.
\end{lemma}

\begin{proof}[Proof of Lemma \protect\ref{lemma leontief}]
Let $\alpha _{ij}^{\ell }$ be the $(i,j)$ element of the matrix $\alpha
^{\ell }$. Since $A=\sum_{\ell =0}^{\infty }\alpha ^{\ell },B=\sum_{\ell
=0}^{\infty }\beta ^{\ell }$ and $\sum_{\ell =1}^{\infty }\ell \theta ^{\ell
-1}=\frac{1}{(1-\theta )^{2}}$, it suffices to show that, for all $\ell \geq
0$ we have $\sum_{j=1}^{n}\beta _{ij}^{\ell }\geq (\sum_{j=1}^{n}\alpha
_{ij}^{\ell })-\ell \theta ^{\ell -1}\epsilon $. We proceed by induction.
The base case ($\ell =1$) is our assumption that $\sum_{j=1}^{n}\beta
_{ij}\geq (\sum_{j=1}^{n}\alpha _{ij})-\epsilon $. To prove the inductive
case, assume we have shown the hypothesis for $\ell $, and we want to show
it for $\ell +1$. Write
\begin{equation*}
\sum_{j}\beta _{ij}^{\ell +1}=\sum_{j}\sum_{k}\beta _{ik}\beta _{kj}^{\ell
}=\sum_{k}\beta _{ik}\sum_{j}\beta _{kj}^{\ell }.
\end{equation*}%
By induction, this is greater than or equal to
\begin{equation*}
\sum_{k}\beta _{ik}\sum_{j}\alpha _{kj}^{\ell }-\sum_{k}\beta _{ik}\ell
\theta ^{\ell -1}\epsilon .
\end{equation*}%
We now use the fact that $\sum_{k}\alpha _{ik}-\epsilon \leq \sum_{k}\beta
_{ik}\leq \theta $ to infer that $\sum_{k}\beta _{ik}\sum_{j}\alpha
_{kj}^{\ell }-\sum_{k}\beta _{ik}\ell \theta ^{\ell -1}\epsilon$ is bounded
below by
\begin{equation*}
\sum_{k}\alpha _{ik}\sum_{j}\alpha _{kj}^{\ell }-\epsilon \sum_{j}\alpha
_{kj}^{\ell }-\theta \ell \theta ^{\ell -1}\epsilon .
\end{equation*}%
The first term in the above expression is equal to $\sum_{j}\sum_{k}\alpha
_{ik}\alpha _{kj}^{\ell }=\sum_{j}\alpha _{ij}^{\ell +1}$. The second term
is bounded below by $-\epsilon \Vert \alpha ^{\ell }\Vert _{\infty }\geq
-\epsilon \theta ^{\ell }$. We conclude that
\begin{equation*}
\sum_{j}\beta _{ij}^{\ell +1}\geq \sum_{j}\alpha _{ij}^{\ell +1}-(\ell
+1)\theta ^{\ell }\epsilon .
\end{equation*}%
Adding up over all $\ell \in \mathbb{N}$, we obtain
\begin{equation*}
\sum_{j}B_{ij}\geq \sum_{j}A_{ij}-\frac{1}{(1-\theta )^{2}}\epsilon
\end{equation*}
\end{proof}
\newline

From this lemma, we can infer that for any $S \supset \{1,...,T(\epsilon)\}$%
, $\sum_{j=1}^{t}\mathcal{L}_{ij}(S)\geq \sum_{j=1}^{t}\mathcal{L}_{ij}-%
\frac{1}{(1-\theta )^{2}}\epsilon $, which we will use in the proof that
follows.

We first prove that $\liminf_{t\rightarrow \infty }-\frac{p_{i}^{\ast }(t)}{%
t\sum_{j=1}^{t}\mathcal{L}_{ij}}\geq D$. We will first show that this is the
case even if industry $i$ chooses a suboptimal set of inputs corresponding
to those with the highest levels of log productivity (rather than the
cost-minimizing bundles), and then infer from this that it is also true for
the equilibrium price sequence. Let us define $S_{i}^{\max }(t)=\arg
\max_{S_{i}\supset \{1,...,T(\epsilon )\}}a_{i}(S_{i})$, $S^{\max
}(t)=\{S_{i}^{\max }(t)\}_{i=1}^{t}$, and define $p_{i}^{\max
}(t)=-\sum_{j=1}^{t}\mathcal{L}_{ij}(S^{\max }(t))(a_{j}(S_{j}^{\max
}(t))-\log (1+\mu _{j}))$. The value $a_{i}(S_{i}^{\max }(t))$ is the
maximum of $2^{t-1-T(\epsilon )}$ random variables drawn jointly from $\Phi
_{i}(t-1-T(\epsilon ))$. Then Assumption \ref{assumption logproductivity
negative} implies that $\lim_{t\rightarrow \infty }\frac{a_{i}(S_{i}^{\max
}(t))}{t-1-T(\epsilon )}=D$ almost surely. Since $T(\epsilon )$ is a
constant independent of $t$, we have $\lim_{t\rightarrow \infty }\frac{%
a_{i}(S_{i}^{\max }(t))}{t}=D$ almost surely. Since a countable intersection
of almost sure events happens almost surely, we also have $%
\lim_{t\rightarrow \infty }\min_{i\leq t}\frac{a_{i}(S_{i}^{\max }(t))}{t}%
=\lim_{t\rightarrow \infty }\max_{i\leq t}\frac{a_{i}(S_{i}^{\max }(t))}{t}%
=D $ almost surely (where the $\min $ and $\max $ are over the set of
industries). Furthermore, since $S_{i}^{\max }(t)\supset \{1,...,T(\epsilon
)\}$, we have $\sum_{j=1}^{t}\mathcal{L}_{ij}(S^{\max }(t))\geq
\sum_{j=1}^{t}\mathcal{L}_{ij}-\frac{1}{(1-\theta )^{2}}\epsilon .$ Plugging
these bounds into the definition of $p_{i}^{\max }$, we obtain%
\begin{eqnarray*}
-p_{i}^{\max }(t)=\sum_{j=1}^{t}\mathcal{L}_{ij}(S(t))(a_{j}(S_{j}^{\max
}(t))-\log (1+\mu _{j}))\geq &&\min_{k\leq t}(a_{k}(S_{k}^{\max }(t))-\log
(1+\mu _{k}))\sum_{j=1}^{t}\mathcal{L}_{ij}(S(t)) \\
&\geq &\min_{k\leq t}(a_{k}(S_{k}^{\max }(t))-\log (1+\mu
_{k}))(\sum_{j=1}^{t}\mathcal{L}_{ij}-\frac{1}{(1-\theta )^{2}}\epsilon ).
\end{eqnarray*}%
Dividing both sides by $t\sum_{j=1}^{t}\mathcal{L}_{ij}$, we obtain%
\begin{equation*}
-\frac{p_{i}^{\max }(t)}{t\sum_{j=1}^{t}\mathcal{L}_{ij}}\geq \frac{%
\min_{k\leq t}(a_{k}(S_{k}^{\max }(t))-\log (1+\mu _{k}))}{t}-\epsilon \frac{%
\min_{k\leq t}(a_{k}(S_{k}^{\max }(t))-\log (1+\mu _{k}))}{t(1-\theta
)^{2}\sum_{j=1}^{t}\mathcal{L}_{ij}}.
\end{equation*}%
Using the fact that $\sum_{j=1}^{t}\mathcal{L}_{ij}\geq 1$, this inequality
can be written as%
\begin{equation*}
-\frac{p_{i}^{\max }(t)}{t\sum_{j=1}^{t}\mathcal{L}_{ij}}\geq \frac{%
\min_{k\leq t}(a_{k}(S_{k}^{\max }(t))-\log (1+\mu _{k}))}{t}-\epsilon \frac{%
\min_{k\leq t}(a_{k}(S_{k}^{\max }(t))-\log (1+\mu _{k}))}{t(1-\theta )^{2}}.
\end{equation*}%
Taking $\liminf $ on both sides, and using the fact that $\mu _{k}$ is a
constant independent of $t$, we obtain%
\begin{equation*}
\liminf_{t\rightarrow \infty }-\frac{p_{i}^{\max }(t)}{t\sum_{j=1}^{t}%
\mathcal{L}_{ij}}\geq D-\epsilon D\frac{1}{(1-\theta )^{2}}.
\end{equation*}%
Since $\epsilon $ is arbitrarily small, we conclude that%
\begin{equation*}
\liminf_{t\rightarrow \infty }-\frac{p_{i}^{\max }(t)}{t\sum_{j=1}^{t}%
\mathcal{L}_{ij}}\geq D.
\end{equation*}

With the same arguments as in the proof of Theorem \ref{existence theorem},
we also have that the function $\kappa (p)=(\min_{S_{1}}\log (1+\mu
_{1})+k_{1}(a_{1}(S_{1}),S_{1},p),...,\min_{S_{t}}\log (1+\mu
_{t})+k_{t}(a_{t}(S_{t}),S_{t},p))$ has a smallest fixed point which gives
the equilibrium log price vector $p^{\ast }(t)$. Starting from $p^{\max}(t)$%
, we can define a decreasing sequence $p^{\tau }(t)=\kappa (p^{\tau -1}(t))$
which converges to a fixed point $p(t)$ of $\kappa $. Since the equilibrium
log price vector is the lowest fixed point of $\kappa $, we have that $%
p^{\ast }(t)\leq p(t)\leq p^{\max }(t)$. Dividing by $t\sum_{j=1}^{t}%
\mathcal{L}_{ij}$ and taking $\lim \inf $ on both sides, we conclude%
\begin{equation}
\liminf_{t\rightarrow \infty }-\frac{p_{i}^{\ast }(t)}{t\sum_{j=1}^{t}%
\mathcal{L}_{ij}}\geq D.  \label{limit inferior}
\end{equation}

To prove that $\limsup_{t\rightarrow \infty }\frac{-p_{i}^{\ast }(t)}{%
t\sum_{j=1}^{t}\mathcal{L}_{ij}}\leq D$, let us write $-p_{i}^{\ast
}(t)=\sum_{j=1}^{t}\mathcal{L}_{ij}(S(t))(a_{j}(S_{j}(t))-\log(1+\mu_j))\leq
\max_{k\leq t}(a_{k}(S_{k}(t))-\log(1+\mu_k))\sum_{j=1}^{t}\mathcal{L}_{ij}.$
The value of $\max_{k\leq t}(a_{k}(S_{k}(t))+\log(1+\mu_k))$ can be upper
bounded by $\max_{k\leq t}\max_{S_{k}^{\prime }} (a_{k}(S_{k}^{\prime }))-
\min_{k} (\log(1+\mu_k))$. As we argued above, $\lim_{t \rightarrow \infty}
\frac{\max_{k\leq t}\max_{S_{k}^{\prime }} (a_{k}(S_{k}^{\prime }))}{t} = D$
almost surely. Furthermore, $\min_{k \leq t} (\log(1+\mu_k))$ is a constant
independent of $t$. Dividing $-p_{i}^{\ast }(t)$ by $t\sum_{j=1}^{t}\mathcal{%
L}_{ij}$ and taking $\lim \sup $ on both sides, we obtain

\begin{equation*}
\limsup_{t\rightarrow \infty }-\frac{p_{i}^{\ast}(t)}{t\sum_{j=1}^{t}%
\mathcal{L}_{ij}}\leq \limsup_{t\rightarrow \infty }\frac{\max_{k\leq
t}\max_{S_{k}^{\prime }}a_{k}(S_{k}^{\prime })-\min_{k \leq t} (1+\mu_k)}{t}%
\leq D\text{ almost surely.}
\end{equation*}%
Combining this with (\ref{limit inferior}), we can thus conclude that
\begin{equation}
\lim_{t\rightarrow \infty }-\frac{p_{i}^{\ast }(t)}{t\sum_{j=1}^{t}\mathcal{L%
}_{ij}}=D\text{ almost surely},  \label{price limit equation}
\end{equation}%
and thus%
\begin{equation*}
g^{\ast }=\lim_{t\rightarrow \infty }\left( -\frac{\pi (t)}{t}\right)
=D\sum_{i,j=1}^{\infty }\beta _{i}\mathcal{L}_{ij}\text{ almost surely}.
\end{equation*}
\end{proof}
\newline

\begin{proof}[Proof of Theorem \protect\ref{theorem hicks neutral growth}]
Since $k_{i}(S_{i},a_{i}(S_{i}),p)=-a_{i}(S_{i})+\overline{k}_{i}(S_{i},p)$,
equilibrium log prices satisfy%
\begin{equation*}
p_{i}^{\ast }=\log (1+\mu _{i})-a_{i}(S_{i}^{\ast })+\overline{k}%
_{i}(S_{i}^{\ast },p^{\ast }),
\end{equation*}%
and%
\begin{equation*}
S_{i}^{\ast }\in \arg \min_{S}-a_{i}(S)+\overline{k_{i}}(S,p^{\ast }).
\end{equation*}%
Let $b$ be a vector such that $b_{i}=\log (1+\mu _{i})-a_{i}(S_{i})$. Then
for any production network $S$,%
\begin{equation*}
p=b+\overline{k}(S,p),
\end{equation*}%
where $\overline{k}(S,p)$ is a vector valued function whose $i^{th}$
coordinate is $\overline{k}_{i}(S_{i},p)$. Since $\frac{d\log \overline{k_{i}%
}}{d\log p_{j}}\geq 0$ and by assumption $\sum_{j=1}^{\infty }\frac{d\log
\overline{k}_{i}}{d\log p_{j}}\leq \theta $ for all $i$, each entry of the
Jacobian of the function $\Phi (p)=p-\overline{k}(S,p)$ is greater than $%
1-\theta >0$, and less than or equal to 1. Let us denote the Jacobian of $%
\overline{k}$ with respect to $p$ when the network is given by $S$ by $J_{%
\overline{k},S,p}$. Then the matrix $(I-J_{\overline{k},S,p})$ is
invertible, and all entries of $(I-J_{\overline{k},S,p})^{-1}$ are
non-negative (equivalently, $J_{\overline{k},S,p}$ is a P-matrix). From Gale
and Nikaido (1965), there exists a \emph{globally defined} function $p(b,S)$
that is continuously differentiable in $b$ such that $p(b,S)-\overline{k}%
(S,p(b,S))=b$. Taking derivatives with respect to $b$,%
\begin{equation*}
J_{p,b,S}-J_{\overline{k},S,p}J_{p,b,S}=I
\end{equation*}%
where $J_{p,b,S}$ is the Jacobian of $p(b,S)$. We can write $J_{p,b,S}=(I-J_{%
\overline{k},S,p})^{-1}$ and observe that each entry of the matrix $%
J_{p,b,S} $ is greater than or equal to 1, and less than or equal to $\frac{1%
}{1-\theta }$.

Then for any $b^{\prime }\geq b\in \mathbb{R}^{t}$, define $b(\tau )=(1-\tau
)b+\tau b^{\prime }$ for $\tau \in \lbrack 0,1]$, and for any $t$%
-dimensional vector $\gamma $ such that $\gamma \geq 0$ and $%
\sum_{i=1}^{\infty }\gamma _{i}=1$, define%
\begin{equation*}
\pi (\tau ,S)=\gamma ^{\prime }p(b(\tau ),S).
\end{equation*}%
This function is differentiable defined on $[0,1]$ with derivative $\gamma
^{\prime }J_{p,b,S}(b^{\prime }-b)$. Then, from the mean value theorem,
there exists $\tau _{0}\in (0,1)$ such that%
\begin{equation*}
\pi (1,S)-\pi (0,S)=\gamma ^{\prime }J_{p,b,S}|_{b=b(\tau _{0})}(b^{\prime
}-b).
\end{equation*}%
Since the coefficients of $J_{p,b,S}$ are bounded between $1$ and $\frac{1}{%
1-\theta }$, and $b^{\prime }\geq b$ we have
\begin{equation}
\gamma ^{\prime }(b^{\prime }-b)\leq \pi (1,S)-\pi (0,S)\leq \frac{1}{%
1-\theta }\gamma ^{\prime }(b^{\prime }-b).
\label{Hicks neutral key inequality}
\end{equation}%
Set $S=S^{\ast }(t)$ (which is the equilibrium network at time $t$), $%
b_{i}^{\prime }=-\overline{k}_{i}(S_{i}^{\ast },0)$ and $b_{i}=\log (1+\mu
_{i})-a_{i}(S_{i}^{\ast }(t))$. Then $p(b,S^{\ast })=p^{\ast }$, where
recall that $p^{\ast }$ is the equilibrium price vector at time $t$.
Moreover, we also have $b^{\prime }+\overline{k}(S^{\ast },0)=0$, so that $%
p(b^{\prime },S^{\ast }(t))=0$, and $\pi (1,S^{\ast }(t))=0$. Then (\ref%
{Hicks neutral key inequality}) implies%
\begin{equation*}
\gamma ^{\prime }(b^{\prime }-b)\leq -\gamma ^{\prime }p^{\ast }(t)\leq
\frac{1}{1-\theta }\gamma ^{\prime }(b^{\prime }-b).
\end{equation*}%
From Assumptions \ref{assumption markups} and \ref{assumption
logproductivity negative}, we have that $\lim_{t\rightarrow \infty }\frac{%
b_{i}}{t}=\lim_{t\rightarrow \infty }\frac{a_{i}(S_{i}(t))}{t}=-D$ almost
surely.

On the other hand, because $S^{\ast }(t)$ is cost-minimizing, $b^{\prime
}\geq -\overline{k}((\varnothing ,...,\varnothing ),0)$, and because the log
cost function is nonincreasing in log prices, we also have $b^{\prime }\leq
-\lim_{p\rightarrow -\infty }\overline{k}(S^{\ast }(t),p)=-\ell (t)$, where $%
\ell (t)=\log \min_{L_{i}:F(X_{i},L_{i},S_{i}^{\ast }(t))=1}L_{i}$. Finally,
because labor is essential from the first part of\ Assumption \ref%
{assumption leontief}, there exists $\underline{\ell }$ such that $\ell
(t)\geq \underline{\ell }$ for all $t\in \mathbb{N}$, and thus$-\overline{k}%
((\varnothing ,...,\varnothing ),0)\leq b^{\prime }\leq -\underline{\ell }$.
Dividing this inequality by $t$, and taking the limit as $t\rightarrow
\infty $, we obtain $\lim_{t\rightarrow \infty }\frac{b^{\prime }}{t}=0$,
and $b\leq b^{\prime }$ almost surely (as $t\rightarrow \infty $).

Taking limits on both sides of (\ref{Hicks neutral key inequality}), we
obtain
\begin{equation*}
D\sum_{i=1}^{\infty }\gamma _{i}=\lim_{t\rightarrow \infty }\frac{\gamma
^{\prime }(b^{\prime }-b)}{t}\leq \lim_{t\rightarrow \infty }-\frac{\gamma
^{\prime }p^{\ast }(t)}{t}\leq \lim_{t\rightarrow \infty }\frac{\frac{1}{%
1-\theta }\gamma ^{\prime }(b^{\prime }-b)}{t}=\frac{D}{1-\theta }.
\end{equation*}%
Now setting $\gamma _{i}=1$ and $\gamma _{j}=0$ for all $j\neq i$, we obtain%
\begin{equation*}
D\sum_{i=1}^{\infty }\gamma _{i}\leq \lim_{t\rightarrow \infty }-\frac{%
p_{i}^{\ast }(t)}{t}\leq \frac{D}{1-\theta }.
\end{equation*}

For the last part of the theorem, first note that with a similar argument
using the Jacobians as here, we can show that nominal GDP is bounded in this
case also, and thus as in Lemma \ref{lemma growth rate}, $g^{\ast
}=\lim_{t\rightarrow \infty }\left( -\frac{\pi (t)}{t}\right) $. Then
setting $\gamma =\beta $ (with $\beta $ as given in Assumption \ref%
{assumption2prime}), we get%
\begin{equation*}
D\sum_{i=1}^{\infty }\beta _{i}=D\leq g^{\ast }\leq \frac{D}{1-\theta }%
\sum_{i=1}^{\infty }\beta _{i}=\frac{D}{1-\theta },
\end{equation*}%
establishing the desired result.
\end{proof}

\begin{proof}[Proof of Theorem \protect\ref{theorem hierarchical}]
The production function%
\begin{equation*}
Y_{i}=L_{i}^{1-\sum_{k\in R_{i}}\sum_{j\in S_{i,k}}\alpha
_{ij}}\prod_{k=1}^{K}(A_{i,k}(S_{i,k})\prod_{j\in S_{i,k}}X_{ij}^{\alpha
_{ij}})
\end{equation*}%
can be recast as a production function with productivity term $%
A_{i}(S_{i})=\prod_{k=1}^{K}A_{i,k}(S_{i,k})$. For this production function,
Assumption \ref{assumption logproductivity negative} is satisfied, with
\begin{equation*}
\lim_{t\rightarrow \infty }\frac{a_{i}(S_{i}(t))}{t}=\lim_{t\rightarrow
\infty }\frac{\sum_{k}a_{i,k}(S_{i,k}(t))}{t}= \sum_{k=1}^K D_k
\end{equation*}%
almost surely. Applying Theorem \ref{proposition growth} to this function,
we obtain the desired result.
\end{proof}
\newline

\begin{proof}[Proof of Lemma \protect\ref{lemma edge probabilities}]
The log unit cost of adopting set $S_{i}$ is $k_{i}(S_{i},a_{i}(S_{i}),p)=%
\sum_{j\in S_{i}}\alpha _{ij}p_{j}-b_j - \epsilon(S_{i})$ where $%
\epsilon(S_{i})$ is distributed according to a Gumbel distribution with
variance parameter $\sigma $. Choosing $S_{i}$ to minimize $%
k_{i}(S_{i},a_{i}(S_{i}),p)$ is equivalent to choosing $S_{i}$ to maximize $%
-k_{i}(S_{i},a_{i}(S_{i}),p)=-\sum_{j\in S_{i}}\alpha _{ij}p_j+b_j +
\epsilon(S_{i})$. Part 1 then follows from the same derivation as that of
Lemma 1 in McFadden (1973).

Part 2 follows because%
\begin{eqnarray*}
\Pr (j\in S_{i}|P)= &&\frac{\sum_{S_{i}:j\in S_{i}}\prod_{j^{\prime }\in
S_{i}} e^{b_j} P_{j^{\prime }}^{-\frac{\alpha _{ij^{\prime }}}{\sigma }}}{%
\sum_{S_{i}}\prod_{j^{\prime }\in S_{i}} e^{b_j}P_{j^{\prime }}^{-\frac{%
\alpha _{ij^{\prime }}}{\sigma }}}=\frac{\sum_{S_{i}:j\in
S_{i}}\prod_{j^{\prime }\in S_{i}}e^{b_j}P_{j^{\prime }}^{-\frac{\alpha
_{ij^{\prime }}}{\sigma }}}{\sum_{S_{i}:j\in S_{i}}\prod_{j^{\prime }\in
S_{i}}e^{b_j}P_{j^{\prime }}^{-\frac{\alpha _{ij^{\prime }}}{\sigma }%
}+\sum_{S_{i}:j\not\in S_{i}}\prod_{j^{\prime }\in S_{i}}e^{b_j}P_{j^{\prime
}}^{-\frac{\alpha _{ij^{\prime }}}{\sigma }}} \\
= &&\frac{e^{b_j}P_{j}^{-\frac{\alpha _{ij}}{\sigma }}\sum_{S_{i}:j\not\in
S_{i}}\prod_{j^{\prime }\in S_{i}}e^{b_j}P_{j^{\prime }}^{-\frac{\alpha
_{ij^{\prime }}}{\sigma }}}{e^{b_j}P_{j}^{-\frac{\alpha _{ij}}{\sigma }%
}\sum_{S_{i}:j\not\in S_{i}}\prod_{j^{\prime }\in S_{i}}e^{b_j}P_{j^{\prime
}}^{-\frac{\alpha _{ij^{\prime }}}{\sigma }}+\sum_{S_{i}:j\not\in
S_{i}}\prod_{j^{\prime }\in S_{i}}e^{b_j}P_{j^{\prime }}^{-\frac{\alpha
_{ij^{\prime }}}{\sigma }}}=e^{b_j}\frac{P_{j}^{-\frac{\alpha _{ij}}{\sigma }%
}}{1+P_{j}^{-\frac{\alpha _{ij}}{\sigma }}}.
\end{eqnarray*}
\end{proof}
\newline

\begin{proof}[Proof of Theorem \protect\ref{theorem 8}]
\textbf{Part 1. }Since $\sum_{j=1}^{n}\alpha _{ij}\leq 1$ for every $i$ from
Assumption \ref{assumption further Leontief}, $\mathcal{I}_{i}(n)=\frac{1}{n}%
\sum_{j=1}^{n}\alpha _{ij}(S(n))\leq \frac{1}{n}$. Thus, for every $\epsilon
>0$, we have $\Vert \mathcal{I}(n)\Vert _{\infty }\leq \frac{1}{n}$. This
implies that $\mathcal{I}(n)$ uniformly converges to 0.

\textbf{Part 2. }We can write $\mathcal{O}_{j}(n)=\frac{1}{n}%
\sum_{i=1}^{n}\alpha _{ij}I(i,j,n)$, where $I(i,j,n)$ is an indicator
function that is equal to 1 if $j\in S_{i}(n)$ and 0 otherwise. Since $%
I(i,j,n)\leq 1$,
\begin{equation*}
\mathcal{O}_{j}(n)\leq \frac{1}{n}\sum_{i=1}^{n}\alpha _{ij}=\alpha_j.
\end{equation*}%
This implies that $\lim \sup_{n\rightarrow \infty }\mathcal{O}_{j}(n)=\bar{%
\mathcal{O}}_{j}\leq \alpha _{j}$ for all $j$. %The convergence is uniform
%since for every $\epsilon >0$, we can choose $N>\frac{\Vert \gamma \Vert
%_{\infty }}{\epsilon }$ so that $\mathcal{O}_{j}(n)\leq \alpha _{j}+\frac{1}{%
%n}\gamma _{j}\leq \alpha _{j}+\epsilon $ for every $j\in \mathbb{N}$ and
%every $n>N$.

\textbf{Part 3. }Let $P(n)$ be the price vector in economy $\mathcal{E}(n)$,
and let $\mathcal{O}_{j}(n)|P(n)$ be the outdegree of $j$ conditional on
prices. From Lemma \ref{lemma edge probabilities}, the decisions of any two
industries $i,i^{\prime }$ on whether or not to choose $j$ as a supplier are
independent given prices. Thus, the sequence of random variables $%
\{I(i,j,n)|P(n)\}_{i=1}^{n}$ is a sequence of independent Bernoulli random
variables with $\Pr (I(i,j,n)|P(n))=\frac{e^{b_j}P_{j}^{-\alpha _{ij}}}{%
1+e^{b_j}P_{j}^{-\alpha _{ij}}}$. The expected outdegree of firm $j$ given a
fixed price vector $P(n)$ is $\mathbb{E}[\mathcal{O}_{j}(n)|P(n)]=\frac{1}{n}%
\sum_{i=1}^{n}\alpha _{ij}\frac{e^{b_j}P_{j}^{-\alpha _{ij}}}{%
1+e^{b_j}P_{j}^{-\alpha _{ij}}}.$ If $P_j^{-\alpha_{ij}} \geq 1$ for every $%
j\in \mathbb{N}$, then we have $\frac{e^{b_j}P_{j}^{-\alpha _{ij}}}{%
1+e^{b_j}P_{j}^{-\alpha _{ij}}}\geq \frac{e^{b_j}}{1+e^{b_j}}$, and $\mathbb{%
E}[\mathcal{O}_{j}(n)|P(n)]\geq \frac{1}{n}\sum_{i=1}^{n}\frac{\alpha _{ij}
e^{b_j}}{1+e^{b_j}}$. Taking $\lim \sup$ on both sides, we obtain
\begin{gather*}
\lim \sup_{n \to \infty} \mathbb{E}[\mathcal{O}_{j}(n)|P(n)] \geq \frac{
\alpha_j e^{b_j}}{1+e^{b_j}}.
\end{gather*}

Recall that if $X_{1},...,X_{n}$ are independent random variables in the
interval $[0,1]$, we have the following Chernoff bound%
\begin{equation*}
\Pr (|\frac{1}{n}\sum_{i=1}^{n}X_{i}-\frac{1}{n}\sum_{i=1}^{n}\mathbb{E}%
[X_{i}]|\geq \epsilon )\leq 2e^{-2n\epsilon ^{2}}.
\end{equation*}%
Using this Chernoff bound and the conditional independence of each $%
I(i,j,n)|P(n)$, we get that for any $\epsilon >0$, we have $\Pr (|\mathcal{O}%
_{j}(n)-\mathbb{E}[\mathcal{O}_{j}(n)|P(n)]|\geq \epsilon )\leq
2e^{-2n\epsilon ^{2}}.$
%Using a union bound\footnote{If $X_1,...,X_n$ are arbitrarily correlated independent random variables , the union bound tells us that
%\begin{equation*}
%\Pr(\max_i X_i \geq \epsilon) = \Pr( X_1 \geq \epsilon \text{ or } X_2 \geq \epsilon \text{ or } ... \text{ or } X_n \geq \epsilon) \leq \sum_{i=1}^n \Pr(X_i \geq \epsilon).
%\end{equation*}
%}
%\begin{equation*}
%\Pr (\max_{j\leq n}|\mathcal{O}_{j}(n)-\mathbb{E}[\mathcal{O}%
%_{j}(n)|P(n)]||\geq \epsilon)\leq 2ne^{-2n\epsilon ^{2}}.
%\end{equation*}%
Using the first Borel-Cantelli Lemma (Lemma \ref{lemmab1}) and the fact that
$\sum_{n=1}^{\infty }2e^{-2n\epsilon ^{2}}<\infty $, we conclude that $\lim
\sup_{n\rightarrow \infty }|\mathcal{O}_{j}(n)-\mathbb{E}[\mathcal{O}%
_{j}(n)|P(n)]|\leq 0$ almost surely. Using the reverse triangle inequality
and the fact that $\mathcal{O}_{j}(n)\geq 0$, this becomes
\begin{equation*}
\lim \sup_{n\rightarrow \infty }\mathbb{E}[\mathcal{O}_{j}(n)|P(n)]\leq \lim
\inf_{n\rightarrow \infty }\mathcal{O}_{j}(n)=\underline{\mathcal{O}_{j}}%
\text{ almost surely.}
\end{equation*}%
Finally, recall from the proof of Corollary \ref{corollary bounds} that $%
\lim \sup_{n\rightarrow \infty }\max_{j\leq n}\frac{p_{j}(n)}{n}\leq 0$
almost surely, and thus $\lim \sup_{n\rightarrow \infty }\max_{j\leq n}
P_{j}(n)^{-\alpha_{ij}} \geq 1 $ almost surely (this result still holds,
since the assumptions imposed here are stronger versions of those in Theorem %
\ref{proposition growth}). Therefore, $\lim \inf_{n\rightarrow \infty
}\min_{j\leq n}\frac{e^{b_j}P_{j}^{-\alpha _{ij}}}{1+e^{b_j}P_{j}^{-\alpha
_{ij}}}\geq \frac{e^{b_j}}{e^{1+b_j}}$ almost surely, and consequently,%
\begin{equation*}
\lim \sup_{n\rightarrow \infty }\frac{e^{b_j}}{1+e^{b_j}}\alpha_{j}\leq \lim
\sup_{n\rightarrow \infty }\mathbb{E}[\mathcal{O}_{j}(n)|P(n)] \leq
\underline{\mathcal{O}_j}
\end{equation*}%
holds almost surely for all $j\in \mathbb{N}$.

Finally, note that if $\bar{\mathcal{O}}$ were a degenerate distribution,
then either $\bar{\mathcal{O}_{j}}=0$ for all $j$ or $\bar{\mathcal{O}_{j}}%
=\rho >0$ for all $j$. In the former case, we would have $\sum_{j=1}^{\infty
}\bar{\mathcal{O}_{j}}=0$, which cannot be the case because $\bar{\mathcal{O}%
}_j>\alpha_j \frac{e^{b_j}}{1+e^{b_j}} $ and $\sum_{j=1}^{\infty }\frac{%
\alpha _{j} e^{b_j}}{1+e^{b_j}}>0$. In the latter case, we would have $%
\sum_{j=1}^{\infty }\bar{\mathcal{O}_{j}}=\infty $, which cannot be the case
because $\bar{\mathcal{O}}\leq \alpha $ and $\sum_{j=1}^{\infty }\alpha
_{j}\leq 1$. The argument that $\underline{\mathcal{O}}$ cannot be a
degenerate distribution is analogous to the argument for $\bar{\mathcal{O}}$.
\end{proof}
\newline

\begin{thebibliography}{99}
\bibitem{Acemoglu2017a} Acemoglu, D., Autor, D. and Patterson C. (2017).
Bottlenecks and Sluggish Productivity Growth. \emph{Manuscript.}

\bibitem{Acemoglu2012} Acemoglu, D., Carvalho, V. M., Ozdaglar, A., and
Tahbaz-Salehi, A. (2012). The network origins of aggregate fluctuations.
\emph{Econometrica}, 80(5), 1977-2016

\bibitem{Acemoglu2017} Acemoglu, D., Ozdaglar, A., and Tahbaz-Salehi, A.
(2017). Microeconomic origins of macroeconomic tail risks. \emph{The
American Economic Review}, 107(1), 54-108.

\bibitem{AghionHowitt} Aghion, P., and Howitt, P. (1992). A model of Growth
through Creative Destruction.\emph{Econometrica}, 60(2), 323-351.

\bibitem{AntrasChor} Antr\`as, P., and Chor, D. (2013). Organizing the
global value chain. \emph{\ Econometrica}, 81(6), 2127-2204.

\bibitem{Antras} Antr\`as, P. Fort T. C. and Tintelnot (2017). The Margins
of Global Sourcing: Theory and Evidence from U.S. Firms, \emph{American
Economic Review}, 107(9), 2514--64.

\bibitem{Auerswald} Auerswald, P., Kauffman, S., Lobo, J., \& Shell, K.
(2000). The production recipes approach to modeling technological
innovation: An application to learning by doing. \emph{Journal of Economic
Dynamics and Control}, 24(3), 389-450.

\bibitem{Baqaee} Baqaee, D. R. (2016). Cascading failures in Production
Networks, forthcoming \emph{Econometrica.}

\bibitem{BaqaeeFarhi} Baqaee, D. R., and Farhi, E. (2017). The Macroeconomic
Impact of Microeconomic Shocks: Beyond Hulten's Theorem . \emph{National
Bureau of Economic Research Working Paper No. w23145.}

\bibitem{BaqaeeFarhi2} Baqaee, D.R., and Farhi, E. (2017). Productivity and
Misallocation in General Equilibrium. \emph{\ National Bureau of Economic
Research Working Paper No. w24007.}

\bibitem{Dominick} Bartelme, D., and Gorodnichenko Y. (2015) Linkages and
Economic Development. \emph{National Bureau of Economic Research Working
Paper No. 21251.}

\bibitem{Bertsekas} Bertsekas, Dimitri, Angelia Nedic and Asuman Ozdaglar
(2003) \emph{Convex Analysis and Optimization}, Athena Scientific, Belmont
Massachusetts.

\bibitem{BiglioLao} Bigio, S., and La'O, J. (2016). Financial frictions in
production networks. \emph{National Bureau of Economic Research Working
Paper No. 22212.}

\bibitem{} Boehm, J. and Oberfield, E. (2018). Misallocation in the Market
for Inputs: Enforcement and the Organization of Production, NBER Working
Paper 24937, August 2018.

\bibitem{Tsvinsky} Caliendo, L., Parro, F., and Tsyvinski, A. (2017).
Distortions and the Structure of the World Economy. \emph{National Bureau of
Economic Research Working Paper No. w23332.}

\bibitem{Chaney} Chaney, T (2014). The network structure of international
trade. \emph{American Economic Review}, 104(11),3600-3634.

\bibitem{De Loecker} De Loecker, J., Eeckhout, J., and Unger G. (2018). The
Rise of Market Power and the Macroeconomic Implications. {\em Manuscript.}

\bibitem{EatonKortum} Eaton, J., and Kortum, S. (2001). Technology, trade,
and growth: A unified framework. \emph{European economic review}, 45(4),
742-755.

%\bibitem{EatonKortumKramarz} Eaton, J., Kortum, S., and Kramarz, F. (2011). An anatomy of international trade: Evidence from French firms. \emph{Econometrica}, 79(5), 1453-1498.

\bibitem{Fadinger} Fadinger, H., Ghiglino C. and Teteryatnikova M. (2016)
Income differences and input-output structure. \emph{CEPR Discussion Paper
No. 11547.}

\bibitem{Gabaix} Gabaix, X. (2011). The granular origins of aggregate
fluctuations. \emph{Econometrica}, 79(3), 733-772.

\bibitem{GaleNikaido} Gale, D., and Nikaido, H. (1965). The Jacobian matrix
and global univalence of mappings. \emph{Mathematische Annalen}, 159(2),
81-93.

\bibitem{Gighlino} Ghiglino, C. (2012). Random walk to innovation: Why
productivity follows a power law. \emph{Journal of Economic Theory}, 147(2),
713-737.

\bibitem{GrossmanHelpman} Grossman, G. M., and Helpman, E. (1991). Quality
ladders in the theory of growth. \emph{The Review of Economic Studies},
58(1), 43-61.

\bibitem{GrossmanHelpman2} Grossman, G. M., and Helpman, E. (1992).
Innovation and growth in the global economy. \emph{MIT press.}

\bibitem{Gualdi} Gualdi, S., and Mandel, A. (2016). Endogenous growth in
production networks. \emph{\ Journal of Evolutionary Economics}, 1-27.

\bibitem{Hawkins} Hawkins, D., and Simon, H. A. (1949). Note: some
conditions of macroeconomic stability. \emph{Econometrica, Journal of the
Econometric Society}, 245-248.

\bibitem{Hulten} Hulten, C. R. (1978). Growth accounting with intermediate
inputs. \emph{The Review of Economic Studies}, 511--518.

\bibitem{IMF} IMF (2015). World Economic Outlook.

\bibitem{Jones1995} Jones, C. I. (1995). R \& D-based models of economic
growth. \emph{Journal of political Economy}, 103(4), 759-784.

\bibitem{Jones2011} Jones, C. I. (2011). Intermediate goods and weak links
in the theory of economic development. \emph{American Economic Journal:
Macroeconomics}, 3(2), 1-28.

\bibitem{Kennan} Kennan, J. (2001). Uniqueness of Positive Fixed Points for
Increasing Concave Functions on $\mathbb{R}^n$: An Elementary Result. {\emph{%
R}eview of Economic Dynamics}, 4(4), 893--899.

\bibitem{KletteKortum} Klette, T. J., and Kortum, S. (2004). Innovating
firms and aggregate innovation. \emph{Journal of political economy}, 112(5),
986-1018.

\bibitem{Lim} Lim, K. (2017). Firm-to-rm Trade in Sticky Production
Networks. \emph{Manuscript.}

\bibitem{Liu} Liu, E. (2017). Industrial policies and economic development.
\emph{Manuscript.}

\bibitem{LongPlosser} Long, J. B., and Plosser, C. I. (1983). Real business
cycles. \emph{Journal of political Economy}, 91(1), 39-69. Chicago

\bibitem{Lucas} Lucas, R. E. (2009). Ideas and growth. \emph{Economica},
76(301), 1-19.

\bibitem{LucasMoll} Lucas, R. E., and Moll, B. (2014). Knowledge growth and
the allocation of time. \emph{Journal of Political Economy}, 122(1), 1-51.

\bibitem{McFadden} McFadden, D. (1973). Conditional logit analysis of
qualitative choice behavior. \emph{Frontiers of Econometrics}, ed. by P.
Zambreka. New York: Academic Press.

\bibitem{Mulla} Mulla, D., and Khosla, R. (2016). Historical Evolution and
Recent Advances in Precision Farming. \emph{Soil-Specific Farming Precision
Agriculture}, ed. by. R Lal, and B.A. Stewart. CRC Press, pp. 1-35.

\bibitem{Oberfield} Oberfield, E. (2017). A theory of input-output
architecture. \emph{Technical report.}

\bibitem{PerlaTonetti} Perla, J., and Tonetti, C. (2014). Equilibrium
imitation and growth. \emph{Journal of Political Economy}, 122(1), 52-76.

\bibitem{Peters} Peters, M. (2013). Heterogeneous mark-ups, growth and
endogenous misallocation.

\bibitem{Romer} Romer, P. M. (1990). Endogenous technological change. \emph{%
Journal of political Economy}, 98(5, Part 2), S71-S102.

\bibitem{Samuelson} Samuelson, P. A. (1951). Abstract of a theorem
concerning substitutability in open Leontief models. \emph{Activity analysis
of production and allocation}, 13, 142-6.

\bibitem{Taschereau} Taschereau-Dumouchel, M. (2017). Cascades and
fluctuations in an economy with an endogenous production network.\emph{%
Manuscript}.

\bibitem{Topkis} Topkis, D. M. (1998). Supermodularity and Complementarity.
\emph{Princeton University Press.}

\bibitem{Tintlenot} Tintelnot, F., Kikkawa, A.K., Mogstad, M and Dhyne, E.
(2017). Trade and Domestic Production Networks. \emph{Manuscript}.

\bibitem{Weitzman} Weitzman, M. L. (1998). Recombinant growth. \emph{The
Quarterly Journal of Economics}, 113(2), 331-360.\newline
\newpage
\end{thebibliography}

\clearpage

\section*{Online Appendix B: Additional Results and Omitted Proofs}

\label{Appendix B} \setcounter{equation}{0}\setcounter{theorem}{0} %
\setcounter{lemma}{0} \setcounter{proposition}{0} \setcounter{corollary}{0} %
\setcounter{example}{0} \renewcommand{\theequation}{B\arabic{equation}} %
\renewcommand{\thelemma}{B\arabic{lemma}} \renewcommand{\theproposition}{B%
\arabic{proposition}} \renewcommand{\thecorollary}{B\arabic{corollary}} %
\renewcommand{\thetheorem}{B\arabic{theorem}}\renewcommand{\theexample}{B%
\arabic{example}} \pagenumbering{arabic}% resets `page` counter to 1
\renewcommand*{\thepage}{B-\arabic{page}}

\subsection*{Properties of Cobb-Douglas Technologies}

\begin{lemma}
\label{lemma CD}The unit cost function of the Cobb-Douglas production
function is%
\begin{equation*}
K_{i}(S_{i},A_{i}(S_{i}),P)=\frac{\prod_{j\in S_{i}}P_{j}^{\alpha _{ij}}}{%
A_{i}(S_{i})}.
\end{equation*}
\end{lemma}

\begin{proof}[Proof of Lemma \protect\ref{lemma CD}]
Let $X_{ij}^{\ast }$ and $L_{i}^{\ast }$ be firm $i$'s optimal choices of
inputs and labor when producing one unit of output. From the firm's
first-order conditions, we have $P_{j}X_{ij}^{\ast }=\alpha _{ij}\frac{P_{i}%
}{1+\mu_i}$ and $L_{i}^{\ast }=(1-\sum_{j\in S_{i}}\alpha _{ij})\frac{P_{i}}{%
1+\mu_i}.$ Dividing the former equation by the latter, we obtain $%
X_{ij}^{\ast }=\frac{\alpha _{ij}L_{i}^{\ast }}{(1-\sum_{j\in S_{i}}\alpha
_{ij})P_{j}}.$ Plugging this into the production function (and recalling
that only one unit of output is produced), we obtain
\begin{gather*}
1=\frac{1}{(1-\sum_{j\in S_{i}}\alpha _{ij})^{1-\sum_{j\in S_{i}}\alpha
_{ij}}\prod_{j\in S_{i}}\alpha _{ij}^{\alpha _{ij}}}A_{i}(S_{i})(L_{i}^{\ast
})^{(1-\sum_{j\in S_{i}}\alpha _{ij})} \prod_{j \in S_i} (\frac{\alpha
_{ij}L_{i}^{\ast }}{(1-\sum_{j\in S_{i}}\alpha _{ij})P_{j}})^{\alpha _{ij}}
\\
1=\frac{1}{(1-\sum_{j\in S_{i}}\alpha _{ij})^{1-\sum_{j\in S_{i}}\alpha
_{ij}}\prod_{j\in S_{i}}\alpha _{ij}^{\alpha _{ij}}}L_{i}^{\ast
}A_{i}(S_{i})(1-\sum_{j\in S_{i}}\alpha _{ij})^{-\sum_{j\in S_{i}}\alpha
_{ij}}\prod\limits_{j\in S_{i}}\frac{\alpha _{ij}^{\alpha _{ij}}}{%
P_{j}^{\alpha _{ij}}} \\
1=\frac{L_{i}^{\ast }A_{i}(S_{i})}{(1-\sum_{j\in S_{i}}\alpha
_{ij})\prod_{j\in S_{i}}P_{j}^{\alpha _{ij}}}.
\end{gather*}%
Therefore, $L_{i}^{\ast }=\frac{(1-\sum_{j\in S_{i}}\alpha _{ij})\prod_{j\in
S_{i}}P_{j}^{\alpha _{ij}}}{A_{i}(S_{i})}.$ Since $%
K_{i}(S_{i},A_{i}(S_{i}),P)=\frac{P_{i}}{1+\mu_i}=\frac{L_{i}^{\ast }}{%
(1-\sum_{j\in S_{i}}\alpha _{ij})},$ we conclude that $%
K_{i}(S_{i},A_{i}(S_{i}),P)=\frac{\prod_{j\in S_{i}}P_{j}^{\alpha _{ij}}}{%
A_{i}(S_{i})}.$
\end{proof}
\newline

\begin{corollary}
\label{corollary CD} When all industries have Cobb-Douglas production
functions and the input-output network is $S$, equilibrium log prices are
given as a solution to the following system of linear equations:%
\begin{equation*}
p=-(I-\alpha (S))^{-1}(a(S)-m)
\end{equation*}
where $m_i = \log(1+\mu_i)$.
\end{corollary}

\begin{proof}[Proof of Corollary \protect\ref{corollary CD}]
From Lemma \ref{lemma CD}, $P_{i}=(1+\mu_i)\frac{\prod_{j\in
S_{i}}P_{j}^{\alpha _{ij}}}{A_{i}(S_{i})}$ for each $i$. Taking logs on both
sides, we obtain
\begin{equation*}
p_{i}=\sum_{j\in S_{i}}\alpha _{ij}p_{j} + \log(1+\mu_i) -a_{i}(S_{i})\text{
for each }i\text{.}
\end{equation*}%
From Assumption \ref{assumption 1}, labor is essential and thus $%
\sum_{j=1}^{n}\alpha _{ij}<1$ for each $i$. Then, from the Perron-Frobenius
Theorem, the matrix $(I-\alpha (S))$ is invertible, and thus $p=-(I-\alpha
(S))^{-1}(a(S)-m)$.
\end{proof}
\newline

\subsection*{Continuity of GDP\newline
}

\begin{theorem}
\label{proposition continuity no markups}\textbf{(Continuity of GDP without
distortions)} Suppose that $\mu _{i}=0$ for all $i=1,2,\ldots ,n$. Then,
(real) GDP is continuous in the log productivity vector $a=\{a_{i}(S)\}_{i%
\in \mathcal{N},S_{i}\subset \mathcal{N}}$.
\end{theorem}

\begin{proof}
Because the utility function is continuous, real GDP $U(C_{1}^{\ast
},...,C_{n}^{\ast })$ is a continuous function of the log price vector $p$
and nominal income of the representative household. Since distortions are
zero, nominal Income of the representative household is equal to labor
income, which is constant and equal to 1. Thus, all we need to show is that
the equilibrium log price vector $p$ varies continuously with $a$. Let $a$
be a log productivity vector and let $S$ be an input-output network (not
necessarily the equilibrium one). Let $p$ be the vector of equilibrium log
prices if the network is exogenously fixed to be $S$, and technology is
given by $a$. Then $p$ is the unique solution to the system of equations
\begin{equation*}
p-k(S,a(S),p)=m
\end{equation*}%
where $m$ is a vector whose $i^{th}$ component is $\log (1+\mu _{i})$. The
left-hand side is a continuously differentiable function of prices whose
Jacobian is equal to $I-J_{k,p}$, where $J_{k,p}$ is the Jacobian of $k$
with respect to $p$. Since labor is essential, there exists $\theta <1$ such
that $\sum_{j=1}^{n}\frac{\partial k_{i}}{\partial p_{j}}<\theta $. Recall
that a P-matrix is a matrix whose principal minors are all positive. Hawkins
and Simon (1948) show that a matrix of the form $B=I-A$ is a P-matrix if and
only if $(I-A)^{-1}$ exists and all of its coefficients are non-negative,
which is true for our matrix $I-J_{k,p}$. Therefore, the matrix $I-J_{k,p}$
is a P-matrix. Then once again from Gale and Nikaido (1965) there exists a
globally defined function $p(a,S)$ that is continuously differentiable in $a$%
.

Now fix an arbitrary $S^{0}$ and let $p^{0}(a)=p(a,S^{0})$. For any $t\geq 1$%
, define $S_{i}^{t}=\arg \min_{S_{i}^{t}}\log (1+\mu
_{i})+k_{i}(S_{i}^{t},a_{i}(S_{i}^{t}),p^{t-1}(a)), p^t(a)=p(a,S^{t})$, and
note that $p^t(a) \leq p^{t-1}(a)$. We have that $p^{\ast}(a) = \lim_{t \to
\infty} p^t(a)$. Since each $p^t$ corresponds to a network $S^t$, and there
are a finite number of possible networks, there must only be a finite number
of vectors in the sequence $\{p^t\}_{t=1}^{\infty}$.\footnote{%
Note there are cannot be cycles in the sequence because $p^t$ is decreasing.}
Eventually we must reach a $T$ such that $p^t = p^T$ for all $t \geq T$.
This implies that $p^{\ast}(a) = p^{T}(a)$. Since $p^{T} = p(a,S^{T})$ is a
continuous function of $a$, we conclude that $p^{\ast}(a)$ is a continuous
function of $a$.
\end{proof}

\begin{theorem}
\label{proposition continuity no network}\textbf{(Continuity of GDP with
exogenous production network)} Suppose that $S$ is an exogenously fixed
network. Then, (real) GDP is continuous in the log productivity vector $%
a=\{a_{i}(S)\}_{i\in \mathcal{N},S_{i}\subset \mathcal{N}}$.
\end{theorem}

\begin{proof}
Because the utility function is continuous, real GDP, given by $%
U(C_{1}^{\ast },...,C_{n}^{\ast })$, is a continuous function of the price
vector $P$ and the nominal income of the representative household, $%
Y^{N}=1+\sum_{i=1}^{n}\lambda _{i}\frac{\mu _{i}}{1+\mu _{i}}P_{i}^{\ast
}Y_{i}^{\ast }$. Thus, it suffices to show that $P^{\ast }$ is a continuous
function of $a$, and nominal income is a continuous function of $a$.

We can use the same argument as in the proof of Theorem \ref{proposition
continuity no markups} to show that $P^{\ast }$ is continuous in $a$.

To show that nominal income is continuous in $a$, let $\hat{Y}%
_{i}=P_{i}^{\ast }Y_{i}^{\ast }$ as in the Proof of Lemma \ref%
{lem:fixedPoint}. We showed in Lemma \ref{lem:fixedPoint} that $\hat{Y}$ is
a fixed point of a contraction mapping $\Phi $, and the Jacobian matrix of $%
\Phi $ is a P-matrix. Gale and Nikaido (1965) show that there exists a
globally defined function $\hat{Y}(a,S)$ that is continuously differentiable
in $a$ and which satisfies $\hat{Y}=\Phi (\hat{Y})$. Since nominal income
can be written as $Y^{N}=1+\sum_{i=1}^{n}\lambda _{i}\frac{\mu _{i}}{1+\mu
_{i}}\hat{Y}_{i}^{\ast }$, we conclude that $Y^{N}$ is a continuous function
of $a$.
\end{proof}

\subsection*{Quasi-Submodularity and the Technology Price-Single-Crossing
Condition}

\begin{example}
\label{example counter example}\textbf{(Quasi-submodularity does not imply
the technology-price single-crossing condition) }Consider an economy with
three industries. Suppose that $\mu _{i}=0$ for all $i$ for simplicity. The
production function in each industry is a Cobb-Douglas production function,
but crucially technology does not take a Hicks-neutral form, and the input
shares of an industry depend on the set of inputs used. Namely,%
\begin{equation*}
Y_{i}=\frac{1}{(1-\sum_{j\in S_{i}}\alpha _{ij}(S_{i}))^{1-\sum_{j\in
S_{i}}\alpha (S_{i})_{ij}}\prod_{j\in S_{i}}\alpha _{ij}(S_{i})^{\alpha
_{ij}(S_{i})}}A_{i}(S_{i})L_{i}^{1-\sum_{j\in S_{i}}\alpha
_{ij}(S_{i})}\prod_{j\in S_{i}}X_{ij}^{\alpha _{ij}(S_{i})},
\end{equation*}%
where the conditioning of $\alpha _{ij}$'s on the set of inputs, $S_{i}$,
emphasizes the difference from the family of Cobb-Douglas production
functions with Hicks-neutral technology. Suppose also that industries 1 and
2 use only labor as input and have production functions $Y_{1}=e^{-\epsilon
}L_{1}$ and $Y_{2}=e^{\epsilon }L_{2}$, where $\epsilon >0$. In equilibrium,
the prices for industries 1 and 2 satisfy $p_{1}=-a_{1}=\epsilon $ and $%
p_{2}=-a_{2}=-\epsilon $, where we have also defined $a_{i}$ (for $i=1,2$)
as the log productivities of these two industries. Industry 3, on the other
hand, can choose any one of $\varnothing ,\{1\},\{2\}$ or $\{1,2\}$ as its
set of inputs, with the following input shares
\begin{equation*}
\alpha _{31}(S)=%
\begin{cases}
0\text{ if }1\not\in S_{3} \\
\frac{2}{3}\text{ if }S_{3}=\{1\} \\
\frac{1}{3}\text{ if }S_{3}=\{1,2\}%
\end{cases}%
\text{ and }\,\alpha _{32}(S)=%
\begin{cases}
0\text{ if }2\not\in S_{3} \\
\frac{2}{3}\text{ if }S_{3}=\{2\} \\
\frac{1}{3}\text{ if }S_{3}=\{1,2\}.%
\end{cases}%
\end{equation*}%
The log productivity for industry 3 is given by $a_{3}(\varnothing
)=a_{3}(\{1\})=a_{3}(\{2\})=0$, and $a_{3}(\{1,2\})=\epsilon $.
Quasi-submodularity then requires that for all equilibrium prices $%
(p_{1},p_{2}),$%
\begin{align*}
\frac{2}{3}p_{2}& \leq 0\implies -\epsilon +\frac{1}{3}p_{1}+\frac{1}{3}%
p_{2}\leq \frac{2}{3}p_{1} \\
\frac{2}{3}p_{1}& \leq 0\implies -\epsilon +\frac{1}{3}p_{1}+\frac{1}{3}%
p_{2}\leq \frac{2}{3}p_{2}
\end{align*}%
(and also with strict inequalities). It is straightforward to verify that
these conditions hold. In particular, because $a_{1}=-\epsilon <0$ and $%
a_{2}=\epsilon >0$, we have $p_{1}=\epsilon >0$ and $p_{2}=-\epsilon <0$,
and thus the first condition is always satisfied as $-\epsilon +\frac{1}{3}%
p_{2}\leq \frac{1}{3}p_{1}$, while the second condition is also always
satisfied because we never have $p_{1}\leq 0$. Hence, the unit cost function
for industry 3 is quasi-submodular.

We next show that it does not satisfy the technology-price single-crossing
property. First note that given the equilibrium prices characterized so far,
it is cost-minimizing for industry 3 to choose $S_{3}=\{1,2\}$, since its
log unit cost with $S_{3}=\varnothing $ is 0, with $S_{3}=\{1\}$, it is $%
\frac{2}{3}\epsilon $, with $S_{3}=\{2\}$, it is $-\frac{2}{3}\epsilon $,
and with $S_{3}=\{1,2\}$, it takes its lowest value, $-\epsilon $. Next
consider a change in the technology of industry 2 so that $a_{2}$ increases
to $a_{2}^{\prime }=3\epsilon $. This can be verified to be a positive
technology shock, since we still have $p_{1}>0$ and $p_{2}<0$, and thus the
quasi-submodularity condition continues to be satisfied. But following this
change, the log unit cost for industry 3 from choosing $S_{3}=\{2\}$
declines to $-2\epsilon $, while the log unit cost from $S_{3}=\{1,2\}$
declines only to $-\epsilon +\frac{1}{3}\epsilon -\epsilon =-\frac{5}{3}%
\epsilon >-2\epsilon $. Therefore, following this positive technology shock
industry 3 chooses a smaller set of input suppliers, switching from $\{1,2\}$
to $\{2\}$.
\end{example}

\bigskip

\subsection*{Proofs of Propositions \protect\ref{prop:supermodular}-\protect
\ref{cs ces}}

\begin{proof}[Proof of Proposition \protect\ref{prop:supermodular}]
We first show that the technology-price single-crossing condition holds for
industry $i$ when $P_{-i}^{\prime} \leq P_i$, but $P_i^{\prime} = P_i$. We
then argue that the technology-price single-crossing condition still applies
even when $P_i^{\prime} < P_i$.

Because $F_{i}(S_{i},A_{i}(S_{i}),L_{i},X_{i})$ is supermodular, the profit
function $\Lambda
_{i}(S_{i},A_{i}(S_{i}),P,L_{i},X_{i})=P_{i}F_{i}(S_{i},A_{i}(S_{i}),L_{i},X_{i})-\sum_{j=1}^{n}P_{j}X_{ij}-L_{i}
$ is supermodular in $L_{i},X_{i},A_{i}(S_{i}),S_{i}$ and $-P_{-i}$. If we
take $P_{i}$ as fixed, Topkis (1998) shows that the function
\begin{equation*}
\tilde{\Lambda}_{i}(S_{i},A_{i}(S_{i}),P)=\max_{X_{i},L_{i}}\Lambda
_{i}(L_{i},X_{i},A_{i}(S_{i}),S_{i},P)
\end{equation*}%
is supermodular in $A_{i}(S_{i}),S_{i}$ and $-P_{-i}$. Thus, $\tilde{\Pi}%
_{i} $ will satisfy the following single-crossing condition. For all $%
S_{i}^{\prime }\supset S_{i}$ and all $P^{\prime }$ such that $%
P_{-i}^{\prime }\leq P_{-i}$ and $P_{i}^{\prime }=P_{i}$, we have

\begin{equation*}
\tilde{\Pi}_{i}(S_{i}^{\prime },A_{i}(S_{i}^{\prime }),P)\geq \tilde{\Pi}%
_{i}(S_{i},A_{i}(S_{i}),P)\implies \tilde{\Pi}_{i}(S_{i}^{\prime
},A_{i}(S_{i}^{\prime }),P^{\prime })\geq \tilde{\Pi}%
_{i}(S_{i},A_{i}(S_{i}),P^{\prime }).
\end{equation*}
Let $Q_{i}(P)$ be the demand for good $i$ when the prices are $P$, and write
$\tilde{\Pi}_{i}(S_{i},A_{i}(S_{i}),P)=Q_{i}(P)(P_i-
K_{i}(S_{i},A_{i}(S_{i}),P))$. The cost function satisfies the
single-crossing condition with the following argument:
\begin{gather*}
K_{i}(S_{i}^{\prime },A_{i}(S_{i}^{\prime }),P)\leq
K_{i}(S_{i},A_{i}(S_{i}),P)\iff \\
Q_{i}(P)(P_i-K_{i}(S_{i}^{\prime },A_{i}(S_{i}^{\prime }),P))\geq
Q_{i}(P)(P_i-K_{i}(S_{i},A_{i}(S_{i}),P)).
\end{gather*}%
But the last inequality implies
\begin{gather*}
Q_{i}(P^{\prime })(P_{i}^{\prime }-K_{i}(S_{i}^{\prime },A_{i}(S_{i}^{\prime
}),P^{\prime }))\geq Q_{i}(P^{\prime })(P_{i}^{\prime
}-K_{i}(S_{i},A_{i}(S_{i}),P^{\prime }))\iff \\
K_{i}(S_{i}^{\prime },A_{i}(S_{i}^{\prime }),P^{\prime })\leq
K_{i}(S_{i},A_{i}(S_{i}),P^{\prime }),
\end{gather*}
which proves that the technology-price single-crossing condition for
industry $i$ when $P_i = P_i^{\prime}$ and $P^{\prime}_{-i} \leq P_{-i}$.

To see that this generalizes to cases where $P_i^{\prime} < P_i$, let $%
P^{\prime \prime}$ be a price vector such that $P^{\prime \prime}_{j} =
P^{\prime}_j$ for all $j \neq i$, and $P^{\prime \prime}_i = P_i$. Assume
that $K_{i}(S_{i}^{\prime },A_{i}(S_{i}^{\prime }),P)\leq
K_{i}(S_{i},A_{i}(S_{i}),P)$, and note that

\begin{enumerate}
\item Since $P^{\prime \prime }\leq P$ and $P_{i}^{\prime \prime }=P_{i}$,
our argument above yields the inequality $K_{i}(S_{i}^{\prime
},A_{i}(S_{i}^{\prime }),P^{\prime \prime })\leq
K_{i}(S_{i},A_{i}(S_{i}),P^{\prime \prime })$.

\item Since $K_{i}$ does not depend on the $i^{th}$ coordinate of the price
vector, we have that $K_{i}(\cdot ,\cdot ,P^{\prime \prime })=K_{i}(\cdot
,\cdot ,P^{\prime })$
\end{enumerate}

From the above two observations, we conclude that $K_{i}(S_{i}^{\prime
},A_{i}(S_{i}^{\prime }),P^{\prime })\leq K_{i}(S_{i},A_{i}(S_{i}),P^{\prime
})$.
\end{proof}
\newline

\begin{proof}[Proof of Proposition \protect\ref{prop:CD}]
Since the price single-crossing condition is preserved by monotonic
transformation, it suffices to show that it is satisfied by the log unit
cost function. To show that the log unit cost function satisfies the
single-crossing conditions, let $S_{i}\subset S_{i}^{\prime }$ and $%
p^{\prime} \leq p$ and note that
\begin{gather*}
k_{i}(S_{i}^{\prime },a_{i},p)-k_{i}(S_{i},a_{i},p)\leq 0\iff \\
\sum_{j\in S_{i}^{\prime }}\alpha _{ij}p_{j}-\sum_{j\in S_{i}}\alpha
_{ij}p_{j}-a_{i}(S_{i}^{\prime })+a_{i}(S_{i})\leq 0\iff \\
\sum_{j\in S_{i}^{\prime }-S_{i}}\alpha _{ij}p_{j}-a_{i}(S_{i}^{\prime
})+a_{i}(S_{i})\leq 0\implies \\
\sum_{j\in S_{i}^{\prime }-S_{i}}\alpha
_{ij}p^{\prime}_{j}-a_{i}(S_{i}^{\prime })+a_{i}(S_{i})\leq 0\iff \\
\sum_{j\in S_{i}^{\prime }}\alpha _{ij}p^{\prime}_{j}-\sum_{j\in
S_{i}}\alpha _{ij}p^{\prime}_{j}-a_{i}(S_{i}^{\prime })+a_{i}(S_{i})\leq
0\iff \\
k_{i}(S_{i}^{\prime },a_{i},p^{\prime})-k_{i}(S_{i},a_{i},p^{\prime})\leq 0.
\end{gather*}
\end{proof}
\newline

\begin{proof}[Proof of Proposition \protect\ref{cs ces}]
In this case, the technology function $A_{i}$ maps a set $S_{i}$ to a vector
$(A_{ij})_{j\in S_{i}}$. Write the CES cost function for firm $i$ as
\begin{equation*}
K_{i}(S_{i},A_{i},P)=((1-\sum_{j\in S_{i}}\alpha _{ij})^{\sigma}+\sum_{j\in
S_{i}}\alpha _{ij}^{\sigma}(\frac{P_{j}}{A_{ij}})^{1-\sigma })^{\frac{1}{1-\sigma }}.
\end{equation*}

Since the single-crossing condition is preserved by monotone
transformations, it suffices to consider a monotone transformation of $K_{i}$%
. We split the analysis into two cases:

Case 1: $\sigma < 1$

In this case, we can raise the cost function to the power $1-\sigma$ to
obtain $(K_i(S_i,A_i,P))^{1-\sigma} = (1-\sum_{j \in S_i} \alpha_{ij})^{\sigma} +
\sum_{ j \in S_i} \alpha_{ij}^{\sigma} (\frac{P_j}{A_{ij}})^{1-\sigma}$. Since $%
1-\sigma > 0$, minimizing $K_i$ is equivalent to minimizing $%
(K_i(S_i,A_i,P))^{1-\sigma}$. We will show that $(K_i(S_i,A_i,P))^{1-\sigma}$
satisfies the single-crossing condition. %
%To show that $K_i^{1-\sigma}$ is submodular in $(S_i,A_i(S_i))$, note that $%
%\sum_{ j \in S_i} \alpha_{ij}^{\sigma} (\frac{P_j}{A_{ij}})^{1-\sigma}$ is
%modular in $S_i$, has decreasing differences in $S_i, A_{ij}$, and satisfies
%$\frac{\partial (K_i^{1-\sigma})^2}{\partial A_{ij} A_{ij^{\prime}}} = 0$
%for $j \neq j^{\prime}$. Thus, it suffices to show that $g(S_{i})\equiv
%(1-\sum_{j\in S_{i}}\alpha _{ij})^{\sigma }$ is submodular as a function of $%
%S_{i}$. Let $T_{i}$ be an arbitrary set and let $s=1-\sum_{j\in S_{i}}\alpha
%_{ij},t=1-\sum_{j\in T_{i}}\alpha _{ij},u=1-\sum_{j\in S_{i}\cup
%T_{i}}\alpha _{ij},v=1-\sum_{j\in S_{i}\cap T_{i}}\alpha _{ij}$. Note that $%
%u\leq s,t\leq v$. Furthermore, $s-u=v-t=\sum_{j\in T_{i}-S_{i}}\alpha _{ij}$%
%. We can write
%\begin{gather*}
%g(S_{i}\cup T_{i})-g(S_{i})=u^{\sigma }-s^{\sigma }=u^{\sigma
%}-(u+(s-u))^{\sigma }\leq  \notag \\
%t^{\sigma }-(t+(s-u))^{\sigma }=t^{\sigma }-(t+(v-t))^{\sigma }=t^{\sigma
%}-v^{\sigma }=g(T_{i})-g(S_{i}\cap T_{i}).
%\end{gather*}%
%where the inequality follows from the fact that the function $x^{\sigma }$
%is concave in $x$ and that $u\leq t$ (see Figure 1).
%%\begin{figure}[tbp]
%\label{tikz picture} \centering
%\begin{tikzpicture}[my plot/.style={
%thick,
%smooth,
%samples=100,
%domain=0.1:5},
%my grid/.style={dashed,opacity=0.5, every node/.style={black,opacity=1}},
%my axis/.style={latex-latex}]
%\draw[my plot] (0,0) plot (\x,{ln(\x)});
%\coordinate (start plot) at (0.1,{ln(0.1)}); % domain start
%\coordinate (end plot) at (5,{ln(5)}); % domain end
%\draw[my axis] ([shift={(-0.5cm,0.5cm)}]start plot |- end plot) node[left] {$g(x)=x^{\sigma}$} |- node[coordinate](origin){} ([shift={(0.5cm,-0.5cm)}]start plot -| end plot); %creates the axis a little bigger than the plot and also sepatared
%\def\x{0.5}\def\t{3}\def\p{0.55}
%\def\s{1.5}\def\y{4}
%% define the x, y and p values
%\coordinate (Ux) at (\x,{ln(\x)}); % set the u(x) coordinate
%\coordinate (Uy) at (\y,{ln(\y)}); % set the u(y) coordinate
%\coordinate (Us) at (\s,{ln(\s)}); % set the u(y) coordinate
%\coordinate (Ut) at (\t,{ln(\t)}); % set the u(y) coordinate
%%\coordinate (Up) at ({\p*\x+(1-\p)*\y},{ln(\p*\x+(1-\p)*\y)}); % set the u(p*x+(1-p)*y) coordinate
%%\draw (Ux) -- coordinate[pos=1-\p] (Up-mid) (Uy); % set the coordinate on the linear curve
%%\path let \p1=(Up-mid), \n1={pow(e,\y1*0.03514)} in (28.4576*\n1,\y1) coordinate (Up-mid2); %this is the most tricky part explained in the answer
%% with everything set, just draw the lines:
%\draw[my grid] (Ux) |- node[below]{$u$} (origin) |- node[left]{$u^{\sigma}$} cycle;
%\draw[my grid] (Uy) |- node[below]{$v$} (origin) |- node[left]{$v^{\sigma}$} cycle;
%\draw[my grid] (Us) |- node[below]{$s$} (origin) |- node[left]{$s^{\sigma}$} cycle;
%\draw[my grid] (Ut) |- node[below]{$t$} (origin) |- node[left]{$t^{\sigma}$} cycle;
%%\draw[my grid] (Up) |- node[below right,font=\scriptsize]{$px+(1-p)y$} (origin) |- node[left]{$u(px+(1-p)y)$} cycle;
%%\draw[my grid] (Up-mid2) |- node[below,font=\scriptsize]{$c(F,u)$} (origin) |- node[left]{$pu(x)+(1-p)u(y)$} cycle;
%%\draw[my grid] (Up-mid) -- (Up-mid2);
%\end{tikzpicture}
%\caption{This figure illustrates that when $g(x) = x^{\protect\sigma}$ with $%
%0 < \protect\sigma <1 $, and $u,s,t,v$ are defined so that $u \leq s,t \leq
%v $, $s-u = t-v$, then $g(u)-g(s) \leq g(t)-g(v)$.}
%\end{figure}
%
%To show that $K_i^{1-\sigma}$ satisfies the technology single-crossing
%condition,
Let $S_i \subset S_i^{\prime}$ and $P^{\prime} \leq P$. We can derive the
chain of implications
\begin{gather*}
(K_i(S_i^{\prime },A_i(S_i^{\prime }),P))^{1-\sigma} -
(K_i(S_i,A_i(S_i),P))^{1-\sigma} \leq 0 \implies  \notag \\
((1-\sum_{j \in S_i^{\prime }} \alpha_{ij})^{\sigma} - (1-\sum_{j \in S_i}
\alpha_{ij})^{\sigma}) + \sum_{j \in S_i^{\prime }-S_i} \alpha_{ij}^{\sigma} (\frac{P_j}{A_{ij}%
})^{1-\sigma} \leq 0 \implies  \notag \\
((1-\sum_{j \in S_i^{\prime }} \alpha_{ij})^{\sigma} - (1-\sum_{j \in S_i}
\alpha_{ij})^{\sigma}) + \sum_{j \in S_i^{\prime }-S_i} \alpha_{ij}^{\sigma} (\frac{%
P^{\prime}_j}{A_{ij}})^{1-\sigma} \leq 0 \implies  \notag \\
(K_i(S_i^{\prime },A_i(S_i^{\prime }),P^{\prime}))^{1-\sigma} -
(K_i(S_i,A_i(S_i),P^{\prime }))^{1-\sigma} \leq 0.
\end{gather*}
so the single-crossing condition is satisfied.
%let $S_i^{\prime }\supset S_i$ and $A^{\prime
%}\geq A$ such that $A_{i0}^{\prime }= A_{i0}$ for all $i$. Write
%\begin{gather}
%K_i^{1-\sigma}(S_i^{\prime },A_i(S_i^{\prime }),P) -
%K_i^{1-\sigma}(S_i,A_i(S_i),P) \leq 0 \implies  \notag \\
%\frac{1}{A_{i0}^{1-\sigma}} ((1-\sum_{j \in S_i^{\prime }}
%\alpha_{ij})^{\sigma} - (1-\sum_{j \in S_i} \alpha_{ij})^{\sigma}) + \sum_{j
%\in S_i^{\prime }-S_i} \alpha_{ij}^{\sigma} (\frac{P_j}{A_{ij}})^{1-\sigma}
%\leq 0 \implies  \notag \\
%\frac{1}{A_{i0}^{^{\prime }1-\sigma}} ((1-\sum_{j \in S_i^{\prime }}
%\alpha_{ij})^{\sigma} - (1-\sum_{j \in S_i} \alpha_{ij})^{\sigma}) + \sum_{j
%\in S_i^{\prime }-S_i} \alpha_{ij}^{\sigma} (\frac{P_j}{A^{\prime }_{ij}}%
%)^{1-\sigma} \leq 0 \implies  \notag \\
%K_i^{1-\sigma}(S_i^{\prime },A^{\prime }_i(S_i^{\prime }),P) -
%K_i^{1-\sigma}(S_i,A^{\prime }_i(S_i),P^{\prime }) \leq 0.
%\end{gather}
%Reasoning analogously, we conclude that if $P^{\prime }\leq P$ we have

Case 2: $\sigma > 1$

In this case, we can raise the cost function to the power $1-\sigma$ to
obtain $(K_i(S_i,A_i,P))^{1-\sigma} = (1-\sum_{j \in S_i} \alpha_{ij})^{\sigma} +
\sum_{ j \in S_i} \alpha_{ij}^{\sigma} (\frac{P_j}{A_{ij}})^{1-\sigma}$. Since $%
1-\sigma < 0$, minimizing $K_i$ is equivalent to \emph{maximizing} $%
(K_i(S_i,A_i,P))^{1-\sigma}$. We need to show that $(K_i(S_i,A_i,P))^{1-%
\sigma}$ satisfies a \emph{reverse} single-crossing condition. For all $S_i
\subset S_i^{\prime}$ and $P^{\prime} \leq P$ $(K_i(S_i^{\prime
},A_i(S_i^{\prime }),P))^{1-\sigma} - (K_i(S_i,A_i(S_i),P))^{1-\sigma} \geq
0 \implies (K_i(S_i^{\prime },A_i(S_i^{\prime }),P^{\prime }))^{1-\sigma} -
(K_i(S_i,A_i(S_i^{\prime }),P^{\prime }))^{1-\sigma} \geq 0.$

Let $S_i \subset S_i^{\prime}$ and $P^{\prime} \leq P$. Since $(\frac{P_j}{%
A_{ij}})^{1-\sigma} \leq (\frac{P_j^{\prime}}{A_{ij}})^{1-\sigma}$, we
obtain the chain of implications

\begin{gather*}
(K_{i}(S_{i}^{\prime },A_{i}(S_{i}^{\prime }),P))^{1-\sigma
}-(K_{i}(S_{i},A_{i}(S_{i}),P))^{1-\sigma }\geq 0\implies \\
((1-\sum_{j\in S_{i}^{\prime }}\alpha _{ij})^{\sigma}-(1-\sum_{j\in S_{i}}\alpha
_{ij})^{\sigma})+\sum_{j\in S_{i}^{\prime }-S_{i}}\alpha _{ij}^{\sigma}(\frac{P_{j}}{A_{ij}}%
)^{1-\sigma }\geq 0\implies \\
((1-\sum_{j\in S_{i}^{\prime }}\alpha _{ij})^{\sigma}-(1-\sum_{j\in S_{i}}\alpha
_{ij})^{\sigma})+\sum_{j\in S_{i}^{\prime }-S_{i}}\alpha _{ij}^{\sigma}(\frac{P_{j}^{\prime }}{%
A_{ij}})^{1-\sigma }\geq 0\implies \\
(K_{i}(S_{i}^{\prime },A_{i}(S_{i}^{\prime }),P^{\prime }))^{1-\sigma
}-(K_{i}(S_{i},A_{i}(S_{i}),P^{\prime }))^{1-\sigma }\leq 0.
\end{gather*}%
so the single-crossing condition is satisfied.
\end{proof}

\subsection*{Borel-Cantelli Lemmas}

\begin{lemma}[First Borel-Cantelli Lemma]
\label{lemmab1} Suppose that $\{Z_{n}\}_{n\in \mathbb{N}}$ is a sequence of
random variables. If for any fixed $\epsilon >0$ we have
\begin{equation*}
\sum_{n=1}^{\infty }\Pr [Z_{n}>\epsilon ]<\infty
\end{equation*}%
then $\limsup_{n\rightarrow \infty }Z_{n}\leq 0$ almost surely.
\end{lemma}

\begin{lemma}[Second Borel-Cantelli Lemma]
\label{lemmab2} Suppose that $\{Z_{n}\}_{n\in \mathbb{N}}$ is a sequence of
independent random variables. If
\begin{equation*}
\sum_{n=1}^{\infty }\Pr [Z_{n}\geq 0]=\infty
\end{equation*}%
then $\limsup_{n\rightarrow \infty }Z_{n}\geq 0$ almost surely.
\end{lemma}

\subsection*{Distributions that Satisfy Assumption \protect\ref{assumption
logproductivity negative}}

\begin{proposition}
\label{proposition gumbel} Let $\{a_{i}(S_{i}(t))\}_{S\subset \{1,...,t\}}$
be a random variable where each $a_{i}(S_{i}(t))$ is an independently drawn
Gumbel random variable with cdf $\Phi (z;\mu ,\sigma )=e^{-e^{-\frac{z}{%
\sigma }}}$. Then Assumption \ref{assumption logproductivity negative} is
satisfied with $D=\sigma \log 2$.
\end{proposition}

\begin{proof}
We can write $\lim_{n\rightarrow \infty }\frac{\max_{S\subset
\{1,...,t\}}a_{i}(S_{i})}{t}$ as $\limsup_{n\rightarrow \infty }\frac{Z_{n}}{%
\log _{2}n+1}$ where $n=2^{t-1}$ and $Z_{i}$ is a Gumbel random variable.
The probability that $Z_{n}$ is above $\mu +\sigma \log n$ is equal to $%
1-e^{-e^{-\log n}}=1-e^{-\frac{1}{n}}$. Since $1-e^{-z}=z+o(z)$, there
exists a constant $\kappa >0$, and an integer $N$ such that for all $n\geq N$
we have $1-e^{-\frac{1}{n}}\geq \frac{\kappa }{n} $. Since $%
\sum_{n=N}^{\infty }\Pr [Z_{n}>\mu +\sigma \log n]>\sum_{n=N}^{\infty }\frac{%
\kappa }{n}=\infty $ and the variables $Z_{1},...,Z_{n}$ are independent, we
can use Lemma \ref{lemmab2} to derive that $\limsup_{n\rightarrow \infty }%
\frac{Z_{n}}{\log n}\geq \sigma $ almost surely. Using the fact that $%
\lim_{n\rightarrow \infty }\frac{\log n}{\log _{2}n+1}=\log 2$, we conclude
that $\frac{Z_{n}}{\log _{2}n+1}\geq \sigma \log 2.$

To prove the reverse inequality, let $\epsilon >0$ be arbitrary. The
probability that $Z_{n}$ is above $\mu +\sigma (1+\epsilon )\log n$ is $%
1-e^{-e^{-(1+\epsilon )\log n}}=1-e^{-n^{-1-\epsilon }}$. Since $%
1-e^{-z}=z+o(z)$, there exists a constant $\kappa >0$ and an integer $N$
such that for all $n\geq N$, we have $1-e^{-\frac{1}{n^{1+\epsilon }}}\leq
\frac{\kappa }{n^{1+\epsilon }}.$ Since $\epsilon >0$ is arbitrary and $%
\sum_{n=N}^{\infty }\Pr [Z_{n}\geq \mu +\sigma (1+\epsilon )\log n]\leq
\sum_{n=N}^{\infty }\kappa n^{-1-\epsilon }<\infty $, Lemma \ref{lemmab1}
implies that $\limsup_{n\rightarrow \infty }\frac{Z_{n}}{\log n}\leq \sigma $
almost surely. The union of two almost-sure events occurrs almost surely, so
we can conclude that $\limsup_{n\rightarrow \infty }\frac{Z_{n}}{\log n}%
=\sigma $ or equivalently $\lim_{n\rightarrow \infty }\frac{\max
(Z_{1},...,Z_{n})}{\log n}=\sigma $ almost surely. Using the fact that $%
\lim_{n\rightarrow \infty }\frac{\log n}{\log _{2}n+1}=\log 2$, we obtain $%
\lim_{n\rightarrow \infty }\frac{\max (Z_{1},...,Z_{n})}{\log _{2}n}%
=\lim_{n\rightarrow \infty }\frac{\max (Z_{1},...,Z_{n})}{\log n}\log
2=\sigma \log 2.$
\end{proof}
\newline

\begin{proposition}
\label{proposition exponential distribution}Let $\{a_{i}(S_{i}(t))\}_{S%
\subset \{1,...,t\}}$ be a random variable where each $a_{i}(S_{i}(t))$ is
an independently drawn exponential random variable with cdf $\Phi (z;\nu
)=1-e^{-\nu z}$. Then Assumption \ref{assumption logproductivity negative}
is satisfied with $D=\frac{1}{\nu }\log 2$.
\end{proposition}

\begin{proof}
We can write $\lim_{n\rightarrow \infty }\frac{\max_{S\subset
\{1,...,t\}}a_{i}(S_{i})}{t}$ as $\limsup_{n\rightarrow \infty }\frac{Z_{n}}{%
\log _{2}n+1}$ where $n=2^{t-1}$ and $Z_{i}$ is an exponential random
variable . The probability that $Z_{n}$ is above $\frac{\log n}{\nu }$ is
equal to $e^{-\log n}=n^{-1}$. Since $\sum_{n=1}^{\infty }n^{-1}=\infty $
and the variables $Z_{1},...,Z_{n}$ are independent, Lemma \ref{lemmab2}
implies that $\limsup_{n\rightarrow \infty }\frac{Z_{n}}{\log n}\geq \frac{1%
}{\nu }$ almost surely. Since $\lim_{n\rightarrow \infty }\frac{\log n}{\log
_{2}n+1}=\log 2$, we conclude that $\limsup_{n\rightarrow \infty }\frac{Z_{n}%
}{\log n}\geq \frac{\log 2}{\nu }$ almost surely.

To prove the reverse inequality, let $\epsilon >0$ be arbitrary. The
probability that $Z_{n}$ is above $\frac{\log n+\epsilon \log n}{\nu }$ is $%
e^{-\log n-\epsilon \log n}=n^{-1-\epsilon }$. Since $\epsilon >0$ is
arbitrary and $\sum_{n=1}^{\infty }n^{-1-\epsilon }<\infty $, Lemma \ref%
{lemmab1} implies $\limsup_{n\rightarrow \infty }\frac{Z_{n}}{\log n}\leq
\frac{1}{\nu }$ almost surely. The intersection of two almost-sure events
occurrs almost surely, so we can conclude that $\limsup_{n\rightarrow \infty
}\frac{Z_{n}}{\log n}=\frac{1}{\nu }$ or equivalently $\lim_{n\rightarrow
\infty }\frac{\max (Z_{1},...,Z_{n})}{\log n}=\frac{1}{\nu }$ almost surely.
Since $\lim_{n\rightarrow \infty }\frac{\log n}{\log _{2}n+1}=\log 2$, we
obtain $\lim_{n\rightarrow \infty }\frac{\max (Z_{1},...,Z_{n})}{\log _{2}n+1%
}=\lim_{n\rightarrow \infty }\frac{\max (Z_{1},...,Z_{n})}{\log n}\log 2=%
\frac{1}{\nu }\log 2.$
\end{proof}
\newline

\begin{proposition}
\label{proposition correlated} Let $\{a_{i}(S_{i}(t))\}_{S\subset
\{1,...,t\}}$ be a random variable where each $a_{i}(S_{i}(t))=\sum_{j\in
S_{i}(t)}\tilde{a}_{j}$ and each $\tilde{a}_{j}$ is an independent random
variable which is equal to $-1$ with probability $\frac{1}{2}$ and equal to $%
1$ with probability $\frac{1}{2}$. Assumption \ref{assumption
logproductivity negative} is satisfied with $D=\frac{1}{2}.$%.
\end{proposition}

\begin{proof}
We can write $a^{\ast }(t)=\max_{S\subset \{1,...,t\}}a_{i}(S_{i}(t))=|j:%
\tilde{a}_{j}=1|=\sum_{j=1}^{t}X_{j}$ where $X_{j}=%
\begin{cases}
1\text{ if }\tilde{a}_{j}\geq 0 \\
0\text{ otherwise.}%
\end{cases}%
$ is an independent Bernoulli random variable taking values 0 or 1 with
probability $\frac{1}{2}$. Recall that if $X_{1},...,X_{n}$ are indpendent
random variables in the interval $[0,1]$, we have the following Chernoff
bound%
\begin{equation*}
\Pr (|\frac{1}{n}\sum_{i=1}^{n}X_{i}-\frac{1}{n}\sum_{i=1}^{n}\mathbb{E}%
[X_{i}]|\geq \epsilon )\leq 2e^{-2n\epsilon ^{2}}.
\end{equation*}%
Using this Chernoff bound, we have%
\begin{equation*}
\Pr (|\frac{1}{t}\sum_{i=1}^{t}\tilde{a}_{i}-\frac{1}{2}|\geq \epsilon )\leq
2e^{-2t\epsilon ^{2}}.
\end{equation*}%
Since $\sum_{t=0}^{\infty }2e^{-2t\epsilon ^{2}}$ converges, from Lemma \ref%
{lemmab1} $\limsup_{t}|\frac{1}{t}\sum_{i=1}^{t}X_{i}-\frac{1}{2}|\leq 0$
almost surely. But since absolute values cannot be negative, we also have $%
\liminf_{t}|\frac{1}{t}\sum_{i=1}^{t}X_{i}-\frac{1}{2}|=0$. Therefore, the
limit $\lim_{t\rightarrow \infty }\frac{\max_{S \subset \{1,...,t\} }
a_{i}(S_{i}(t))}{t}=\frac{1}{2}$ almost surely.
\end{proof}

\bigskip %
%\begin{proposition}
%\label{proposition gumbel2} Let $\{a_{i}(S_{i}(t)):S_{i}(t)=\cup
%_{k=1}^{K}S_{i,k}(t),S_{i,k}(t)\subset V_{k}\text{ and }S_{i,k^{\prime
%}}(t)\neq \varnothing \text{ for each }k^{\prime }\in R_{i}\backslash
%\{i\}\} $ be a random variable where each $a_{i}(S_{i}(t))$ is an
%independently drawn Gumbel random variable with cdf $\Phi (z;\mu ,\sigma
%)=e^{-e^{-\frac{z}{\sigma }}}$. Then
%\begin{equation*}
%\lim_{t\rightarrow \infty }\frac{\max_{S_{i}(t)}a_{i}(S_{i}(t))}{t}=K\sigma
%\log 2\text{ almost surely.}
%\end{equation*}
%\end{proposition}
%
%\begin{proof}
%The value $\max_{S_{j}(t)}a_{j}(S_{j}(t))$ is the maximum of the $%
%\sum_{r=0}^{|R_{j}|}(-1)^{r}\binom{|R_{j}|}{r}2^{(t-T(\epsilon ))(K-r)}$
%independently distributed Gumbel random variables.\footnote{%
%The expression $\sum_{r=0}^{|R_{i}|}(-1)^{r}\binom{|R_{i}|}{r}%
%2^{(t-T(\epsilon ))K-r}$ is derived through the exclusion-inclusion
%principle. To count the sets that firm $j$ can choose, start from all
%possible $2^{(t-T(\epsilon ))K}$ subsets of inputs that contain $%
%\{1,...,KT(\epsilon )\}$. However, we must remove the $\binom{|R_{j}|}{1}%
%2^{(t-T(\epsilon ))(K-1)}$ sets which do not contain an element in some
%category $k\in R_{i}$ that the firm needs to produce. If we stopped here, we
%would be removing sets where two or more categories are absent multiple
%times, and thus miscounting the number of available sets. Thus, we must add
%the $\binom{|R_{j}|}{2}2^{(t-T(\epsilon ))(K-2)}$ with two or more sets
%being absent back into the sum of total feasible sets. Again, this leads to
%a miscount because sets where three or more categories are absent are being
%added back to the sum multiple times. Thus, we must remove the $\binom{%
%|R_{j}|}{3}2^{(t-T(\epsilon ))(K-3)}$ sets where 3 or more categories are
%absent. We continue with this procedure until we reach the set where all
%elements in all categories in $R_{j}$ are absent.} Since $|R_{j}|\leq K$,
%and $K$ is a constant independent of $t$, we have that for large $t$ the
%expression $\sum_{r=0}^{|R_{j}|}(-1)^{r}\binom{|R_{j}|}{r}2^{(t-T(\epsilon
%))(K-r)}$ is bounded below by $E2^{tK}$ where $E$ is a constant independent
%of $t$. Thus, the value $\max_{S_{j}(t)}a_{j}(S_{j}(t))$ is the maximum of $%
%\psi (t)$ independent Gumbel random variables, where $E2^{(t-T(\epsilon
%))K}\leq \psi (t)\leq 2^{(t-T(\epsilon ))K}$.
%
%Proposition \ref{proposition gumbel} implies that $\lim_{t\rightarrow \infty
%}\frac{\max_{S_{j}(t)}a_{j}(S_{j}(t))}{\log \psi (t)}=\sigma \log 2$ almost
%surely. Since $E2^{(t-T(\epsilon ))K}\leq \psi (t)\leq 2^{(t-T(\epsilon ))K}$%
%, we have
%
%\begin{equation*}
%\lim_{t\rightarrow \infty }\frac{\max_{S_{j}(t)}a_{j}(S_{j}(t))}{%
%(t-T(\epsilon ))K\log 2}\leq \lim_{t\rightarrow \infty }\frac{%
%\max_{S_{j}(t)}a_{j}(S_{j}(t))}{\log \psi (t)}\leq \lim_{t\rightarrow \infty
%}\frac{\max_{S_{j}(t)}a_{j}(S_{j}(t))}{\log E+(t-T(\epsilon ))K\log 2}.
%\end{equation*}%
%Since $T(\epsilon )$ and $E$ are constants independent of $t$,%
%\begin{equation*}
%\lim_{t\rightarrow \infty }\frac{\max_{S_{j}(t)}a_{j}(S_{j}(t))}{\log \psi
%(t)}=\lim_{t\rightarrow \infty }\frac{\max_{S_{j}(t)}a_{j}(S_{j}(t))}{tK\log
%2},
%\end{equation*}%
%and we can therefore conclude that%
%\begin{equation*}
%\lim_{t\rightarrow \infty }\frac{\max_{S_{j}(t)}a_{j}(S_{j}(t))}{t}=K\sigma
%\log 2\text{ almost surely.}
%\end{equation*}
%\end{proof}%\newline
%
%\begin{proposition}
%\label{proposition exponential 2}Let $\{a_{i}(S_{i}(t)):S_{i}(t)=\cup
%_{k=1}^{K}S_{i,k}(t),S_{i,k}(t)\subset V_{k}\text{ and }S_{i,k^{\prime
%}}(t)\neq \varnothing \text{ for each }k^{\prime }\in R_{i}\backslash
%\{i\}\} $ be a random variable where each $a_{i}(S_{i}(t))$ is an
%independently drawn exponential random variable with cdf $\Phi (z;\lambda
%)=1-e^{-\lambda z}$. Then
%\begin{equation*}
%\lim_{t\rightarrow \infty }\frac{\max_{S_{i}(t)}a_{i}(S_{i}(t))}{t}=K\frac{1%
%}{\lambda }\log 2\text{ almost surely.}
%\end{equation*}
%\end{proposition}
%
%\begin{proof}
%The proof is analogous to the proof of Proposition \ref{proposition gumbel2}.
%\end{proof}

\begin{proposition}
\label{proposition gumbel3} Let $a_{i}(S_{i})=\sum_{j\in
S_{i}}b_{j}+\epsilon _{i}(S_{i})$ where each $b_{j}$ is drawn identically
and independently from the same distribution, and where $\epsilon (S)$ is an
independently drawn Gumbel random variable with cdf $\Phi (z;\mu ,\sigma
)=e^{-e^{-\frac{z}{\sigma }}}$. Assume that $\mathbb{E}[b_{j}|b_{j}\geq 0]$
is finite. Then
\begin{equation*}
\Pr (b\geq 0)\sigma \log 2\leq \lim \inf_{t\rightarrow \infty }\frac{%
\max_{S_{i}(t)}a_{i}(S_{i}(t))}{t}
\end{equation*}%
\begin{equation*}
\lim \sup_{t\rightarrow \infty }\frac{\max_{S_{i}(t)}a_{i}(S_{i}(t))}{t}\leq
\mathbb{E}[b_{j}|b_{j}\geq 0]+\sigma \log 2
\end{equation*}%
almost surely.
\end{proposition}

\begin{proof}
Let $\mathcal{S}^{+}(t)$ be the collection of sets $S_{i}(t)$ such that $%
b_{j}\geq 0$ and $j\leq t$ for all $j\in S_{i}(t)$. That is, there are no
negative elements $b_{j}$ for any set $S_{i}\in \mathcal{S}^{+}$. Let $\chi
(t)=|\{j:b_{j}\geq 0\}|$ and note that the size of $\mathcal{S}^{+}(t)$ is $%
2^{\chi (t)}$. Applying a Chernoff bound, we obtain that $\lim_{t\rightarrow
\infty }\frac{\chi (t)}{t}\geq \Pr (b_{j}\geq 0)$ almost surely. Using
proposition \ref{proposition gumbel}, we obtain that
\begin{equation*}
\lim_{t\rightarrow \infty }\frac{\max_{S_{i}(t)\in \mathcal{S}%
^{+}(t)}\epsilon _{i}(S_{i}(t))}{t}\geq \Pr (b\geq 0)\sigma \log 2.
\end{equation*}

Since $a_{i}(S_{i}(t))=\sum_{j\in S_{i}(t)}b_{j}+\epsilon _{i}(S_{i}(t))\geq
\epsilon _{i}(S_{i}(t))$ for every $S_{i}(t)\in \mathcal{S}^{+}$, we have
that $\lim \inf_{t\rightarrow \infty }\frac{\max_{S_{i}(t)\in \mathcal{S}%
^{+}(t)}a_{i}(S_{i}(t))}{t}\geq \lim_{t\rightarrow \infty }\frac{%
\max_{S_{i}(t)\in \mathcal{S}^{+}(t)}\epsilon _{i}(S_{i}(t))}{t}$. We
conclude that $\Pr (b\geq 0)\sigma \log 2\leq \lim \inf_{t\rightarrow \infty
}\frac{\max_{S_{i}(t)}a_{i}(S_{i}(t))}{t}.$

To prove the other side of the inequality, note that $\frac{%
\max_{S_{i}(t)}a_{i}(S_{i}(t))}{t}\leq \frac{\max_{S_{i}(t)}\sum_{j\in
S_{i}(t)}b_{j}}{t}+\frac{\max_{S_{i}(t)}\epsilon_{i}(S_{i}(t))}{t}$. The
first term converges to $\mathbb{E}[b_{j}|b_{j}\geq 0]$ by the law of large
numbers. The second term converges to $\sigma \log 2$ by Proposition \ref%
{proposition gumbel}. Thus, $\lim \sup_{t\rightarrow \infty }\frac{%
\max_{S_{i}(t)}a_{i}(S_{i}(t))}{t}\leq \mathbb{E}[b_{j}|b_{j}\geq 0]+\sigma
\log 2$.
\end{proof}

\begin{corollary}
\label{proposition gumbel4} If $b_{j}$ is defined as in Proposition \ref%
{proposition gumbel3} is drawn from a distribution satisfying $\Pr
(b_{j}\geq 0)>0,\mathbb{E}[b_{j}|b_{j}\geq 0]<\infty $, then there exist
finite and positive constants $\overline{D}>\underline{D}$ such that
\begin{equation*}
\underline{D}\leq \lim \inf_{t\rightarrow \infty }\frac{%
\max_{S_{i}(t)}a_{i}(S_{i}(t))}{t}
\end{equation*}%
\begin{equation*}
\lim \sup_{t\rightarrow \infty }\frac{\max_{S_{i}(t)}a_{i}(S_{i}(t))}{t}\leq
\overline{D}
\end{equation*}%
almost surely.
\end{corollary}

\subsection*{No Growth without Choice of Input Combinations}

We next state and prove a theorem that shows that, in contrast to our main
growth result, Theorem \ref{proposition growth}, when new goods are
introduced into the supply chain at random (or with minimal choice), there
will be zero growth in the long run.

\begin{theorem}
\label{theorem no growth without endogenous choice}\textbf{(No growth
without selection) }Suppose that Assumptions \ref{assumption1prime}, \ref%
{assumption2prime}, \ref{assumption logproductivity negative} and \ref%
{assumption leontief} hold. At each time $t\geq 1$, a set of suppliers $%
S_{i}^{O}(t)\subset \{1,\ldots ,t\}$ for each $i=1,2,\ldots ,n$ is selected
uniformly at random. Then each industry $i$ chooses between its existing set
of suppliers, $S_{i}^{\ast }(t-1)$, and $S_{i}^{O}(t)$. Then $g^{\ast }=0$
almost surely.
\end{theorem}

\begin{proof}[Proof of Theorem \protect\ref{theorem no growth without
endogenous choice}]
Let $S_{i}^{O}(t)$ be the input combination available to industry $i$ at
time $t$. Let $S_{i}^{\ast }(t)$ be the set that minimizes industry $i$'s
unit cost when it chooses between $S_{i}^{\ast }(t-1)$ and $S_{i}^{O}(t)$.
Clearly,%
\begin{equation*}
a_{i}(S_{i}^{\ast }(t))\leq \max_{j\in \{1,...,t\}}\max_{\tau \in
\{1,...,t\}}a_{j}(S_{j}^{O}(\tau )).
\end{equation*}%
Therefore, denoting the equilibrium log productivity sequence by $a(S^{\ast
}(t))$, we have%
\begin{equation*}
-\frac{\pi (t)}{t}=\frac{1}{t}\beta (t)^{\prime }\mathcal{L}(t)a(S^{\ast
}(t))\leq \frac{1}{t}\max_{j\in \{1,...,t\}}\max_{\tau \in
\{1,...,t\}}a_{j}(S_{j}^{O}(\tau ))\beta (t)^{\prime }\mathcal{L}(t)1(t),
\end{equation*}%
where $1(t)$ is a $t\times 1$ vector all of whose components are ones. Since
$\beta (t)^{\prime }\mathcal{L}(t)1(t)=\sum_{i,j=1}^{n}\beta _{j}\mathcal{L}%
_{ij}$ and $\sum_{j=1}^{\infty }\beta _{j}=1$, this implies%
\begin{eqnarray*}
\lim \sup_{t\rightarrow \infty }\left( -\frac{\pi (t)}{t}\right)
&=&\limsup_{t\rightarrow \infty }\frac{1}{t}\max_{j\in
\{1,...,t\}}\max_{\tau \in \{1,...,t\}}a_{j}(S_{j}^{O}(\tau ))\beta
(t)^{\prime }\mathcal{L}(t)1(t) \\
&\leq &\limsup_{t\rightarrow \infty }\frac{1}{1-\theta }\frac{1}{t}%
\max_{j\in \{1,...,t\}}\max_{\tau \in \{1,...,t\}}a_{j}(S_{j}^{O}(\tau )) \\
&=&\limsup_{t\rightarrow \infty }\frac{D}{1-\theta }\frac{\log _{2}(t^{2})}{t%
}=0\text{ almost surely,}
\end{eqnarray*}%
where the last equality follows from Assumption \ref{assumption
logproductivity negative}.

Since $\lim \inf_{t\rightarrow \infty }$ $\left( -\frac{\pi (t)}{t}\right)
\geq 0$ (as additional technology choices cannot increase prices), the
previous argument establishes that $g^{\ast }=\lim_{t\rightarrow \infty
} \left( -\frac{\pi (t)}{t}\right) =0$.
\end{proof}

\subsection*{Growth with Harrod-Neutral Technology and CES Production
Functions}

Consider the family of (modified) constant elasticity of substitution production functions with Harrod-neutral technology:\footnote{The qualifier ``modified'' refers to the fact that we are raising the distribution parameters, the $\alpha_{ij} $'s, to the power $1/\sigma $, which ensures that the unit cost functions are linear in the $\alpha_{ij}$'s.}
\begin{equation}
F_{i}(S_{i},A_{i}(S_{i}),L_{i},X_{i})=[(1-\sum_{j\in S_{i}}\alpha _{ij})^{%
\frac{1}{\sigma }}(A_{i}(S_{i})L_{i})^{\frac{\sigma -1}{\sigma }}+\sum_{j\in
S_{i}}\alpha _{ij}^{\frac{1}{\sigma }}X_{ij}^{\frac{\sigma -1}{\sigma }}]^{%
\frac{\sigma }{\sigma -1}}.  \label{Appendix CES}
\end{equation}

We next state and prove a theorem that shows that, when the production
functions are given by (\ref{Appendix CES}), the economy grows at a constant
rate. Even though in this case the asymptotic growth rate turns out to be
independent of the structure of the input-output network, the level of GDP
still depends on it. In this section of the Appendix, we also set
distortions equal to zero, i.e., $\mu =0$.

\begin{theorem}
\label{theorem CES appendix} Suppose that Assumptions \ref{assumption1prime}%
, \ref{assumption2prime} and \ref{assumption logproductivity negative} hold,
and that production functions are given by (\ref{Appendix CES}). Assume
further that distortions are zero and that each industry chooses its set of
suppliers $S_{i}^{\ast }(t)\subset \{1,\ldots ,t\}$. Then for each $%
i=1,2,\ldots ,t$, the equilibrium log price vector $p^{\ast }(t)$ satisfies,%
\begin{equation*}
\lim_{t\rightarrow \infty }-\frac{p_{i}^{\ast }(t)}{t}=D>0\text{ almost
surely},
\end{equation*}%
and thus%
\begin{equation*}
g^{\ast }=D\text{ almost surely.}
\end{equation*}
\end{theorem}

\begin{proof}[Proof of Theorem \protect\ref{theorem CES appendix}]
The cost function for industry $i$ is
\begin{equation*}
K_{i}(S_{i},A_{i}(S_i),P)=( (1-\sum_{j\in S_{i}}\alpha _{ij}) (\frac{1}{%
A_i(S_i)})^{1-\sigma} +\sum_{j\in S_{i}}\alpha _{ij}P_j^{1-\sigma })^{\frac{1%
}{1-\sigma }}.
\end{equation*}

Since distortions are equal to zero, we have $P^{\ast }=K(S^{\ast
},A(S^{\ast }),P^{\ast })$ so that $P_{i}^{\ast }=((1-\sum_{j\in
S_{i}}\alpha _{ij})(\frac{1}{A_{i}(S_{i})})^{1-\sigma }+\sum_{j\in
S_{i}}\alpha _{ij}(P_{j}^{\ast })^{1-\sigma })^{\frac{1}{1-\sigma }}$. It is
convenient to raise both sides in the previous equation to the $1-\sigma $
power to obtain the following system of linear equations in $Q^{\ast
}=((P_{1}^{\ast })^{1-\sigma },...,(P_{n}^{\ast })^{1-\sigma })$:%
\begin{equation*}
Q_{i}^{\ast }=(1-\sum_{j\in S_{i}^{\ast }}\alpha _{ij})(\frac{1}{%
A_{i}(S_{i}^{\ast })})^{1-\sigma }+\sum_{j\in S_{i}^{\ast }}\alpha
_{ij}Q_{j}^{\ast }.
\end{equation*}%
The solution to this set of equations can be written as%
\begin{equation*}
Q^{\ast }=(I-\alpha (S^{\ast }))^{-1}B
\end{equation*}%
where $B_{i}=(1-\sum_{j\in S_{i}^{\ast }}\alpha _{ij})(\frac{1}{%
A_{i}(S_{i}^{\ast })})^{1-\sigma }$. Write $A_{i}(S_{i}^{\ast
}(t))=e^{Dt+\epsilon _{i}(t)}$, where $D$ is as in Assumption \ref%
{assumption logproductivity negative} and $\lim_{t\rightarrow \infty }\frac{%
\epsilon _{i}(t)}{t}=0$ almost surely. We can use this to write $Q_{i}^{\ast
}(t)$ as
\begin{equation*}
Q_{i}^{\ast }(t)=\sum_{j=1}^{t}\mathcal{L}_{ij}(S^{\ast }(t))(1-\sum_{k\in
S_{j}^{\ast }(t)}\alpha _{jk})(e^{-(1-\sigma )(Dt+\epsilon _{j}(t))}).
\end{equation*}%
Since $1\leq \sum_{j=1}^{t}\mathcal{L}_{ij}(S^{\ast }(t))\leq \frac{1}{%
1-\theta }$ and $1-\theta \leq (1-\sum_{k\in S_{j}^{\ast }(t)}\alpha
_{jk})\leq 1$, we have that
\begin{equation*}
e^{-(1-\sigma )Dt-\max_{k\leq t}|(1-\sigma )\epsilon _{j}(t)|}(1-\theta
)\leq Q_{i}^{\ast }(t)\leq e^{-(1-\sigma )Dt+\max_{k\leq t}|(1-\sigma
)\epsilon _{j}(t)|}\frac{1}{1-\theta }.
\end{equation*}%
Taking logarithms, we obtain
\begin{equation*}
-(1-\sigma )Dt-\max_{k\leq t}|(1-\sigma )\epsilon _{j}(t)|+\log (1-\theta
)\leq (1-\sigma )p_{i}^{\ast }(t)\leq -(1-\sigma )Dt+\max_{k\leq
t}|(1-\sigma )\epsilon _{j}(t)|-\log (1-\theta ).
\end{equation*}%
Dividing by $t$ and taking the limit as $t$ goes to infinity, we obtain
\begin{equation*}
-(1-\sigma )D\leq (1-\sigma )\lim_{t\rightarrow \infty }\frac{p_{i}^{\ast
}(t)}{t}\leq -(1-\sigma )D
\end{equation*}%
almost surely. We conclude that $\lim_{t\rightarrow \infty }\frac{%
p_{i}^{\ast }(t)}{t}=D$ almost surely, and therefore $g^{\ast }=D$ almost
surely.
\end{proof}

\clearpage

\section*{Online Appendix C: Robustness Results}

\label{Appendix C} \setcounter{equation}{0}\setcounter{theorem}{0} %
\setcounter{lemma}{0} \setcounter{proposition}{0} \setcounter{corollary}{0} %
\renewcommand{\theequation}{C\arabic{equation}} \renewcommand{\thelemma}{C%
\arabic{lemma}} \renewcommand{\theproposition}{C\arabic{proposition}} %
\renewcommand{\thecorollary}{C\arabic{corollary}} \renewcommand{%
\thetheorem}{C\arabic{theorem}}

\pagenumbering{arabic}% resets `page` counter to 1
\renewcommand*{\thepage}{C-\arabic{page}}

In this Appendix, we report four sets of robustness checks on the results
presented in Table 1 in the text. First, we repeat the same regressions
using alternative definitions of significant change in input structure ---
dummies $J_{i,10}(t)$ and $J_{i,30}(t)$ computed analogously, but with
thresholds corresponding to the 10th and 30th percentiles of the
distribution of the Jaccard distance in that year. Next, we report
regressions that are weighted by the value added of the industry in question
in 1987 to give greater weight to larger industries. Finally, we limit the
sample to 1997-2002 so as to focus on the period in which the data are
consistently from the NAICS classification system. The results are broadly
similar to those reported in the text and imply similar counterfactual
aggregate TFP growth estimates.


\begin{table}[tbph]
\centering
{\small {  \input{../ReplicationFiles/Tables/big_table_tfp_10.tex}} }
\caption*{ \textbf{ Table C1: New input combinations and TFP (10th percentile threshold). }The table presents OLS estimates of the regression equation $\Delta \log
TFP_{i}(t)=\beta J_{i,10}(t)+\gamma _{i}+\nu (t)+\epsilon _{i}(t)$
using a dataset of five-year stacked-differences for 488 industries between
1987 and 2007. $J_{i,10}(t)$ is a dummy indicating the Jaccard distance
between the sets of inputs $S_{i}(t)$ and $S_{i}(t-1)$ being above the $%
10^{th}$ percentile of its distribution in that year. Column 1 only includes
period dummies. Column 2 adds industry-specific linear trends, the $\gamma
_{i}$'s. Column 3 adds lagged change in log TFP, $\Delta \log TFP_{i}(t-1)$.
Panel A is for the entire sample. Panel B focuses on manufacturing
industries and Panel C excludes computer industries (those within the three-digit SIC industries 357 and 367). Standard errors that are robust against arbitrary heteroscedasticity
and serial correlation at the level of industry are reported in parentheses.  }
\label{table big table}
\end{table}


\begin{table}[tbph]
\centering
{\small { \input{../ReplicationFiles/Tables/big_table_tfp_30.tex}} }
\caption*{ \textbf{Table C2: New input combinations and TFP ( 30th percentile  threshold ).} The table presents OLS estimates of the regression equation $\Delta \log
TFP_{i}(t)=\beta J_{i,30}(t)+\gamma _{i}+\nu (t)+\epsilon _{i}(t)$
using a dataset of five-year stacked-differences for 488 industries between
1987 and 2007. $J_{i,30}(t)$ is a dummy indicating the Jaccard distance
between the sets of inputs $S_{i}(t)$ and $S_{i}(t-1)$ being above the $%
30^{th}$ percentile of its distribution in that year. Column 1 only includes
period dummies. Column 2 adds industry-specific linear trends, the $\gamma
_{i}$'s. Column 3 adds lagged change in log TFP, $\Delta \log TFP_{i}(t-1)$.
Panel A is for the entire sample. Panel B focuses on manufacturing
industries and Panel C excludes computer industries (those within the three-digit SIC industries 357 and 367). Standard errors that are robust against arbitrary heteroscedasticity
and serial correlation at the level of industry are reported in parentheses.   }
\label{table big table}
\end{table}

\begin{table}[tbp]
\centering
{\small { \input{../ReplicationFiles/Tables/big_table_tfp_weighted.tex}} }
\caption*{\textbf{Table C3: New input combinations and TFP (Value-Added Weighted Regressions).} The table presents weighted OLS estimates of the regression equation $\Delta
\log TFP_{i}(t)=\beta J_{i,20}(t)+\gamma _{i}+\nu (t)+\epsilon
_{i}(t)$ using a dataset of five-year stacked-differences for 488 industries
between 1987 and 2007 and value added of the industry in 1987 as weight. $%
J_{i,20}(t)$ is a dummy indicating the Jaccard distance between the sets of
inputs $S_{i}(t)$ and $S_{i}(t-1)$ being above the $20^{th}$ percentile of
its distribution in that year. Column 1 only includes period dummies. Column
2 adds industry-specific linear trends, the $\gamma _{i}$'s. Column 3 adds
lagged change in log TFP, $\Delta \log TFP_{i}(t-1)$. Panel A is for the
entire sample. Panel B focuses on manufacturing industries and Panel C
excludes computer industries (those within the three-digit SIC industries 357 and 367). Standard errors
that are robust against arbitrary heteroscedasticity and serial correlation
at the level of industry are reported in parentheses. }
\label{table big table}
\end{table}

\begin{table}[tbp]
\centering
{\small {  \input{../ReplicationFiles/Tables/big_table_tfp_1997.tex}} }
\caption*{\textbf{Table C4: New input combinations and TFP (1997-2007). }The table presents OLS estimates of the regression equation $\Delta \log
TFP_{i}(t)=\beta J_{i,20}(t)+\gamma _{i}+\nu (t)+\epsilon _{i}(t)$
using a dataset of five-year stacked-differences for 488 industries between
1997 and 2007. $J_{i,20}(t)$ is a dummy indicating the Jaccard distance
between the sets of inputs $S_{i}(t)$ and $S_{i}(t-1)$ being above the $%
20^{th}$ percentile of its distribution in that year. Column 1 only includes
period dummies. Column 2 adds industry-specific linear trends, the $\gamma
_{i}$'s. Column 3 adds lagged change in log TFP, $\Delta \log TFP_{i}(t-1)$.
Panel A is for the entire sample. Panel B focuses on manufacturing
industries and Panel C excludes computer industries (those within the three-digit SIC industries 357 and 367). Standard errors that are robust against arbitrary heteroscedasticity
and serial correlation at the level of industry are reported in parentheses.   }
\label{table big table}
\end{table}


\clearpage

\section*{Online Appendix D: Details of the Quantitative Exercise}

\setcounter{equation}{0}\setcounter{theorem}{0} \setcounter{lemma}{0} %
\setcounter{proposition}{0} \setcounter{corollary}{0} \renewcommand{%
\theequation}{D\arabic{equation}} \renewcommand{\thelemma}{C\arabic{lemma}} %
\renewcommand{\theproposition}{C\arabic{proposition}} \renewcommand{%
\thecorollary}{C\arabic{corollary}} \renewcommand{\thetheorem}{C%
\arabic{theorem}}

\pagenumbering{arabic}% resets `page` counter to 1
\renewcommand*{\thepage}{D-\arabic{page}}

We now describe the details of the quantitative exercise discussed in
Section \ref{section comparative statics}. We start with a disaggregated
economy with Cobb-Douglas sectoral technologies and an endogenous
input-output structure (with extensive margin choices about inputs)
calibrated to the 2007 US input-output tables from the BEA. We then compare the response of this economy to
an increase in the TFP of a sector to the response of more aggregated models
(using both Cobb-Douglas and CES production functions) calibrated to the
same data.

In this quantitative exercise, we parametrize the sectoral production
functions as follows:
\begin{equation*}
Y_{i}=A_{i}(S_{i})F_{i}(X_{i},L_{i},S_{i})
\end{equation*}%
and
\begin{equation*}
A_{i}(S_{i})=B_{i0}\prod_{j\in S_{i}}B_{ij}.
\end{equation*}%
Then, denoting $b_{ij}=\log B_{ij}$, the log cost function for industry $i$
is%
\begin{equation*}
k_{i}(p,a_{i}(S_{i}))=\sum_{j\in S_{i}}(p_{j}\alpha _{ij}-b_{ij})-b_{i0}.
\end{equation*}

With this parametrization, industry $i$ will adopt industry $j$ as a
supplier if and only if $b_{ij}\geq p_{j}\alpha _{ij}$ (adopting the
convention that an industry adopts an input when indifferent).\footnote{%
We also note that, though we are keeping the number of industry fixed here,
this specification is consistent with sustained growth when the number of
industries changes as in Section \ref{section dynamic}. In particular, if $%
b_{i0},b_{i1},b_{i2},b_{i3},...$ are drawn independently so that $%
\Pr (b_{ij}>\delta _{i})>\epsilon _{i}$ for some constants $\delta
_{i},\epsilon _{i}>0$, then Assumption \ref{assumption logproductivity negative} is satisfied
(because $\liminf_{t\rightarrow \infty }a_{i}(S_{i}(t))\geq \delta
_{i}\epsilon _{i}>0$ almost surely).}

We further assume that in both the disaggregated and the aggregated economies,
the preferences of the representative household are Cobb-Douglas as in
Assumption \ref{assumption1prime}.

\subsection*{Disaggregated Economy with Endogenous Production Network}

We calibrate our model economy to 2007 US input-output tables from the BEA,
which comprise 391 sectors. As noted in footnote \ref{footnote data}, we
exclude the government sector (consisting of nine industries in the input-output tables), privately-owned residential property and the
sector made up of custom duties (the latter two have zero labor share).
Throughout, GDP refers to the sum of value added of the remaining sectors.

We choose the parameters of the model as follows: for any edge $(i,j)$
observed in the input-output matrix, $\alpha _{ij}$ is set equal to the
observed $(i,j)^{th}$ entry in the input-output matrix. For any edge $(i,j)$
not observed in the data, $\alpha _{ij}$ is set equal to $\alpha
_{ij}=0.95\cdot (1-\sum_{j^{\prime }\in S_{i}}\alpha _{ij^{\prime }})\frac{%
\sum_{i^{\prime }:j\in S_{i^{\prime }}}\alpha _{i^{\prime }j}}{%
\sum_{i^{\prime },j^{\prime }:j^{\prime }\in S_{i^{\prime }}}\alpha
_{i^{\prime }j^{\prime }}}$. This choice ensures that (1) all observed edges
have cost shares equal to the cost shares in the data; (2) all edges that
are absent in the 2007 input-output matrix have cost shares proportional to
the observed outdegree of the supplier; and (3) the row sums of the full
input-output matrix (including the edges that are absent in 2007) are less
than 1 so that labor is an essential input as required by Assumption \ref%
{assumption 1}.

The $\beta _{i}$'s are set equal to each industry's consumption share.

We take distortions from the markup estimates of De Loecker, Eeckhout and
Unger (2018), which are at the two-digit level. We apply the same distortion
to all subindustries in the same two-digit industry.

We assume that the $b_{ij}$'s are drawn from truncated Normal distributions,
where the truncation ensures that the productivity of the edge is consistent
with its presence or absence in the input-output tables. More specifically,
each $b_{i0}$ is drawn independently from a Normal prior with mean $m$ and
standard deviation 1. The parameter $m$ is chosen so that equilibrium GDP
matches US GDP (which is computed from the BEA input-output tables as
11.563 trillion 2007 dollars, excluding the government sector, owner-occupied residential
housing and custom duties).

We implement the truncation procedure for $b_{ij}$'s as follows. Each is
drawn independently from a Normal prior with mean $\frac{m}{n}$ and standard
deviation $\frac{1}{n}$ (where $n=391$ is the number of industries).

\begin{enumerate}
\item We draw $b_{i0},b_{i1},\ldots ,b_{in}$ from the prior distributions
described above.

\item We set $a_{i}(S_{i})=b_{i0}+\sum_{j\in S_{i}}b_{ij}$.

\item We compute $p=(I-\alpha )^{-1}(\log (1+\mu )-a(S))$.

\item We repeat the following steps until $b_{ij}\geq \alpha _{ij}p_{j}$ for
all $i\in \{1,...n\}$ and all $j\in S_{i}$, and $b_{ij}<\alpha _{ij}p_{j}$
for all $i\in \{1,...,n\}$ and all $j\not\in S_{i}$:

\begin{enumerate}
\item If $j \in S_i$ and $b_{ij} < \alpha_{ij} p_j$, then redraw $b_{ij}$
from a truncated Normal distribution (with the same parameters as above) with support over the interval $%
[\alpha_{ij}p_j,\infty)$.

\item If $j\not\in S_{i}$ and $b_{ij}>\alpha _{ij}p_{j}$, then redraw $%
b_{ij} $ from a truncated Normal distribution with the same parameters as above but now with support over the interval $(-\infty
,\alpha _{ij}p_{j}]$.

\item If $j\in S_{i}$ and $b_{ij}\geq \alpha _{ij}p_{j}$ or $j\not\in S_{i}$
and $b_{ij}\leq \alpha _{ij}p_{j}$, then keep $b_{ij}$.

\item Recompute $a_i(S_i) = b_{i0} + \sum_{j \in S_i} b_{ij}$ and $p =
(I-\alpha)^{-1} ( \log(1+\mu)-a(S))$.
\end{enumerate}
\end{enumerate}

This procedure yields two posterior distributions, one for $b_{ij}$
conditional on $j\in S_{i}$ and another for $b_{ij}$ conditional on $%
j\not\in S_{i}$. Figures 3 and 4 depict these conditional distributions.

\begin{figure}[tbph]
\label{histogram positive}
\centering
\includegraphics{../ReplicationFiles/Figures/PositiveHistogram.jpg}

{\textbf{\ Figure
3: Distribution of edge-specific productivities $b_{ij}$ conditional on edge $(i,j)$ being observed in the 2007 US input-output matrix.}}
\end{figure}

\begin{figure}[tbph]
\label{histogram negative}
\centering
\includegraphics{../ReplicationFiles/Figures/NegativeHistogram.jpg}

{\textbf{\ Figure
4: Distribution of edge-specific productivities $b_{ij}$ conditional on edge $(i,j)$ being absent in the 2007 US input-output matrix.}}
\end{figure}




Once the productivity parameters, the $a_{i}(S_{i})$'s, have been sampled,
we compute log prices from equation (\ref{Leontief equation}) in the text as%
\begin{equation*}
p=-(I-\alpha )^{-1}(a-\log (1+\mu )).
\end{equation*}

To compute nominal GDP, we assume that all revenues generated by distortions
are rebated to households (i.e., $\lambda _{j}=1$ for all industries), which
is consistent with our use of markup data to choose the level of
distortions. Since utility and production functions are Cobb-Douglas, we
have
\begin{equation}
P_{i}C_{i}=\beta _{i}(1+\sum_{i=1}^{n}\frac{\mu _{i}}{1+\mu _{i}}P_{i}Y_{i}).
\label{consumption expenditure in appendix}
\end{equation}%
Letting $GDP^{N}$ denote nominal $GDP$ and $d_{i}=\frac{P_{i}Y_{i}}{GDP^{N}}$
denote the Domar weight for industry $i$, (\ref{consumption expenditure in
appendix}) can be written as

\begin{equation*}
P_{i}C_{i}=\beta _{i}(1+\sum_{i=1}^{n}\frac{\mu _{i}}{1+\mu _{i}}d_{i}\cdot
GDP^{N}).
\end{equation*}%
Summing this equation over all industries, we obtain nominal GDP in terms of
Domar weights as%
\begin{equation}
\label{equation nominal}
GDP^{N}=\frac{1}{1-\sum_{i=1}^{n}\frac{\mu _{i}}{1+\mu _{i}}d_{i}}.
\end{equation}%
Moreover, we can also express the Domar weights in terms of input-output
entries. Let $\hat{\alpha}_{ji}=\frac{P_{j}X_{ji}}{P_{i}Y_{i}}$ denote the
amount (in dollars) of good $j$ necessary to produce one dollar's worth of
good $i$. Note that this is different from $\alpha _{ji}$ which is the cost
share of input $i$ in the production of good $j$ (the fraction of the cost
of good $j$ that goes to input $i$). In particular,
\begin{gather}\hat{\alpha}_{ji}=\frac{%
\alpha _{ji}}{1+\mu _{j}}.
\end{gather}
 Rearranging the market-clearing condition for
industry $i$,
\begin{equation*}
P_{i}Y_{i}=P_{i}C_{i}+\sum_{j\in S_{i}}P_{j}X_{ji},
\end{equation*}%
we can write

\begin{equation}
P_{i}Y_{i}=P_{i}C_{i}+\sum_{j\in S_{i}}\hat{\alpha}_{ji}P_{i}Y_{i}.
\label{expenditure equation appendix}
\end{equation}%
Dividing both sides of equation (\ref{expenditure equation appendix}) by
nominal GDP, we get%
\begin{equation*}
d_{i}=\beta _{i}+\sum_{j\in S_{i}}\hat{\alpha}_{ji}d_{i}
\end{equation*}%
\begin{equation}
\label{equation domar cost}
d=(I-\hat{\alpha}^{\prime })^{-1}\beta .
\end{equation}

Given the Domar weights and nominal GDP, we can compute real GDP from
equation (\ref{real GDP}) as%
\begin{equation*}
Y(t)=\frac{Y^{N}(t)}{\prod_{i=1}^{t}P_{i}(t)^{\beta _{i}}}.
\end{equation*}
Taking logarithms on both sides, and using equations  (\ref{Leontief equation}), (\ref{equation nominal}) and (\ref{equation domar cost}), this becomes
\begin{equation}
\log Y(t) =  \beta^{\prime} (I-\alpha)^{-1} (a-\log(1+\mu)) - \log(1-\sum_{i=1}^{n}\frac{\mu _{i}}{1+\mu _{i}}d_{i}).
\end{equation}

We use this formula to compute real GDP and calibrate the parameter $m$ to
match GDP in 2007. In this equilibrium, as in the US input-output tables in 2007, 32.54\% of all
possible edges are present.

We then increase TFP in the computer and electronic product manufacturing sector (NAICS 334). As noted in the text, this sector makes up 1.98\% of US GDP in 2007. More specifically, for each one of the 20 detailed industries in the
(two-digit) computer and electronic product manufacturing sector, we
increase $b_{0,i}$ by 1\%. We then compute the implied increase in real GDP.

Our algorithm for computing new equilibrium prices and input-output matrix
is as follows:

\begin{enumerate}
\item Let $\alpha (0)$ be the input-output matrix in the original economy,
and let $S_{i}(0)$ be the original network. Let $a_{i}(0)=b_{i0}+\sum_{j\in
S_{i}(0)}$ and let $p(0)=(I-\alpha (0))^{-1}(a(0)-\log (1+\mu ))$. Finally,
let $\alpha $ (with no time argument) denote the full input-output matrix,
including the entries for edges not observed in the 2007 data.

\item Initialize at $t=0$ and repeat until prices converge. In particular:

\begin{enumerate}
\item If $b_{ij}>p_{j}(t)\alpha _{ij}$, set $\alpha _{ij}(t+1)=\alpha _{ij}$%
. Update $S_{i}(t+1)$ to include $j$. Note that $S_{i}(t+1)\supseteq
S_{i}(t) $ because, from Theorem \ref{theorem network increasing}, an
increase in $b_{i0}$ (which is a positive technology shock) or a decrease in
prices always expands the equilibrium production network.

\item Set $a(t+1)=b_{i0}+\sum_{j\in S_{i}(t+1)}b_{ij}$.

\item Set $p(t+1)=(I-\alpha (t+1))^{-1}(a(t+1)-\log (1+\mu ))$.
\end{enumerate}
\end{enumerate}


We find that the new equilibrium has 288  additional edges, so that now
32.73\% of edges in the input-output matrix are present. In this new
equilibrium, real GDP increases by 0.72\%. Of this increase, 0.13 percentage
points come from greater value added in the computer and electronic product
manufacturing sector. The remaining 0.59 percentage points originate from
other sectors expanding their output as they face lower prices and add
additional suppliers.

\subsection*{Aggregate Economies with Exogenous Production Network}

We now repeat the same exercise but for three more aggregated economies with
84 industries (at the three-digit NAICS level). Crucially, these aggregated
economies do not feature an extensive margin of adjustment in their
input-output structure (hence \textquotedblleft exogenous\textquotedblright\
production networks). One of those economies has Cobb-Douglas production
technologies and the other two have CES technologies, with elasticity of
substitution parameters $\sigma =1/2$ and $\sigma =2$, respectively. All
three economies are calibrated to the 2007 US input-output tables (at the
same level of aggregation) and thus have the same baseline equilibrium as
our disaggregated economy. Nevertheless, we show that they generate very
different responses to the increase in the TFP of the computers and
electronic product manufacturing sector --- because there are no extensive
margin changes in the production network.

Even though there are no such extensive margin changes in input-output
linkages, in the CES economy changes in equilibrium prices will lead to
changes in equilibrium input quantities and thus in the entries of the
input-output matrix. Nevertheless, the increase in equilibrium GDP will be
small in both the Cobb-Douglas and CES aggregated economies.

\paragraph{Aggregation Procedure:}

We first describe how we consistently aggregate from the more disaggregated
economy. Our procedure closely follows Acemoglu, Ozdaglar and Tahbaz-Salehi
(2017), except that we adapt their formulae to include markups.\footnote{%
We are grateful to Alireza Tahbaz-Salehi for help and suggestions on this
point.} We use capital letters $(I,J,...)$ to denote \textquotedblleft
sectors\textquotedblright\ (short for aggregated sectors) and lowercase
levels $(i,j,...)$ to denote \textquotedblleft industries\textquotedblright\
(short for disaggregated industries), with the convention that $i$ is a
disaggregated industry that is part of the aggregated sector $I$. We denote
the number of aggregated sectors by $N$ and the number of industries by $n$.

We begin by aggregating the BEA input-output tables, markups and our imputed
values for $b_{i0},b_{ij}$. For each sector $I$, the aggregation process
should not change the following quantities: (1) households' consumption expenditure on sector $I$;
(2) sector $I$'s total output; (3) sector $I$'s profits; (4) sector $I$'s
expenditure on intermediate goods from sector $J$; (5) sector $I$'s
expenditure on labor; and (6) real GDP. More formally, denoting value added
in industry $i$ by $v_{i}$, these requirements imply:

\begin{align}
P_I C_I &= \sum_{i \in I} P_i C_i  \label{appendix consumption} \\
P_I Y_I &= \sum_{i \in I} P_i Y_i  \label{appendix output} \\
\Pi_I &= \sum_{i \in I} \Pi_i  \label{appendix profit} \\
P_J X_{IJ} &= \sum_{i \in I, j \in J} P_j X_{ij}
\label{appendix intermediate} \\
W L_I &= \sum_{i \in I} W L_i  \label{appendix labor} \\
\sum_{I=1}^N v_I &= \sum_{i=1}^n v_i  \label{appendix value added}
\end{align}

Because the household's utility is Cobb-Douglas, equation (\ref{appendix
consumption}) implies that
\begin{gather}\beta _{I}=\sum_{i\in I}\beta _{i}.\end{gather} Let $d_{i}=%
\frac{P_{i}Y_{i}}{GDP}$ and let $d_{I}=\frac{P_{I}Y_{I}}{GDP}$ represent
industry and sector-level Domar weights. Equation (\ref{appendix output})
then implies that
\begin{gather}d_{I}=\sum_{i\in I}d_{i}.\end{gather} Denoting sectoral markups by $%
\mu _{I}$, equation (\ref{appendix profit}) yields:%
\begin{equation*}
\frac{\mu _{I}}{1+\mu _{I}}P_{I}Y_{I}=\sum_{i\in I}\frac{\mu _{i}}{1+\mu _{i}%
}P_{i}Y_{i}.
\end{equation*}%
Dividing both sides by $GDP$ and rearranging terms, we obtain
\begin{equation}
\frac{\mu _{I}}{1+\mu _{I}}=\frac{1}{d_{I}}\sum_{i\in I}\frac{\mu _{i}}{%
1+\mu _{i}}d_{i}.
\end{equation}

To derive the aggregate input-output matrix $(\alpha _{IJ})_{I,J=1}^{N}$,
begin with equation (\ref{appendix intermediate}) and multiply both sides by
$\frac{1}{P_{I}Y_{I}}\frac{P_{I}Y_{I}}{GDP}$, which gives

\begin{equation}
\frac{\alpha _{IJ}}{1+\mu _{I}}d_{I}=\sum_{i\in I,j\in J}\frac{\alpha _{ij}}{%
1+\mu _{i}}d_{i},
\end{equation}%
where we have use the fact that $\frac{\alpha _{ij}}{1+\mu _{i}}=\frac{%
P_{j}X_{ij}}{P_{i}Y_{i}}$.

The labor aggregation condition (\ref{appendix labor}) implies $%
L_{I}=\sum_{i\in I}L_{i}$.

Finally, we need to derive sectoral-level TFPs from industry-level TFPs. In
doing this, $GDP$ and the price deflator $e^{-\beta ^{\prime }\mathcal{L}%
(a-\log (1+\mu ))}$ have to be invariant to aggregation.\footnote{%
Nominal GDP is also invariant to aggregation since $d_{I}\frac{\mu _{I}}{%
1+\mu _{I}}=\sum_{i\in I}\frac{\mu _{i}}{1+\mu _{i}}d_{i}$ and $GDP^{N}=%
\frac{1}{1-\sum_{i=1}^{n}\frac{\mu _{i}}{1+\mu _{i}}d_{i}}=\frac{1}{%
1-\sum_{I=1}^{N}\frac{\mu _{I}}{1+\mu _{I}}d_{I}}$.} Let
\begin{gather}\tilde{d}%
_{j}=\sum_{i=1}^{n}\beta _{i}\mathcal{L}_{ij},\tilde{d}_{J}=\sum_{I=1}^{N}%
\beta _{I}\mathcal{L}_{IJ}\end{gather}
represent the industry and sectoral \emph{%
cost-based Domar weights}.\footnote{%
We borrow this terminology from Baqaee and Farhi (2017). One can show that
the standard Domar weights can be computed as $\beta ^{\prime }(I-\hat{\alpha%
})^{-1}$, where $\hat{\alpha}_{ij}=\frac{P_{j}X_{ij}}{P_{i}Y_{i}}$ is the
revenue-based input-output matrix. The cost-based Domar weights are computed
with the analogous formula, but using the cost-based input-output matrix
instead. When distortions/markups are zero, the two types of Domar weights
coincide.} These can be computed from the industry and sectoral cost-based
input-output matrices, $(\alpha _{ij})_{i,j=1}^{n},(\alpha
_{I,J})_{I,J=1}^{N})$, respectively. Then the price deflators for the
disaggregated and aggregate economies are, respectively, $e^{-\sum_{i=1}^{n}%
\tilde{d_{i}}(a_{i}-\log (1+\mu _{i}))}$ and $e^{-\sum_{I=1}^{N}\tilde{d_{I}}%
a_{I}-\log (1+\mu _{I})}$. Because of these two expressions have to
coincide, we derive our last restriction as%
\begin{equation}
a_{I}=\frac{1}{\tilde{d}_{I}}\sum_{i\in I}\tilde{d}_{i}a_{i}-\frac{1}{\tilde{%
d}_{I}}\tilde{d}_{i}\log (1+\mu _{i})+\log (1+\mu _{I}).
\end{equation}

\paragraph{The Aggregated Cobb-Douglas Economy:}

The above aggregation procedure conserves household and firm expenditures,
firm profits and GDP. We use it to aggregate the input-output matrix from
the BEA data.\footnote{%
Our markups are already at the two-digit level, so do not need to be
aggregated.} We also aggregate sectoral TFPs to three-digit NAICS sectoral
level with the procedure described above. We then compute equilibrium prices
and GDP for the aggregated Cobb-Douglas economy (which naturally coincide
with GDP in the disaggregated economy).

We then treat this aggregated Cobb-Douglas economy as primitive and
introduce the same 1\% TFP increase in the (two-digit) computer and
electronic product manufacturing sector. Following this change in TFP, there
is no extensive margin change in the input-output structure of the economy (by construction),
and since we have Cobb-Douglas production technologies, the entries of the
input-output matrix do not change either. We then compute the resulting
changes in prices and quantities and real GDP. We find that real GDP
increases by 0.04\% (as compared to 0.72\% in the disaggregated economy with
endogenous input-output linkages).

\paragraph{Aggregated CES Economies:}

We repeat the same procedure for aggregated CES economies. In this case, we
use the aggregated $\alpha _{IJ}$'s as parameters for constant elasticity of
substitution sectoral production functions as in equation (11) in the text. We initialize sectoral TFPs at the levels computed for the
aggregated Cobb-Douglas economy. We then raise the TFP of the computer and
electronic product manufacturing sector by 1\% and compute the change in
real GDP in the same way.

Following the TFP shock, there is again by construction no extensive margin change in
the input-output structure of the economy, but because the elasticity of
substitution between inputs is no longer equal to 1, entries of the
input-output matrix change as prices change. Nevertheless, we find that the
implied increases in real GDP are again small (as in the aggregated
Cobb-Douglas economy). In particular, when the elasticity of substitution
between inputs is $\sigma =1/2$, the 1\% TFP increase in the computer and electronic product manufacturing sector leads to a $0.09\%$ increase in real GDP. The same shock leads to a $0.01\%$ increase in GDP when $\sigma = 2$.

\end{document}
